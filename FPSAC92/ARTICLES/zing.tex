\input amstex.tex
\documentstyle{amsppt}
\NoRunningHeads
\nopagenumbers
\catcode`\@=11
\font@\twelverm=cmr10 scaled\magstep1
\font@\twelveit=cmti10 scaled\magstep1
\font@\twelvesl=cmsl10 scaled\magstep1
\font@\twelvesmc=cmcsc10 scaled\magstep1
\font@\twelvett=cmtt10 scaled\magstep1
\font@\twelvebf=cmbx10 scaled\magstep1
\font@\twelvei=cmmi10 scaled\magstep1
\font@\twelvesy=cmsy10 scaled\magstep1
\font@\twelveex=cmex10 scaled\magstep1
\font@\twelvemsa=msam10 scaled\magstep1
\font@\twelveeufm=eufm10 scaled\magstep1
\font@\twelvemsb=msbm10 scaled\magstep1
\newtoks\twelvepoint@
\font@\fourteenrm=cmr10 scaled\magstep2
\font@\fourteenit=cmti10 scaled\magstep2
\font@\fourteensl=cmsl10 scaled\magstep2
\font@\fourteensmc=cmcsc10 scaled\magstep2
\font@\fourteentt=cmtt10 scaled\magstep2
\font@\fourteenbf=cmbx10 scaled\magstep2
\font@\fourteeni=cmmi10 scaled\magstep2
\font@\fourteensy=cmsy10 scaled\magstep2
\font@\fourteenex=cmex10 scaled\magstep2
\font@\fourteenmsa=msam10 scaled\magstep2
\font@\fourteeneufm=eufm10 scaled\magstep2
\font@\fourteenmsb=msbm10 scaled\magstep2
\newtoks\fourteenpoint@
\def\fourteenpoint{\normalbaselineskip15\p@
 \abovedisplayskip18\p@ plus4.3\p@ minus12.9\p@
 \belowdisplayskip\abovedisplayskip
 \abovedisplayshortskip\z@ plus4.3\p@
 \belowdisplayshortskip10.1\p@ plus4.3\p@ minus5.8\p@
 \textonlyfont@\rm\fourteenrm \textonlyfont@\it\fourteenit
 \textonlyfont@\sl\fourteensl \textonlyfont@\bf\fourteenbf
 \textonlyfont@\smc\fourteensmc \textonlyfont@\tt\fourteentt
 \ifsyntax@ \def\big##1{{\hbox{$\left##1\right.$}}}%
  \let\Big\big \let\bigg\big \let\Bigg\big
 \else
  \textfont\z@=\fourteenrm  \scriptfont\z@=\twelverm  
\scriptscriptfont\z@=\tenrm
  \textfont\@ne=\fourteeni  \scriptfont\@ne=\twelvei  
\scriptscriptfont\@ne=\teni
  \textfont\tw@=\fourteensy \scriptfont\tw@=\twelvesy 
\scriptscriptfont\tw@=\tensy
  \textfont\thr@@=\fourteenex \scriptfont\thr@@=\twelveex
        \scriptscriptfont\thr@@=\twelveex
  \textfont\itfam=\fourteenit \scriptfont\itfam=\twelveit
        \scriptscriptfont\itfam=\twelveit
  \textfont\bffam=\fourteenbf \scriptfont\bffam=\twelvebf
        \scriptscriptfont\bffam=\tenbf
  \setbox\strutbox\hbox{\vrule height12.2\p@ depth5\p@ width\z@}%
  \setbox\strutbox@\hbox{\lower.72\normallineskiplimit\vbox{%
        \kern-\normallineskiplimit\copy\strutbox}}%
 \setbox\z@\vbox{\hbox{$($}\kern\z@}\bigsize@=1.7\ht\z@
 \fi
 \normalbaselines\rm\ex@.2326ex\jot4.3\ex@\the\fourteenpoint@}

\font@\seventeenrm=cmr10 scaled\magstep3
\font@\seventeenit=cmti10 scaled\magstep3
\font@\seventeensl=cmsl10 scaled\magstep3
\font@\seventeensmc=cmcsc10 scaled\magstep3
\font@\seventeentt=cmtt10 scaled\magstep3
\font@\seventeenbf=cmbx10 scaled\magstep3
\font@\seventeeni=cmmi10 scaled\magstep3
\font@\seventeensy=cmsy10 scaled\magstep3
\font@\seventeenex=cmex10 scaled\magstep3
\font@\seventeenmsa=msam10 scaled\magstep3
\font@\seventeeneufm=eufm10 scaled\magstep3
\font@\seventeenmsb=msbm10 scaled\magstep3
\newtoks\seventeenpoint@
\def\seventeenpoint{\normalbaselineskip18\p@
 \abovedisplayskip21.6\p@ plus5.2\p@ minus15.4\p@
 \belowdisplayskip\abovedisplayskip
 \abovedisplayshortskip\z@ plus5.2\p@
 \belowdisplayshortskip12.1\p@ plus5.2\p@ minus7\p@
 \textonlyfont@\rm\seventeenrm \textonlyfont@\it\seventeenit
 \textonlyfont@\sl\seventeensl \textonlyfont@\bf\seventeenbf
 \textonlyfont@\smc\seventeensmc \textonlyfont@\tt\seventeentt
 \ifsyntax@ \def\big##1{{\hbox{$\left##1\right.$}}}%
  \let\Big\big \let\bigg\big \let\Bigg\big
 \else
  \textfont\z@=\seventeenrm  \scriptfont\z@=\fourteenrm  
\scriptscriptfont\z@=\twelverm
  \textfont\@ne=\seventeeni  \scriptfont\@ne=\fourteeni  
\scriptscriptfont\@ne=\twelvei
  \textfont\tw@=\seventeensy \scriptfont\tw@=\fourteensy 
\scriptscriptfont\tw@=\twelvesy
  \textfont\thr@@=\seventeenex \scriptfont\thr@@=\fourteenex
        \scriptscriptfont\thr@@=\fourteenex
  \textfont\itfam=\seventeenit \scriptfont\itfam=\fourteenit
        \scriptscriptfont\itfam=\fourteenit
  \textfont\bffam=\seventeenbf \scriptfont\bffam=\fourteenbf
        \scriptscriptfont\bffam=\twelvebf
  \setbox\strutbox\hbox{\vrule height14.6\p@ depth6\p@ width\z@}%
  \setbox\strutbox@\hbox{\lower.86\normallineskiplimit\vbox{%
        \kern-\normallineskiplimit\copy\strutbox}}%
 \setbox\z@\vbox{\hbox{$($}\kern\z@}\bigsize@=2\ht\z@
 \fi
 \normalbaselines\rm\ex@.2326ex\jot5.2\ex@\the\seventeenpoint@}

\newif\iftab@\tab@false
\newif\ifvtab@\vtab@false
\def\tab{\bgroup\tab@true\vtab@false\vst@bfalse\Strich@false%
\def\\{\global\hline@@false%
\ifhline@\global\hline@false\global\hline@@true\fi\cr}
   \edef\l@{\the\leftskip}\ialign\bgroup\hskip\l@##\hfil&&##\hfil\cr}
\def\endtab{\cr\egroup\egroup}
\def\vtab{\vtop\bgroup\vst@bfalse\vtab@true\tab@true\Strich@false%
   \bgroup\def\\{\cr}\ialign\bgroup&##\hfil\cr}
\def\endvtab{\cr\egroup\egroup\egroup}
\def\stab{\D@cke0.5pt\catcode`\^^I=4 
 \bgroup\tab@true\vtab@false\vst@bfalse\Strich@true\Let@@\vspace@
 \normalbaselines\offinterlineskip
  \openup\spreadmlines@
 \edef\l@{\the\leftskip}\ialign
 \bgroup\hskip\l@##\hfil&&##\hfil\crcr}
\def\endstab{\crcr\egroup
 \egroup\catcode`\^^I=10 }
\newif\ifvst@b\vst@bfalse
\def\vstab{\D@cke0.5pt\catcode`\^^I=4 
 \vtop\bgroup\tab@true\vtab@false\vst@btrue\Strich@true\bgroup\Let@@\vspace@
 \normalbaselines\offinterlineskip
  \openup\spreadmlines@\bgroup}
\def\endvstab{\crcr\egroup\egroup
 \egroup\tab@false\Strich@false\catcode`\^^I=10 }

\newdimen\htstrut@
\htstrut@8.5\p@
\newdimen\htStrut@
\htStrut@12\p@
\newdimen\dpstrut@
\dpstrut@3.5\p@
\newdimen\dpStrut@
\dpStrut@3.5\p@
\def\openup{\afterassignment\@penup\dimen@=}
\def\@penup{\advance\lineskip\dimen@
  \advance\baselineskip\dimen@
  \advance\lineskiplimit\dimen@
  \divide\dimen@ by2
  \advance\htstrut@\dimen@
  \advance\htStrut@\dimen@
  \advance\dpstrut@\dimen@
  \advance\dpStrut@\dimen@}
\def\Let@@{\relax\iffalse{\fi%
    \def\\{\global\hline@@false%
     \ifhline@\global\hline@false\global\hline@@true\fi\cr}%
    \iffalse}\fi}
\def\matrix{\null\,\vcenter\bgroup
 \tab@false\vtab@false\vst@bfalse\Strich@false\Let@@\vspace@
 \normalbaselines\openup\spreadmlines@\ialign
 \bgroup\hfil$\m@th##$\hfil&&\quad\hfil$\m@th##$\hfil\crcr
 \Mathstrut@\crcr\noalign{\kern-\baselineskip}}
\def\endmatrix{\crcr\Mathstrut@\crcr\noalign{\kern-\baselineskip}\egroup
 \egroup\,}
\def\smatrix{\D@cke0.5pt\null\,
 \vcenter\bgroup\tab@false\vtab@false\vst@bfalse\Strich@true\Let@@\vspace@
 \normalbaselines\offinterlineskip
  \openup\spreadmlines@\ialign
 \bgroup\hfil$\m@th##$\hfil&&\quad\hfil$\m@th##$\hfil\crcr}
\def\endsmatrix{\crcr\egroup
 \egroup\,\Strich@false}
\newdimen\D@cke
\def\Dicke#1{\global\D@cke#1}
\newtoks\tabs@\tabs@{&}
\newif\ifStrich@\Strich@false
\newif\iff@rst

\def\Stricherr@{\iftab@\ifvtab@\errmessage{\noexpand\s not allowed
     here. Use \noexpand\vstab!}%
  \else\errmessage{\noexpand\s not allowed here. Use \noexpand\stab!}%
  \fi\else\errmessage{\noexpand\s not allowed
     here. Use \noexpand\smatrix!}\fi}
\def\format{\ifvst@b\else\crcr\fi\egroup\iffalse{\fi\ifnum`}=0 \fi\format@}
\def\format@#1\\{\def\preamble@{#1}%
 \def\Str@chfehlt##1{\ifx##1\s\Stricherr@\fi\ifx##1\\\let\Next\relax%
   \else\let\Next\Str@chfehlt\fi\Next}%
 \def\c{\hfil\noexpand\ifhline@@\hbox{\vrule height\htStrut@%
   depth\dpstrut@ width\z@}\noexpand\fi%
   \ifStrich@\hbox{\vrule height\htstrut@ depth\dpstrut@ width\z@}%
   \fi\iftab@\else$\m@th\fi\the\hashtoks@\iftab@\else$\fi\hfil}%
 \def\r{\hfil\noexpand\ifhline@@\hbox{\vrule height\htStrut@%
   depth\dpstrut@ width\z@}\noexpand\fi%
   \ifStrich@\hbox{\vrule height\htstrut@ depth\dpstrut@ width\z@}%
   \fi\iftab@\else$\m@th\fi\the\hashtoks@\iftab@\else$\fi}%
 \def\l{\noexpand\ifhline@@\hbox{\vrule height\htStrut@%
   depth\dpstrut@ width\z@}\noexpand\fi%
   \ifStrich@\hbox{\vrule height\htstrut@ depth\dpstrut@ width\z@}%
   \fi\iftab@\else$\m@th\fi\the\hashtoks@\iftab@\else$\fi\hfil}%
 \def\s{\ifStrich@\ \the\tabs@\vrule width\D@cke\the\hashtoks@%
          \fi\the\tabs@\ }%
 \def\sa{\ifStrich@\vrule width\D@cke\the\hashtoks@%
            \the\tabs@\ %
            \fi}%
 \def\se{\ifStrich@\ \the\tabs@\vrule width\D@cke\the\hashtoks@\fi}%
 \def\cd{\hfil\noexpand\ifhline@@\hbox{\vrule height\htStrut@%
   depth\dpstrut@ width\z@}\noexpand\fi%
   \ifStrich@\hbox{\vrule height\htstrut@ depth\dpstrut@ width\z@}%
   \fi$\dsize\m@th\the\hashtoks@$\hfil}%
 \def\rd{\hfil\noexpand\ifhline@@\hbox{\vrule height\htStrut@%
   depth\dpstrut@ width\z@}\noexpand\fi%
   \ifStrich@\hbox{\vrule height\htstrut@ depth\dpstrut@ width\z@}%
   \fi$\dsize\m@th\the\hashtoks@$}%
 \def\ld{\noexpand\ifhline@@\hbox{\vrule height\htStrut@%
   depth\dpstrut@ width\z@}\noexpand\fi%
   \ifStrich@\hbox{\vrule height\htstrut@ depth\dpstrut@ width\z@}%
   \fi$\dsize\m@th\the\hashtoks@$\hfil}%
 \ifStrich@\else\Str@chfehlt#1\\\fi%
 \setbox\z@\hbox{\xdef\Preamble@{\preamble@}}\ifnum`{=0 \fi\iffalse}\fi
 \ialign\bgroup\span\Preamble@\crcr}
\newif\ifhline@\hline@false
\newif\ifhline@@\hline@@false
\def\hlinefor#1{\multispan@{\strip@#1 }\leaders\hrule height\D@cke\hfill%
    \global\hline@true\ignorespaces}
\catcode`\@=\active
\font\Bf=cmbx12
\font\Rm=cmr12
\def\LL{\leavevmode\setbox0=\hbox{L}\hbox to\wd0{\hss\char'40L}}
\def\al{\alpha}
\def\be{\beta}
\def\ga{\gamma}
\def\de{\delta}
\def\ep{\varepsilon}
\def\ze{\zeta}
\def\et{\eta}
\def\th{\theta}
\def\vt{\vartheta}
\def\io{\iota}
\def\ka{\kappa}
\def\la{\lambda}
\def\rh{\rho}
\def\si{\sigma}
\def\ta{\tau}
\def\ph{\varphi}
\def\ch{\chi}
\def\ps{\psi}
\def\om{\omega}
\def\Ga{\Gamma}
\def\De{\Delta}
\def\Th{\Theta}
\def\La{\Lambda}
\def\Si{\Sigma}
\def\Ph{\Phi}
\def\Ps{\Psi}
\def\Om{\Omega}
\def\row#1#2#3{#1_{#2},\ldots,#1_{#3}}
\def\rowup#1#2#3{#1^{#2},\ldots,#1^{#3}}
\def\x{\times}
\def\crf{}            %used for crossreferencing, Tex should ignore.
\def\rf{}             %used for refencing (section-numbers)
\def\rfnew{}          %used for new-section numbers
\def\P{{\Bbb P}}
\def\R{{\Bbb R}}
\def\X{{\Cal X}}
\def\C{{\Bbb C}}
\def\Mf{{\Cal Mf}}
\def\FM{{\Cal F\Cal M}}
\def\F{{\Cal F}}
\def\G{{\Cal G}}
\def\V{{\Cal V}}
\def\T{{\Cal T}}
\def\A{{\Cal A}}
\def\N{{\Bbb N}}
\def\Z{{\Bbb Z}}
\def\Q{{\Bbb Q}}
\def\ddt{\left.\tfrac \partial{\partial t}\right\vert_0}
\def\dd#1{\tfrac \partial{\partial #1}}
\def\today{\ifcase\month\or
 January\or February\or March\or April\or May\or June\or
 July\or August\or September\or October\or November\or December\fi
 \space\number\day, \number\year}
\def\nmb#1#2{#2} %zum Nummerieren
\def\dfrac#1#2{{\displaystyle{#1\over#2}}}
\def\tfrac#1#2{{\textstyle{#1\over#2}}}
\def\iprod#1#2{\langle#1,#2\rangle}
\def\pder#1#2{\frac{\partial #1}{\partial #2}}
\def\iint{\int\!\!\int}
\def\({\left(}
\def\){\right)}
\def\[{\left[}
\def\]{\right]}
\def\supp{\operatorname{supp}}
\def\Df{\operatorname{Df}}
\def\dom{\operatorname{dom}}
\def\Ker{\operatorname{Ker}}
\def\Tr{\operatorname{Tr}}
\def\Res{\operatorname{Res}}
\def\Aut{\operatorname{Aut}}
\def\kgV{\operatorname{kgV}}
\def\ggT{\operatorname{ggT}}
\def\diam{\operatorname{diam}}
\def\Im{\operatorname{Im}}
\def\Re{\operatorname{Re}}
\def\ord{\operatorname{ord}}
\def\rang{\operatorname{rang}}
\def\rng{\operatorname{rng}}
\def\grd{\operatorname{grd}}
\def\inv{\operatorname{inv}}
\def\maj{\operatorname{maj}}
\def\des{\operatorname{des}}
\def\varmaj{\operatorname{\overline{maj}}}
\def\vardes{\operatorname{\overline{des}}}
\def\pvarmaj{\operatorname{\overline{maj}'}}
\def\pmaj{\operatorname{maj'}}
\def\ln{\operatorname{ln}}
\def\der{\operatorname{der}}
\def\Hom{\operatorname{Hom}}
\def\tr{\operatorname{tr}}
\def\Span{\operatorname{Span}}
\def\grad{\operatorname{grad}}
\def\div{\operatorname{div}}
\def\rot{\operatorname{rot}}
\def\Sp{\operatorname{Sp}}
\def\sgn{\operatorname{sgn}}
\def\liml{\lim\limits}
\def\supl{\sup\limits}
\def\bigcupl{\bigcup\limits}
\def\bigcapl{\bigcap\limits}
\def\limsupl{\limsup\limits}
\def\liminfl{\liminf\limits}
\def\intl{\int\limits}
\def\suml{\sum\limits}
\def\maxl{\max\limits}
\def\minl{\min\limits}
\def\prodl{\prod\limits}
\def\tg{\operatorname{tan}}
\def\ctg{\operatorname{cot}}
\def\arctg{\operatorname{arctan}}
\def\arccot{\operatorname{arccot}}
\def\arcctg{\operatorname{arccot}}
\def\tgh{\operatorname{tanh}}
\def\ctgh{\operatorname{coth}}
\def\arcsinh{\operatorname{arcsinh}}
\def\arccosh{\operatorname{arccosh}}
\def\arctgh{\operatorname{arctanh}}
\def\arcctgh{\operatorname{arccoth}}
\def\3{\ss}
\catcode`\@=11
\def\dddot#1{\vbox{\ialign{##\crcr
      .\hskip-.5pt.\hskip-.5pt.\crcr\noalign{\kern1.5\p@\nointerlineskip}
      $\hfil\displaystyle{#1}\hfil$\crcr}}}
\catcode`\@=13



\magnification1200
\hsize15.6truecm
\vsize22.8truecm
\catcode`\@=11
\def\iddots{\mathinner{\mkern1mu\raise\p@\hbox{.}\mkern2mu
    \raise4\p@\hbox{.}\mkern2mu\raise7\p@\vbox{\kern7\p@\hbox{.}}\mkern1mu}}
\catcode`\@=13
\topmatter 
\title Counting tableaux with row and column bounds
\endtitle 
\author C.~Krattenthaler\footnote"$\dagger$" {Institut f\"ur Mathematik der
 Universit\"at Wien,
Strudlhofgasse 4, A-1090 Wien, Austria.\newline \hbox{\hskip8pt{\it
 E-mail\/}:
KRATT\@PAP.UNIVIE.AC.AT}
}\\S.~G.~Mohanty\footnote"$\ddagger$"{McMaster University,
Hamilton, Ontario, Canada L8S\,4K1.\newline \hbox{\hskip8pt{\it E-mail\/}:
MOHANTY\@MCMASTER.CA}}
\endauthor 
%\affil 
%Institut f\"ur Mathematik der Universit\"at Wien,\\
%Strudlhofgasse 4, A-1090 Wien, Austria.\\
%\vskip6pt
%McMaster University,\\
%Hamilton, Ontario, Canada L8S\,4K1
%\endaffil 
%\email KRATT\@AWIRAP.BITNET\endemail
%\dedicatory \enddedicatory
%\date \enddate
%\thanks \endthanks
%\subjclass Primary 05A15;
% Secondary 05A17, 05E10, 11P81.
%\endsubjclass
%\keywords tableaux, generating functions, plane partitions,
%nonintersecting lattice paths\endkeywords
\abstract It is well-known that the 
generating function for tableaux
of a given skew shape with $r$ rows where the parts in the $i$'th row are 
bounded
by some nondecreasing upper and lower bounds which 
depend on $i$ can be written in form 
of a
determinant of size $r$. 
We show that the generating function for tableaux of a given
skew shape with $r$ rows and $c$ columns where the parts in the $i$'th row
are bounded by nondecreasing upper and lower bounds which depend on $i$ 
{\it and} the parts
in the $j$'th column are bounded by nondecreasing upper and lower 
bounds which
depend on $j$ can also be given in determinantal form. The size of
the determinant now is $r+2c$. We also show that determinants can be
obtained when the nondecreasingness is dropped.
Subsequently, analogous results are
derived for $(\alpha
,\beta
)$-plane partitions.
\endabstract
\endtopmatter

\newskip\Einheit \Einheit=0.5cm
\newcount\xcoord \newcount\ycoord
\newdimen\xdim \newdimen\ydim
\long\def\LOOP#1\REPEAT{\def\BODY{#1}\ITERATE}
\def\ITERATE{\BODY \let\next\ITERATE \else\let\next\relax\fi \next}
\let\REPEAT=\fi
\def\Punkt{\hbox{\raise-2pt\hbox to0pt{\hss$\ssize\bullet$\hss}}}
\def\DickPunkt(#1,#2){\unskip
  \raise#2 \Einheit\hbox to0pt{\hskip#1 \Einheit
          \raise-4pt\hbox to0pt{\hss\fourteenpoint$\bullet$\hss}\hss}}
\def\Kreis(#1,#2){\unskip
  \raise#2 \Einheit\hbox to0pt{\hskip#1 \Einheit
          \raise-4pt\hbox to0pt{\hss\fourteenpoint$\circ$\hss}\hss}}
\def\Pfad(#1,#2),#3\endPfad{\unskip\leavevmode
  \xcoord#1 \ycoord#2 \ZeichnePfad#3\endPfad}
\def\ZeichnePfad#1{\ifx#1\endPfad\let\next\relax
  \else\let\next\ZeichnePfad
    \ifnum#1=1
      \raise\ycoord \Einheit\hbox to0pt{\hskip\xcoord \Einheit
         \vrule height0.5pt width1 \Einheit depth0.5pt\hss}%
      \advance\xcoord by 1
    \else
      \raise\ycoord \Einheit\hbox to0pt{\hskip\xcoord \Einheit
        \hbox{\hskip-1pt\vrule height1 \Einheit width1pt depth0pt}\hss}%
      \advance\ycoord by 1
    \fi
  \fi\next}
\def\Koordinatenachsen(#1,#2){\unskip
 \hbox to0pt{\hskip-.5pt\vrule height#2 \Einheit width.5pt depth1 \Einheit}%
 \hbox to0pt{\hskip-1 \Einheit \xcoord#1 \advance\xcoord by1
    \vrule height0.25pt width\xcoord \Einheit depth0.25pt\hss}}
\def\Gitter(#1,#2){\unskip \xcoord0 \ycoord0 \leavevmode
  \LOOP\ifnum\ycoord<#2
    \loop\ifnum\xcoord<#1
      \raise\ycoord \Einheit\hbox to0pt{\hskip\xcoord \Einheit\Punkt\hss}%
      \advance\xcoord by1
    \repeat
    \xcoord0
    \advance\ycoord by1
  \REPEAT}
\def\Label#1#2(#3,#4){\unskip \xdim#3 \Einheit \ydim#4 \Einheit
  \def\lo{\advance\xdim by-.5 \Einheit \advance\ydim by.5 \Einheit}%
  \def\o{\advance\ydim by.5 \Einheit}%
  \def\ro{\advance\xdim by.5 \Einheit \advance\ydim by.5 \Einheit}%
  \def\l{\advance\xdim by-.5 \Einheit}%
  \def\r{\advance\xdim by.5 \Einheit}%
  \def\lu{\advance\xdim by-.5 \Einheit \advance\ydim by-.5 \Einheit}%
  \def\u{\advance\ydim by-.5 \Einheit}%
  \def\ru{\advance\xdim by.5 \Einheit \advance\ydim by-.5 \Einheit}%
  #1\raise\ydim\hbox to0pt{\hskip\xdim
     \vbox to0pt{\vss\hbox to0pt{\hss$#2$\hss}\vss}\hss}%
}

\document
%\baselineskip24pt
%\normalbaselineskip24pt

\subheading{1. Introduction and Definitions}
Let $\pmb\la=(\la_1,\dots,\la_r)$ and $\pmb\mu=(\mu_1,\dots,\mu_r)$ be
$r$-tupels of integers satisfying $\la_1\ge\dots\ge\la_r$,
$\mu_1\ge\dots\ge\mu_r$, and $\pmb \la\ge \pmb\mu$, meaning $\la_i\ge
\mu_i$ for all $i$. A {\it  tableau of shape
$\pmb\la/\pmb\mu$} is an array $\pi$
$$\matrix 
&&&\pi_{1,\mu_1+1}&\innerhdotsfor3\after\quad &\pi_{1,\la_1}\\
&&\pi_{2,\mu_2+1}\quad \dots&\pi_{2,\mu_1+1}&\innerhdotsfor2\after\quad 
&\pi_{2,\la_2}\\
&\iddots&&\vdots&&\iddots\\
\pi_{r,\mu_r+1}&\innerhdotsfor3\after\quad &\pi_{r,\la_r}
\endmatrix\tag1.1$$
of integers $\pi_{ij}$, $1\le i\le r$, $\mu_i+1\le j\le \la_i$,
such that the rows are weakly and the columns are strictly increasing. 
The number of entries in the tableau in (1.1) is
$(\la_1-\mu_1)+\dots+(\la_r-\mu_r)$, for which we write
$\vert\pmb\la-\pmb\mu\vert$. The entries will be called
{\it parts} of the tableau. The sum of all parts of a tableau $\pi$
is called the {\it norm}, in symbols $n(\pi)$, of the tableau. 

In order to make $\pmb\la$ and $\pmb\mu$ unique, we always assume that
$\mu_r=0$. Sometimes we will call $\la_1$ the {\it width}, and $r$
the {\it depth} of a shape $\pmb\la/\pmb\mu$.
If $\pmb\mu=0$ we shortly
write $\pmb\la$ for the shape $\pmb\la/\pmb\mu$.

The weight $w(\pi)$ of a tableau $\pi$ under consideration will be $\prod
x_{\pi_{ij}}$ where the product is over all parts $\pi_{ij}$ of
$\pi$.

It is well-known (cf\. \cite{2,4,5,9}) that the generating function 
$\sum _{} ^{}w(\pi)$
summed over all tableaux $\pi$ of shape $\pmb\la/\pmb\mu$ where the parts
in row $i$ are at most $a_i$ and at least $b_i$ for some $r$-tupels
$\pmb a=(a_1,\dots,a_r)$ and $\pmb b=(b_1,\dots,b_r)$ satisfying
$a_1\le a_2\le\dots\le a_r$, $b_1\le b_2\le\dots\le b_r$, and $\pmb
a\ge \pmb b$, can be
written in form of an $r\times r$-determinant,
$$\det_{1\le s,t\le r}(h_{\la_s-s-\mu_t+t}(\pmb x;a_s,b_t))\ ,$$
where  $h_n(\pmb x;A,B)$ is the complete homogenous symmetric
functions of order $n$ in the variables $x_B,x_{B+1},\dots,x_A$,
$$h_n(\pmb x;A,B):=\sum _{B\le i_1\le i_2\le\dots\le i_n\le
A}x_{i_1}x_{i_2}\dotsb x_{i_n}\ .$$

%\tracingmacros=2 \tracingcommands=2
A natural generalization of this problem is to ask for the generating
function $\sum _{} ^{}w(\pi)$
for tableaux with row bounds {\it and} column
bounds, to be precise, for the generating function
for tableaux of shape $\pmb\la/\pmb\mu$ where
the parts in row $i$ are at most $a_i$ and at least $b_i$, and the
parts
in column $j$ are at most $c_j$ and at least $d_j$, for some tupels
$\pmb a=(a_1,\dots,a_r)$, $\pmb b=(b_1,\dots,b_r)$,
$\pmb c=(c_1,\dots,c_{\la_1})$, and $\pmb d=(d_1,\dots,d_{\la_1})$ satisfying
$a_1\le a_2\le\dots\le a_r$, $b_1\le b_2\le\dots\le b_r$,
$c_1\le c_2\le\dots\le c_{\la_1}$, $d_1\le d_2\le\dots\le d_{\la_1}$, $\pmb
a\ge\pmb b$, and $\pmb c\ge\pmb d$. In Theorem~1 we show that this generating
function can also be written in form of a determinant whose entries are 
complete
homogenous symmetric functions. The size of the determinant is
$r+2\la_1$, i.e\. it is the depth plus twice the width of the shape.
Also in section~2, from this theorem we deduce determinant formulas
for the norm generating function of $(\alpha
,\beta
)$-reverse 
plane partitions
(which are generalizations of tableaux, cf\. section~2 for the
definition) with row and
column bounds, thus obtaining Corollary~2. Finally, in section~3 we
show how to get determinants if the monotonicity of the row and
column bounds is dropped. Now in adverse choices of the shape and the
row and column bounds, the size of the determinant might explode.
\newpage



\subheading{2. Monotone row and column bounds}
 Recall
(Gessel and Viennot \cite{3,4}) that a tableau $\pi$ of shape
$\pmb\la/\pmb\mu$ with
the parts in row $i$ being at most $a_i$ and at least $b_i$ can be
bijectively mapped onto a family $(P_1,\dots,P_r)$ of nonintersecting
lattice paths. ``Nonintersecting" in this context
 means that each two paths of this family have no
point in common. This correspondence maps the $i$-th row of $\pi$ to
the $i$-th path $P_i$ in the family such that $P_i$ starts at
$(\mu_i+r+1-i,b_i)$
and terminates at $(\la_i+r+1-i,a_i)$, and such that the parts of 
the $i$-th row can
be read off the heights of the horizontal steps in $P_i$. 
Figure~1 gives a simple example for $r=3$, $\pmb\la=(5,4,4)$,
$\pmb\mu=(2,1,0)$,
$\pmb a=(8,9,12)$, and $\pmb b=(1,3,6)$.

$$
\smatrix \format\r\s\c\s\c\s\c\s\c\s\c\s\l\\
\omit&\omit&\omit&\omit&\omit&\hlinefor7\\
1\le&\omit&\omit&\omit&\omit&&\hbox to10pt{\hss4\hss}&&%
\hbox to10pt{\hss6\hss}&&\hbox to10pt{\hss6\hss}&&\le8\\
\omit&\omit&\omit&\hlinefor9\\
3\le&\omit&\omit&&5&&7&&9&&\omit&\omit&\le9\\
\omit&\hlinefor9\\
6\le&&\vphantom
{g}\hbox to10pt{\hss6\hss}&&\hbox to10pt{\hss8\hss}&&9&&11&&\omit
&\omit&\le12\\
\omit&\hlinefor9
\endsmatrix
\quad \longleftrightarrow \quad 
%
\vcenter{\hsize4cm\leavevmode\Einheit0.4cm
\Gitter(9,13)
\Koordinatenachsen(10,13)
\Pfad(5,1),2221221122\endPfad
\Pfad(3,3),221221221\endPfad
\Pfad(1,6),1221212212\endPfad
\DickPunkt(5,1)
\DickPunkt(3,3)
\DickPunkt(1,6)
\DickPunkt(8,8)
\DickPunkt(5,12)
\DickPunkt(6,9)
\Label\lo{4}(6,4)
\Label\lo{6}(7,6)
\Label\lo{6}(8,6)
\Label\lo{5}(4,5)
\Label\lo{7}(5,7)
\Label\lo{9}(6,9)
\Label\lo{6}(2,6)
\Label\lo{8}(3,8)
\Label\lo{9}(4,9)
\Label\lo{11}(5,11)
\Label\r{P_1}(6,5)
\Label\l{P_2}(3,4)
\Label\l{P_3}(2,8)
\par}$$
\nobreak
$$\text {\eightpoint Figure 1}$$

The
condition that $\pi$ has strictly increasing columns corresponds to
the condition that the paths are nonintersecting. 
In addition, this bijection is weight-preserving if we define the
weight of a family $\Cal P=(P_1,\dots,P_r)$ of paths to be
$$w(\Cal P)=\prod _{} ^{}x_{h}\ ,$$
where the product is over the heights $h$ of all the horizontal steps
of the paths.

We want to compute the generating
function $\sum _{} ^{}w(\pi)$ for tableaux $\pi$ of shape
$\pmb\la/\pmb\mu$ where the parts in row $i$ are at most $a_i$ and at
least $b_i$, $i=1,2,\dots,r$, and the parts in column $j$ are at most
$c_j$ and at least $d_j$, $j=1,2,\dots,\la_1$. Note that we always
assume $\mu_r=0$ so that there are lower and upper bounds for each
column. To abbreviate the notation, for these tableaux we shall often use the
terminology {\it tableaux which obey the row bounds $\pmb a,\pmb b$ and
the column bounds $\pmb c, \pmb d\,$}, or even shorter {\it tableaux
obeying $\pmb a,\pmb b,\pmb c,\pmb d\,$}, always assuming that $\pmb a$
and $\pmb b$ are the upper and lower row bounds while $\pmb c$ and
$\pmb d$ are the upper and lower column bounds, respectively.


A simple trick enables us to use nonintersecting paths 
for this generalized problem also. 
This is accomplished by adding $2\la_1$ ``dummy paths"
of length 0. In fact, tableaux $\pi$ of shape $\pmb\la/\pmb\mu$
obeying $\pmb a,\pmb b,\pmb c,\pmb d$ bijectively
correspond to families 
$(P_1,\dots,P_r,\mathbreak P_{r+1},\dots,P_{r+2\la_1})$ of nonintersecting
lattice paths where for $i=1,\dots,r$ by using the Gessel/Viennot bijection 
the path $P_i$ is obtained from
the $i$'th row of $\pi$ respecting the row bounds $\pmb a$ and $\pmb
b$, $P_i:(\mu_i+r+1-i,b_i)\to(\la_i+r+1-i,a_i)$, while the paths $P_l$,
$l=r+1,\dots,r+2\la_1$, are dummy paths of length 0, the
starting and final points of which are given below.
$$P_l:\ (l'+M(l'),d_{l'}-1)\to(l'+M(l'),d_{l'}-1)\quad \quad
l=r+1,\dots,r+\la_1,\hphantom{\la_1+2}$$
with $l'=l-r$, and
$$P_{l}:\ (l''+\Lambda
(l''),c_{l''}+1)
\to(l''+\Lambda
(l''),c_{l''}+1)\quad \quad
l=r+\la_1+1,\dots,r+2\la_1,
$$
with $l''=l-r-\la_1$. The auxiliary functions $M$ and $\Lambda
$ are
defined by
$$M(I)=\sum _{j=1} ^{r}\chi
(I>\mu_j)\quad \text {and}\quad 
\Lambda
(I)=\sum _{j=1} ^{r}\chi
(I>\la_j)\ ,$$
where $\chi
(\Cal A)$ is the usual truth function ($\chi
(\Cal A)=1$ if
$\Cal A$ is true, and $\chi
(\Cal A)$=0 otherwise). Figure~2 gives an
example for this correspondence with $r=4$, $\la=(8,6,4,3)$,
$\mu=(3,2,0,0)$, $\pmb a=(10,12,13,13)$, $\pmb b=(1,1,2,3)$, $\pmb
c=(6,6,9,10,11,11,12,12)$, $\pmb d=(2,2,3,3,3,4,6,7)$.
\vskip-10pt
\vbox {
{%
$$\gather
\smatrix \format\r\s\c\s\c\s\c\s\c\s\c\s\c\s\c\s\c\s\l\\
&\omit&2\le&\omit&2\le&\omit&3\le&\omit&3\le&\omit&3\le&\omit
&4\le&\omit&6\le&\omit&7\le&\omit\\
\omit&\omit&\omit&\omit&\omit&\omit&\omit&\hlinefor{11}&\omit\\
1\le&\omit&\omit&\omit&\omit&\omit&\omit&&3&&4&&6&&7&&7&&\le10\\
\omit&\omit&\omit&\omit&\omit&\hlinefor{13}&\omit\\
1\le&\omit&\omit&\omit&\omit&&5&&5&&8&&9&&\omit&\omit&\omit&\omit
&\le12\\
\omit&\hlinefor{13}\\
2\le&&2&&3&&6&&6&&\omit&\omit&\omit&\omit&\omit&\omit&\omit&\omit
&\le13\\
\omit&\hlinefor{9}\\
3\le&&4&&5&&8&&\omit&\omit&\omit&\omit&\omit&\omit&\omit&\omit&\omit&\omit
&\le13\\
\omit&\hlinefor{7}\\
&\omit&\le6&\omit&\le6&\omit&\le9&\omit&\le10&\omit&\le11&\omit&\le11
&\omit&\le12&\omit&\le12
\endsmatrix\\
\updownarrow\\
\Einheit0.4cm
\Gitter(14,14)
\Koordinatenachsen(14,14)
\Pfad(7,1),22121221211222\endPfad
\Pfad(5,1),222211222121222\endPfad
\Pfad(2,2),121222112222222\endPfad
\Pfad(1,3),2121222122222\endPfad
\DickPunkt(7,1)
\DickPunkt(5,1)
\DickPunkt(2,2)
\DickPunkt(1,3)
\DickPunkt(3,1)
\DickPunkt(4,1)
\DickPunkt(6,2)
\DickPunkt(8,2)
\DickPunkt(9,2)
\DickPunkt(10,3)
\DickPunkt(11,5)
\DickPunkt(12,6)
\DickPunkt(12,10)
\DickPunkt(9,12)
\DickPunkt(6,13)
\DickPunkt(4,13)
\DickPunkt(1,7)
\DickPunkt(2,7)
\DickPunkt(3,10)
\DickPunkt(5,11)
\DickPunkt(7,12)
\DickPunkt(8,12)
\DickPunkt(10,13)
\DickPunkt(11,13)
\Label\lo{3}(8,3)
\Label\lo{4}(9,4)
\Label\lo{6}(10,6)
\Label\lo{7}(11,7)
\Label\lo{7}(12,7)
\Label\lo{5}(6,5)
\Label\lo{5}(7,5)
\Label\lo{8}(8,8)
\Label\lo{9}(9,9)
\Label\lo{2}(3,2)
\Label\lo{3}(4,3)
\Label\lo{6}(5,6)
\Label\lo{6}(6,6)
\Label\lo{4}(2,4)
\Label\lo{5}(3,5)
\Label\lo{8}(4,8)
\Label\r{P_{1}}(9,5)
\Label\r{P_{2}}(7,7)
\Label\l{P_{3}}(6,8)
\Label\r{P_{4}}(3,7)
\Label\u{\raise-15pt\hbox{$ P_{5}$}}(3,1)
\Label\u{\raise-15pt\hbox{$ P_{6}$}}(4,1)
\Label\u{\raise-15pt\hbox{$ P_{7}$}}(6,2)
\Label\u{\raise-15pt\hbox{$ P_{8}$}}(8,2)
\Label\ru{P_{9}}(9,2)
\Label\ru{\hskip6ptP_{10}}(10,3)
\Label\u{P_{11}}(11,5)
\Label\u{P_{12}}(12,6)
\Label\o{P_{13}\hskip8pt}(1,7)
\Label\o{P_{14}}(2,7)
\Label\l{P_{15}\hskip10pt}(3,10)
\Label\o{P_{16}}(5,11)
\Label\u{\raise-15pt\hbox{$ P_{17}$}}(7,12)
\Label\u{\raise-15pt\hbox{$ \hskip5pt P_{18}$}}(8,12)
\Label\u{\raise-15pt\hbox{$ P_{19}$}}(10,13)
\Label\u{\raise-15pt\hbox{$\hskip5pt P_{20}$}}(11,13)
\hskip6cm
\endgather$$
\vskip-20pt
$$\text {\eightpoint Figure 2}$$}
}

One easily gets convinced that $P_{r+1},\dots,P_{r+\la_1}$ 
force $P_1,\dots,P_r$
to stay above them, and, analogously, that $P_{r+\la_1+1},\dots,P_{r+2\la_1}$
force $P_1,\dots,P_r$ to stay below them; otherwise
$(P_1,\dots,P_r,P_{r+1},\dots, P_{r+2\la_1})$ would not be
nonintersecting. But that means that the corresponding tableau obeys
the column bounds $\pmb d$ and $\pmb c$. 

Now, since we have reduced our tableaux counting problem to the
problem of counting families of nonintersecting paths, we may apply
the main theorem of Gessel and Viennot.
\proclaim{Proposition 1} (Gessel, Viennot \cite{4, sec.~2})
The generating function $\sum w(\Cal P)$ for families $\Cal
P=(P_1,\dots,P_n)$ of nonintersecting paths,
where $P_i$ runs from $(C_i,D_i)$ to $(A_i,B_i)$ is given by
$$\det_{1\le s,t\le n}(h_{A_s-C_t}(\pmb x;B_s,D_t))\ ,
\tag1.2$$
whenever the location of the points $(A_i,B_i)$, $(C_i,D_i)$,
$i=1,\dots,n$, guarantees the following property: 
Let the path $Q_i$, $i=1,\dots,n$,  
start at $(C_i,D_i)$ and terminate at any of the points $(A_j,B_j)$,
$j=1,\dots,n$. If $(Q_1,\dots,Q_n)$ is a family of
{\rm nonintersecting} lattice paths, then for all $i=1,\dots,n$ the path
 $Q_i$ terminates at $(A_i,B_i)$.\quad \quad \qed
\endproclaim

This yields the following theorem:
\proclaim{Theorem~1}Let $\pmb a,\pmb b,\pmb c,\pmb d$ as above.
The generating function $\sum w(\pi)$ for tableaux $\pi$ of shape 
$\pmb\la/\pmb\mu$ where
the parts in row $i$ are at most $a_i$ and at least $b_i$, and the parts
in column $j$ are at most $c_j$ and at least $d_j$ is given by
$$\det_{1\le s,t\le r+2\la_1}(h_{A_s-C_t}(\pmb
x;B_s,D_t))\ ,\tag1.3$$
where $A_i,B_i,C_i,D_i$ are displayed in the table 
below.
{\eightpoint
$$\smatrix \format\c\s\c\s\c\s\c\\
&&1\le i\le r&&r+1\le i\le r+\la_1&&r+\la_1+1\le i\le r+2\la_1\\
&& &&i'=i-r&&i''=i-r-\la_1\\
\hlinefor7\\
A_i&&\la_i+r+1-i&&i'+M(i')&&i''+\Lambda
(i'')\\
B_i&&a_i&&d_{i'}-1&&c_{i''}+1\\
C_i&&\mu_i+r+1-i&&i'+M(i')&&i''+\Lambda
(i'')\\
D_i&&b_i&&d_{i'}-1&&c_{i''}+1
\endsmatrix$$
\nobreak
$$\text {\eightpoint Table 1}$$}
\qed\endproclaim

Setting $x_i=q^i$ for all $i$, we obtain as a corollary the norm
generating function for tableaux with row and column bounds.
\proclaim{Corollary 1}Let $\pmb a,\pmb b,\pmb c,\pmb d$ as in
Theorem~1.
The generating function $\sum q^{n(\pi)}$ for tableaux $\pi$ of shape
$\pmb\la/\pmb\mu$ where
the parts in row $i$ are at most $a_i$ and at least $b_i$, and the parts
in column $j$ are at most $c_j$ and at least $d_j$ is given by
$$\det_{1\le s,t\le r+2\la_1}\left(q^{D_t(A_s-C_t)}
\left[\matrix A_s+B_s-C_t-D_t\\
B_s-D_t\endmatrix\right]\right)\ ,\tag1.4$$
where $A_i,B_i,C_i,D_i$ are displayed in Table~1. 
The $q$-binomial coefficient is defined by
$$\left[\matrix n\\k\endmatrix\right]=\frac
{(1-q^n)(1-q^{n-1})\dotsb(1-q^{n-k+1})}
{(1-q^k)(1-q^{k-1})\dotsb(1-q)}\ ,$$
if $n\ge k\ge0$, and $0$ otherwise.\quad \quad \qed
\endproclaim

Applying this result to tableaux 
of shape $(8,6,4,3)/(3,2,0,0)$
obeying $\pmb a=(10,12,\mathbreak 13,13)$, $\pmb b=(1,1,2,3)$, $\pmb
c=(6,6,9,10,11,11,12,12)$, and $\pmb d=(2,2,3,3,3,4,6,7)$, a 
typical
example of which is displayed in Figure~2, we obtain that the norm
generating function for the 196.650.160 tableaux of this type is
given by

\NoBlackBoxes
{\eightpoint
$$\align
\sssize{q^{63}} + &\sssize7\,{q^{64}} + 31\,{q^{65}} + 108\,{q^{66}} + 
318\,{q^{67}} + 
  830\,{q^{68}} + 1967\,{q^{69}}
 + 4309\,{q^{70}} + 8825\,{q^{71}} + 
  17054\,{q^{72}} + 31300\,{q^{73}} + 54857\,{q^{74}} + 92202\,{q^{75}}
\\
&\sssize+ 
  149150\,{q^{76}} + 232896\,{q^{77}} + 351926\,{q^{78}} + 515736\,{q^{79}} 
+ 
  734334\,{q^{80}} + 1017550\,{q^{81}} + 1374118\,{q^{82}} + 
  1810680\,{q^{83}} + 2330671\,{q^{84}} \\
&\sssize+ 2933374\,{q^{85}} + 
  3613028\,{q^{86}} + 4358418\,{q^{87}} + 5152677\,{q^{88}} + 
  5973797\,{q^{89}} + 6795371\,{q^{90}} + 7588103\,{q^{91}} + 
  8321344\,{q^{92}} \\
&\sssize+ 8965234\,{q^{93}} + 9492509\,{q^{94}} + 
  9880603\,{q^{95}} + 10112996\,{q^{96}} + 10180583\,{q^{97}} + 
  10081990\,{q^{98}} + 9823778\,{q^{99}} + 9419561\,{q^{100}} \\
&\sssize+ 
  8889120\,{q^{101}} + 8256606\,{q^{102}} + 7549054\,{q^{103}} + 
  6794310\,{q^{104}} + 6019584\,{q^{105}} + 5249728\,{q^{106}} + 
  4506378\,{q^{107}} + 3806938\,{q^{108}} \\
&\sssize+ 3164499\,{q^{109}} + 
  2587580\,{q^{110}} + 2080681\,{q^{111}} + 1644535\,{q^{112}} + 
  1277020\,{q^{113}} + 973600\,{q^{114}} + 728280\,{q^{115}} + 
  534013\,{q^{116}} \\
&\sssize+ 383492\,{q^{117}} + 269382\,{q^{118}} + 
  184885\,{q^{119}} + 123772\,{q^{120}} + 80708\,{q^{121}} + 
  51140\,{q^{122}} + 31435\,{q^{123}} + 18679\,{q^{124}} + 10708\,{q^{125}}
\\
&\sssize+ 
  5890\,{q^{126}} + 3102\,{q^{127}} + 1549\,{q^{128}} + 733\,{q^{129}} + 
  322\,{q^{130}} + 132\,{q^{131}} + 48\,{q^{132}} + 16\,{q^{133}} + 
  4\,{q^{134}} + {q^{135}}.
\endalign$$\par}

Corollary~1 can be generalized in the following way. Call an array
$\overline{\pi}$ of
the form (1.1) an {\it $(\alpha
,\beta
)$-reverse plane partition of shape
$\pmb\la/\pmb\mu$} if 
$$\align 
&\overline{\pi}_{ij}+\alpha
\le \overline{\pi}_{i,j+1}
\quad \quad 1\le i\le r,\ \mu_i+1\le j<\la_i\\
\text {and}&\tag1.5\\
&\overline{\pi}_{ij}+\beta
\le \overline{\pi}_{i+1,j}
\quad \quad 1\le i< r,\ \mu_i+1\le j\le\la_{i+1}\ .
\endalign$$
This definition comprises several classes of reverse plane
partitions. In particular, tableaux are $(0,1)$-reverse plane partitions. 

Given a tableau $\pi$ of shape $\pmb\la/\pmb\mu$, the transformation
$$\pi_{ij}\to \pi_{ij}+i(\beta
-1)+j\alpha
\ ,$$
applied to every part $\pi_{ij}$ of $\pi$, maps $\pi$ to an
$(\alpha
,\beta
)$-reverse plane partition. Clearly, this mapping is a
bijection between tableaux and $(\alpha
,\beta
)$-reverse plane partitions.
This bijection does not preserve $w(\pi)$ nor the norm $n(\pi)$. But
for the norm we have the simple assertion that the norm of the
$(\alpha
,\beta
)$-reverse plane partition which was obtained from a certain
tableau under this transformation differs from the norm of this
tableau by $\sum _{} ^{}\big(i(\beta
-1)+j\alpha
\big)$, where the sum is
over all $i,j$ with $1\le i\le r$ and $\mu_i+1\le j\le\la_i$. This
quantity only depends on the shape $\pmb\la/\pmb\mu$ and not on the
tableau involved.
Therefore, using this bijection
 it is an easy task to transfer Corollary~1 to
the more general case of $(\alpha
,\beta
)$-reverse plane partitions. One
only has to find out how the row and column bounds change under this
transformation.
\proclaim{Corollary~2}
Let $\pmb a,\pmb b$ be $r$-tupels and $\pmb c,\pmb d$
be $\la_1$-tupels of integers satisfying
$$\gather  a_i+\alpha
(\la_{i+1}-\la_i)+(\beta
-1)\le a_{i+1},\quad b_i+\alpha
(\mu_{i+1}-\mu_i)+(\beta
-1)\le b_{i+1}\tag1.6\\
 c_i+(\beta
-1)(\la'_{i+1}-\la'_i)+\alpha
\le c_{i+1},\quad d_i+(\beta
-1)(\mu'_{i+1}-\mu'_i)+\alpha
\le d_{i+1}\ .\endgather$$
The generating function $\sum q^{n(\overline{\pi})}$ for $(\alpha
,\beta
)$-reverse
plane partitions $\overline{\pi}$ of shape $\pmb\la/\pmb\mu$ where
the last part in row $i$ is at most $a_i$ and the first part in row
$i$ is at least $b_i$, and where the down-most part
in column $j$ is at most $c_j$ and the upper-most part in column $j$
is at least $d_j$, is given by
$$\multline
q^{(\beta
-1)\sum _{i=1} ^{r}i(\la_i-\mu_i)+\alpha
\sum _{i=1}
^{r}\left[\binom{\la_i+1}2-\binom{\mu_i+1}2\right]}\\
\times\det_{1\le s,t\le r+2\la_1}\left(q^{D_t(A_s-C_t)}
\left[\matrix A_s+B_s-C_t-D_t\\
B_s-D_t\endmatrix\right]\right)\ ,
\endmultline\tag1.7$$
where $A_i,B_i,C_i,D_i$ are displayed in
Table~2.
{\eightpoint
$$\smatrix \format\c\s\c\s\c\s\c\\
&&1\le i\le r&&r+1\le i\le r+\la_1&&r+\la_1+1\le i\le r+2\la_1\\
&& &&i'=i-r&&i''=i-r-\la_1\\
\hlinefor7\\
A_i&&\la_i+r+1-i&&i'+M(i')&&i''+\Lambda
(i'')\\
B_i&&a_i-(\beta
-1)i-\alpha
\la_i&&d_{i'}-(\beta
-1)(\mu'_{i''}+1)-
\alpha
 i''-1&&c_{i''}-(\beta
-1)\la'_{i''}-
\alpha
 i''+1\\
C_i&&\mu_i+r+1-i&&i'+M(i')&&i''+\Lambda
(i'')\\
D_i&&b_i-(\beta
-1)i-\alpha
(\mu_i+1)&&d_{i'}-(\beta
-1)(\mu'_{i''}+1)-
\alpha
 i''-1&&c_{i''}-(\beta
-1)\la'_{i''}-
\alpha
 i''+1
\endsmatrix$$
\nobreak
$$\text {\eightpoint Table 2}$$
}
$\pmb\la'/\pmb\mu'$ is the conjugate shape of $\pmb\la/\pmb\mu$.
\qed\endproclaim

\remark{Remarks}
1) Theorem~1 and Corollary~2 immediately can be used for
column-strict plane partitions and $(\alpha
,\beta
)$-plane
partitions (cf\. \cite{5}), 
respectively, since rotation by $180^\circ$ turns them
into tableaux and $(\alpha
,\beta
)$-reverse plane partitions, respectively.

2) Very often the size of the determinant can be reduced by ignoring
superflous paths. (In Figure~2 $P_5,P_6,P_{17},P_{19},P_{20}$ are
superflous.) Also, 
the entries in the determinant can be reduced by
replacing $a_i$ by $\min\{a_i,c_{\la_i}\}$, and $b_i$ by
$\max\{b_i,d_{\mu_i+1}\}$. For a rectangular shape
$(c^r)$ it is easily seen that after these manipulations $P_{r+1}$
and $P_{r+2c}$ are always superflous so that in this case the
size of the determinant in fact is at most $r+2c-2$.

3) Reflection in the main diagonal turns an
$(\alpha
,\beta
)$-reverse plane partition of shape $\pmb\la/\pmb\mu$ into an 
$(\beta
,\alpha
)$-reverse plane partition of shape $\pmb\la'/\pmb\mu'$. 
Therefore the
norm generating function  for $(\alpha
,\beta
)$-reverse plane partitions of a
given rectangular shape $(c^r)$ and with given row and column bounds can be
computed in two different ways using Corollary~2. The first way is to
use Corollary~2 directly, thus (by Remark~2) obtaining a determinant of size
$(r+2c-2)$. Or, secondly, one could use the reflection which
exchanges $\alpha
$ and $\beta
$, $r$ and $c$, $\pmb a$ and $\pmb c$, and
$\pmb b$ and $\pmb d$, and then apply Corollary~2 with these new
parameters. This time a determinant of size $(c+2r-2)$ is obtained.
Therefore, in order to minimize the size of the determinant, one
should first check whether $r\ge c$ or not, in the first case
use Corollary~2 directly, in the latter Corollary~2 should be
used only after first having performed the reflection.
With a skew shape, in general Corollary~2 will not be 
applicable in two ways as
described above because after the reflection the new
bounds might not satisfy (1.6). But as will be seen later, it is
possible to give a determinant for the ``reflected" problem, also.

4) Of course, our formula can also be used if there are bounds only on three
sides. In fact, in this case the determinants in Theorem~1 or
Corollaries~1,2 trivially reduce to determinants of size $r+\la_1$.
It should be noted that three bound counting for rectangular shapes
 has been
considered earlier by Chorneyko and Zing \cite{1, 8, Theorems~1.2.1
and 1.2.2}. Using the Narayana/Mohanty \cite{cf.~7} method of successively
building up determinants, they derived determinants for the number of
rectangular tableaux with bounds $\pmb a$, $\pmb b$, $\pmb c$ or 
$\pmb a$, $\pmb b$, $\pmb d$, respectively. 
However, Chorneyko and Zing's determinants slightly differ
from our $q=1$-results. But they can also be explained by nonintersecting 
lattice paths. 
These determinants correspond to a slightly different choice of
the dummy paths. For example, to obtain Chorneyko and Zing's
 determinant in the
$(\pmb a,\pmb b,\pmb d)$-case for a shape $(c^r)$ 
\cite{8, Theorem~1.2.1}, one had to
take the dummy paths $P_l:(l,b_{1}-1)\to (l,\max\{b_1,d_{l-r}\}-1)$,
$l=r+1,\dots,r+c$, instead of $P_l:(l,d_{l-r}-1)\to (l,d_{l-r}-1)$.
The corresponding picture of a typical
example is given in the 
figure below.
$$
\smatrix \format\r\s\c\s\c\s\c\s\c\s\l\\
&\omit&2\le&\omit&4\le&\omit&4\le&\omit&5\le\\
\omit&\hlinefor9\\
1\le&&3&&4&&6&&7&&\le8\\
\omit&\hlinefor9\\
3\le&&5&&5&&7&&9&&\le9\\
\omit&\hlinefor9\\
6\le&&6&&8&&9&&11&&\le12\\
\omit&\hlinefor9\\
\endsmatrix
\quad \longleftrightarrow \quad 
%
\vcenter{\hsize4cm\leavevmode\Einheit0.4cm
\Gitter(9,14)
\Koordinatenachsen(10,15)
\Pfad(3,1),22121221212\endPfad
\Pfad(2,3),2211221221\endPfad
\Pfad(1,6),1221212212\endPfad
\Pfad(4,0),2\endPfad
\Pfad(5,0),222\endPfad
\Pfad(6,0),222\endPfad
\Pfad(7,0),2222\endPfad
\DickPunkt(3,1)
\DickPunkt(4,1)
\DickPunkt(2,3)
\DickPunkt(5,3)
\DickPunkt(6,3)
\DickPunkt(7,4)
\DickPunkt(1,6)
\DickPunkt(7,8)
\DickPunkt(5,12)
\DickPunkt(6,9)
\DickPunkt(4,0)
\DickPunkt(5,0)
\DickPunkt(6,0)
\DickPunkt(7,0)
\Label\lo{3}(4,3)
\Label\lo{4}(5,4)
\Label\lo{6}(6,6)
\Label\lo{7}(7,7)
\Label\lo{5}(3,5)
\Label\lo{5}(4,5)
\Label\lo{7}(5,7)
\Label\lo{9}(6,9)
\Label\lo{6}(2,6)
\Label\lo{8}(3,8)
\Label\lo{9}(4,9)
\Label\lo{11}(5,11)
\Label\u{\raise-15pt\hbox{$ P_4$}}(4,0)
\Label\u{\raise-15pt\hbox{$ P_5$}}(5,0)
\Label\u{\raise-15pt\hbox{$ P_6$}}(6,0)
\Label\u{\raise-15pt\hbox{$ P_7$}}(7,0)
\Label\r{P_1}(5,5)
\Label\l{P_2}(2,4)
\Label\l{P_3}(2,8)
\par}$$
\nobreak
$$\text {\eightpoint Figure 3}$$
Clearly, the dummy paths in Figure~3 have just the same effect as
those in Figure~2. We decided to take paths of length 0 because this
choice causes the entries in the determinant to become smaller.
 \quad \quad \qed
\endremark
\newpage

\subheading{3. Unrestricted row and column bounds}
It turns out that even if we drop the condition of the bounds being
nondecreasing, the generating functions for tableaux
 can be given in determinantal
form. Let $\pmb a,\pmb b$ be arbitrary $r$-tupels of integers and
$\pmb c,\pmb d$ be arbitrary $\la_1$-tupels of integers only
satisfying
$\pmb a\ge\pmb b$ and $\pmb c\ge\pmb d$. We want to compute the
generating function for tableaux of shape $\pmb\la/\pmb\mu$ where
the parts in row $i$ are at most $a_i$ and at least $b_i$, and the
parts in column $j$ are at most $c_j$ and at least $d_j$. 
Since now a formula for
the entries of the resulting determinant would involve very clumsy
expressions, it is more convenient to only give a rough description of the
procedure which finally leads to the determinantal formula. Moreover,
as will be seen later, it is better not to rigorously 
stick to a formula because
very often in the last step of this procedure the size of the
determinant can be significantly reduced, which would not be observed
when a general formula is directly used.


Let us give a sketch of this procedure, which is performed in four steps.
First, consider the case $\pmb\la=(5,4,3)$, $\pmb\mu=(2,0,0)$, $\pmb
a=(14,10,13)$, $\pmb b=(4,1,4)$, $\pmb c=(6,8,16,15,12)$, $\pmb
d=(2,7,1,7,5)$ which is illustrated on the left-hand side of Figure~4.

{\eightpoint
$$
\smatrix \format\r\s\c\s\c\s\c\s\c\s\c\s\l\\
&\omit&2\le&\omit&7\le&\omit&1\le&\omit&7\le&\omit&5\le&\omit&\\
\omit&\omit&\omit&\omit&\omit&\hlinefor7\\
4\le&\omit&&\omit&&&&&&&&&\le 14\\
\omit&\hlinefor{11}\\
1\le&&&&&&&&&& &\omit&\le10\\
\omit&\hlinefor9\\
4\le&&&&&&&&&\omit&&\omit&\le13\\
\omit&\hlinefor7\\
&\omit&\le6&\omit&\le8&\omit&\le16&\omit&\le15&\omit&\le12
\endsmatrix
%
\overset\text {first step}\to\longrightarrow
%
\smatrix \format\r\s\c\s\c\s\c\s\c\s\c\s\l\\
&\omit&2\le&\omit&7\le&\omit&4\le&\omit&7\le&\omit&7\le&\omit&\\
\omit&\omit&\omit&\omit&\omit&\hlinefor7\\
4\le&\omit&&\omit&&&&&&&&&\le 12\\
\omit&\hlinefor{11}\\
2\le&&&&&&&&&& &\omit&\le10\\
\omit&\hlinefor9\\
4\le&&&&&&&&&\omit&&\omit&\le13\\
\omit&\hlinefor7\\
&\omit&\le6&\omit&\le8&\omit&\le13&\omit&\le10&\omit&\le12
\endsmatrix
$$}
\nopagebreak
$$\text {\eightpoint Figure 4}$$

Obviously, some entries in $\pmb a,\pmb b,\pmb c$, or $\pmb d$,
respectively, could be replaced by greater respectively smaller
 ones without changing the
set of tableaux obeying these bounds. For example, $d_5=5$ could be
replaced by $7$, $a_1=14$ by $12$, etc. Formally, we may replace
$\pmb a$ by $\bar{\pmb a}$, $\pmb b$ by $\bar{\pmb b}$, \dots, where

{\eightpoint
$$\gather  \bar{a}_i=\cases \min\{a_i,a_{i+1},\dots,a_{\la'_{\la_i}}\}
&\la_{i+1}=\la_i\\
\min\{a_i,c_{\la_i}\}&\la_{i+1}<\la_i\endcases\ ,\quad 
\bar{b}_i=\cases \max\{b_{\mu'_{\mu_i}+1},\dots,b_{i-1},b_i\}
&\mu_{i-1}=\mu_i\\
\max\{b_i,d_{\mu_i+1}\}&\mu_{i-1}>\mu_i\endcases\ ,\\
\bar{c}_i=\cases \min\{c_i,c_{i+1},\dots,c_{\la_{\la'_i}}\}
&\la'_{i+1}=\la'_i\\
\min\{c_i,a_{\la'_i}\}&\la'_{i+1}<\la'_i\endcases\ ,\quad 
\bar{d}_i=\cases \max\{d_{\mu_{\mu'_i}+1},\dots,d_{i-1},d_i\}
&\mu'_{i-1}=\mu'_i\\
\max\{d_i,b_{\mu'_i+1}\}&\mu'_{i-1}>\mu'_i\endcases\ .\endgather$$}
In our example, this normalization of the bounds is displayed on the
right-hand side of Figure~4.

In the second step, tableaux of shape $\pmb\la/\pmb\mu$ obeying
$\bar{\pmb a},\bar{\pmb b},\bar{\pmb c},\bar{\pmb d}$ are interpreted
as families of lattice paths as was done before in order to prove
Theorem~1.

\newpage
$$\gather
\smatrix \format\r\s\c\s\c\s\c\s\c\s\c\s\l\\
&\omit&2\le&\omit&7\le&\omit&4\le&\omit&7\le&\omit&7\le&\omit&\\
\omit&\omit&\omit&\omit&\omit&\hlinefor7\\
4\le&\omit&&\omit&&&4&&9&&9&&\le 12\\
\omit&\hlinefor{11}\\
2\le&&3&&7&&7&&10&& &\omit&\le10\\
\omit&\hlinefor9\\
4\le&&4&&8&&8&&&\omit&&\omit&\le13\\
\omit&\hlinefor7\\
&\omit&\le6&\omit&\le8&\omit&\le13&\omit&\le10&\omit&\le12
\endsmatrix\hskip6cm
\\
\vphantom{fg}\\
\downarrow\text {\eightpoint second step}\hskip4cm\\
\endgather$$
\nopagebreak
\vskip-8pt
\nopagebreak
$$
\vcenter{\hsize4cm\leavevmode\Einheit0.4cm
\Gitter(9,15)
\Koordinatenachsen(9,15)
\Pfad(5,4),12222211222\endPfad
\Pfad(2,2),212222112221\endPfad
\Pfad(1,4),122221122222\endPfad
\DickPunkt(5,4)
\DickPunkt(2,2)
\DickPunkt(1,4)
\DickPunkt(3,1)
\DickPunkt(4,6)
\DickPunkt(6,3)
\DickPunkt(7,6)
\DickPunkt(8,6)
\DickPunkt(1,7)
\DickPunkt(2,9)
\DickPunkt(3,14)
\DickPunkt(4,13)
\DickPunkt(5,11)
\DickPunkt(6,10)
\DickPunkt(7,13)
\DickPunkt(8,12)
\Label\lo{3}(3,3)
\Label\lo{7}(4,7)
\Label\lo{7}(5,7)
\Label\lu{10}(6,10)
\Label\lo{4}(6,4)
\Label\lu{9}(7,9)
\Label\lu{9}(8,9)
\Label\lo{4}(2,4)
\Label\lo{8}(3,8)
\Label\lo{8}(4,8)
}
\overset\text {third step}\to\longrightarrow
%
\vcenter{\hsize4cm\leavevmode\Einheit0.4cm
\Gitter(9,15)
\Koordinatenachsen(9,15)
\Pfad(5,4),12222211222\endPfad
\Pfad(2,2),212222112221\endPfad
\Pfad(1,4),122221122222\endPfad
\DickPunkt(5,4)
\DickPunkt(2,2)
\DickPunkt(1,4)
\DickPunkt(3,1)
\DickPunkt(4,6)
\DickPunkt(6,3)
\DickPunkt(7,6)
\DickPunkt(8,6)
\DickPunkt(1,7)
\DickPunkt(2,9)
\DickPunkt(3,14)
\DickPunkt(4,13)
\DickPunkt(5,11)
\DickPunkt(6,10)
\DickPunkt(7,13)
\DickPunkt(8,12)
\DickPunkt(4,2)
\DickPunkt(4,3)
\DickPunkt(4,4)
\DickPunkt(4,5)
\DickPunkt(6,12)
\DickPunkt(6,11)
\Label\lo{3}(3,3)
\Label\lo{7}(4,7)
\Label\lo{7}(5,7)
\Label\lu{10}(6,10)
\Label\lo{4}(6,4)
\Label\lu{9}(7,9)
\Label\lu{9}(8,9)
\Label\lo{4}(2,4)
\Label\lo{8}(3,8)
\Label\lo{8}(4,8)
\Label\l{(}(3,1)
\Label\r{)}(3,1)
\Label\l{(}(4,2)
\Label\r{)}(4,2)
\Label\l{(}(4,3)
\Label\r{)}(4,3)
\Label\l{(}(4,4)
\Label\r{)}(4,4)
\Label\l{(}(6,11)
\Label\r{)}(6,11)
\Label\l{(}(7,13)
\Label\r{)}(7,13)
\Label\l{(}(3,14)
\Label\r{)}(3,14)
}$$
\nopagebreak
$$\text {\eightpoint Figure 5}$$

But now there are not enough dummy paths to guarantee that every
family of nonintersecting paths corresponds to a tableaux of the
desired type. Therefore, in the third step we have to insert
additional dummy paths to build up ``barriers" at some places. Let us
consider the starting points $S_i,S_{i+1}$ of the paths $P_i,P_{i+1}$
which correspond to the $i$'th and $(i+1)$'th row of
the tableau, respectively. Depending on which of the relations 
$\mu_i=\mu_{i+1}$,
$\bar b_i\le\bar b_{i+1}$, $\bar d_{\mu_i}\le\bar b_i$ hold or not,
we obtain four cases which schematically are described in Figure~6.
$$
\underset\text {Case~1: $\mu_i=\mu_{i+1}$, $\bar b_i\le\bar
b_{i+1}$}\to
{\vcenter{\hsize4cm\leavevmode\Einheit0.4cm
\DickPunkt(0,5)
\DickPunkt(1,0)
\Pfad(0,5),121112\endPfad
\Pfad(1,0),22211211\endPfad
\Label\lu{S_{i+1}}(0,5)
\Label\lu{S_{i}}(1,0)
\vskip0.3cm}}
\underset\text {Case~2: $\mu_i>\mu_{i+1}$, $\bar d_{\mu_i}>\bar b_i\le\bar
b_{i+1}$}\to
{\vcenter{\hsize4cm\leavevmode\Einheit0.4cm
\DickPunkt(0,3)
\DickPunkt(5,0)
\Pfad(0,3),22221122212\endPfad
\Pfad(5,0),22211221\endPfad
\DickPunkt(1,2)
\DickPunkt(2,5)
\DickPunkt(3,8)
\DickPunkt(4,10)
\Kreis(2,3)
\Kreis(2,4)
\Kreis(3,6)
\Kreis(3,7)
\Kreis(4,9)
\Label\lu{S_{i+1}}(0,3)
\Label\lu{S_{i}}(5,0)
\vskip10pt}}
$$
$$
\underset\text {Case~3: $\bar d_{\mu_i}\le\bar b_i>\bar
b_{i+1}$}\to
{\vcenter{\hsize4cm\leavevmode\Einheit0.4cm
\DickPunkt(0,0)
\DickPunkt(4,10)
\Pfad(0,0),222221222122\endPfad
\Pfad(4,10),11\endPfad
\Label\lu{S_{i+1}}(0,0)
\Label\lo{S_{i}}(4,10)
\DickPunkt(1,-1)
\DickPunkt(2,3)
\DickPunkt(3,6)
\Kreis(2,0)
\Kreis(2,1)
\Kreis(2,2)
\Kreis(3,4)
\Kreis(3,5)
\Kreis(4,7)
\Kreis(4,8)
\Kreis(4,9)
}}
\underset\text {Case~4: $\bar d_{\mu_i}>\bar b_i>\bar
b_{i+1}$}\to
{\vcenter{\hsize4cm\leavevmode\Einheit0.4cm
\DickPunkt(0,0)
\DickPunkt(5,5)
\Pfad(0,0),222211222222112\endPfad
\Pfad(5,5),21\endPfad
\Label\lu{S_{i+1}}(0,0)
\Label\lu{S_{i}}(5,5)
\DickPunkt(1,-1)
\DickPunkt(2,2)
\DickPunkt(3,7)
\DickPunkt(4,9)
\Kreis(2,0)
\Kreis(2,1)
\Kreis(3,3)
\Kreis(3,4)
\Kreis(3,5)
\Kreis(3,6)
\Kreis(4,8)
}}$$
\nopagebreak
$$\text {\eightpoint Figure 6}$$
The boldface dots different from $S_i$ and $S_{i+1}$ indicate the
dummy paths which were inserted during Step~2. The circles
indicate the dummy paths which have to be added to build up barriers
which prevent $P_{i+1}$ from going below $P_i$ or not obeying $\pmb
d$, respectively. Only in Case~1 nothing has to be done. Considering
the end points of $P_i$ and $P_{i+1}$, dummy paths are added in an
analogous manner in order to guarantee that the corresponding tableau
is of the desired type. The result of this third step applied to our
example is the family of paths on the right-hand side of Figure~5.


Finally, in the fourth step we look after paths which are superflous.
These subsequently are dropped. In the right-hand side
 family of lattice paths in Figure~5 the superflous paths are put
into parentheses. (It should be observed that paths should only be
dropped sequentially because the same path could be superflous with
respect to some set of paths which already has been dropped but not
superflous with respect to another set. There can be several
different sets of paths which can be legitimately 
dropped. One will choose that
one which causes the corresponding determinant to become as ``small" as
possible.)

Now Proposition~1 can be applied thus again obtaining a determinant
for the generating function. In our running example
(Figures~4,5) this procedure yields that there are 17.163 tableaux of
the desired type, the norm generating function of which is given by
$$\align 
\ssize
{q^{62}} &
\ssize+ 6\,{q^{63}} + 21\,{q^{64}} + 55\,{q^{65}} + 118\,{q^{66}} + 
  221\,{q^{67}} + 372\,{q^{68}} + 572\,{q^{69}} + 812\,{q^{70}} + 
  1072\,{q^{71}} + 1322\,{q^{72}} + 1530\,{q^{73}}\\
& \ssize+ 1665\,{q^{74}} + 
  1707\,{q^{75}} + 1650\,{q^{76}} + 1503\,{q^{77}} + 1290\,{q^{78}} + 
  1041\,{q^{79}} + 789\,{q^{80}} + 561\,{q^{81}} + 373\,{q^{82}} + 
  231\,{q^{83}} \\
&\ssize+ 132\,{q^{84}} + 69\,{q^{85}} + 32\,{q^{86}} + 
  13\,{q^{87}} + 4\,{q^{88}} + {q^{89}}.
\endalign$$
Of course, with the help of this procedure the norm generating function
for $(\alpha
,\beta
)$-plane partitions or $(\alpha
,\beta
)$-reverse plane
partitions of shape $\pmb\la/\pmb\mu$ which obey arbitrary $\pmb a,\pmb
b,\pmb
c,\pmb
d,$ can also be computed by transforming the problem to the
corresponding tableaux problem.

\Refs
\ref\no 1\by I. Z. Chorneyko, L. K. K. Zing
\paper $K$-sample analogues of two-sample
Kolmogorov--Smirnov stastistics\jour Selecta Statistica Canadiana\vol 6
\yr 1982\pages 37--62\endref
\ref\no 2\by W. Chu\paper Dominance method for plane partitions. IV:
Enumeration of flagged skew tableaux\jour 
Linear and Multilinear Algebra\vol 28\yr 1990\pages 185--191\endref
\ref\no3\by Gessel, I., G. Viennot \paper Binomial
determinants, paths, and hook length formulae\jour Adv\. in Math\. 
\vol 58\yr 1985\pages 300--321\endref
\ref\no 4\by I. Gessel, G. Viennot\paper Determinants, paths, and
plane partitions\paperinfo preprint, 1989\endref
\ref\no5\by C. Krattenthaler\paper   Generating functions for plane 
partitions of  a  given 
shape\jour  Manuscripta Math\.\vol 69\yr1990\pages 173--202\endref
\ref\no 6 \by I. G. Macdonald \book Symmetric Functions and Hall 
Polynomials \publ Oxford University Press\publaddr New 
York/London\yr1979\endref
\ref\no 7\by S.~G.~Mohanty\book Lattice path counting and
applications\publ Academic Press\publaddr New York\yr 1979\endref
\ref\no 8\by L. K. K. Zing\book $K$-sample analogues of
Kolmogorov--Smirnov statistics and group tests
\bookinfo Ph\. D. Thesis, McMaster University, 1978\endref
\ref\no 9\by M.~Wachs\paper Flagged Schur functions, Schubert
polynomials, and symmetrizing operators\jour 
J. Combin\. Theory~A\vol 40\yr 1985\pages 276--289\endref
\endRefs
\vskip20pt

\eightpoint\hangindent10pt\hangafter1\noindent
1991 {\it Mathematics subject classifications}: Primary 05A15;
 Secondary 05A17, 05E10, 11P81.

\hangindent10pt\hangafter1\noindent
{\it Keywords and phrases}: tableaux, generating functions, plane partitions,
nonintersecting lattice\linebreak paths.

\enddocument
