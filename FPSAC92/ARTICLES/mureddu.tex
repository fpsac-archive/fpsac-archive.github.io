\documentstyle[leqno]{article}
\begin{document}
\title{From algebraic sets to monomial linear bases
\\ by means of combinatorial algorithms.}
\author{L. Cerlienco 
and M. Mureddu}
\date{}
\maketitle

\newtheorem{prop}{Prop.}
\newtheorem{cor}{Corollary}
\newtheorem{lemma}{Lemma}
\newcommand{\prova}{\noindent {\bf Proof.} \indent }
\ 

\vskip .7in
\noindent {\bf Summary.}\indent Let $K$ be a field; let ${\cal 
P}\subset K^n$ be a finite set and let $\Im({\cal P})\subset 
K[x_1,\ldots,x_n]$ be the ideal of ${\cal P}$. A purely 
combinatorial algorithm to obtain a linear basis of the quotient 
algebra $K[x_1,\ldots,x_n]/\Im({\cal P})$ is given. Such a basis 
is represented by an $n$-dimensional Ferrers diagram of monomials 
which is minimal with respect 
to the inverse lexicographical order $\preceq_{i.l.}$. 
It is also shown how this algorithm can be extended to the case in 
which ${\cal P}$ is an 
{\em algebraic multiset}. A few  applications  are  stated  (among 
them,  how 
to determine a reduced Gr\"obner basis  of  $\Im({\cal  P})$  with 
respect to $\preceq_{i.l.}$ without using Buchberger's algorithm).
\vskip .3in

\noindent 
\section{Introduction}
\vskip .12in

\noindent {\bf 1.1} \indent
Let {\bf N} be the monoid  of   non-negative   integers.   Denote   by
${\bf i}:=(i_1,\ldots ,i_n)$   an   arbitrary   element   in   the
power
${\bf N}^n$. The usual order on {\bf N}, as  well   as   the   partial
order it induces on ${\bf N}^n$, will be denoted by $\leq$.

Define an {\it n-dimensional Ferrers diagram} to be any finite ideal of
the poset ${\bf N}^n$, i.e.\ any non-empty finite 
subset ${\cal F}\subseteq {\bf N}^n$ such that  ${\bf j}<{\bf i}\in
{\cal F}\Longrightarrow {\bf j}\in {\cal  F}$.
An element
${\bf i}=(i_1,\ldots,i_n)\not\in {\cal  F}$ is said to be a
{\em co-minimal element}  for  the
Ferrers diagram ${\cal  F}$ if it is a minimal element of the
complementary filter ${\bf N}^n\setminus {\cal  F}$, i.e.\
if $(i_1,\ldots,i_{r-1}, i_r-1,i_{r+1},\ldots,i_n)\in {\cal  F}$
for each $r$ such that $i_r\geq 1$.
Of course, ${\bf i}\not\in {\cal  F}$ is a co-minimal element iff
${\cal  F'}:= \{{\bf i}\}\cup {\cal
F}$ is a Ferrers diagram.

We will write $\preceq$ for any {\it term-ordering} on
${\bf N}^n$, i.e.\ a linear ordering which  is compatible with the
monoid structure on ${\bf N}^n $:
$$
{\bf 0}\prec {\bf i} \indent \mbox{ for every }{\bf i}\neq {\bf 0}
\mbox{ in }{\bf N}^n,
$$
$$
{\bf i}\preceq{\bf j}\Longrightarrow {\bf i}+{\bf r}\preceq
{\bf j}+{\bf r}
 \indent \mbox{ for every } {\bf i},{\bf j},{\bf r}\in
{\bf N}^n.
$$
It is well known that every term-ordering on ${\bf N}^n$ is also a
well-ordering.
\vskip .1in

\noindent {\bf 1.2} \indent
Let $K$ be a field and let
$X:=\{x_1,x_2,\ldots ,x_n\}$
be a given set
 of indeter\-minates. Let us consider the usual polynomial algebra
$K[X]:=K[x_1,\ldots ,x_n]$. Denote by $M_X\subseteq  K[X]$  the  free
 abelian  monoid  on  $X$.  The  elements  of  $M_X$  (i.e.\  the
monic  monomials) will  be called  {\it terms} of $K[X]$ and denoted
by
${\bf x^i}:=x_1^{i_1}\cdots x_n^{i_n}$ with
${\bf i}:=(i_1,\ldots ,i_n)\in {\bf N}^n$. The orders $\leq$ and
$\preceq$ on ${\bf N}^n$, as well as the notion of Ferrers diagram,
extend to $M_X$ in an obvious way.

An ideal $J$ of the algebra $K[X]$ is said to be {\it cofinite} if
$$
{\rm codim}(J):={\rm dim}\left(K[X]/J\right)<\infty.
$$
Given a finite set ${\cal P}:=\{P_1,\ldots ,P_N\}\subseteq  K^n$,  the
ideal
$$
\Im({\cal P}):=\{p\in K[X]\mid (\forall P\in {\cal P})(p(P)=0)\}
$$
of the algebraic set ${\cal P}$ is cofinite. More generally, when $K$ is
algebraically closed, from the {\it Nullstellensatz} it follows that
an ideal $J\subseteq K[X]$ is a cofinite ideal iff the algebraic set
of $J$, i.e.\
$$
{\cal V}(J):=\left\{ P\in K^n\mid (\forall p\in J)(p(P)=0)
\right\},
$$
is finite. In this case we have
$\#{\cal V}(J)={\rm codim}(\sqrt{J})\leq {\rm codim}(J)$ (cfr. [5] p.23).

For a given ideal $J$, any linear basis ${\cal L}_J$ of the
quotient algebra
$K[X]/J$ whose elements are of the form $[{\bf x^i}]_J:={\bf x^i}+J$ 
will be called a {\it monomial basis}. 
If ${\cal L}_J=\{{\bf x^i}+J \mid {\bf i}\in L\subseteq {\bf N}^n\}$ 
is a monomial basis, we shall say that 
$\{{\bf x^i} \mid {\bf i}\in L\}$ is a 
{\it system of representatives\/} for the monomial basis
${\cal L}_J$. Obviously, if 
${\cal L}_J=\{{\bf x^i}+J \mid {\bf i}\in L\}$  is  a 
monomial basis  of  $K[X]/J$,  then  any  polynomial  $p\in  K[X]$  is 
congruent modulo $J$ to exactly one polynomial of the form 
$\sum_{{\bf i}\in L}a_{\bf i}{\bf x^i}$, $a_{\bf i}\in  K$.  Given  an 
arbitrary term-ordering $\preceq$ on $M_X$, a  monomial  basis  
${\cal L}_J= \{ {\bf x}^{{\bf i}_1}+J, \ldots, 
{\bf x}^{{\bf i}_N}+J \}$ with 
${\bf x}^{{\bf i}_1}\prec \ldots \prec {\bf x}^{{\bf i}_N}$ 
is said to be {\it minimal} with respect to 
$\preceq$ if for any other monomial basis 
${\cal L'}_J= \{ {\bf x}^{{\bf i'}_1}+J,$ $\ldots,
{\bf x}^{{\bf i'}_N}+J \}$ with 
${\bf x}^{{\bf i'}_1}\prec \ldots \prec {\bf x}^{{\bf i'}_N}$ 
we have ${\bf x}^{{\bf i}_j}\preceq {\bf x}^{{\bf i'}_j}$ 
for $j=1,\ldots,N$. Equivalently, 
${\cal L}_J= \{ {\bf x}^{{\bf i}_1}+J, \ldots, 
{\bf x}^{{\bf i}_N}+J \}$, 
${\bf x}^{{\bf i}_1}\prec \ldots \prec {\bf x}^{{\bf i}_N}$, 
is  {\it minimal} with respect to $\preceq$ if every term 
${\bf x}^{{\bf i}}$, such that 
${\bf x}^{{\bf i}}\prec {\bf x}^{{\bf i}_j}$,
 is linearly dependent (modulo $J$) on 
${\bf x}^{{\bf i}_1}, \ldots,{\bf x}^{{\bf i}_{j-1}}$. 
Of course, both ${\cal L}_J$ and 
${\cal L'}_J$, 
${\cal L}_J\neq {\cal L'}_J$,  could be  minimal  with  respect  to 
different term-orderings. 
It is not hard to prove that if
${\cal L}_J=\{{\bf x}^{{\bf i}_j}+J \mid j=1,\ldots,N\}$  
is  a  minimal  monomial 
basis  then  
$L=\{ {\bf i}_1,\ldots,{\bf i}_N\}\subseteq  {\bf  N}^n$  
is  an  $n$-dimensional   Ferrers diagram. In fact, if 
${\bf x^i}\not \in {\cal L}_J$, then 
$$
{\bf x^i} = \sum_{ {\bf x}^{{\bf i}_j} \prec {\bf x^i}} 
\lambda_j {\bf x}^{{\bf i}_j} \indent {\rm modulo } \, J.
$$
Thus,
$$
{\bf x^{i+r}} = \sum_{ {\bf x}^{{\bf i}_j} \prec {\bf x^i}} 
\lambda_j {\bf x}^{{\bf i}_j+{\bf r}} \indent {\rm modulo}\, J
$$
where ${\bf r}\in {\bf N}^n$ and 
${\bf x}^{{\bf i}_j}\prec {\bf x}^{{\bf i}_j+{\bf r}}\prec 
{\bf x^{i+r}}$ as $\preceq$ is a term-ordering. Since either 
${\bf x}^{{\bf i}_j+{\bf r}}+J$ belongs to ${\cal L}_J$ or 
${\bf x}^{{\bf i}_j+{\bf r}}+J$ is a linear combination  of  smaller 
elements in ${\cal L}_J$, from the above congruence we deduce that 
the same is true  for 
${\bf x^{i+r}}+J$: 
$$
{\bf x^{i+r}} = \sum_{ {\bf x}^{{\bf i}_j} \prec {\bf x^{i+r}}} 
\mu_j {\bf x}^{{\bf i}_j} \indent {\rm modulo} \, J.
$$
Hence, ${\bf i}\not\in L \Rightarrow {\bf i}+{\bf r}\not\in L$,
i.e. ${\bf i}+{\bf r}\in L \Rightarrow {\bf i}\in L$; thus, $L$ is 
a Ferrers diagram. 
\vskip .1in

\noindent {\bf 1.3} \indent
In the search for a minimal monomial basis  ${\cal  L}_{\cal  P}$, 
in the following section we  
present a purely combinatorial algorithm to get it   
from ${\cal  P}$. In fact, making use of the  {\bf  Algorithm  MB} 
below, we associate  a Ferrers diagram 
${\cal MB(P)}=\{ {\bf d}_1,\ldots ,{\bf d}_N\}\subseteq {\bf N}^n$
to any finite set 
${\cal P}:=\{P_1,\ldots ,P_N\}\subseteq  K^n$.
The Ferrers diagram ${\cal MB(P)}$ 
gives the monomial basis ${\cal L}_{\cal P}=\{{\bf x^d}+\Im({\cal 
P}) \mid {\bf d}\in {\cal MB(P)}\}$ which is minimal with 
respect to the inverse lexicographical ordering $\preceq_{i.l.}$ with
$x_1\prec_{i.l.} x_2\prec_{i.l.} \ldots \prec_{i.l.} x_n$ .

The implementation of  {\bf  Algorithm  MB}, as well  as  that  of 
its applications described in section {\bf 4}, presents no serious 
difficulties. 
\vskip .17in

\noindent 
\section{The Algorithm MB}
\vskip .12in

\noindent {\bf 2.1} \indent Let 
${\cal P}:=\{P_1,\ldots ,P_N\} \subseteq K^n $. For 
$1\leq  j\leq N$ and for $0\leq  s\leq n$, put 

$\underline{\cal P} :=(P_1,\ldots ,P_N)$;

${\bf d}_j = (d_{j,1},\ldots,d_{j,n})\in {\bf N}^n$;

$\pi_s \colon  K^n  \longrightarrow  K^s$, \indent
$(x_1,\ldots, x_n)  \mapsto  (x_1,\ldots, x_s)$
\indent ($\pi_0(P)$ is assumed to be 

the empty sequence);

$\pi^s \colon  K^n  \longrightarrow  K^{n-s+1}$, \indent 
$(x_1,\ldots, x_n)  \mapsto  (x_s,\ldots, x_n)$. 
\vskip .1in 

\noindent Given a list $\underline{\cal P}:=(P_1,\ldots,P_N)$ of points, 
we determine an ordered set 
$\underline{\cal F}:=({\bf d}_1,\ldots,{\bf d}_N)$.
\vskip .1in
\begin{description}
\item[MB1.][Initialize.] \\
   Set ${\bf d}_1\leftarrow {\bf d}_2\leftarrow \ldots {\bf d}_N
   \leftarrow  (0,\ldots, 0)$;   $i \leftarrow 1$.\\
   ({\em In the beginning the coordinates of all the elements of 
   $\underline{\cal F}$ are zero.}) 


\item[MB2.][Put the first $i$ points of the  list 
     $\underline{\cal  P}$  in $\cal Q$.] \\
    Set ${\cal Q} \leftarrow \{P_j \mid 1\leq j\leq i\}$.

\item[MB3.][Find which coordinate of ${\bf d}_{i+1}$ has to be 
    changed, say  $ d_{i+1,s}$.] \\
    Set $s \leftarrow \max \{k\geq 1 \mid
    \pi_{k-1} (P_j) = \pi_{k-1}(P_{i+1})$, for some $P_j \in {\cal Q} \}$.\\
    ({\em $s-1$ is the length of the longest initial segment 
    shared 
    by $P_{i+1}$ and some point $P_j \in  {\cal  Q}$.   If  $s>1$, 
    then in successive steps this value decreases.})

\item[MB4.][Find the points that determine 
      the $s$-th coordinate of ${\bf d}_{i+1}$.]\\
     Set ${\cal E} \leftarrow \{j \mid P_j \in {\cal Q}$, \, 
     $\pi_{s-1} (P_j) = \pi_{s-1}(P_{i+1})$, \, 
     $\pi^{s+1} ({\bf d}_j) = \pi^{s+1}({\bf d}_{i+1})\}$.\\
({\em ${\cal E}$ contains the indices of the points of ${\cal Q}$ which have 
the first $s-1$
    coordinates equal to those of $P_{i+1}$ and whose 
    corresponding 
    elements in $\underline{\cal F}$ have the $n-s$ rightmost 
    coordinates equal 
    to those of ${\bf d}_{i+1}$. ${\cal E}$ is always non-empty}.)

\item[MB5.][Assign the value to the $s$-th coordinate of 
    ${\bf d}_{i+1}$.]\\
    Set $d_{i+1,s} \leftarrow (1+ \max\{d_{j,s} \mid j\in {\cal E} \})$.

\item[MB6.][Did you determine the first coordinate of
         ${\bf d}_{i+1}$?]\\
    If $s>1$
\begin{description}
\item[MB6.1.][Find the points that determine 
     another coordinate of ${\bf d}_{i+1}$.]\\
     Set 
     $ {\cal Q} \leftarrow \{P_j \mid 1 \leq j \leq i, \, \pi^s({\bf d}_j)
    = \pi^s({\bf d}_{i+1}) = (d_{i+1,s},\ldots,d_{i+1,n}) \}$.
    ({\em ${\cal Q}$ contains the
    points of $\underline{\cal P}$ different from $P_{i+1}$ whose 
    corresponding elements in
    $\underline{\cal F}$ have the $n-s+1$ rightmost coordinates 
    equal to those of ${\bf d}_{i+1}$}.) 

\item[MB6.2.][Is ${\cal Q}$ empty?]\\
    If ${\cal Q} \not= \emptyset$, return to step {\bf MB3.}
\end{description}

\item[MB7.][Increase $i$.] \\
    Set $i \leftarrow i+1$. If $i<N,$ return to step {\bf MB2.}
\item[MB8.][Done.] Terminate the algorithm. 
\end{description}
\vskip .1in

\noindent {\bf 2.2} \indent 
Let $\underline{\cal P}:=(P_1,\ldots,P_N)$ 
and $\underline{\cal F}:=({\bf d}_1,\ldots,{\bf d}_N)$ as in  {\bf 
2.1}. 
We put: \, ${\cal MB(P)}:=\{ {\bf d}_1,\ldots ,{\bf d}_N\}$; 
$\delta_{\underline{\cal P}}\colon {\cal P}\to {\cal MB(P)}$, 
$P_i\mapsto {\bf d}_i$. Notice  that  now  and  in  the  following 
notations such as $\underline{\cal P}$  and  $\underline{\cal F}$ 
always denote {\it ordered sets} whose underlying sets are ${\cal 
P}$ and ${\cal F}$, respectively; 
moreover, the order on $\underline{\cal F}$, as well as the map 
$\delta_{\underline{\cal P}}$, depends on both the ordered set 
$\underline{\cal P}$ and the {\bf Algorithm MB}. 
One could think that  ${\cal MB(P)}$ is ill-defined, namely that 
it depends on the order which has been used for arranging the 
points $P_1,\ldots,P_N$ when starting {\bf Algorithm MB} (i.e.\ on 
the list $\underline{\cal P}$) rather than on the set
${\cal P}=\{ P_1,\ldots,P_N\}$ itself. Well, this is not true. In fact, 
the following properties hold. 

%Lemma 1
\begin{lemma}
Let $\underline{\cal P} :=(P_1,\ldots ,P_i,P_{i+1},\ldots ,P_N)$, 
$\underline{\cal P'} :=(P_1,\ldots ,P_{i+1},P_i,\ldots ,P_N)$. 
Let 
$\underline{\cal F}:=({\bf d}_1,\ldots ,{\bf d}_i,
{\bf d}_{i+1},\ldots ,{\bf d}_N)$ and 
$\underline{\cal F'}:=({\bf d'}_1,\ldots ,{\bf d'}_i,
{\bf d'}_{i+1},\ldots ,{\bf d'}_N)$ be the lists associated 
by {\bf  Algorithm  MB} with $\underline{\cal P}$ and 
$\underline{\cal P'}$ respectively. Then, 
$\{ {\bf d}_1,\ldots ,{\bf d}_N\}=
\{ {\bf d'}_1,\ldots, {\bf d'}_N\}$.
\end{lemma}

\prova 
Consider 
$$P_i=(a_{i,1},\ldots,a_{i,r-1},a_{i,r},\ldots,a_{i,n}),$$
$$P_{i+1}=
(a_{i+1,1},\ldots,a_{i+1,r-1},a_{i+1,r},\ldots,a_{i+1,n}),$$ 
$${\bf d}_i=(d_{i,1},\ldots,d_{i,r-1},d_{i,r},\ldots,d_{i,n}),$$ 
$${\bf d}_{i+1}=
(d_{i+1,1},\ldots,d_{i+1,r-1},d_{i+1,r},\ldots,d_{i+1,n})$$ 
where  $r$  is  such  that  $d_{i,r}\not   =d_{i+1,r},   d_{i,r+1} 
=d_{i+1,r+1}, \dots, d_{i,n} =d_{i+1,n}$. It may happen that either 
i) $(a_{i,1},\ldots,a_{i,r-1})\not =  
(a_{i+1,1},\ldots,a_{i+1,r-1})$ 
(in which case 
${\bf d'}_{i} ={\bf d}_{i+1}$ and 
${\bf d'}_{i+1} ={\bf d}_{i}$),
\indent or 
ii) $(a_{i,1},\ldots,a_{i,r-1}) =  
(a_{i+1,1},\ldots,a_{i+1,r-1})$ 
(in which case we have $a_{i,r}\not =a_{i+1,r}$, since otherwise 
$d_{i,r+q}\not =d_{i+1,r+q}$ for some  $q\geq  1$.  From  this  we 
deduce 
${\bf d'}_{i} ={\bf d}_{i}$ 
and 
${\bf d'}_{i+1} ={\bf d}_{i+1}$).

\noindent 
First notice that obviously 
${\bf d}_{1}={\bf d'}_{1}, \ldots,{\bf d}_{i-1}={\bf d'}_{i-1}$. 
We prove that we  also have 
${\bf d}_{i+2}={\bf d'}_{i+2},  \ldots,{\bf  d}_{N}={\bf  d'}_{N}$ 
by  distinguishing the case i) from the case ii). 
In the first case the proof is easy. In fact, when calculating 
${\bf d}_{i+t}$  and/or  ${\bf  d'}_{i+t}$,  the  set  ${\cal  E}$ 
defined in {\bf MB4} cannot contain both $i$ and $i+1$ since 
$\pi^{s+1}({\bf d}_{i})\not = \pi^{s+1}({\bf d}_{i+1})$ 
for every $s\leq r$ and 
$\pi_{s-1}(P_{i})\not = \pi_{s-1}(P_{i+1})$ for every $s> r$. 

\noindent
In case ii), suppose that 
${\bf   d}_{i+2}={\bf   d'}_{i+2},    \ldots,{\bf    d}_{N-1}={\bf 
d'}_{N-1}$; consider 
${\bf d}_N=(d_{N,1}, 
\ldots,d_{N,t-1},d_{N,t}\not =0,d_{N,t+1}=0,\ldots,d_{N,n}=0
)$, so that $(a_{N,1},\ldots,a_{N,t-1})$ is  the  longest  initial 
segment of the coordinates that  $P_N$  shares  with  some  $P_j$, 
$j<N$. By arguing separately in the three cases $t=r$,  $t<r$  and 
$t>r$, it is not difficult to prove that ${\bf d'}_N={\bf d}_N$. 
\hfill $\Box$ 
\vskip .1in

As a straightforward consequence of {\bf Lemma 1}  we  obtain  the 
following proposition. 

%Prop.1
\begin{prop}
Let $\underline{\cal P}$ and $\underline{\cal F}$ be  as  in  {\bf 
Lemma 1}. 
Let $\underline{\cal F'}:=({\bf d'}_1,\ldots ,{\bf d'}_N)$ 
be the list associated  by {\bf  Algorithm  MB} with the list of points
$\underline{\cal P'}=( P_{\sigma(1)},\ldots,P_{\sigma(N)})$, 
where $\sigma\in S_N$. ($S_N$ is the symmetric group.) 
Then,  for  some $\tau\in S_N$ we have 
$({\bf d'}_1,\ldots ,{\bf d'}_N) =
({\bf d}_{\tau(1)},\ldots ,{\bf d}_{\tau(N)})$. \hfill $\Box$
\end{prop}

%Cor.1
\begin{cor} 
If ${\cal P}\subseteq {\cal Q}$, then  ${\cal MB(P)}\subseteq 
{\cal MB(Q)}$.  \hfill $\Box$
\end{cor}

%Prop.2
\begin{prop}
Let $\underline{\cal P}$ and $\underline{\cal F}$ be  as  in  {\bf 
Lemma 1}. 
It is possible to rearrange the points $ P_1,\ldots,P_N$ in a 
suitable list $\underline{\cal P'}=( P_{\sigma(1)},\ldots,P_{\sigma(N)})$ 
such that the elements in the corresponding list
$\underline{\cal F'}=({\bf d}_{\tau(1)},\ldots $   
$\ldots ,{\bf d}_{\tau(N)})$ 
are in inverse lexicographical order 
$\preceq_{i.l.}$: \, 
$s<t \Rightarrow {\bf d}_{\tau(s)}\prec_{i.l.} {\bf d}_{\tau(t)}$. 
\end{prop}

\prova 
Let 
$P_j=(a_{j,1},\ldots,a_{j,r-1},a_{j,r},\ldots,a_{j,n})$, 
${\bf d}_j=(d_{j,1},\ldots,d_{j,r-1},d_{j,r},$ $\ldots,d_{j,n})$, 
$j=1,\ldots,N$. It suffices to observe that if, for some $i$, 
${\bf d}_{i+1} := \delta_{\underline{{\cal P}}}(P_{i+1})
\prec_{i.l.}{\bf d}_{i} := \delta_{\underline{{\cal P}}}(P_{i})$ (i.e. 
$d_{i,r}>d_{i+1,r}, d_{i,r+1}=d_{i+1,r+1}, \ldots, 
d_{i,n}=d_{i+1,n}$ for some $r\leq n$), then we have necessarily 
the case i) considered in the proof of {\bf Lemma 1}.  
\hfill $\Box$

%Lemma 2
\begin{lemma}
Let $\underline{\cal P}$ and $\underline{\cal F}$ be  as  in  {\bf 
Lemma 1}. 
Let  ${\bf d}_N= (d_{N,1},\ldots, d_{N,n})$. If 
$d_{N,i}\not = 0$, then there is some $k<N$ such that 
${\bf d}_k= (d_{N,1},\ldots, d_{N,i-1}, d_{N,i}-1,$ $d_{N,i+1}, 
\ldots, d_{N,n})$.
\end{lemma}

\prova  When calculating  the first $i-1$ coordinates
$d_{N,1},\ldots, d_{N,i-1}$ of 
${\bf d}_N=\delta_{\underline{\cal P}}(P_N)$, we have to consider 
the set 
$\{ P_{j_1}, \ldots, P_{j_s}, P_{j_{s+1}}=P_N\}\subseteq {\cal 
P}$, 
$j_1< \ldots < j_s < N$, of all points  
$P_{j_r}\in {\cal P}$ such that $\pi^i(\delta_{\underline{\cal 
P}}(P_{j_r}))=(d_{N,i},\ldots, d_{N,n})$. 
Putting 
$\underline{\tilde{\cal P}}:=( \pi_{i-1}(P_{j_1}), \ldots, 
\pi_{i-1}(P_{j_s}), \pi_{i-1}(P_N))$, we have 
$\pi_{i-1}(\delta_{\underline{\cal P}}(P_N))=
\delta_{\underline{\tilde{\cal P}}}(\pi_{i-1}(P_N))= 
(d_{N,1},\ldots, d_{N,i-1})$.
For every 
$r\in \{ 1,\ldots, s+1\}$, there is a point $P_{j'_r}\in {\cal P}$, 
$j'_r<j_r$, such that 
$d_{j'_r,i}= d_{j_r,i}-1=d_{N,i}-1, 
d_{j'_r,i+1}=d_{j_r,i+1}=d_{N,i+1},\ldots, 
d_{j'_r,n}=d_{j_r,n}=d_{N,n}$ and
$\pi_{i-1}(P_{j'_r})=\pi_{i-1}(P_{j_r})$. 
Up to a suitable rearrangement of the points in 
$\underline{\cal P}$, we may 
assume that $j'_1< \ldots < j'_{s+1}$. 
In order to calculate 
$\pi_{i-1}(\delta_{\underline{\cal P}}(P_{j'_{s+1}}))$ 
we have to consider the set 
${\cal Q}:=\{ P_l\in {\cal P}\mid l\leq j'_{s+1}, \, 
\pi^i({\bf d}_l)=(d_{j'_{s+1},i},\ldots, d_{j'_{s+1},n})=
(d_{N,i}-1,d_{N,i+1}\ldots, d_{N,n})\}$. Of course 
$P_{j'_r}\in {\cal Q}$. Let 
$\tilde{\cal Q}:=\{ \pi_{i-1}(P)\mid P\in {\cal Q}\}\subseteq
K^{i-1}$; we have $\tilde{\cal P}\subseteq \tilde{\cal Q}$. 
Hence ${\cal MB(}\tilde{{\cal P}}{\cal )}
\subseteq {\cal MB(}\tilde{{\cal Q}}{\cal )}\subseteq 
{\bf N}^{i-1}$; in particular 
$(d_{N,1},\ldots, d_{N,i-1})\in {\cal MB(}\tilde{{\cal Q}}{\cal 
)}$. It follows that there exists a point $P_k\in {\cal Q}$ such
that 
$\delta_{\underline{\cal P}}(P_k)=(d_{N,1},\ldots, d_{N,i-1},d_{N,i}-1, 
d_{N,i+1}\ldots, d_{N,n})$. \hfill $\Box$
\vskip .1in

As a straightforward consequence of {\bf Lemma 2} we get: 

%Prop.3 
\begin{prop}
The set  ${\cal MB(P)}$ is an $n$-dimensional Ferrers diagram. 
\hfill $\Box$
\end{prop}

%Prop.4 
\begin{prop}
The set
${\cal L}_{\cal  P}=\{{\bf x^d}+\Im({\cal P}) \mid {\bf d}\in 
{\cal MB(P)}\}$ is a monomial  linear  basis  for 
$K[X]/\Im({\cal P})$.
\end{prop} 
\prova 
By induction on the dimension $n$ of $K^n$.

For $n=1$, we have ${\cal P}=\{\rho_1,\ldots,\rho_N\}$, $\rho_i\in 
K$, and $\Im({\cal P})=(g)$, with $g=\Pi_{i=1}^N  (x-\rho_i)$.  {\bf 
Algorithm MB} gives ${\cal MB(P)}=\{0,1,\ldots,N-1\}$; hence 
${\cal L}_{\cal  P}=\{{\bf x^d}+\Im({\cal P}) \mid 
{\bf d}\in {\cal  MB(P)}\}=\{1+(g),x+(g),\ldots,x^{N-1}+(g)\}$, 
which  is  a monomial basis for $K[X]/(g)$. 

Suppose now that the statement is true for every finite subset  of 
$K^{n'}$, $n'<n$, and prove it  for  ${\cal  P}=\{P_1,\ldots,P_N\} 
\subset K^n$. As $\dim K[X]/\Im({\cal P})=N=\#{\cal  MB(P)}$, it
remains to prove that the residue classes modulo $\Im({\cal P})$ of 
the monomials ${\bf x^d}$, ${\bf d}\in {\cal  MB(P)}$, are 
linearly independent over $K[X]/\Im({\cal P})$; in other words, we 
have to prove that any polynomial 
\begin{equation}
p(x_1,\ldots,x_n)=\sum_{{\bf d}\in {\cal  MB(P)}} \alpha_{\bf d}
x^{\bf d},  \indent \alpha_{\bf d}\in K
\end{equation}
belonging to $\Im({\cal P})$ is the zero polynomial. 

Putting ${\cal F}:={\cal  MB(P)}$, 
${\cal F}_r:=\{{\bf d}=(d_1,\ldots,d_n)\in {\cal F}\mid d_n=r\}$ and
${\cal P}_r:=\{P\in {\cal P}\mid \delta_{\underline{\cal P}}(P) \in
{\cal F}_r\}$, it is easy to check that

\begin{equation}
{\cal MB}(\pi_{n-1}({\cal P}_r))= \pi_{n-1}({\cal F}_r).
\end{equation}
Let us write polynomial (1) in the form
\begin{equation}
p(x_1,\ldots,x_n)=\sum_{r=0}^h p_r(x_1,\ldots,x_{n-1})x_n^r
\end{equation}
where $h=$max$\{ d_n\mid {\bf d}=(d_1,\ldots,d_n)\in {\cal F}\}$ and
\begin{equation}
p_r(x_1,\ldots,x_{n-1})=
\sum_{(d_1,\ldots,d_{n-1})\in \pi_{n-1}({\cal F}_r)}
\alpha_{(d_1,\ldots,d_{n-1},r)} x_1^{d_1}\cdots x_{n-1}^{d_{n-1}}.
\end{equation}

The polynomial $p(x_1,\ldots,x_n)\in \Im({\cal P})$ 
has to vanish at every point in
${\cal P}$. Consider a point $P=(a_1,\ldots, a_n)\in {\cal
P}_h\subseteq {\cal P}$; because of {\bf MB5}, there exists in ${\cal P}$ 
exactly $h+1$
points that have the same first $n-1$ coordinates as
$P=(a_1,\ldots, a_n)$. It follows that the polynomial
$$
p(a_1,\ldots,a_{n-1},x_n)=\sum_{r=0}^h p_r(a_1,\ldots,a_{n-1})x_n^r
$$
vanishes identically. In particular $p_h(a_1,\ldots,a_{n-1})=0$
for every $(a_1,\ldots,a_{n-1})\in$ $\in {\cal Q}_h:=\pi_{n-1}({\cal
P}_h)$. Hence
\begin{equation}
p_h(x_1,\ldots,x_{n-1})\in \Im({\cal Q}_h)\subseteq
K[x_1,\ldots,x_{n-1}].
\end{equation}
By (2) and (4) we have
\begin{equation}
p_h(x_1,\ldots,x_{n-1})=
\sum_{(d_1,\ldots,d_{n-1})\in {\cal MB}({\cal Q}_h)}
\alpha_{(d_1,\ldots,d_{n-1},h)} x_1^{d_1}\cdots x_{n-1}^{d_{n-1}}.
\end{equation}
Because of the inductive hypothesis, the set
$\{x_1^{d_1}\cdots x_{n-1}^{d_{n-1}} + \Im({\cal Q}_h)\mid
(d_1,\ldots$ $\ldots, d_{n-1})\in {\cal MB}({\cal Q}_h)\}$
is a monomial basis for $K[x_1,\ldots, x_{n-1}]/\Im({\cal Q}_h)$.
>From this and from (5) we deduce that polynomial (6) vanishes
identically. Hence

\begin{equation}
p(x_1,\ldots,x_n)=\sum_{r=0}^{h-1} p_r(x_1,\ldots,x_{n-1})x_n^r.
\end{equation}

Arguing for $r=h-1, h-2,\ldots, 1,0$ as for $r=h$, we conclude
that $p_r(x_1,\ldots,x_{n-1})$ is the zero polynomial for every
$r\in \{0,\ldots,h\}$.\hfill $\Box$

%Prop.5 
\begin{prop}
The  monomial linear basis
${\cal L}_{\cal  P}=\{{\bf x^d}+\Im({\cal P}) \mid {\bf d}\in 
{\cal MB(P)}\}$  for $K[X]/\Im({\cal P})$ 
is minimal with respect to inverse lexicographical order 
$\preceq_{i.l.}$ on $M_X$.
\end{prop}
\prova
Let $\underline{\cal P}=(P_1,\ldots,P_N)$, 
$\underline{\cal F}:=\underline{\cal MB(P)}=({\bf d}_1,\ldots,{\bf d}_N)$; 
${\bf d}_i = (d_{i,1},\ldots,$ $ d_{i,n})$, 
$P_i=(a_{i,1},\ldots,a_{i,n})$ for $i = 1,\dots,N$.
Assume that ${\bf d}_1\prec_{i.l.}\ldots\prec_{i.l.}{\bf d}_N$. 
Let $N'\leq N$ and let ${\bf d} = 
(d_{1},\ldots,d_{n})\prec_{i.l.}{\bf d}_{N'}$. 
We have to prove that ${\bf x}^{\bf d}$ is linearly dependent 
(modulo $\Im({\cal P})$) on 
${\bf x}^{{\bf d}_1},\ldots,{\bf x}^{{\bf d}_{N'-1}}$, i.e. there 
exists in $\Im({\cal P})$ a non-zero polynomial 
$p(x_1,\ldots,x_n)$ of the form 
%(8)
\begin{equation}
p(x_1,\ldots,x_n)=
\sum_{i=1}^{N'-1}\alpha_i{\bf x}^{{\bf d}_i}+\alpha {\bf x^d}. 
\end{equation}
Obviously, we may assume that ${\bf d}_{N'-1}\prec_{i.l.}{\bf d}$. 
Because of {\bf MB6.1} of   {\bf Algorithm MB}, 
without loss of generality we may also assume that 
$d_n<d_{N',n}$. Thus, $h:=d_{N',n}>0$ 
and $d_{j,n}<h$ for every $j\in \{1,\ldots,N'-1\}$.


Let us first prove that there is a polynomial of the form (8) 
which belongs to $\Im({\cal P}')\supseteq\Im({\cal P})$,  where 
${\cal P}'=\{P_1,\ldots,P_{N'}\}$. Equivalently, let us show that 
the linear homogeneous system 
%(9)
\begin{equation}
\sum_{i=1}^{N'-1}\alpha_i a_{s,1}^{d_{i,1}} \cdots a_{s,n}^{d_{i,n}}
+\alpha a_{s,1}^{d_1} \cdots a_{s,n}^{d_n} = 0, \indent \indent
(s = 1,\ldots,N') 
\end{equation}
admits a non-zero solution 
$(\alpha_1,\ldots,\alpha_{N'-1},\alpha)$. In fact the $N'$ by $N'$ 
matrix $A$ whose $s-th$ row is
$$
a_{s,1}^{d_{1,1}} \cdots a_{s,n}^{d_{1,n}}
\indent \ldots  \indent 
a_{s,1}^{d_{N'-1,1}} \cdots a_{s,n}^{d_{N'-1,n}}\indent 
a_{s,1}^{d_1} \cdots a_{s,n}^{d_n} 
$$
(that is, the evaluation at $P_s$ of the list of monomials 
${\bf x}^{{\bf d}_1},\ldots,{\bf x}^{{\bf d}_{N'-1}},
{\bf  x}^{\bf d}$) is a singular matrix.
In order to prove this, consider the $h+1$ points 
$P_{t_1},\ldots,P_{t_h}$ and $P_{t_{h+1}}=P_{N'}$ ($1\leq t_j\leq N'$) 
which have the same $n-1$ first coordinates as $P_{N'}$ ({\bf MB5} 
ensures that ${\cal P}$ contains $h+1$ such points). 
Let $A'$ 
be the $h+1$ by $N'$ submatrix of $A$ whose rows correspond to 
these points. It is enough to show that $A'$ has rank less than 
$h+1$. In fact, any $h+1$ by $h+1$ minor $C$ of $A'$ has 
columns which correspond to monomials 
${\bf x^i} = x_1^{i_1} \cdots x_{n-1}^{i_{n-1}} x_n^{i_n}$, 
${\bf i}\in \{ {\bf d}_1,\ldots,{\bf d}_{N'-1},{\bf d}\}$, 
so that $i_n$ may assume only the $h$ values $0,1,\ldots,h-1$; 
since  the $(n-1)$-tuples $(a_{j,1},\ldots,a_{j,n-1})$ are always 
the same for $j\in \{t_1,\ldots, t_{h+1}\}$, we conclude 
that $C=0$. This completes the proof that there is a polynomial 
$p(x_1,\ldots,x_n)$ of the form (8) which belongs to $\Im({\cal P'})$.

Let us now prove that we have $p(P_l)=0$  for every 
$P_l\in {\cal P}\setminus {\cal P'}$. Let 
$P_l=(a_{l,1},\ldots,a_{l,n-1}, a_{l,n})$ be such a point; 
consider the polynomial 
$$
q_l(x_n):= p(a_{l,1},\ldots,a_{l,n-1},x_n)=\sum_{s=0}^{h-1} 
q_{l,s} x_n^s. 
$$
Since ${\bf d}_{N'}\prec_{i.l.} {\bf d}_l$, i.e. 
$h= d_{N',n}\leq d_{l,n}$,
there are in ${\cal P'}$ at least $h$ points which have the 
same $n-1$ first coordinates as $P_l$; it follows that the 
polynomial $q_l(x_n)$ 
vanishes identically. Thus, $p(P_l)=0$. 
\ \hfill $\Box$

\vskip .1in

\noindent {\bf 2.3} \indent 
In the case where $n=2$ {\bf Algorithm MB} can be given the 
following simplified form.

Assume that 
$$
{\cal P}=\{(a_1,b_{1,1}), \ldots, (a_1,b_{1,h_1}),\ldots,
(a_m,b_{m,1}), \ldots, (a_m,b_{m,h_m})\}
$$
with $h_1+\ldots +h_m=N$, $h_1\geq \ldots \geq h_m$ and 
$i\not = j \Rightarrow a_i\not =a_j$. Then,
$$
{\cal MB(P)}=\{(p,q)\mid 0\leq p<m, \, 0\leq q<h_p\}.
$$
\vskip .17in

\noindent 
\section{Generalization to the algebraic multisets}
\vskip .12in

\noindent {\bf 3.1} \indent
All that we have stated up to now can be generalized to what  we  might 
call {\em algebraic multisets} in the following sense. 

Consider the linear map 
\begin{eqnarray*}
{\bf D_i}\colon  K[X] & \longrightarrow &  K[X] \\
{\bf x^h}  & \longmapsto & {{\bf h}\choose {\bf i}}{\bf x^{h-i}}
\end{eqnarray*}
where 
${\bf h}=(h_1,\ldots,h_n)$, ${\bf i}=(i_1,\ldots,i_n)\in  {\bf  N}$ 
and ${{\bf h}\choose  {\bf  i}}:={  h_1\choose   i_1}\cdot  \ldots
\cdot { h_n\choose  i_n}$. We have the formula 
%(10)
\begin{equation}
{\bf D_i}(fg)= \sum_{{\bf h}+{\bf  k}={\bf  i}} 
{\bf  D_h}(f){\bf D_k}(g).
\end{equation}

\noindent Observe that when  the  field  $K$  has 
characteristic zero, then 
$$
{\bf D_i}=\frac{1}{\bf i!}{\bf D^i}:=
\frac{1}{\bf i!} 
\frac{\partial^{i_1+\cdots +i_n}}{\partial x_1^{i_1}\cdots \partial
x_n^{i_n}}
$$
where  ${\bf i!}:=i_1!\cdots i_n!$.

\noindent Let $v_P$ be the evaluation map at the point $P$:
\begin{eqnarray*}
v_P\colon  K[X] & \longrightarrow &  K \\
q  & \longmapsto & q(P).
\end{eqnarray*}
Define the linear map  $v^{\bf i}_P$ as the composition 
$v_P\circ{\bf D_i}$, i.e.\ 
\begin{eqnarray*}
v^{\bf i}_P \colon K[X] & \longrightarrow &  K \\
q  & \longmapsto & ({\bf D_i}q)(P).
\end{eqnarray*}
For every ideal $J$ of $K[X]$ and every 
$P\in {\cal V}(J)$, put 
$$
{\cal F}_J(P):= \left\{
{\bf i}\in  {\bf  N}^n\big|\; (\forall 
p\in J)\left(v^{\bf i}_P(p)=0\right)\right\}. 
$$

%Prop.6 
\begin{prop}
i) ${\cal F}_J(P)$ is an $n$-dimensional Ferrers diagram;  ii)  if 
$(g_1,\ldots ,g_s)$ is a system of generators of $J$, then 
${\cal F}_J(P)$ is the largest $n$-dimensional 
Ferrers diagram  contained  in  the 
set 
$$
 \left\{{\bf i}\in  {\bf  N}^n\big|\;
v^{\bf i}_P(g_j)=0
\;  \mbox{for every}\; 1\leq j\leq s \right\}. 
$$
\end{prop}

\prova 
Let $P=(a_1,\ldots,a_n)$. Let ${\bf i}\in {\cal F}_J(P)$ and let 
${\bf j}<{\bf i}$. Let us prove that ${\bf j}\in {\cal F}_J(P)$. 
Let $f:=({\bf x}-P)^{{\bf i}-{\bf j}}= 
(x_1-a_1)^{i_1-j_1}\cdots (x_n-a_n)^{i_n-j_n}$. We have 
$v^{\bf h}_P(f)=({\bf D_h}f)(P)=\delta^{{\bf i}-{\bf j}}_{\bf h}$. 
>From (10) we deduce that for every $g\in J$ we have 
$$
0=v^{\bf i}_P(fg)=
\sum_{{\bf h}+{\bf  k}={\bf  i}} 
v^{\bf h}_P(f) \cdot v^{\bf k}_P(g)= 
v^{\bf j}_P(g).
$$
Hence ${\bf j}\in {\cal F}_J(P)$. 
The second part of the statement is a straightforward  consequence 
of i) and of (10). \hfill $\Box$


We define a {\em  finite  $n$-dimensional  algebraic  multiset},  or 
simply an  {\em  algebraic multiset}, to be  a set
$\wp  :=\{(P_1,{\cal   F}_1),\ldots  ,(P_N,{\cal   F}_N)\}$;  each 
element $(P_j,{\cal   F}_j)\in \wp$ consists of a point $P_j$ of  $K^n$ 
together with an $n$-dimensional 
Ferrers diagram ${\cal   F}_j\subset  {\bf  N}^n$, 
which will be called the {\em  algebraic  diagram}  of  the  point 
$P_j$. We shall freely make use of the notation $(P,{\bf i})\in  \wp$,  or 
also $P\in \wp$, to mean that $P=P_j$ and ${\bf i}\in {\cal F}_j$
for  some  $j\in \{1,\ldots,N\}$. To every algebraic multiset
$\wp  =\{(P_1,{\cal    F}_1),\ldots   ,(P_N,{\cal    F}_N)\}$   we 
associate the set 
$$
\Im(\wp):=\left\{ p\in K[X]\,\big|
\;(\forall j)(\forall {\bf i}\in {\cal F}_j)\left(
v^{\bf i}_{P_j}(p)=0\right)\right\}.
$$
It is not difficult to prove that $\Im(\wp)$ is a  cofinite  ideal 
of  $K[X]$  and  that  ${\cal  F}_{\Im(\wp)}(P_j)={\cal  F}_j$ 
for every $j\in \{1,\ldots,N\}$. Moreover, one can prove that 
$$
{\rm codim} \ \Im(\wp)=\#\wp:=\sum_{j=1}^N\#{\cal  F}_j .
$$ 

The question now is: how do we get a monomial linear basis for 
$K[X]/\Im(\wp)$? Once more the problem can be solved by applying 
a slightly modified version of {\bf Algorithm MB} (in fact 
{\bf Algorithm MB} itself with a few obvious changes) to a suitable 
set ${\cal R}(\wp)\subset (K\times {\bf N})^n$ associated with the 
algebraic multiset $\wp$. The set ${\cal R}(\wp)$ will be called the  {\em
canonical representation} of the algebraic multiset $\wp$. 
To be precise, consider the bijection 
\begin{eqnarray*}
u\colon K^n\times {\bf N}^n \indent \indent & \longrightarrow &  
\indent \indent (K\times {\bf N})^n \\
((a_1,\ldots,a_n),(i_1,\ldots,i_n))      &      \longmapsto      & 
((a_1,i_1),\ldots,(a_n,i_n)).
\end{eqnarray*}
Put
$$
{\cal R}={\cal R}(\wp):=\{u(P,{\bf  i})\mid (P,{\bf  i})\in \wp\}.
$$
and 
$$
{\cal MB}(\wp):={\cal MB}({\cal R}).
$$
It is  possible (though by no means trivial) to  prove 
that ${\cal MB}(\wp)$ satisfies properties analogous to those of 
${\cal MB}({\cal P})$; in particular, i) ${\cal MB}(\wp)$ is 
an $n$-dimensional   Ferrers diagram and ii) the set
${\cal L}_\wp :=\{{\bf x^i}+{\Im(\wp)}\mid {\bf i}\in {\cal 
MB}(\wp)\}$  is a monomial linear basis of  $K[X]/\Im(\wp)$  which 
is minimal with respect to the inverse lexicographical order. 
\vskip .17in

\noindent 
\section{Applications}
\vskip .12in

Both  {\bf Algorithm MB} and its generalization may  come  in  
handy  for  solving  various problems. Let us examine a few of 
them.
\vskip .05in 

\noindent {\bf 4.1} \indent 
First of all, let us see how  to  determine  a 
system of generators $(\gamma_1,\ldots,\gamma_r)$ for the  ideal  $\Im(\wp)$ 
of a finite algebraic multiset 
$\wp  :=\{(P_1,{\cal   F}_1),\ldots   ,(P_N,{\cal    F}_N)\}$.  In 
fact, the set $\{\gamma_1,\ldots,\gamma_r\}$ we shall obtain is  also  a  
{\em reduced  Gr\"obner  basis}  of  $\Im(\wp)$ with respect to 
$\preceq_{i.l.}$.  
(The  notion  of 
reduced Gr\"obner basis as well as related topics can be found  in 
[1], [3], [8].) It  goes  without 
saying that the  same  procedure  works  also  when  $\wp$  is  a  
finite algebraic set. 

Let 
${\cal L}_\wp :=\{{\bf x^d}+{\Im(\wp)}\mid {\bf d}\in 
{\cal MB}(\wp)\}=\{ {\bf x}^{{\bf d}_1},\ldots,{\bf x}^{{\bf d}_m} 
\}$, 
where $m=\# \wp={\rm codim}(\Im(\wp))$, 
be the monomial linear basis obtained in {\bf 3.1} and let
$\{ {{\bf r}_1},\ldots,{\bf r}_r \}$ 
be the  set of the co-minimal elements of the Ferrers diagram 
${\cal MB}(\wp)$. For each ${\bf r}_i$, $i=1,\ldots,s$, 
define the polynomial $\gamma_i\in 
K[X]$ to be the following determinant. The  
first  row
of $\gamma_i$ is the list 
$({\bf x}^{{\bf d}_1},\ldots,{\bf x}^{{\bf d}_m}, 
{\bf x}^{{\bf r}_i})$, 
while the successive rows are lists of  the form 
$(v^{\bf i}_{P_j}({\bf x}^{{\bf d}_1}),\ldots,
v^{\bf i}_{P_j}({\bf x}^{{\bf d}_m}), 
v^{\bf i}_{P_j}({\bf x}^{{\bf r}_i}))$, 
one for each $(P_j,{\bf i})\in \wp$. It can easily be proved that the list 
$(\gamma_1,\ldots,\gamma_r)$ 
is a reduced Gr\"obner basis of $\Im(\wp)$ with respect to the inverse 
lexicographical order. Once the basis $(\gamma_1,\ldots,\gamma_r)$ 
has been found, a Gr\"obner basis with respect to  a  different 
term-ordering may be computed by making use of the  methods  given 
in [4].
\vskip .1in

\noindent {\bf 4.2} \indent 
Consider  a  linear  form  $f\in  K[X]^{\ast}\cong 
K[[X]]$. If Ker$(f)$ contains a cofinite ideal $J$ of $K[X]$, then
$f$ is said to be an {\em $n$-linearly recursive function}  and  $J$ 
is called a {\em characteristic ideal} of  $f$.  This  notion  has 
been introduced in [2] as a generalization of that of {\em
linearly  recursive 
sequence},  to  which  it  reduces  when  $n=1$. 
An $n$-linearly recursive function $f$ is uniquely determined by 
its {\em characteristic ideal} $J$ and by its {\em initial 
values}, which are the values assumed by $f$ on a system of 
representatives of a linear basis of $K[X]/J$. 
The set of $n$-linearly recursive functions may   also   be 
regarded as  the dual bialgebra of the usual polynomial 
bialgebra on $K[X]$. When working on these subjects, it may  happen 
that examples (perhaps, {\em suitable} examples) of 
$n$-linearly recursive functions are needed. How to construct  them? 
It is convenient to divide the answer to this  question  into  two 
parts. 

\noindent (A) Let us first suppose that we know a system of generators 
$(g_1,\ldots,g_s)$ of the characteristic ideal $J$ of the
$n$-linearly recursive functions we are considering. In this case we 
may calculate a reduced Gr\"obner basis 
$G_{\preceq}:= RGB(g_1,\ldots,g_s)$ 
of $(g_1,\ldots,g_s)$ 
with respect to some term-ordering $\preceq$ on $M_X$.
Let $G_{\preceq}=(\gamma_1,\ldots,\gamma_r)$,     $\gamma_j\in
K[X]$, and let ${\bf y}_j\in M_X$ be the leading term (with  respect  to 
$\preceq$) of the polynomial $\gamma_j$. Then the set 
$\{ {\bf x^d} + J\mid {\bf x^d}\in B \}$, where 
$B=M_X\setminus \cup_{j=1}^r {\bf y}_j\cdot M_X$,  is  a  monomial 
linear basis for $K[X]/J$, which is minimal with respect to  
$\preceq$. It follows that any $n$-linearly recursive function whose 
characteristic ideal is $J$ is uniquely determined by the set 
of  its  {\em  initial values} 
$\{f({\bf x^d})\mid {\bf x^d}\in  B\}$, all the other values 
$f({\bf x^i})$, ${\bf x^i}\not \in B$, 
being calculated by making use of the polynomials 
$\gamma_j\in G_{\preceq}$ as {\em scales of recurrence}. 

\noindent (B) If instead  the  characteristic ideal is  not 
given, we are quite unlikely to obtain one of them (remember it 
must be cofinite!) just choosing at random  a  set  of  generators: 
most of the time we would  get  a  non-cofinite ideal  or,  when 
cofinite, a trivial one. The  correct  way  to  do  this  consists 
instead in choosing a finite algebraic multiset (possibly, a finite
algebraic set) $\wp$ and then determining both the monomial linear
basis ${\cal L}_\wp$ of $K[X]/\Im(\wp)$ and  a set of generators for 
$\Im(\wp)$ by means of the machinery described above.
\vskip .1in 

\noindent {\bf 4.3} \indent 
Lastly, consider the following  interpolation 
problem. Given a finite $n$-dimensional algebraic multiset 
$\wp  :=\{(P_1,{\cal   F}_1),\ldots   ,(P_N,{\cal    F}_N)\}$  and 
a set of values $\{\alpha_{j,{\bf i}}\mid j=1,\ldots,N\, {\rm and} \; 
{\bf i}\in {\cal F}_j\}\subset K$ determine the {\em unique} 
polynomial $p$ of the form
$\sum_{{\bf x^i}\in B}a_{\bf i}{\bf x^i}$ ($a_{\bf  i}\in  K$  and 
$B$ is a system of representatives of a monomial linear basis of 
$K[X]/\Im(\wp)$) such that 
$v^{\bf i}_{P_j}(p)=\alpha_{j,{\bf i}}$. 

This problem appears to be an $n$-dimensional natural generalization  of 
the unidimensional one which is solved by means of  
the Lagrange-Hermite  interpolation
formula.
Once  more,  the 
key point for solving this problem is to  determine  the  monomial
basis ${\cal L}_\wp$. Let 
${\cal L}_\wp=
\{ {\bf x}^{{\bf d}_1}+\Im(\wp),\ldots,{\bf x}^{{\bf d}_m}+\Im(\wp) \}$ 
and let 
${\cal A}$ be the $m\times m$ matrix whose rows are of the form 
$(v^{\bf i}_{P_j}({\bf x}^{{\bf d}_1}),\ldots,
(v^{\bf i}_{P_j}({\bf x}^{{\bf d}_m}))$, 
one for each $(P_j,{\bf i})\in  \wp$.  Since  ${\cal  L}_\wp$ is  a 
linear  basis  of  $K[X]/\Im(\wp)$,  the  matrix  ${\cal  A}$   is 
non-singular. Consider the vector 
$\overline{\alpha}$ whose components are the values 
$\alpha_{j,{\bf i}}=v^{\bf i}_{P_j}(p)$ (arranged according to the 
order that has been used for the rows of ${\cal A}$). The components of
the vector $\overline{\beta}:={\cal A}^{-1}\cdot \overline{\alpha}$ 
are the coefficients of the desired polynomial. 
\vskip .2in 

\noindent {\bf Appendix.}
\vskip .2in

\noindent 1) Example of {\bf Algorithm  MB}  for  a  4-dimensional 
set.

$$
\begin{array}{ccc}
{\cal  P}&\indent   \longleftrightarrow   &\indent 
{\cal  F}={\cal MB(P)}\\ 
\ &\ &\ \\
P_{1}=(2,3,9,4)&\indent\longleftrightarrow&\indent {\bf d}_{1}=(0,0,0,0)\\
P_{2}=(2,5,7,3)&\indent\longleftrightarrow&\indent {\bf d}_{2}=(0,1,0,0)\\
P_{3}=(2,3,3,2)&\indent\longleftrightarrow&\indent {\bf d}_{3}=(0,0,1,0)\\
P_{4}=(2,5,5,1)&\indent\longleftrightarrow&\indent {\bf d}_{4}=(0,1,1,0)\\
P_{5}=(6,1,1,3)&\indent\longleftrightarrow&\indent {\bf d}_{5}=(1,0,0,0)\\
P_{6}=(2,3,3,6)&\indent\longleftrightarrow&\indent {\bf d}_{6}=(0,0,0,1)\\
P_{7}=(8,3,4,0)&\indent\longleftrightarrow&\indent {\bf d}_{7}=(2,0,0,0)\\
P_{8}=(6,5,6,3)&\indent\longleftrightarrow&\indent {\bf d}_{8}=(1,1,0,0)\\
P_{9}=(4,7,6,6)&\indent\longleftrightarrow&\indent {\bf d}_{9}=(3,0,0,0)\\
P_{10}=(4,1,7,7)&\indent\longleftrightarrow&\indent {\bf d}_{10}=(2,1,0,0)\\
P_{11}=(2,5,7,8)&\indent\longleftrightarrow&\indent {\bf d}_{11}=(0,1,0,1)\\
P_{12}=(4,3,0,3)&\indent\longleftrightarrow&\indent {\bf d}_{12}=(0,2,0,0)\\
P_{13}=(1,1,2,5)&\indent\longleftrightarrow&\indent {\bf d}_{13}=(4,0,0,0)\\
P_{14}=(2,1,9,0)&\indent\longleftrightarrow&\indent {\bf d}_{14}=(1,2,0,0)\\
P_{15}=(4,1,7,6)&\indent\longleftrightarrow&\indent {\bf d}_{15}=(1,0,0,1)\\
P_{16}=(8,1,8,0)&\indent\longleftrightarrow&\indent {\bf d}_{16}=(3,1,0,0)\\
P_{17}=(2,1,7,6)&\indent\longleftrightarrow&\indent {\bf d}_{17}=(0,2,1,0)\\
P_{18}=(4,3,3,3)&\indent\longleftrightarrow&\indent {\bf d}_{18}=(1,0,1,0)\\
P_{19}=(4,1,1,0)&\indent\longleftrightarrow&\indent {\bf d}_{19}=(1,1,1,0)\\
P_{20}=(8,1,2,4)&\indent\longleftrightarrow&\indent {\bf d}_{20}=(2,0,1,0)\\
P_{21}=(6,3,4,8)&\indent\longleftrightarrow&\indent {\bf d}_{21}=(2,2,0,0)\\
P_{22}=(2,9,6,7)&\indent\longleftrightarrow&\indent {\bf d}_{22}=(0,3,0,0)\\
P_{23}=(2,3,7,1)&\indent\longleftrightarrow&\indent {\bf d}_{23}=(0,0,2,0)\\
P_{24}=(2,1,7,5)&\indent\longleftrightarrow&\indent {\bf d}_{24}=(0,2,0,1)\\
P_{25}=(4,1,1,2)&\indent\longleftrightarrow&\indent {\bf d}_{25}=(0,0,1,1)\\
P_{26}=(2,7,6,5)&\indent\longleftrightarrow&\indent {\bf d}_{26}=(0,4,0,0)\\
P_{27}=(2,5,5,4)&\indent\longleftrightarrow&\indent {\bf d}_{27}=(1,0,1,1)\\
P_{28}=(2,3,7,7)&\indent\longleftrightarrow&\indent {\bf d}_{28}=(0,1,1,1)\\
P_{29}=(2,3,0,2)&\indent\longleftrightarrow&\indent {\bf d}_{29}=(0,0,3,0)\\
P_{30}=(2,1,1,1)&\indent\longleftrightarrow&\indent {\bf d}_{30}=(0,1,2,0)\\
P_{31}=(4,3,3,7)&\indent\longleftrightarrow&\indent {\bf d}_{31}=(1,1,0,1)\\
P_{32}=(8,5,6,4)&\indent\longleftrightarrow&\indent {\bf d}_{32}=(3,2,0,0)\\
P_{33}=(4,5,5,2)&\indent\longleftrightarrow&\indent {\bf d}_{33}=(1,3,0,0)\\
P_{34}=(2,1,1,8)&\indent\longleftrightarrow&\indent {\bf d}_{34}=(0,2,1,1)\\
P_{35}=(8,1,1,1)&\indent\longleftrightarrow&\indent {\bf d}_{35}=(1,0,2,0)\\
P_{36}=(4,5,5,5)&\indent\longleftrightarrow&\indent {\bf d}_{36}=(1,2,0,1)\\
P_{37}=(6,1,8,8)&\indent\longleftrightarrow&\indent {\bf d}_{37}=(3,0,1,0)\\
P_{38}=(2,7,8,4)&\indent\longleftrightarrow&\indent {\bf d}_{38}=(0,3,1,0)
\end{array}
$$
\vskip .2in


The system of representatives for the corresponding monomial basis
is given in the following table. 
\vskip .2in
\noindent \begin{tabular}{|c|cccccc|cccc} \hline
\multicolumn{11}{|c}{} \\
\cline{2-6}  \cline{8-11}
\ & \ & \ & \ & \ & \  & \ & \ & \ & \ & \  \\

 & $1$ & $x_1$ & $x_1^2$ & $x_1^3$ & $x_1^4$ &
          & $x_3$ & $x_1x_3$ & $x_1^2x_3$ & $x_1^3x_3$ \\
\ & \ & \ & \ & \ & \  & \ & \ & \ & \ & \  \\

 & $x_2$ & $x_1x_2$ & $x_1^2x_2$ & $x_1^3x_2$ & &
                   & $x_2x_3 $ & $x_1 x_2 x_3$ &  & \\
\ & \ & \ & \ & \ & \  & \ & \ & \ & \ & \  \\

 & $x_2^2$ & $x_1x_2^2$ & $x_1^2x_2^2$ & $x_1^3x_2^2$ &  &
                                   & $x_2^2 x_3$ &  &  & \\
\ & \ & \ & \ & \ & \  & \ & \ & \ & \ & \  \\


 & $x_2^3$ & $x_1 x_2^3 $ &  &  &  &  & $x_2^3 x_3$ &  &  & \\
\ & \ & \ & \ & \ & \  & \ & \ & \ & \ & \  \\

 & $x_2^4 $ &  &  &  &  &  &  &  &  &  \\

 &  &  &  &  &  &  &  &  &  &   \\
\multicolumn{11}{|c}{} \\
\multicolumn{11}{|c}{} \\
\cline{2-6}  \cline{8-11}
 &  &  &  &  &  &  &  &  &  &   \\

 & $x_3^2$ & $x_1 x_3^2$ &  &  &  &  & $x_3^3$ &  &  & \\
\ & \ & \ & \ & \ & \  & \ & \ & \ & \ & \  \\

 & $x_2 x_3^2$ &  &  &  &  &  &  &  &  &  \\

 &  &  &  &  &  &  &  &  &  & \\
\multicolumn{11}{|c}{} \\
\multicolumn{11}{}{} \\
\multicolumn{11}{}{} \\
\hline
\multicolumn{11}{|c}{} \\
\cline{2-6}  \cline{8-11}
 &  &  &  &  &  &  &  &  &  &   \\
 & $x_4$ & $x_1 x_4$ &  &  &  &  & $x_3 x_4$ & $x_1x_3x_4$ &  & \\

 \ & \ & \ & \ & \ & \  & \ & \ & \ & \ & \  \\

 & $x_2x_4$ & $x_1x_2x_4$ &  &  &  &  & $x_2x_3x_4$ &  &  &  \\
 \ & \ & \ & \ & \ & \  & \ & \ & \ & \ & \  \\


 & $x_2^2x_4$ & $x_1 x_2^2x_4 $ &  &  &  &  &  $x_2^2 x_3x_4$ &  &  & \\

 &  &  &  &  &  &  &  &  &  &   \\
\multicolumn{11}{|c}{}
\end{tabular}

\eject
\noindent 2) Example of {\bf Algorithm MB} for a 3-dimensional
algebraic multiset.
\vskip .2in

Consider the algebraic multiset $\wp$ given by
$$
\begin{array}[t]{llll}
P_1=(0,0,0)\;
&
{\cal F}_1=
\left[
\begin{array}{ccc}
(0,0,0)& (1,0,0)&\ \\
(0,1,0)& \ &\
\end{array}
\right.
&
\left.
\begin{array}{c}
(0,0,1) \\
\
\end{array}
\right]
&
\
\\
\ & \ & \ & \ \\
P_2=(0,0,1)\;
&
{\cal F}_2=
\left[
\begin{array}{ccc}
(0,0,0)& (1,0,0)& (2,0,0)
\end{array}
\right.
&
\left.
\begin{array}{cc}
(0,0,1)& (1,0,1)
\end{array}
\right]
&
\
\\
\ & \ & \ & \ \\
P_3=(1,1,0)\;
&
{\cal F}_3=
\left[
\begin{array}{ccc}
(0,0,0)& (1,0,0)& \ \\
(0,1,0)& (1,1,0)& \
\end{array}
\right.
&
\begin{array}{cc}
(0,0,1)& (1,0,1) \\
(0,1,1) & \
\end{array}
&
\left.
\begin{array}{c}
(0,0,2) \\
(0,1,2)
\end{array}
\right]
\\
\ & \ & \ & \ \\
P_4=(1,1,1)\;
&
{\cal F}_4=
\left[
\begin{array}{c}
(0,0,0)\\
(0,1,0)
\end{array}
\right]
&
\
&
\
\end{array}
$$

The canonical representation   ${\cal  R}={\cal  R}(\wp)$  of  $\wp$
(whose elements are intentionally arranged at random)
as well as the  corresponding $3$-dimensional Ferrers diagram  ${\cal  
MB}({\cal  R})$  are as follows. 

$$
\begin{array}{ccccc}
\wp &\;   \longleftrightarrow   &\;
{\cal  R}&\;   \longleftrightarrow   &\;
{\cal F}={\cal  MB}({\cal  R})\\
\ &\ &\ &\ &\ \\
(1,1,0),(0,0,2)
&\;   \longleftrightarrow   &\;
R_1=(10,\; 10,\; 02)&\; \longleftrightarrow &\;
{\bf d}_1=(0,\;  0,\; 0)\\
(1,1,0),(0,1,2)
&\;   \longleftrightarrow   &\;
R_2=(10,\; 11,\; 02)&\; \longleftrightarrow &\;
{\bf d}_2=(0,\; 1,\; 0)\\
(0,0,1),(1,0,1)
&\;   \longleftrightarrow   &\;
R_3=(01,\; 00,\; 11)&\; \longleftrightarrow &\;
{\bf d}_3=(1,\; 0,\; 0)\\
(1,1,0),(1,0,1)
&\;   \longleftrightarrow   &\;
R_4=(11,\; 10,\; 01)&\; \longleftrightarrow &\;
{\bf d}_4=(2,\; 0,\; 0)\\
(0,0,1),(0,0,1)
&\;   \longleftrightarrow   &\;
R_5=(00,\; 00,\; 11)&\; \longleftrightarrow &\;
{\bf d}_5=(3,\; 0,\; 0)\\
(1,1,0),(0,0,1)
&\;   \longleftrightarrow   &\;
R_6=(10,\; 10,\; 01)&\; \longleftrightarrow &\;
{\bf d}_6=(0,\; 0,\; 1)\\
(0,0,1),(2,0,0)
&\;   \longleftrightarrow   &\;
R_7=(02,\; 00,\; 10)&\; \longleftrightarrow &\;
{\bf d}_7=(4,\; 0,\; 0)\\
(0,0,0),(1,0,0)
&\;   \longleftrightarrow   &\;
R_8=(01,\; 00,\; 00)&\; \longleftrightarrow &\;
{\bf d}_8=(1,\; 0,\; 1)\\
(1,1,0),(0,0,0)
&\;   \longleftrightarrow   &\;
R_9=(10,\; 10,\; 00)&\; \longleftrightarrow &\;
{\bf d}_9=(0,\; 0,\; 2)\\
(0,0,0),(0,1,0)
&\;   \longleftrightarrow   &\;
R_{10}=(00,\; 01,\; 00)&\; \longleftrightarrow &\;
{\bf d}_{10}=(1,\; 1,\; 0)\\
(0,0,0),(0,0,0)
&\;   \longleftrightarrow   &\;
R_{11}=(00,\; 00,\; 00)&\; \longleftrightarrow &\;
{\bf d}_{11}=(2,\; 0,\; 1)\\
(0,0,0),(0,0,1)
&\;   \longleftrightarrow   &\;
R_{12}=(00,\; 00,\; 01)&\; \longleftrightarrow &\;
{\bf d}_{12}=(1,\; 0,\; 2)\\
(1,1,0),(0,1,0)
&\;   \longleftrightarrow   &\;
R_{13}=(10,\; 11,\; 00)&\; \longleftrightarrow &\;
{\bf d}_{13}=(0,\; 1,\; 1)\\
(1,1,0),(1,0,0)
&\;   \longleftrightarrow   &\;
R_{14}=(11,\; 10,\; 00)&\; \longleftrightarrow &\;
{\bf d}_{14}=(3,\; 0,\; 1)\\
(0,0,1),(0,0,0)
&\;   \longleftrightarrow   &\;
R_{15}=(00,\; 00,\; 10)&\; \longleftrightarrow &\;
{\bf d}_{15}=(0,\; 0,\; 3)\\
(1,1,0),(0,1,1)
&\;   \longleftrightarrow   &\;
R_{16}=(10,\; 11,\; 01)&\; \longleftrightarrow &\;
{\bf d}_{16}=(0,\; 1,\; 2)\\
(1,1,0),(1,1,0)
&\;   \longleftrightarrow   &\;
R_{17}=(11,\; 11,\; 00)&\; \longleftrightarrow &\;
{\bf d}_{17}=(2,\; 1,\; 0)\\
(0,0,1),(1,0,0)
&\;   \longleftrightarrow   &\;
R_{18}=(01,\; 00,\; 10)&\; \longleftrightarrow &\;
{\bf d}_{18}=(2,\; 0,\; 2)\\
(1,1,1),(0,1,0)
&\;   \longleftrightarrow   &\;
R_{19}=(10,\; 11,\; 10)&\; \longleftrightarrow &\;
{\bf d}_{19}=(1,\; 0,\; 3)\\
(1,1,1),(0,0,0)
&\;   \longleftrightarrow   &\;
R_{20}=(10,\; 10,\; 10)&\; \longleftrightarrow &\;
{\bf d}_{20}=(0,\; 1,\; 3)
\end{array}
$$
Therefore, a monomial linear basis of  $K[X]/\Im(\wp)$ consists  of
the equivalence classes (modulo $\Im(\wp)$) of the following monomials

\noindent \begin{tabular}{|cccccc|cccc}
\cline{1-5}  \cline{7-10}
 \ & \ & \ & \ & \  & \ & \ & \ & \ & \  \\

 $1$ & $x_1$ & $x_1^2$ & $x_1^3$ & $x_1^4$ &
          & $x_3$ & $x_1x_3$ & $x_1^2x_3$ & $x_1^3x_3$ \\
 \ & \ & \ & \ & \  & \ & \ & \ & \ & \  \\

 $x_2$ & $x_1x_2$ & $x_1^2x_2$ &  & &
                   & $x_2x_3 $ &  & & \\
 \ & \ & \ & \ & \  & \ & \ & \ & \ & \  \\
\multicolumn{10}{}{} \\
\multicolumn{10}{}{} \\
\cline{1-5}  \cline{7-10}
 \ & \ & \ & \ & \  & \ & \ & \ & \ & \  \\

$x_3^2$ & $x_1x_3^2$ & $x_1^2x_3^2$ & & & & $x_3^3$ & $x_1x_3^3$ &  &  \\
 \ & \ & \ & \ & \  & \ & \ & \ & \ & \  \\

 $x_2x_3^2$ &  &  &  & & & $x_2x_3^3 $ &  & & \\

  &  &  &  &  &  &  &  &  &
  \end{tabular}

\vskip .05in
The  reduced  Gr\"obner  basis  of  $\Im(\wp)$  with  respect  to 
$\preceq_{i.l}$ consists of the following polynomials: 
\vskip .02in
$g_1 = x^5-2x^4+x^3$; 

$g_2 = x^3y-2x^2y+xy$; 

$g_3 = y^2+2x^2y-4xy-3x^4+4x^3$; 

$g_4 = x^4z-2x^3z+x^2z-x^4+2x^3-x^2$; 

$g_5 = xyz -yz -x^3z+x^2z-x^4+2x^3-x^2$; 

$g_6 = x^3z^2-x^2z^2+x^4-2x^3+x^2$; 

$g_7 = x^2z^3-xz^3-2x^2z^2+2xz^2-x^3z+2x^2z-xz-x^4+2x^3-x^2$; 

$g_8 = z^4+xz^3-2z^3+x^2z^2-2xz^2+z^2+x^3z-2x^2z+xz+x^4-2x^3+x^2$. 
\vskip .2in


\noindent {\bf Acknowledgements}
\vskip .1in

The authors thank T.Mora, O.Murru, F.Piras and R.Pirastu for their 
advice during several discussions. We are  also  indebted  to  the 
referees of the ``Acts the 4th Conference {\it Formal Power Series 
and Algebraic Combinatorics}, Montreal, 1992'' for  a  few  useful 
suggestions. 






\vskip .3in \noindent

\begin{thebibliography}{99}

\bibitem{[1]}  B.~Buchberger,  {\em  An  Algorithmic   Method   in
Polynomial Ideal Theory}, in ``N.~K.~Bose (ed): Recent  Trends  in
Multidimensional  System  Theory'',  D.~Reidel  Publishing   Co.~, 
Dordrecht-Boston, 1985,
pp. 184-232
\bibitem{[2]}  L.~Cerlienco, F.~Piras, {\em  On the Continuous Dual of a
Polynomial Bialgebra},  Communications in Algebra, 19(10),
2707--2727 (1991)
\bibitem{[3]}   D.~Cox,  J.~Little,  D.~O'Shea,     {\em   Ideals, 
Varieties and   Algorithms}, Springer-Verlag, New York,  1992 
\bibitem{[4]}  J.C.~Faug\`ere,  P.~Gianni, D.~Lazar,  T.~Mora, 
{\em  Efficient computation  of  zero-dimensional-Gr\"obner  bases 
by  change  of ordering}, Submitted to J.~Symb.~Comp.~Technical 
Report LITP 89-52 (1989)
\bibitem{[5]}   W.~Fulton,    {\em    Algebraic    curves},    The  
Benjamin/Cummings, Reading, Mass.,  1981
\bibitem{[6]}  M.G.~Marinari,  H.M.~M\"oller,  T.~Mora,  {\em Gr\"obner 
bases of ideals defined by  functionals  with  an  application  to 
ideals of projective points}, J.~Appl.~Alg.~4 (1993), 1-43
\bibitem{[7]} H.M.~M\"oller, B.~Buchberger,  {\em The  construction
of multivariate polynomials with preassigned zeros}, in
``Computer Algebra (Marseille, 1982)'',  Lecture  Notes  in 
Comp.~Sci., 144, Springer, Berlin-New York, 1982, pp.24-31
\bibitem{[8]}L.~Robbiano,     {\em     Introduzione     all'algebra
computazionale.} Appunti per il corso INDAM. (Roma 1986/87)

\end{thebibliography}

\vskip 1in



\vskip .3in    \noindent
Authors address:

Luigi Cerlienco, Marina Mureddu

Dipartimento di Matematica.

Via Ospedale 72 --- I-09124 Cagliari (Italy)

Tel. 0039-070-6758529/36 --- Fax 0039-070-6758511

e-mail: 

CERLIENCO@VAXCA1.UNICA.IT

MUREDDU@VAXCA1.UNICA.IT

\end{document}
