\documentclass[proceedings,submission
%
% if you want to use pdftex, uncomment the following line
%,pdftex%
% if you have difficulties with hyperref uncomment the following line
%,nohyperref%
% if you have difficulties with fonts uncomment the following line
%,notimes%
]{dmtcs}

\usepackage[latin1]{inputenc}
\usepackage{subfigure}

% graphicx is now loaded automatically no need to put this in here anymore.
% 
%\usepackage{graphicx}

% just comment this out if you don't have natbib
\usepackage[round]{natbib}

\RCSdef$Revision: 1.3 $\endRCSdef
\rcsMajMin


\author{Jens Gustedt\addressmark{1}\thanks{I am not supported.}
  \and Somebody Who\addressmark{2}\thanks{But he is!}
  \and Some Dummy\addressmark{2}}
\title[Formatting a submission for DMTCS]{How to format a submission
  for DMTCS\\
  with the journal's own \LaTeX2e-style}
\address{\addressmark{1}INRIA \& LORIA, campus scientifique, BP 239,
  F-54506 Vand{\oe}uvre l�s Nancy, France\\
\addressmark{2}Alma Mater, campus universalis, terra incognita}
\keywords{some well classifying words}
\revision{\rcsMaj}
% don't try to cheat here, we will check the dates!
\received{14 Oct 1998}
\revised{\today}
\accepted{tomorrow}
\begin{document}
\maketitle
\begin{abstract}
  \begin{quotation}
    This is revision {\rcsMaj.\rcsMin} of this document.
  \end{quotation} 
  This abstract should be some brief description of
  your work and how it advances our science. For journal submissions,
  this is the same text that you fill in the oline form.
  \begin{description}
  \item[Don't] put complicated mathematical expressions here.
  \item[Don't] use commands you defined yourself here.
  \item[Don't] put \verb!\cite! commands here. If you feel that a
    reference is important for the overall estimation of your work,
    spell that reference out by using something like 
    ``\emph{This result has been conjectured by Erd\H{o}s et al. (1931)}''.
  \end{description}
\end{abstract}

For long papers, you may include a table of contents, like the
following. But in most cases this is not necessary. In addition, PDF
documents will have an internal table of contents. Usually this table
is shown by PDF browsers in a different panel than the text itself.

\clearpage
\tableofcontents
\section{Introduction}
\label{sec:in}
The \LaTeX2e-style for DMTCS is derived straight forward from the
usual \texttt{article.sty}. Its main purpose is to ensure a common
layout policy of all articles in DMTCS and to provide editors,
referees and readers with the necessary information. If you think you
need an introduction to \LaTeX2e or search for pointers to other
literature on that, you should consider
reference~\cite{oetiker99:_not_so_short_introd_latex} given at the
end.

Before you read on to know how to use our style file please ensure
that the manuscript fits well into DMTCS.
\begin{enumerate}
\item DMTCS covers \emph{Discrete Mathematics} and \emph{Theoretical 
    Computer Science} as domain of interest. 
\item DMTCS is a \emph{scientific journal}. This requires that your
  work is

  \begin{description}
  \item[original] The main results of your paper must not have
    appeared elsewhere in a journal, neither by yourself nor by
    somebody else.

    There is no excuse for plagiarism, not even self-plagiarism. We
    have a good record in tracking such things down, be warned.
  \item[important] Your results must be of importance to a wider
    public and should be of interest for more readers than just the
    referees. 
  \item[self contained] Other than for conference proceedings, we
    don't have any restrictions on the number of pages for an
    article. So there is no excuse to suppress proofs or not to give
    verbose examples. On the other hand, self containing short notes
    are highly welcome.
  \item[complete] Your work must relate to the current state of the
    art of the domain in question. In particular, foreign and own
    results external to the manuscript must be correctly credited and
    complete reference to such related work must be given. 
  \item[legible] The journal's language is \emph{English}, all
    conventions for scientific work in that language apply. 

  \item[correct] Your work must be mathematical correct \emph{and}
    its quality of writing must be such that the referees will agree
    upon this fact.

    Your writing must be grammatically correct. Be ensured, that
    especially authors that are non-native speakers of English will
    receive all possible help to correct flaws. But also have in mind,
    that incorrect grammar might be the cause of severe
    misunderstandings and finally result in a rejection of the
    paper.  
  \end{description}
\item DMTCS is 
  \begin{description}
  \item[no] circular letter, and
  \item[no] preprint server.
  \end{description}
  If you are looking for that, please consider the wide
  possibilities that the web offers nowadays.
\end{enumerate}


\section{Providing Information for the first Page}
\label{sec:first}

First of all, for a correct submission we need some basic information.
Consider this file here itself as an example how this should be done.
We need the following type of information:

\begin{itemize}
\item The name(s) of the author(s), provided by the \verb!\author!
  command.

  This is about the same as for standard \LaTeX2e. Please refer to
  your \LaTeX\ book to see how this is usually done, or look at the
  examples given here in this file. 

  There is some speciality, though, for sets of authors with different 
  affiliations. In that case, put an appropriate
  \verb!\addressmark{i}! \emph{directly} after the last name of an
  author, where $i$ corresponds to affiliation number $i$. See below,
  on how to include several affiliations into the address command. 
\item The title of the manuscript, provided by the \verb!\title!
  command.

  The title command may be given in two different forms. The first is 
\begin{verbatim}
\title{Your title goes here}
\end{verbatim}
  If done like that, the title that you give is used as running head
  for the odd numbered pages as well. If your title is too long such
  that it doesn't fit into the running head you should use the
  alternative form
\begin{verbatim}
\title[Formatting a submission for DMTCS]{How to format a submission 
for DMTCS with the journal's own \LaTeX2e-style}
\end{verbatim}
  Here the string that inside the \verb![ ]! is used in the
  running head.
\item The address(es) of the authors, provided by the \verb!\address!
  command. 

  If different authors have different affiliatiions, put each 
  such address on a line of its own, separate the lines by a \verb!\\!
  command, and start line number $i$ with the command
  \verb!\addressmark{i}!. For an example, look at the \LaTeX-source of 
  this file here. 
\item The revision n$^0$ of the manuscript, provided by the \verb!\revision!
  command.

  This is to distinguish different versions of your manuscript during
  the refereeing process. The use of the command is as simple as in 
\begin{verbatim}
\revision{1}
\end{verbatim}
\item Some keywords that classify your work, provided by the \verb!\keywords!
  command. Be careful on the choice of these keywords, you are the
  author, you should know best what is adequate such that your
  article can be easily and correctly identified by search engines and 
  alike. Give it in the form
\begin{verbatim}
\keywords{first item, second, third}
\end{verbatim}
  So each ``\emph{key word}'' might consist of several words in the
  usual sense. To separate several key words use commas.

  These keywords must be the same as the ones that are given when you
  fill out the http-form for submission.
\item An abstract of you manuscript, provided by the \verb!abstract!
  environment. This should be no longer than a paragraph and concisely 
  reflect the main contributions of your work. 
  
  Many readers (such as editors) base their selection whether to look
  at a paper more closely on that abstract. In particular there are
  high chances that the decision which referees are assigned to your
  manuscript is mainly based on that abstract. \emph{You have been
    warned.}
  
  This abstract must be the same as the one that is given when you
  fill out the http-form for submission.
\end{itemize}


\section{Hints for the manuscript itself}
\label{sec:hints}
I suppose in the following that you write a paper since you want to
\emph{publish} it, \textit{i.e.}, make it publically available, and
that you want it to be read and understood. Therefore it is imperative
that you stay inside the established conventions for mathematical or
TCS texts. People are used to these conventions.  They help them to
easily and quickly access the real contents of your text and to not to
be diverted by its appearance. 

\subsection{Numbering commands}
\label{sec:numbering}

Please use the standard conventions for all commands and environments
that provide a numbering such as theoremlike environments or
sections. In particular usual counting starts at $1$ and not at
$0$. In particular, we think that an introduction is an integral part
of a paper and should be counted as one ($=1$!). 

\emph{Never ever number items, paragraphs, equations, cases theorems
  or whatsoever manually.} This is the age of computers, use them
before they use you. We need automatically produced numbers to put
hyperlinks into your text, such that reading a paper in DMTCS becomes
a real `electronic' experience.

\subsection{Markup commands}
\label{sec:markup}

Don't use markup of text according to some layout or style, but to
stress semantical differences. The correct way to \emph{emphazise} a
certain part of your text is \verb!\emph{emphazise}!. Don't use the
\verb!\text..! family for that purpose, in particular don't use
\verb!\textit! or \verb!\it!. \emph{They have \emph{different}
  meanings} and rendering. \textit{E.g.}, observe the word
\emph{different} in the previous phrase: this is rendered in an
upright font since this is an \verb!\emph! inside another
\verb!\emph!. On the other hand \verb!\texit! for the abreviations
``\textit{e.g.}'' and ``\textit{i.e.}'' is appropriate. This is
because they are Latin (for \textit{exempli gratia} and \textit{id
  est}), and all foreign text inside English text has to be put in
italics (\textit{sic}!). 

If ever you use commands that change the font use the modern form
\verb!\text..! for them, such as \verb!\textbf{text}!,
\verb!\textit{text}! or \verb!\textsc{text}!. These commands know
better what has to be done when they switch back to normal than the
ancient commands \verb!{\bf text}!, \verb!{\it text\/}! or
\verb!{\sc text}!.

\subsection{Headings}
\label{sec:headings}
Use the standard heading and structuring commands \verb!\section!,
\verb!\subsection! \textit{etc.} to structure your document. For
theorems use the corresponding environments that you may define by
means of the \verb!\newtheorem! command. 

\subsection{Proper Names}
\label{sec:names}
Names of theorems and alike are considered as proper names. In
English (\textit{sic}!) proper names are capitalized. So please write something 
like ``\emph{In Section~\ref{sec:first} we have seen...}'' and
``\emph{By the Main~Theorem we know...}''. But distinguish properly
from the use of the word ``theorem'' as ordinary noun as it is for
example in ``\emph{In the following theorem we prove ...}''.

Please also be careful in the writing of personal names. Customs in
different countries are different! Be sure to use a standard
transcription of names that use a different alphabet than English, and 
also be sure to use the full capabilities of \LaTeX2e for accentuated
character sets that are based on the Latin alphabet. Be sure to catch
the correct concept of ``last name'' in that language.


\subsection{Use a Spell Checker}
\label{sec:check}

It is considered as being very impolite to leave obvious spelling
errors in the manuscript before sending it out. Computers are made for 
these, \emph{use them}.

You might either use the North American variant for spelling or the
British one, but please don't mix them in one paper. The same holds
for different possible spellings for the same word as for example
``\emph{acknowledg(e)ment}'' or ``\emph{formulae}'' versus
``\emph{formulas}''. Be coherent.

\subsection{Mathematics}
\label{sec:math}
Running text must always constitute correct English phrases.  An
English phrase needs a verb and an `$=$'-sign can not be a replacement
for it.
  
All complicated mathematical formulae should be given on separate
lines and should not be spread out into the running text.  Never use
the \verb!$! form of the math enviroment for these. Human or automatic
taggers have a hard time to recognize which is an opening or a closing
\verb!$!. Use
\begin{verbatim}
\begin{math}...\end{math} 
\end{verbatim}
for all formulas that spread over several words and
\begin{verbatim}
\begin{displaymath}...\end{displaymath}
\end{verbatim}
(or \texttt{equation} \textit{etc.}) that should be rendered on a line
of their own. Using the old fashioned double-dollar environment
\begin{verbatim}
$$ some complicated formula $$ 
\end{verbatim}
is frowned upon.

You should use \LaTeX2e environments that provide a numbering for all
formulae that are rendered on line of the their own. Use environmenst
such as \texttt{equation} or \texttt{eqnarray}. Such numbers ease the
referee process very much, and after eventual publication easily allow
readers to refer to in their own work.
  
  
The quantifiers ``$\exists$'' and ``$\forall$'' don't stand as abbreviations of
the partial phrases ``\emph{there is}'' and ``\emph{for all}''. They
are reserved for logical formulae as \emph{such}, that is for work
that talks itself of logical formulae as a subject.
  
  
The equal sign ``$=$'' has different meanings in parts of the two
communities that DMTCS addresses.
\begin{enumerate}
\item It might stand for mathematical identity that is discovered
  \emph{a posteriori}. As an example take the following phrase:
  \begin{center}
    \emph{An easy computation shows that $4!=24$}.
  \end{center}
\item It might stand for a \emph{definition}, as in
  \begin{center}
    \emph{For convenience, set $0!=1$.}
  \end{center}
\end{enumerate}
For the later use of ``$=$'' Computer Scientist often tend to use
``$:=$''. Referees should be tolerant to these different customs.


\subsection{Cross-references and citations}
\label{sec:cross}
Use cross-references throughout your whole paper. Use \verb!\label! and
\verb!\ref! for that and don't do the work of the computer by
yourself. Not only that it is easier (\emph{believe me!}) but also
it helps to insert hyperlinks across the final document in the pdf
version, see Section~\ref{sec:pdf}. 

The same holds for citations. \emph{Never ever number citations by
  hand.} This only can go wrong and it will. Use \LaTeX{}'
\verb!\cite! command. Again, in the pdf version this will have the
advantage of a hyperlink that lets you jump directly to the
bibliography item. 

Use \texttt{bibtex} to produce your bibliography. With a little bit of
initial overhead it lets you easily maintain your references. This
pays off when you will write more than one article in your life...
Have a look into \cite{oetiker99:_not_so_short_introd_latex} and to
the
\href{http://www.dmtcs.org/abstracts/bib/DMTCS.bib}{\texttt{.bib}-file}
of DMTCS to see how this works.

I personally prefer the so-called \emph{natural} citation style as it
is used herein (via \texttt{natbib}). It has the advatage that the
author names of the work that is cited appear properly. Papers are to
the merits of people. In addition, such a citation by
name has the advantage of being easily recognizable without looking in
the bibliography. 


\section{PDF Files}
\label{sec:pdf}
The style now supports an option \texttt{pdftex} to use in combination
with \texttt{pdflatex}. This is now relatively stable. If you
are viewing this document in its pdf form you may see some of the
advantages this has: in particular pdf documents produced in that way
have included \emph{hyperlinks}. If you want to know more about these
features please refer to
\href{http://xxx.lanl.gov/hypertex/}{\texttt{http://xxx.lanl.gov/hypertex/}}.

If your installation doesn't support the package \texttt{hyperref},
you should switch of these features by giving the option
\texttt{nohyperref} in the \verb!\documentclass! declaration at the
beginning of your manuscript.  To give you the possibility to include
hyperlinks even if your local installation doesn't support this, we
provide the command \verb!\href{URL}{text}! in any case.


\begin{figure}[htbp]
  \begin{center}
%    \includegraphics{dmtcs}
    \caption{The logo of \protect\href{http://www.dmtcs.org/}{DMTCS}.}
    \label{fig:logo}
  \end{center}
\end{figure}
\section{Graphics}
\label{sec:graphics}

Please use the (standard) packages \texttt{graphics} or
\texttt{graphicx} to include graphical data and not
\texttt{epsf} or similar. Something like the PostScript picture in the
title of this document can be produced as simple as this
\verb!\includegraphics[width=0.13\textwidth]{}!. Note that in
this command the width is given in relation to the width of the text
and not in an absolut measure and that the file name is given without
extension. \emph{Don't include the extension of the graphic file in
  the command!}  If accepted, we will have to process your file for
PostScript and PDF. Please, leave the choice of the desired format to
the \texttt{graphicx} package.


For a realistic graphic of your paper you should chose a
\texttt{figure} environment as is done with the following for 
Figure~\ref{fig:logo} 
\begin{verbatim}
\begin{figure}[htbp]
  \begin{center}
    \includegraphics{}
    \caption{The logo of DMTCS.}
    \label{fig:logo}
  \end{center}
\end{figure}
\end{verbatim}

If you have several (small) figures that are related to each other you may
place them inside one figure environment. As you may see in
Figure~\ref{fig:logo2} how this may look like and consider the source
of this description here on how to refer to each part (that is
Figure~\ref{logo1cm} and \ref{logo3cm}) individually.
\begin{figure}[htbp]
  \begin{center}
    \subfigure[height of 1 cm\label{logo1cm}]{%\includegraphics[height=1cm]{}
}
    \hfil
    \subfigure[width of 3 cm\label{logo3cm}]{%\includegraphics[width=3cm]{}
}
    \caption{The scaled logo of \protect\href{http://www.dmtcs.org/}{DMTCS}.}
    \label{fig:logo2}
  \end{center}
\end{figure}

\section{Summary of Options}
\label{sec:options}
\begin{tabular}{|l|p{6cm}|}
\hline
option & description\\
\hline
submission & whether or not this is considered as a \\
final & submission or being the final document\\
\hline
pdftex & to be enabled when processing with pdflatex\\
nohyperref & switch reference to \texttt{hyperref} off\\
notimes & switch selection of the \texttt{times} package off\\
\hline
\end{tabular}


\section{Summary of Relevant Commands}
\label{sec:commands}

\begin{tabular}{|l|p{10cm}|}
\hline
command & description\\
\hline
\verb!\address! & the affiliation and address of the authors\\
\verb!\addressmark! & to number different affiliations for different authors\\
\hline
\verb!\revision! & the revision number, defaults to $1$ if this
command is omitted\\
\hline
\verb!\keywords! & a comma separated list of keywords\\
\hline
\verb!\qed! & produces \qed. See \texttt{proof}-environment below.\\
\hline
\verb!\acknowledgements! & \\
\hline
& Use the following commands only for the indicated
  purpose. If, \textit{e.g}, you wand to use a $\mathbb{P}$ that does
  not denote the set of prime numbers use \verb!\mathbb{P}!\\
\hline
\verb!\naturals! & the positive integers or `naturals', \naturals\\
\verb!\integers! & the ring of the integers, \integers\\
\verb!\rationals! & the field of the rational numbers, \rationals\\
\verb!\reals! & the field of the real numbers, \reals\\
\verb!\complexes! & the field of the complex numbers, \complexes\\
\verb!\primes! & the set of prime numbers, \primes\\
\hline
\end{tabular}

\section{Summary of Relevant Environments}
\label{sec:environments}

\begin{tabular}{lp{8cm}}
  \texttt{abstract} & The abstract of your paper, see Section~\ref{sec:first}.\\
  \texttt{proof} &
  \begin{proof}
    Use this for proofs. This helps the reader to understand
    the general structure of your paper easily.
  \end{proof}
\end{tabular}

\acknowledgements
\label{sec:ack}
At the end of the manuscript, right before the bibliography you might
want to place an acknowledgement. This can be easily done by using the 
command \verb!\acknowledgements! as you can see here.

\nocite{*}
\begin{thebibliography}{1}
\providecommand{\natexlab}[1]{#1}
\providecommand{\url}[1]{\texttt{#1}}
\expandafter\ifx\csname urlstyle\endcsname\relax
  \providecommand{\doi}[1]{doi: #1}\else
  \providecommand{\doi}{doi: \begingroup \urlstyle{rm}\Url}\fi

\bibitem[Oetiker et~al.(1999)Oetiker, Partl, Hyna, and
  Schlegl]{oetiker99:_not_so_short_introd_latex}
T.~Oetiker, H.~Partl, I.~Hyna, and E.~Schlegl.
\newblock \emph{The Not So Short Introduction to \LaTeX2e}, 3.3 edition, 1999.
\newblock available at
  \href{http://www.loria.fr/services/tex/general/lshort2e.pdf}{\texttt{http://%
www.loria.fr/services/tex/general/lshort2e.pdf}}.

\end{thebibliography}


\end{document}
