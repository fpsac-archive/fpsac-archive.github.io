%       A Geometric Approach to the Combinatorics of Schubert
%       Polynomials
%
%       Frank Sottile 
%
%       Extended Abstract of Talk Delivered at SFCA'95 - Paris
%
%       Published in Conference Proceedings, 29 May, 1995
%       pp 501--510

\documentstyle[amssymb,12pt]{amsart}


\addtolength{\evensidemargin}{-.6in}\addtolength{\oddsidemargin}{-.6in}
\addtolength{\textwidth}{1.2in}
%\addtolength{\textheight}{.2in} \addtolength{\topmargin}{-.1in}

\newtheorem{lemma}{Lemma}
\newtheorem{thm}[lemma]{Theorem}
\newtheorem{prop}[lemma]{Proposition}
\newtheorem{cor}[lemma]{Corollary}
\newtheorem{example}[lemma]{Example}
\newtheorem{conjecture}[lemma]{Conjecture}



\newcommand{\Edot}{{E\!_{\mbox{\Huge $.$}}}\!}
\newcommand{\Fdot}{{F\!\!_{\mbox{\Huge $.$}}}}
\newcommand{\Fpdot}{{{F\!\!_{\mbox{\Huge $.$}}}\!'}}
\newcommand{\Gdot}{{G\!_{\mbox{\Huge $.$}}}\!}
\newcommand{\Gpdot}{{G\!_{\mbox{\Huge $.$}}}\!'\!}
\newcommand{\llra}{\relbar\joinrel\relbar\joinrel\longrightarrow}
\newcommand{\lllra}{\relbar\joinrel\llra}
\newcommand{\Span}{\mbox{span}}

\newcommand{\rkm}{\stackrel{r[k,m]}{\llra}}
\newcommand{\ckm}{\stackrel{c[k,m]}{\llra}}

\newcommand{\QED}{
\setlength{\unitlength}{1.0pt}%
\begin{picture}(7.5,7.5)
\put(0,-2.5){\rule{2.5pt}{2.5pt}}
\put(0,0){\rule{5pt}{2.5pt}}
\put(0,2.5){\rule{7.5pt}{2.5pt}}
\end{picture}\vspace{10pt}}



\begin{document}

\mbox{ }\vspace{-10pt}

\title[Combinatorics of Schubert Polynomials]{A 
Geometric Approach to the Combinatorics of Schubert Polynomials}

\author{Frank Sottile}
\address{Department of Mathematics\\
        University of Toronto\\
        100 St.~George Street\\
        Toronto, Ontario  M5S 3G3\\
        Canada\\
        (416) 978-4031}
\email{sottile@@math.toronto.edu}
\date{8 March, 1995}

\thanks{Extended abstract of talk delivered at SFCA'95 29 
May -- 2 June, 1995, Paris.
Published in conference abstracts, pp 501--510.}

\maketitle
\vspace{-.2in}

\section*{English Summary}

Schubert polynomials, which had their origins in 
the cohomology of flag varieties, have recently been the subject 
of much interest in algebraic combinatorics.
This scrutiny has led to an elucidation of many of their 
properties.
A basic open problem is to give a rule for multiplying two 
Schubert polynomials, that is,  find an analog of the 
Littlewood-Richardson rule for Schubert polynomials.

Our talk would discuss some recent work on this problem using 
ideas from algebraic geometry, as well as some implications of this 
work for the combinatorics of the symmetric group.
Specifically, we will describe a geometric proof of an
analog of Pieri's rule for Schubert 
polynomials.
This was stated by Lascoux and 
Sch\"utzenberger in~\cite{Lascoux_Schutzenberger_polynomes_schubert}, 
where an algebraic proof was suggested.
Interpreting this formula geometrically exposes a new and striking link to 
the Littlewood-Richardson rule for Schur polynomials, and indicates 
possible extensions of this result.
While our approach uses ideas and methods from algebraic geometry, 
we will present proofs involving little more than elementary (albeit 
complicated) linear algebra.

\section*{R\'esum\'e} 

Les polyn\^omes de Schubert, qui sont issus
de la cohomologie des vari\'et\'es de drapeaux, ont sucit\'e depuis
quelques temps un grand int\'er\^et en combinatoire alg\'ebrique.
Un examen attentif a conduit \`a l'\'elucidation de beaucoup de leurs
propri\'et\'es. Un probl\`eme ouvert fondamental est de trouver
une r\`egle pour multiplier deux polyn\^omes de Schubert, autrement dit,
un analogue de la r\`egle de Littlewood-Richardson. 

Notre expos\'e discutera certains travaux r\'ecents sur ce probl\`eme
en faisant appel \`a des id\'ees provenant de la g\'eom\'etrie
alg\'ebrique, et aussi certaines cons\'equences de ces travaux pour
la combinatoire du groupe sym\'etrique. Plus pr\'ecis\'ement, nous
d\'ecrirons une preuve g\'eom\'etrique d'un analogue de la r\`egle de
Pieri pour les polyn\^omes de Schubert. Cette r\`egle a \'et\'e
\'enonc\'ee dans \cite{Lascoux_Schutzenberger_polynomes_schubert},
o\`u une preuve alg\'ebrique \'etait sugg\'er\'ee. L'interpretation
g\'eom\'etrique de cette formule met en \'evidence un lien nouveau
et surprenant avec la r\`egle de Littlewood-Richardson, et indique
des extensions possibles de ce r\'esultat.
Bien que notre approche fasse appel \`a des id\'ees et des m\'ethodes de
g\'eom\'etrie alg\'ebrique, nous pr\'esenterons des preuves qui
n'utilisent gu\`ere que de l'alg\`ebre lin\'eaire \'el\'ementaire
(quoique compliqu\'ee).


\section{Structure Constants for Schubert Polynomials}


The Appendix contains a brief introduction to Schubert polynomials.
Let $S_n$ be the symmetric group of permutations on the set 
$\{1,2,\ldots,n\}$ and
$S_{\infty} = \bigcup_{n=1}^{\infty} S_n$, the group of 
permutations of the positive integers that fix all but finitely many 
integers.
The collection of Schubert polynomials 
$\{{\frak S}_{w}\,|\,w\in S_{\infty}\}$ forms a basis 
for the polynomial ring 
$R_\infty={\Bbb Z}[x_1,x_2,\ldots]$.
A basic open question is the analog of the Littlewood-Richardson rule:
For $u,v,w \in S_n$, find integer constants $c^{w}_{uv}$ 
such that
\begin{equation}
{\frak S}_{u}\cdot {\frak S}_{v}\  = \ 
\sum_{w\in S_n}c^{w}_{uv}{\frak S}_{w}.
\label{eq:structure}
\end{equation}
These structure constants are positive and 
are known only in certain special cases.

For example, if both $u$ and $v$ are Grassmannian permutations with 
descent $k$ so that ${\frak S}_{u}$ and ${\frak S}_{v}$ are 
symmetric polynomials in the variables $x_1,\ldots,x_k$, then
(\ref{eq:structure}) reduces to the classical Littlewood-Richardson rule.

An important case is when one of
$u$ or $v$ is an adjacent 
transposition, $t_{k\,k{+}1}$.
This is usually attributed to Monk~\cite{Monk}.
However,  at the same time
Chevalley established the analogous formula for generalized
flag varieties in an unpublished manuscript~\cite{Chevalley58}.   
For $w\in S_{\infty}$,
let $\ell(w)$ be the length of $w$.   
Monk's rule states:
\begin{equation}\label{eq:Monk}
{\frak S}_{w}\cdot {\frak S}_{t_{k\,k{+}1}}
\ =\  \sum {\frak S}_{w\cdot t_{a\,b}},
\end{equation}
the sum over all $a\leq k <b$ with 
$\ell(w t_{a\,b})=\ell(w)+1$.
We use geometry to prove a similar result, which is the analog for
Schubert polynomials of the classical 
Pieri's rule.
This analog of Pieri's rule was announced by Lascoux and Sch\"utzenberger
in~\cite{Lascoux_Schutzenberger_polynomes_schubert}, where an algebraic 
proof was suggested.
\smallskip

%We begin with some definitions. 
Permutations $w$ are represented by the sequence 
$(w(1),w(2),\ldots)$ of their values.
For positive integers $k,m$, we define the permutations:
\begin{eqnarray*}
r[k,m] &=&
(1,2,\ldots,k{-}1,\,\,k{+}m,\,\,k,\,\,k{+}1,\ldots,k{+}m{-}1,
\,\,k{+}m{+}1,\ldots)\\
c[k,m] &=&
(1,2,\ldots,k{-}m,\,\,k{-}m{+}2,\ldots,k{+}1,\,\,k{-}m{+}1,\,\,k{+}2,\ldots)
\end{eqnarray*}
%In cycle notation, these are the $m{+}1$-cycles  
%\begin{eqnarray*}
%r[k,m] &=&
%(k{+}m\,\,\,\,k{+}m{-}1\,\ldots\,k{+}2\,\,\,\,k{+}1\,\,\,\,k)\\
%c[k,m]  &=&
%(k{-}m{+}1\,\,\,\,k{-}m{+}2\,\ldots\,k{-}1\,\,\,\,k\,\,\,\,k{+}1)
%\end{eqnarray*}

Let $w,w' \in S_{\infty}$.
Write $w \rkm w'$ if
there exist $a_1,b_1,\ldots,a_m,b_m$ such that
\begin{enumerate}
\item[(a)] $a_i\leq k <b_i$ for $1\leq i\leq m$ and
$w' = w\cdot t_{a_1\,b_1}\cdots t_{a_m\,b_m}$,
 \smallskip
\item[(b)]  $\ell(w^{(i)}) = \ell(w) +i$,
where $w^{(0)} = w$ and 
$w^{(i)}=w^{(i-1)}\cdot t_{a_i\,b_i}$,
 \smallskip
\item[(c)] $w^{(1)}(a_1) <w^{(2)}(a_2)<\cdots <
w^{(m)}(a_m)$.
\end{enumerate}
Equivalently, given any $a_1,\ldots,b_m$ satisfying (a) and (b), 
$b_1,\ldots,b_m$ are distinct.
Similarly, write $w\ckm w'$ 
if we have such $a_1,b_1,\ldots,a_m,b_m$ where now
$$
w^{(1)}(a_1) >w^{(2)}(a_2)>\cdots >
w^{(m)}(a_m).
$$

The range of summation in Monk's rule (\ref{eq:Monk}) generates a partial 
order $\leq_k$ on $S_n$ by $w \leq_k w  t_{a\,b}$ whenever
$a\leq k<b$ and $\ell(w t_{a\,b}) = \ell(w)+1$.
We call it the {\em $k$-Bruhat order} 
($k$-colored Ehresmano\"edre 
in~\cite{Lascoux_Schutzenberger_Symmetry}).
The data $(a_1,b_1), \ldots,(a_m,b_m)$ in the definition
of $w\rkm w'$ describe a particular path from 
$w$ to $w'$ in the $k$-Bruhat order.
\bigskip


\begin{thm} \label{thm:pieri_analog}
Let $w\in S_{\infty}$ and $k,m$ be positive integers.
Then we have
\begin{enumerate}

\item ${\displaystyle 
{\frak S}_{w}\cdot {\frak S}_{r[k,m]} \ 
= \sum_{w \rkm w'}
{\frak S}_{w'}}$.
\smallskip

\item ${\displaystyle 
{\frak S}_{w}\cdot {\frak S}_{c[k,m]} \ 
= \sum_{w \ckm w'}
{\frak S}_{w'}}$.

\end{enumerate}
\end{thm}

This is in a different form than the original 
statement in~\cite{Lascoux_Schutzenberger_polynomes_schubert}.
Bergeron and Billey~\cite{Bergeron_Billey} 
independently conjectured the above form.

This form exposes a link between multiplying 
Schubert polynomials and paths in the Bruhat order.
Such a link is not unexpected.
The Littlewood-Richardson rule
for multiplying Schur functions may be expressed
as a sum over certain paths in Young's lattice of 
partitions.
Lascoux and 
Sch\"utzenberger~\cite{Lascoux_Schutzenberger_schubert_polynials_LR_rule} 
give a procedure for multiplying Schur polynomials 
based upon paths in the Bruhat order on $S_n$.
A connection between paths in the Bruhat order and the 
intersection theory of Schubert varieties is described
in~\cite{Hiller_intersections}.
We believe the eventual description of the structure 
constants $c^w_{uv}$ will be in terms
of paths of certain types in the  Bruhat order
on $S_n$.



We first establish the
equivalence of the two formulas given in Theorem~\ref{thm:pieri_analog},  
then prove the first formula by reducing it to the classical Pieri's rule.
Our approach illustrates an unexpected link to the classical Pieri's rule, 
allowing the elucidation of more structure constants.
We conclude with a `path counting formula' generalizing
Theorem~\ref{thm:pieri_analog} where
${\frak S}_{r[k,m]}$ and ${\frak S}_{c[k,m]}$ are replaced by
hook Schur polynomials.
\smallskip

We would like to thank Sara Billey who suggested these problems to us,
Jean-Yves Thibon for indicating to us the work of Lascoux and 
Sch\"utzenberger, and Nantel Bergeron for many stimulating discussions 
about these results and possible extensions.

\section{The Flag Variety}

Let ${\Bbb F}(n)$ denote the variety of complete flags in ${\Bbb C}\,^n$, 
that is
$$
{\Bbb F}(n) = 
\{ \Edot : E_1\subset E_2\subset \cdots\subset E_n = {\Bbb C}\,^n \,|\,
\dim{E_j} = j\}.
$$
A fixed flag $\Fdot$ determines a cell decomposition of ${\Bbb F}(n)$
due to Ehresmann~\cite{Ehresmann}, which may be
indexed by elements of $S_n$.
The cell determined by $w \in S_n$ is 
\medskip

\noindent $\{ \Edot \,|\, \exists\, f_1,\ldots,f_n \mbox{ with }
f_i \in F_{n+1-w(i)}-
F_{n-w(i)} \mbox{ for }  1\leq i\leq n \mbox{ and }$

\hfill
$E_j = \Span\{f_1,\ldots,f_j\},\mbox{ for }  1\leq j\leq n\},$\medskip

\noindent
which has codimension $\ell(w)$.
Its closure is the Schubert variety   $X_{w}\Fdot$.


The cohomology classes\footnote{Strictly speaking, we mean the 
classes Poincar\'e dual to the fundamental cycles in homology.}
[$X_w\Fdot$] of the Schubert varieties $X_w\Fdot$ give 
an integral basis for the cohomology ring of the flag 
variety~\cite{Ehresmann}.
Using Chern classes, Borel~\cite{Borel} gave an alternate description 
of this ring as $H_n = {\Bbb Z}[x_1,\ldots,x_n]/{\cal S}$, 
where ${\cal S}$ is the ideal generated by the symmetric polynomials.
These two descriptions were reconciled by Demazure~\cite{Demazure}
and Berstein-Gelfand-Gelfand~\cite{BGG}, who described 
representatives for the cohomology classes of Schubert varieties.
Later, Lascoux and 
Sch\"utzenberger~\cite{Lascoux_Schutzenberger_polynomes_schubert}
gave explicit polynomial representatives ${\frak S}_{w}$, called 
Schubert polynomials.

We utilize a few algebraic facts about these cohomology rings.
Let $w_0$ be the longest element of $S_n$, that is,
$w_0(j) = n{+}1{-}j$.
Then 
$X_{w_0}\Fdot = \{\Fdot\}$, so ${\frak S}_{w_0}$
is the class of a point.
The Schubert polynomial basis is a Poincar\'e dual basis;
by this we mean  if 
$\ell(w) + \ell(w') = \ell(w_0)$, then 
$$
{\frak S}_{w} \cdot {\frak S}_{w'}
= \left\{\begin{array}{ll} {\frak S}_{w_0}&\mbox{ if }
w' = w_0 w\\
0 & \mbox{ otherwise} \end{array}\right.. %}
$$
There is also an involution induced by the map
${\frak S}_{w} \longmapsto
{\frak S}_{w_0ww_0}$.
\smallskip


We first observe that the two formulas in 
Theorem~\ref{thm:pieri_analog} are equivalent.
This follows easily from the next lemma.

\begin{lemma}\mbox{ }

\begin{enumerate}
\item
Let $w, w' \in S_n$.
Then 
$w \rkm w'$
if and only if \ 
$w_0ww_0 
\stackrel{c[n-k,m]}{\relbar\joinrel\lllra} 
w_0\,w'\,w_0$.
\item $w_0\, r[k,m]\, w_0 = c[n-k,m]$.
\end{enumerate}
\end{lemma}

Using Poincar\'e duality, formula (1) of Theorem~\ref{thm:pieri_analog}
is equivalent to 
$$
{\frak S}_{w} \cdot {\frak S}_{w_0w'}
\cdot {\frak S}_{r[k,m]} = 
\left\{
\begin{array}{ll}{\frak S}_{w_0}&\mbox{ if }
w  \rkm  w'\\
0 & \mbox{ otherwise} \end{array}\right. . %}
$$
Alternatively, 
\begin{equation} \label{eq:structure_constants}
c_{w\,r[k,m]}^{w'} = \left\{
\begin{array}{ll} 1&\mbox{ if }
w  \rkm  w'\\
0 & \mbox{ otherwise} \end{array}\right. . %}
\end{equation}

To prove (3), fix $w$ and $w'$ in $S_n$.
Note that
$r[k,m] = t_{k\,k+1}\, t_{k\,k+2}\cdots t_{k\,k+m}$.
Iterating Monk's rule  shows that ${\frak S}_{r[k,m]}$ is a summand of
${\frak S}_{t_{k\,k+1}}^m$ with coefficient 1.
Thus the coefficient of ${\frak S}_{w'}$ in the expansion of
${\frak S}_w \cdot {\frak S}_{t_{k\,k+1}}^m$  exceeds the 
coefficient of ${\frak S}_{w'}$ in ${\frak S}_w \cdot {\frak S}_{r[k,m]}$.
The first coefficient is zero unless 
$w\leq_k w'$ and $\ell(w') = \ell(w) +m$.
Thus $c_{w\,r[k,m]}^{w'} = 0$ unless 
$w\leq_k w'$ and $\ell(w') = \ell(w) +m$.
\smallskip


For the rest of this section, we will assume that $w\leq_k w'$ and 
$\ell(w') = \ell(w) +m$.
We will also refer to the accompanying data:
integers $a_1,b_1,a_2,\ldots,a_m,b_m$ with $a_i\leq k <b_i$ where
$w' = wt_{a_1\,b_1}\cdots t_{a_m\,b_m}$ and 
$\ell(wt_{a_1\,b_1}\cdots t_{a_i\,b_i}) = \ell(w) +i$.
\medskip

We establish (\ref{eq:structure_constants}) using the cohomology
of Grassmann varieties.
The association of a flag $\Edot$ to its $k$-dimensional 
part gives a map $\pi$ from the flag variety to the
Grassmannian of $k$-dimensional subspaces in ${\Bbb C}\,^n$,
$G(k,n)$.
This induces a ring homomorphism $\pi^*$ from the 
cohomology of $G(k,n)$ to that of the flag 
variety.
Algebraically, $\pi^*$ is  induced by the inclusion of symmetric polynomials
in $x_1,\ldots,x_k$ into the ring ${\Bbb Z}[x_1,\ldots,x_n]$.
There is also a covariant pushforward map $\pi_*$
induced via Poincar\'e duality from the functorial map on homology.
While $\pi_*$ is not a ring homomorphism,
it does satisfy the following projection formula
(see Example~8.17~of~\cite{Fulton_intersection}):
For cohomology classes $\alpha$ on the flag variety and $\beta$ on
$G(k,n)$, we have:
\begin{equation}\label{eq:projection}
\pi_* (\alpha \cdot \pi^* \beta) = \pi_*(\alpha) \cdot \beta.
\end{equation}

The cohomology ring of $G(k,n)$ is isomorphic to 
a quotient of the ring of symmetric polynomials in 
$x_1,\ldots,x_k$ by the ideal generated by all Schur polynomials
$s_{\lambda}$ where the partition $\lambda$ satisfies
$\lambda_1 > n-k$.
The class of a point is $s_{(n-k)^k}$, where $(n{-}k)^k$ is 
the partition with $k$ parts  each equal to $n-k$.
For a partition $\lambda$ with $\lambda_1\leq n-k$ and 
$\lambda_{k+1} = 0$, let $\lambda^c$ be the partition
$(n-k-\lambda_k,\ldots,n-k-\lambda_1)$.
Then $s_\lambda$ and $s_{\lambda^c}$ are Poincar\'e dual 
cohomology classes.
Let $\underline{m}$ be the partition $(m,0,\ldots,0)$.
A key fact is that  $\pi^*(s_{\underline{m}}) = {\frak S}_{r[k,m]}$.
For partitions $\mu$ and $\lambda$, write 
$\lambda \stackrel{m}{\longrightarrow} \mu$ if $|\lambda|+m=|\mu|$
and 
$$
\mu_1\geq\lambda_1\geq\mu_2\geq\cdots\geq\mu_k\geq\lambda_k.
$$
That is, if $\mu\supset \lambda$ and the skew diagram 
$\mu/\lambda$ is a skew row of length $m$.

The classical Pieri's rule states that for any 
partitions $\lambda$ and $\mu$ with 
$|\lambda|+m = |\mu|$,
$$
s_{\lambda}\cdot s_{\mu^c} \cdot s_{\underline{m}}\  = \ \left\{
\begin{array}{ll} s_{(n-k)^k} & \mbox{ if }
\lambda \stackrel{m}{\longrightarrow} \mu\\
0 & \mbox{ otherwise}\end{array}
\right . .
$$

We use geometry to prove the following:


\begin{lemma}\label{lemma:pushforward}
 Let $w<_k w'$ be permutations in $S_n$.
Suppose  $w' = w t_{a_1\,b_1}\cdots t_{a_m\,b_m}$,
where $a_i\leq k<b_i$, and 
$\ell(w t_{a_1\,b_1}\cdots t_{a_i\,b_i}) = \ell(w)+i$.
Then
\begin{enumerate}
\item There is a cohomology class $\delta$ on $G(k,n)$
such that
$\pi_*({\frak S}_w\cdot {\frak S}_{w_0w'}) = \delta \cdot s_{d^k}$,
where $d = \#\{j>k\,|\, w(j) = w'(j)\} = 
n-k-\#\{b_1,\ldots,b_m\}$.
\smallskip
\item If $w\rkm w'$, then there are partitions 
$\mu \supset \lambda$ where $\mu/\lambda$ is a skew row of length $m$
whose $j^{\mbox{\scriptsize th}}$ row 
is equal to $\#\{i\,|\, a_i = j\}$
and
$\pi_*({\frak S}_w\cdot {\frak S}_{w_0w'}) = s_{\lambda}\cdot s_{\mu^c}$.
\end{enumerate}
\end{lemma}

In our talk, we will not present a proof of this Lemma.
A proof involving a mixture of 
combinatorics, linear algebra and geometry may be found 
in~\cite{sottile_Pieri_flag}.

We deduce Theorem~\ref{thm:pieri_analog} from Lemma 3.
Suppose $w\leq_k w'$ and $\ell(w') = \ell(w) +m$.
We use $\pi_*$  to evaluate $c^{w'}_{w\,r[k,m]}$.
Recall that $ c^{w'}_{w \,r[k,m]}$ is defined by 
$$
 c^{w'}_{w \,r[k,m]} {\frak S}_{w_0}
\ = \ 
{\frak S}_w\cdot {\frak S}_{w_0w'} \cdot {\frak S}_{r[k,m]}.
$$
As  ${\frak S}_{w_0}$ is the class of a point, 
$\pi_*({\frak S}_{w_0})= s_{(n-k)^k}$. 
Apply $\pi_*$ to obtain
$$
 c^{w'}_{w \,r[k,m]} s_{(n-k)^k} 
\ =\ 
\pi_*({\frak S}_w\cdot {\frak S}_{w_0w'} \cdot {\frak S}_{r[k,m]}).
$$
Since  $\pi^*(s_{\underline{m}}) = {\frak S}_{r[k,m]}$,
we apply the projection formula (\ref{eq:projection}) to obtain
$$
c^{w'}_{w \,r[k,m]} s_{(n-k)^k} \ = \ 
\pi_*({\frak S}_w\cdot {\frak S}_{w_0w'}) \cdot s_{\underline{m}}.
$$
Since $s_{\underline{m}}  s_{d^k} = 0$ unless $d+m \leq n-k$, we 
apply part (1) of Lemma~\ref{lemma:pushforward} to see that 
$$
\pi_*({\frak S}_w\cdot {\frak S}_{w_0w'}) \cdot s_{\underline{m}}
\  =\   \delta \cdot s_{d^k}\cdot s_{\underline{m}}\   =\   0,
$$
unless $\#\{b_1,\ldots,b_m\} = m$, that is unless
$w\rkm w'$.
Supposing $w\rkm w'$, part (2) of  Lemma~\ref{lemma:pushforward}
and the classical Pieri's rule shows that 
$$
\pi_*({\frak S}_w\cdot {\frak S}_{w_0w'}) \cdot s_{\underline{m}}
\  =\   s_{\lambda}\cdot s_{\mu^c}\cdot s_{\underline{m}}\   =
\   s_{(n-k)^k},
$$
as $\mu/\lambda$ is a skew row of length $m$.
This establishes Theorem~\ref{thm:pieri_analog}.


\section{Connection to Pieri's Rule and Extensions}

The formulas in Theorem~\ref{thm:pieri_analog} are the analogs of Pieri's rule
for several reasons:
\begin{enumerate}
\item The Schubert polynomial ${\frak S}_{r[k,m]}$ 
equals  $\pi^*(s_{\underline{m}})$.
\item The structure constants in both are either 1 or 0.
\item Theorem~\ref{thm:pieri_analog} is proven by reduction to 
Pieri's rule.
\end{enumerate}
In~\cite{sottile_Pieri_flag}, we show the geometry of
the classical Pieri's rule and Theorem~\ref{thm:pieri_analog} to be 
nearly identical.   This is the unexpected link to Pieri's rule
mentioned in \S 1.
\medskip


The computation of $\pi_*({\frak S}_w \cdot{\frak S}_{w_0w'})$
in Lemma~\ref{lemma:pushforward} allows us to determine 
more structure constants.
To any partition $\nu$ with at most $k$ parts, associate
a permutation 
$$
w(\nu)\  = \ 
(\nu_k+1,\,\nu_{k-1}+2,\,\ldots,\,\nu_2+k-1,\,\nu_1+k,\,\ldots),
$$
the remaining entries written in increasing order.
Then $\pi^*(s_{\nu}) = {\frak S}_{w(\nu)}$.

\begin{thm} Let $w\in S_n$ and $k,m$ be integers. 
Suppose  $w\leq_k w'$ and $\ell(w') = \ell(w) +m$.
Let $a_1,b_1,\ldots,a_m,b_m$ be such that 
$a_i\leq k <b_i$ where $w' = w t_{a_1\,b_1}\cdots t_{a_m\,b_m}$ 
and $\ell(w t_{a_1\,b_1}\cdots t_{a_i\,b_i}) = \ell(w) +i$.
Then
\begin{enumerate}
\item  Suppose $w\rkm w'$.
For any partition
$\nu$, the structure constant $c^{w'}_{w\, w(\nu)}$
equals the Littlewood-Richardson coefficient
$c^\mu_{\lambda\,\nu}$, where $\mu/\lambda$ is a skew row of length
$m$, whose $j^{\mbox{\scriptsize th}}$ row has length 
$\mu_j - \lambda_j \ = \ \#\{i \,|\, a_i = j\}$.
\item
 Suppose $w\ckm w'$.
For any partition
$\nu$, the structure constant $c^{w'}_{w\, w(\nu)}$
equals the Littlewood-Richardson coefficient
$c^\mu_{\lambda\,\nu}$, where $\mu/\lambda$ is a skew column of length
$m$, whose $j^{\mbox{\scriptsize th}}$ column has length
$\#\{i \,|\, b_i=j\}$.
\end{enumerate}
\end{thm}

\noindent{\bf Proof:} Using the involution
${\frak S}_{w} \longmapsto {\frak S}_{w_0\,w\,w_0}$, it suffices to 
prove part (1).
We use  part (2) of Lemma~\ref{lemma:pushforward} to 
evaluate $c^{w'}_{w\,w(\nu)}$. 
Recall that ${\frak S}_{w(\nu)} = \pi^*(s_\nu)$.  Then 
\begin{eqnarray*}
c^{w'}_{w\,w(\nu)}\, s_{(n-k)^k}\   =\  
\pi_*(c^{w'}_{w\,w(\nu)}\,  {\frak S}_{w_0}) &=&
\pi_*({\frak S}_w\cdot{\frak S}_{w_0w'}\cdot {\frak S}_{w(\nu)})\\
%&=& \pi_*({\frak S}_w\cdot{\frak S}_{w_0w'}\cdot \pi^* s_\nu)\\
&=& \pi_*({\frak S}_w\cdot{\frak S}_{w_0w'}) \cdot s_\nu\\
&=& s_\lambda \cdot s_{\mu^c}\cdot s_\nu\\
&=& c^{\mu}_{\lambda\nu}\,  s_{(n-k)^k}. \ \ \ \  \QED
\end{eqnarray*}

In multiset notation for partitions,  $(p,1^{(q-1)})$ 
is the hook shape partition whose first row has length $p$ and 
first column has length $q$.
Define
$$
h[k;\,p,q] \   =\   w(p,1^{(q-1)}).
$$
%In cycle notation,
%$$
%h[k;p,q] = (k{-}q{+}1\,\,\,k{-}q{+}2\,\ldots\,k{-}1\,\,\,
%k\,\,\,k{+}p\,\,\,k{+}p{-}1\,\ldots\,k{+}1).
%$$

We use Theorem~\ref{thm:pieri_analog} to deduce the following
formula for multiplication of an arbitrary Schubert polynomial
${\frak S}_w$ by the hook symmetric function 
${\frak S}_{h[k;\,p,q]} =  \pi^*(s_{(p,1^{(q-1)})})$.

\begin{thm}
Let $q\leq k$ and $k{+}p \leq n$ be integers.
Set $m = p{+}q{-}1$.
For $w\in S_n$,
$$
{\frak S}_w \cdot {\frak S}_{h[k;\,p,q]} \   =\  
\sum {\frak S}_{w\, t_{a_1\,b_1}\cdots t_{a_m\,b_m}},
$$
the sum over all $a_1,b_1,\ldots,a_m,b_m$ with 
$w^{(i)} = w t_{a_1\,b_1}\cdots t_{a_m\,b_m}$ where
\begin{enumerate}
\item[(a)] For $1\leq i\leq m$, we have $a_i \leq k < b_i$
and $\ell(w^{(i)}) = \ell(w) +i$.
\smallskip
\item[(b)] $w^{(1)}(a_1) < \cdots < w^{(p)}(a_p)$ and
$w^{(p)}(a_p) > w^{(p{+}1)}(a_{p{+}1})>\cdots > w^{(m)}(a_m)$.
\end{enumerate}
Alternatively, condition $(\mbox{\rm b})$ for the summation may be replaced by
\begin{enumerate}
\item[(b$'$)] 
$w^{(1)}(a_1) > \cdots > w^{(q)}(a_q)$ and
$w^{(q)}(a_q) <\cdots < w^{(m)}(a_m)$.
\end{enumerate}
\end{thm}

\noindent{\bf Proof:}
Consider the formula involving Schur polynomials
in variables $x_1,\ldots, x_k$:
$$
s_{\underline{p}}\, \cdot \, s_{1^{(q-1)}} \   = \  
s_{(p{+}1,1^{(q-2)})} \, + \, s_{(p,1^{(q-1)})}.
$$
Expressing these as Schubert polynomials (applying $\pi^*$), we have:
$$
{\frak S}_{r[k,p]}\, \cdot\, {\frak S}_{c[k,q{-}1]}
\   =\   {\frak S}_{h[k;\,p{+}1,q{-}1]}  +
{\frak S}_{h[k;\,p,q]}.
$$

To establish the Theorem, fix $m$ and use downward induction  
on either $p$ or $q$, 
using this formula and Theorem~\ref{thm:pieri_analog}.
\QED




\section*{Appendix: Schubert Polynomials}

In~\cite{BGG,Demazure} cohomology classes of Schubert subvarieties of 
the flag manifold were obtained from the class of a point using repeated 
correspondences in ${\Bbb P}^1$-bundles.
Subsequently, Lascoux and 
Sch\"utzenberger~\cite{Lascoux_Schutzenberger_polynomes_schubert} 
showed it was possible to find explicit polynomial representatives.

This has given rise to the present algebraic and combinatorial 
theory of Schubert polynomials.
We outline their construction of Schubert polynomials; for a 
more complete 
account see~\cite{Macdonald_schubert}.


For each integer $n>1$, let $R_n = {\Bbb Z}[x_1,\ldots,x_n]$.
The polynomial ring $R_n$ is graded by the total degree of a monomial.
The symmetric group $S_n$ acts on $R_n$ by permuting the variables.
Let $f\in R_n$  and  $s_i = t_{i\,i{+}1}$ be an adjacent 
transposition.
The polynomial
$f - s_i f$ is anti-symmetric in $x_i$ and $x_{i+1}$,
and so is 
divisible by $x_i - x_{i+1}$.
Thus we define the linear {\em divided difference} operator 
$$
\partial_i = (x_i-x_{i+1})^{-1} (1 - s_i).
$$
This operator has degree $-1$. 
If $f$ is symmetric in $x_i$ and $x_{i+1}$, then $\partial_i f$ is zero.
Otherwise, $\partial_i f$ is symmetric in $x_i$ and $x_{i+1}$.
The divided differences satisfy
\begin{eqnarray*}
\partial_i\circ \partial_i & = & 0 \\
\partial_i \circ\partial_j &=& \partial_j \circ \partial_i \ \ \ 
\ \ \ \ \ \ \ \ \ \ \mbox{ if } |i-j|\geq 2\\
\partial_i\circ\partial_{i+1}\circ\partial_i & = &
\partial_{i+1}\circ\partial_i\circ\partial_{i+1}
\end{eqnarray*}
Thus, if $a = (a_1,\ldots,a_p)$ is a reduced word for a permutation
$w$, then the composition of divided differences 
$\partial_a = \partial_{a_1}\circ\cdots\circ\partial_{a_p}$
depends only upon $w$ and not upon $a$.  
This defines operators $\partial_{w}$ for each $w\in S_n$.

Let $w_0$ be the longest permutation in $S_n$, that is
$w_0(j) = n{+}1{-}j$.
For $w \in S_n$, define the Schubert polynomial ${\frak S}_{w}$
by
$$
{\frak S}_{w} = \partial_{w^{-1}w_0}
\left( x_1^{n-1} x_2^{n-2}\cdots x_{n-1} \right).
$$
Recently other, more combinatorial descriptions have been discovered 
for the Schubert 
polynomials~\cite{Bergeron,BJS,Fomin_Kirillov}.
In each of these a class of combinatorial objects is 
defined with a rule for associating
a monomial to each object.
Given a permutation $w$, a finite set of these combinatorial
objects is constructed.
Then it is proven that the Schubert polynomial ${\frak S}_{w}$
is the sum of all monomials obtained from that set
(counting multiplicities).
This is in the same spirit as the tableaux 
theoretic description of 
symmetric polynomials given by Lascoux and 
Sch\"utzenberger~\cite{Lascoux_Schutzenberger_monoid_plactic}.

The polynomial ${\frak S}_{w}$ is homogeneous of degree
$\ell(w)$, and is independent of which $n$ was chosen, thus
${\frak S}_{w}$ is well defined for each $w\in S_{\infty}$.
Here $S_\infty = \bigcup_{n=1}^\infty S_n$, the group of permutations of the 
positive integers which fix all but finitely many integers.

If $w$ has a unique {\em descent} ($j$ such that $w(j) >
w(j{+}1)$) at $k$, then $w$ is said to be {\em Grassmannian} 
with descent $k$
and ${\frak S}_{w}$ is the Schur polynomial 
$s_{\lambda(w)}(x_1,\ldots,x_k)$.
Here $\lambda(w)$ is the partition 
$(\lambda_1\geq\cdots\geq\lambda_k)$ 
with $k$ parts where 
$\lambda_{k-j+1} = w(j){-}j$.
A permutation $w\in S_n$ is represented as the sequence 
$(w(1),w(2),\ldots)$ of its values.
A partition $\lambda$ with at most $k$ parts determines
a Grassmannian permutation $w(\lambda)$ with descent at $k$: 
$$
w(\lambda) = 
(1{+}\lambda_k,2{+}\lambda_{k-1},\ldots,k{+}\lambda_1,\ldots),
$$
the remaining entries written in increasing order.
If we define
\begin{eqnarray*}
r[k,m] &=&
(1,2,\ldots,k{-}1,\,k{+}m,\,\,k,\,k{+}1,\ldots,k{+}m{-}1,
\,k{+}m{+}1,\ldots)\\
c[k,m] &=&
(1,2,\ldots,k{-}m,\,k{-}m{+}2,\ldots,k{+}1,\,\,k{-}m{+}1,\,k{+}2,\ldots),
\end{eqnarray*}
then $r[k,m]$ and $c[k,m]$ are Grassmannian permutations
with descent at $k$,  and we have 
$\lambda(r[k,m]) = (m,0,\ldots,0)$ and
$\lambda(c[k,m]) = (1^m)$, a single column of length $m$.

The set $\{{\frak S}_{w}\,|\, w\in S_n\}$ is an
integral basis for 
the $\Bbb Z$-module 
$$
H_n \ =\  {\Bbb Z}\langle x_1^{i_1}\cdots x_{n-1}^{i_{n-1}}\,|\, i_j \leq n{-}j
\rangle \   \subset \   {\Bbb Z}[x_1,\ldots,x_n],
$$ 
which is a complete transversal to the ideal ${\cal S}$ generated by the 
non-constant symmetric functions.
As $\Bbb Z$-modules, $H_n \simeq {\Bbb Z}[x_1,\ldots,x_n]/{\cal S}$.
Allowing $n$ to increase shows 
 $\{{\frak S}_{w}\,|\, w\in S_\infty\}$ is an integral 
basis for $R_\infty = {\Bbb Z}[x_1,x_2,\ldots]$.
Given any formula involving finitely many Schubert polynomials
(as in the statement of Theorem~\ref{thm:pieri_analog}),
there is a positive integer $n$ such that $H_n$ contains all the Schubert 
polynomials appearing in that formula.
Thus it is no loss of generality in proving formulas in the rings 
${\Bbb Z}[x_1,\ldots,x_n]/{\cal S}$.
This is the approach we take in \S\S1--3.

\begin{thebibliography}{10}

\bibitem[1]{Bergeron}
{\sc N.~Bergeron}, {\em A combinatorial construction of the {S}chubert
  polynomials}, J. Combin. Theory, Ser. A, 60 (1992), pp.~168--182.

\bibitem[2]{Bergeron_Billey}
{\sc N.~Bergeron and S.~Billey}, {\em {R}{C}-{G}raphs and {S}chubert
  polynomials}, Experimental Math., 2 (1993), pp.~257--269.

\bibitem[3]{BGG}
{\sc I.~N. Bernstein, I.~M. Gelfand, and S.~I. Gelfand}, {\em Schubert cells
  and cohomology of the spaces {$G/P$}}, Russian Mathematical Surveys, 28
  (1973), pp.~1--26.

\bibitem[4]{BJS}
{\sc S.~Billey, W.~Jockush, and R.~Stanley}, {\em Some combinatorial properties
  of {S}chubert polynomials}, J. Algebraic Combinatorics, 2 (1993),
  pp.~345--374.

\bibitem[5]{Borel}
{\sc A.~Borel}, {\em Sur la cohomologie des espaces fibr\'es principaux et des
  espaces homog\`enes des groupes de {L}ie compacts}, Ann. Math., 57 (1953),
  pp.~115--207.

\bibitem[6]{Chevalley58}
{\sc C.~Chevalley}, {\em Sur les d\'ecompositions cellulaires des espaces
  $g/b$}.
\newblock unpublished manuscript, 1958.

\bibitem[7]{Demazure}
{\sc M.~Demazure}, {\em D\'esingularization des vari\'et\'es de {S}chubert
  g\'en\'eralis\'ees}, Ann. Sc. E. N. S. (4), 7 (1974), pp.~53--88.

\bibitem[8]{Ehresmann}
{\sc C.~Ehresmann}, {\em Sur la topologie de certains espaces homog\`enes},
  Ann. Math., 35 (1934), pp.~396--443.

\bibitem[9]{Fomin_Kirillov}
{\sc S.~Fomin and A.~Kirillov}, {\em {Y}ang-{B}axter equation, symmetric
  functions and {S}chubert polynomials}, Adv. Math., 103 (1994), pp.~196--207.

\bibitem[10]{Fulton_intersection}
{\sc W.~Fulton}, {\em Intersection Theory}, Ergebnisse der Math. 2,
  Springer-Verlag, 1984.

\bibitem[11]{Hiller_intersections}
{\sc H.~Hiller}, {\em Combinatorics and intersections of {S}chubert varieties},
  Comment. Math. Helvetica, 57 (1982), pp.~41--59.

\bibitem[12]{Lascoux_Schutzenberger_monoid_plactic}
{\sc A.~Lascoux and M.-P. Sch{\"u}tzenberger}, {\em Le m{ono\"{\i}d} plaxique},
  in Non-Commutative Structures in Algebra and Geometric Combinatorics,
  Quaderni de {``la ricerca scientifica,''} {\bf 109} Roma, CNR, 1981,
  pp.~129--156.

\bibitem[13]{Lascoux_Schutzenberger_polynomes_schubert}
\leavevmode\vrule height 2pt depth -1.6pt width 23pt, {\em Polyn{\^o}mes de
  {S}chubert}, C. R. Acad. Sci. Paris, 294 (1982), pp.~447--450.

\bibitem[14]{Lascoux_Schutzenberger_Symmetry}
\leavevmode\vrule height 2pt depth -1.6pt width 23pt, {\em Symmetry and flag
  manifolds}, in Invariant Theory, ({M}ontecatini, 1982), Springer--Verlag,
  1983, pp.~118--144.
\newblock {L}ecture {N}otes in {M}athematics 996.

\bibitem[15]{Lascoux_Schutzenberger_schubert_polynials_LR_rule}
\leavevmode\vrule height 2pt depth -1.6pt width 23pt, {\em {S}chubert
  polynomials and the {L}ittlewood-{R}ichardson rule}, Lett. Math. Phys., 10
  (1985), pp.~111--124.

\bibitem[16]{Macdonald_schubert}
{\sc I.~G. Macdonald}, {\em Notes on {S}chubert Polynomials}, Laboratoire de
  combinatoire et d'informatique math\'ematique {(LACIM)}, Universit\'e du
  Qu\'ebec \`a Montr\'eal, Montr\'eal, 1991.

\bibitem[17]{Monk}
{\sc D.~Monk}, {\em The geometry of flag manifolds}, Proc. London Math. Soc., 9
  (1959), pp.~253--286.

\bibitem[18]{sottile_Pieri_flag}
{\sc F.~Sottile}, {\em {P}ieri's rule for flag manifolds and 
Schubert polynomials},
\newblock Annales d'Institut Fourier, Grenoble, 46 no. 1 (1996), pp. 1--22.

\end{thebibliography}

\end{document}


