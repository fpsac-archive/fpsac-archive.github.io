\magnification=\magstep1
\baselineskip=14pt  \def \Box {\hbox{}\nobreak
       \vrule width 1.6mm height 1.6mm depth 0mm  \par \goodbreak \smallskip}

\centerline{\bf  Solving Algebraic Equations in Terms of
                 ${\cal A}$-Hypergeometric Series}                    
                                                                    \vskip .5cm
\centerline{     Bernd Sturmfels}
\centerline{     Department of Mathematics }
\centerline{     University of California }
\centerline{     Berkeley, CA 94720, U.S.A. }
\centerline{\tt  bernd@math.berkeley.edu}
                                                        \vskip .6cm \vskip 10pt
                     \midinsert  \narrower {\sl \centerline{Abstract} \noindent
The roots of the general equation of degree $n$ satisfy an 
${\cal A}$-hypergeometric system of differential equations in the 
sense of Gel'fand, Kapranov and Zelevinsky. We construct 
the $n$ distinct ${\cal A}$-hypergeometric series solutions for each of 
the $2^{n-1}$ triangulations of the Newton segment. This works
over any field whose characteristic is
relatively prime to the lengths of the segments in the triangulation.}

\endinsert \vskip .8cm

\beginsection 1. Solving the Quintic

A classical problem in mathematics is to find a 
formula for the roots of the general equation of degree $n$
in terms of its $n+1$ coefficients. While there are formulas
in terms of radicals for $n \leq 4$, Galois theory teaches us
that no such formula exists for the general quintic
$$ a_5 x^5 \, + \, a_4 x^4 \, + \, a_3 x^3 \, + \, a_2 x^2 \, + \,
a_1 x \, + \, a_0  \quad = \quad 0. \eqno (1.1) $$
An alternative approach is to expand
the roots into fractional power series
(or Puiseux series). In 1757 Johann Lambert expressed the roots
of the trinomial equation $\,x^p + x + r \,$ as a 
Gauss hypergeometric function in the parameter $r$.
Series expansions of more general algebraic functions
were subsequently given by Euler, Chebyshev and Eisenstein,
among others. The poster ``Solving the Quintic with Mathematica'' [12]
gives a nice introduction to these classical techniques
and underlines their relevance for symbolic computation.

The state of the art in the first half of our century 
appears in works of Richard Birkeland [2] and Karl Mayr [10]. They
proved that the roots are multivariate hypergeometric functions
(in the sense of Horn) in all of the coefficients and they gave
series expansions for the roots and their powers.
The purpose of this note is to refine the these results.

Our point of departure is the fact
that the roots satisfy the ${\cal A}$-hypergeometric 
differential equations introduced by
Gel'fand, Kapranov and Zelevinsky [6],[7].
Here ${\cal A}$ denotes the configuration
of $n+1$ equidistant points on the affine line.
It follows from recent work of McDonald [11]
that there are $2^{n-1}$ distinct complete sets of series
solutions, one for each of the $2^{n-1}$
triangulations of ${\cal A}$. We shall construct these 
series solutions explicitly.

Let us illustrate our general construction for the example of the
quintic (1.1). Here the set ${\cal A}$ has 
$16$ distinct triangulations.
The finest triangulation divides ${\cal A}$ into five segments
of unit length. The coarsest triangulation of ${\cal A}$
is just a single segment of length $5$.

For the finest triangulation we get the following expressions
for the five roots of (1.1):
$$  \eqalign{
 X_{1,-1} \,=\, - \bigl[{ a_0 \over a_1 }\bigr]  ,\qquad
 X_{2,-1} & \,=\, - \bigl[{ a_1 \over a_2 }\bigr] \,+ \,
\bigl[{a_0 \over a_1} \bigr]   ,\qquad
 X_{3,-1} \,=\, - \bigl[{a_2 \over a_3} \bigr] \,+ \,
\bigl[{a_1 \over a_2} \bigr]   ,\cr
 X_{4,-1} & \,=\, - 
\bigl[{a_3 \over a_4} \bigr] \,+\, \bigl[{a_2 \over a_3} \bigr]   ,\qquad
 X_{5,-1} \,=\, - \bigl[{a_4 \over a_5} \bigr] \,+\,
 \bigl[{a_3 \over a_4} \bigr] .\cr} $$
Each bracket represents a power series
having the monomial in the bracket as its first term:
$$ \eqalign{
\bigl[ {a_0 \over a_1} \bigr] \quad & = \quad
 {a_0 \over a_1} 
+ {a_0^2 a_2 \over a_1^3} 
- {a_0^3 a_3 \over a_1^4}
+ 2 {a_0^3 a_2^2 \over a_1^5}
+ {a_0^4 a_4 \over a_1^5}
- 5 { a_0^4 a_2 a_3 \over a_1^6 } 
- {a_0^5 a_5 \over a_1^6}
+ \cdots \cr
\bigl[ {a_1 \over a_2} \bigr] \quad & = \quad
  {a_1 \over a_2} 
+ {a_1^2 a_3 \over a_2^3}
- {a_1^3 a_4 \over a_2^4}
- 3 {a_0 a_1^2 a_5 \over a_2^4}
+ 2 {a_1^3 a_3^3 \over a_2^5 }
 +  {a_1^4 a_5 \over a_2^5 }
 - 5 {a_1^4 a_3 a_4 \over a_2^6}
+ \cdots \cr
\bigl[ {a_2 \over a_3 } \bigr] \quad & = \quad
       {a_2 \over a_3} 
- { a_0 a_5 \over a_3^2}
- { a_1 a_4 \over a_3^2}
+ 2 { a_1 a_2 a_5 \over a_3^3 }
+ { a_2^2 a_4 \over a_3^3 }
- { a_2^3 a_5 \over a_3^4 }
+ 2 { a_2^3 a_4^2 \over a_3^5 }
+ \cdots \cr
\bigl[ {a_3 \over a_4 } \bigr] \quad & = \quad
       {a_3 \over a_4} 
- { a_2 a_5 \over a_4^2 }
+ { a_3^2 a_5 \over a_4^3 }
+ { a_1 a_5^2 \over a_4^3}
- 3 { a_2 a_3 a_5^2 \over a_4^4 }
- { a_0 a_5^3 \over a_4^4 }
+ 4 { a_1 a_3 a_5^3 \over a_4^5} + \cdots \cr
\bigl[ {a_4 \over a_5} \bigr] \quad & = \quad
  {a_4 \over a_5}  \cr}$$
Note that the last bracket is just a single Laurent monomial.
The other four brackets $\,\bigl[ { a_{i-1} \over a_i } \bigr]\,$
can easily be written as an explicit sum over ${\bf N}^4$.  For instance, 
$$ \bigl[ { a_0 \over a_1} \bigr] \quad = \quad
\sum_{i,j,k,l \geq 0 }
{ (-1)^{2i+3j+4k+5 l} \,(2 i \! + \! 3 j \! + \! 4 k \! + \!5 l)! \over
i\,! \, j \,! \, k\,! \, l\,! \,  (i \!+ \!2 j \!+ \!3 k\! + \!4 l + 1)!} \cdot
{ a_0^{i+2j+3k+4l+1}    a_2^i  a_3^j  a_4^k a_5^l \over
               a_1^{2 i+3 j+4 k+5 l+1}} $$
Each coefficient appearing in one of
these series is integral. Therefore
our five series solutions of the general quintic are 
characteristic-free. They work over any base field.

The situation is different for the coarsest triangulation
of ${\cal A}$. Here we must assume that the characteristic is
different from $5$. The five series solutions of (1.1) are
$$ X_{5,\xi} \,\,\, =\,\,\,
\xi \cdot \bigl[{a_0^{1/5}  \over a_5^{1/5}}\bigr]
\,+ \, {1 \over 5} \cdot \biggl(
\xi^2  \bigl[{a_1 \over a_0^{3/5} a_5^{2/5}} \bigr]
\,+ \, \xi^3  \bigl[{a_2 \over a_0^{2/5} a_5^{3/5}} \bigr]
\,+ \, \xi^4  \bigl[{a_3 \over a_0^{1/5} a_5^{4/5}} \bigr]
\,- \, \bigl[{a_4 \over           a_5 } \bigr]
\biggr)  $$
where $\xi$ runs over the five roots of the
equation $ \, \xi^{5} = - 1 $. The brackets denote the series
$$ \eqalign{
\bigl[ {a_0^{1/5} \over a_5^{1/5}} \bigr] \quad & = \,\,\,
  {a_0^{1/5} \over a_5^{1/5}} 
- {1 \over 25} { a_1 a_4 \over a_0^{4/5} a_5^{6/5}}
- {1 \over 25} { a_2 a_3 \over a_0^{4/5} a_5^{6/5}}
+ {2 \over 125} {a_1^2 a_3 \over a_0^{9/5} a_5^{6/5}}
+ {3 \over 125} {a_2 a_4^2 \over a_0^{4/5} a_5^{11/5}} + \cdots \cr
\bigl[ {a_1 \over a_0^{3/5} a_5^{2/5}} \bigr] \quad & = \,\,\,
  {a_1 \over a_0^{3/5} a_5^{2/5}} 
- {1 \over 5} { a_3^2 \over a_0^{3/5} a_5^{7/5}}
- {2 \over 5} {a_2 a_4 \over a_0^{3/5} a_5^{7/5}}
+ {7 \over 25} {a_3 a_4^2 \over a_0^{3/5} a_5^{12/5}}
+ {6 \over 25} {a_1 a_2 a_3 \over a_0^{8/5} a_5^{7/5}} + \cdots \cr
\bigl[ {a_2 \over a_0^{2/5} a_5^{3/5}} \bigr] \quad & = \,\,\,
  {a_2 \over a_0^{2/5} a_5^{3/5}}
- {1 \over 5} { a_1^2 \over a_0^{7/5} a_5^{3/5} }
- {3 \over 5} { a_3 a_4 \over a_0^{2/5} a_5^{8/5}}
+ {6 \over 25} {a_1 a_2 a_4 \over a_0^{7/5} a_5^{8/5} }
+ {3 \over 25} { a_1 a_3^2 \over  a_0^{7/5} a_5^{8/5} } + \cdots \cr
\bigl[ {a_3 \over a_0^{1/5} a_5^{4/5}} \bigr] \quad & = \,\,\,
  {a_3 \over a_0^{1/5} a_5^{4/5}}
- {1 \over 5} { a_1 a_2 \over a_0^{6/5} a_5^{4/5}}
- {2 \over 5} { a_4^2 \over a_0^{1/5} a_5^{9/5}}
+ {1 \over 25} {a_1^3 \over a_0^{11/5} a_5^{4/5}}
+ {4 \over 25} {a_1 a_3 a_4 \over a_0^{6/5} a_5^{9/5}} + \cdots \cr} $$
Each of these four series can be expressed as an explicit sum
over the lattice points in a $4$-dimensional polyhedron.
The general formula will be presented in Theorem 3.2 below.

\beginsection 2. The Roots are ${\cal A}$-hypergeometric 

Our problem is to compute the roots of the general equation of degree $n$,
$$ f(x) \quad = \quad
  a_0 \, + \,a_1 x \,+ \,a_2 x^2 \,+\,
\ldots \,+\, a_{n-1} x^{n-1} \,+ \, a_n x^n . \eqno (2.1) $$
Each root of $f(x)$ is an algebraic function in the
indeterminate coefficients:
$$ X \quad =  \quad X(a_0,a_1,a_2,\ldots,a_{n-1},a_n). $$

\proclaim Proposition 2.1. {\rm (Karl Mayr [10, p.~284]) \ }
The roots of the general equation of degree $n$ satisfy
the following system of linear partial differential equations:
$$  {\partial^2 X \over \partial a_i \partial a_j} \, = \,
{\partial^2 X \over \partial a_k \partial a_l} \qquad
\hbox{whenever} \quad i+j = k+l, \eqno (2.2) $$
$$ \sum_{i=0}^n i a_i {\partial X \over \partial a_i} \,\,\, = \,\,\, - X
\qquad \hbox{and} \qquad
 \sum_{i=0}^n  a_i {\partial X \over \partial a_i} \,\,\, = \,\,\, 0 .
\eqno (2.3) $$

\vskip .1cm

The system (2.2)-(2.3) is a special instance of the class of
${\cal A}$-hypergeometric differential equations introduced
by Gel'fand, Kapranov and Zelevinsky [5],[7]. Namely, 
(2.2)-(2.3) is the ${\cal A}$-hypergeometric system
with parameters  ${-1 \choose \phantom{-}0 }$ associated with the
integer matrix
$$ {\cal A} \quad := \quad 
\pmatrix{ 0 & 1 & 2 & 3 &  \cdots & n-1 & n  \cr
          1 & 1 & 1 & 1 & \cdots & 1 & 1 \cr}.  \eqno (2.4) $$
The column vectors of ${\cal A}$ are homogeneous coordinates
of $n+1$ equidistant points on the line.
Their convex hull is a line segment: it is 
the Newton polytope of $f(x)$.

The Euler-type equations (2.3) follow readily from the 
homogeneity relations
$$ \eqalign{ 
X(a_0, t a_1, t^2 a_2, \ldots, t^{n-1} a_{n-1}, t^n a_n) & \quad = 
\quad {1 \over t} \cdot X(a_0,a_1,a_2,\ldots,a_{n-1},a_n) , \cr
X(t a_0, t a_1, t a_2, \ldots, t a_{n-1}, t a_n) & \quad = 
\quad X(a_0,a_1,a_2,\ldots,a_{n-1},a_n) .\cr }$$
The equations (2.2) appeared in 
Karl Mayr's 1937 paper [10, equation (2) on page 284]. \break
I shall present two proofs different proofs.
The first one was shown to me in the spring of 1992 by 
Jean-Luc Brylinski. See [3] for an appearance of (2.2)
in differential geometry.

\vskip .2cm

\noindent {\sl Brylinsky's proof of (2.2):}
It uses implicit differentiation and works over any base field.
We consider the first derivative $\,f'(x) \,=\, \sum_{i=1}^n
i a_i x^{i-1} \,$ and the second derivative
$\,f''(x) \,=\, \sum_{i=2}^n i (i-1) a_i x^{i-2} $. 
Note that $\,f'(X) \not= 0 $, since $a_0,\ldots,a_n$
are indeterminates.
Differentiating the defining identity
$\, \sum_{i=0}^n a_i X(a_0,a_1,\ldots,a_n)^i \, = \, 0 \,$
with respect to $a_j$, we get
$$ X^j \, + \, f'(X) \cdot {\partial X \over \partial a_j} 
\quad = \quad 0 . \eqno (2.5) $$
We next differentiate $\, {\partial X / \partial a_j }\,$
with respect to the indeterminate $a_i$:
$$ {\partial^2 X \over \partial a_i \partial a_j } \quad = \quad
{\partial \over \partial a_i } \bigl( - { X^j \over f'(X) } \bigr)
\quad = \quad
{\partial f'(X) \over \partial a_i } X^j f'(X)^{-2}  \, - \,
j X^{j-1} {\partial X \over \partial a_i } f'(X)^{-1} . 
\eqno (2.6) $$
Using (2.5) and the identity
$\,{\partial f'(X) \over \partial a_i } \,=\,
- {f''(X) \over f'(X)} \cdot X^i + i X^{i-1} $, we can rewrite (2.6) as 
$$ {\partial^2 X \over \partial a_i \partial a_j } \quad = \quad
- f''(X) X^{i+j} f'(X)^{-3} \,\, + \,\,
(i+j) X^{i+j-1} f'(X)^{-2}. \eqno (2.7) $$
The expression (2.7) depends only on the sum of indices $i+j$.
This proves (2.2). \Box

\vskip .1cm

\noindent {\sl A complex analysis proof of (2.2): }
Suppose we are working over the field of complex numbers ${\bf C}$.
Consider the logarithmic derivative
$\,[log(f(x))]' \,= \, f'(x)/f(x) $. We view it as a rational
function in $ahabe_0,a_1,\ldots,a_n$ and differentiate with respect
to these variables:
$$ {\partial^2 \over \partial a_i \partial a_j} [log(f(x))]'  
\quad = \quad 
\bigl( { - x^{i+j} \over f^2(x) } \bigr)' .$$
This shows that $[log(f(x))]'  $ satisfies the
quadratic ${\cal A}$-hypergeometric
equations (2.2). Proposition 2.1 follows by differentiating under the
integral sign in Cauchy's formula
$$ X \quad = \quad {1 \over 2 \pi i}
 \int_\Gamma {z f'(z) \over f(z)} dz, \eqno (2.8) $$
where $\Gamma$ is a sufficiently small loop in the complex plane. \Box

\beginsection 3. Series Expansions

For any rational number $u$ and any integer $v$ we abbreviate
$$ \gamma(u,v) \,\,\, := \,\,\, \cases{ \qquad
   1         &  if $v = 0$, \cr
   u (u  -  1) (u  - 2)  \cdots  (u+v+1)    & if $v  < 0$, \cr
   \qquad 0     & if $u$ is a negative integer and $u \geq -v $, \cr \quad
  {1 \over (u+1)(u+2) \cdots (u+v)} & otherwise. \cr }$$
If $u$ is not a negative integer then
$\gamma(u,v-1) = (u+v) \cdot \gamma(u,v)$ and we have
$\gamma(u,v) = \Gamma(u+1) / \Gamma(u+v+1)$, where
$\Gamma$ is the usual Gamma function.

Let ${\cal L} $ denote the integer kernel of ${\cal A}$.
This is the $(n-1)$-dimensional sublattice of ${\bf Z}^{n+1}$
spanned by $\{ e_{i-1} - 2 e_i + e_{i+1} \,: \,
i = 1,\ldots,n \!- \!1\}$.
Consider any monomial $ a_0^{u_0} a_1^{u_1} \cdots a_n^{u_n}$
with rational exponents in the coefficients of $f(x)$.
We define the formal power series
$$
\bigl[ a_0^{u_0} a_1^{u_1} \cdots a_n^{u_n} \bigr] \quad := \quad
\sum_{(v_0,\ldots,v_n) \in {\cal L}}
\prod_{i=0}^n  \, \bigl(\,\gamma(u_i,v_i) \,a_i^{u_i+v_i} \bigr). 
\eqno (3.1) $$
This series satisfies the linear differential equations (2.3) with
their right hand sides $-X$ and $0$ replaced by
$ \gamma_1 X$ and $ \gamma_2 X$, and it
satisfies the quadratic equations (2.2)
in the following two cases. To prove the
second case it is best to first derive formula (4.2) below.

\proclaim Lemma 3.1. {\rm ([6, Lemma 1]) }
Let $(u_0,u_1,\ldots,u_n)$ be a rational vector which either
has no negative integer coordinate or has the form
$(0,\ldots,0,1,-1,0,\ldots,0)$. Then
the series $\bigl[ a_0^{u_0}\cdots a_n^{u_n} \bigr]$ 
is a formal solution of the  ${\cal A}$-hypergeometric system
with parameters $\,{ \gamma_1 \choose \gamma_2 } \,= \, 
{\cal A} \cdot (u_0,u_1,\ldots,u_n)^T$.

Gel'fand, Kapranov and Zelevinsky constructed
a complete set of series solutions for each
regular triangulations of the set ${\cal A}$.
We shall adapt their general construction to our special case.
Extra care must be taken, however, because our equations
(2.3) do not satisfy the non-resonance hypothesis which is
necessary for [6, Theorem 3] to hold.

We write $0,1,\ldots,n$ for the points in our configuration ${\cal A}$ 
in (2.4). It has $2^{n-1}$ triangulations, all of which are regular. 
Each triangulation is indexed by a subset $I$ of $\{1,\ldots,n-1\}$.
Writing the complementary subset as
$\, \{0,1,\ldots,n\} \backslash I \,\, =\,\,
\{0= i_0 < i_1 < i_2 < \cdots < i_{r-1} < i_r = n \}$,
the triangulation of ${\cal A}$ indexed by $I$
consists of the $r$ segments
$[i_0,i_1],[i_1,i_2],\ldots,[i_{r-1},i_r]$.
See [8, Sections 7.3.A and 12.2.A] for details.
If $r=n$ (resp.~$r=1$) then this is the finest
(resp.~coarsest) triangulation referred to in Section 1.

We are now prepared to present the main result of this note. 
In the remainder of Section 3 we shall be
working over the field of complex numbers ${\bf C}$.
Fix any of the $2^{n-1}$ triangulations of ${\cal A}$.
For $j = 1, \ldots ,r $ we write
$\,d_j \,:=  \,i_j - i_{j-1} \,$ for
the length of the $j$-th segment in that triangulation. 
Clearly, $d_1 + d_2 + \cdots + d_r = n$.
Let $\,\xi \, = \, (-1)^{1/d_j }\,$ be any of the $d_j $-th roots of $-1$.
We define the ${\cal A}$-hypergeometric series
$$ X_{j,\xi} \quad := \quad
\xi \cdot \biggl[ {a_{i_{j-1}}^{1 \over d_j}  a_{i_j}^{-{1 \over d_j}}} 
\biggr] \quad + \quad {1 \over d_j} \cdot
\sum_{k = 2}^{d_j} \xi^k \cdot \biggl[
a_{i_{j-1}+k-1} \,a_{i_{j-1}}^{{k-d_j \over d_j}}  a_{i_j}^{-{k \over d_j}}
\biggl]
\quad + \quad 
{1 \over d_j} \cdot
 \biggl[ {a_{i_{j-1}-1} \over a_{i_{j-1}} } \biggr]
$$
If $j=1$ then the expression
$\bigl[ {a_{-1} \over a_{0} }\bigr]$ appears
in the rightmost summand. We define it to 
be zero. Note that by varying $j$ and $\xi$ we
have defined $n$ distinct series in total.

\proclaim Theorem 3.2.
The $n$ series $X_{j,\xi}$ are roots of the
general equation of order $n$, that is,
 $f( X_{j,\xi} ) = 0$. There exists a constant $M$ such that
all $n$ series $X_{j,\xi}$ converge whenever
$$\, |a_{i_{j-1}}|^{i_j - k} \cdot |a_{i_j}|^{k - i_{j-1}} 
 \,\, \leq  \,\,  M \cdot |a_k|^{d_j} \quad
\hbox{for all} \,\,\, 1 \leq j \leq r \, \hbox{ and } \, 
k \not\in \{ i_{j-1} , i_j \}.
\eqno (3.2) $$

\noindent {\sl Proof: }
Consider the open convex cone
$$ C \,\, = \,\, \bigl\{ \,w \in {\bf R}^{n+1} \,:\,
(i_j - k) \cdot w_{i_{j-1}} \, + \,(k - i_{j-1}) \cdot w_{i_j}
 < d_j \cdot w_k \,\,\,
\hbox{for} \,\, j=1,\ldots \!,r, \,
k \not\in \{i_{j-1},i_j \} \bigl\}. $$
This is the outer normal cone to the secondary polytope 
$\Sigma({\cal A})$ at the vertex corresponding to the triangulation 
$I$ of ${\cal A}\,$ (cf.~[8, Theorem 12.2.2]). Equivalently, the 
cone $C$ consists  of all vectors $w = (w_0,w_1,\ldots,w_n)$ 
which induce the triangulation in question.

Let ${\cal U}$ be the region in the coefficient space ${\bf C}^{n+1}$
defined by the inequalities (3.2)
for $M \gg 0$. There exists a vector 
$V \in {\bf R}^{n+1}$ such 
that $\,(log(|a_0|),\ldots,log(|a_n|))-V \,\in \, C $
for all $(a_0,\ldots,a_n)$ in ${\cal U}$. Let ${\cal H}$
be the space of all complex-valued functions on ${\cal U}$
which are ${\cal A}$-hypergeometric with parameters
${-1 \choose \phantom{-} 0 }$. The ${\cal A}$-hypergeometric
system is holonomic of rank $n$ and its singular locus,
the discriminantal locus of $f$,
is disjoint from ${\cal U}\,$ (see [6]). Hence ${\cal H}$ is a
complex vector space of dimension at most $n$.
We shall identify $n$ linearly independent elements 
in ${\cal H}$, which will imply that ${\cal H}$ has dimension exactly $n$.

It follows from [6, Proposition 2] that the series
$\bigl[ a_0^{u_0} \cdots a_n^{u_n} \bigr]$ defined in (3.1) converges 
in ${\cal U}$ provided {\sl at most two} of the exponents $u_i$
are non-integers. Each of the summands in the definition
of $X_{j,\xi}$ has this property. Therefore $X_{j,\xi}$ converges 
in ${\cal U}$. Using Lemma 3.1, we conclude that the $n$ 
series $X_{j,\xi}$ lie in the vector space ${\cal H}$.

We fix an vector $w=(w_0,\ldots,w_n)$ in $C$ such that
all coordinates $w_i$ are integers and such that
no two of the lines spanned by pairs 
$\{ (w_i,i), (w_j,j) \}$ in ${\bf R}^2$ are parallel.
The weight of a monomial  $\,a_0^{i_0} a_1^{i_1} \cdots a_n^{i_n}\,$ 
(with rational exponents) is defined
to be $w_0 i_0 + w_1 i_1 + \cdots + w_n i_n$.
We replace the input equation by its {\it toric deformation}
$$ f_t(x) \quad = \quad
  a_0 t^{w_0} \, + \,a_1 t^{w_1} x  \,+ \,a_2 t^{w_2} x^2 \,+\,
\ldots \,+\, a_{n-1} t^{w_{n-1}} x^{n-1} \,+ \, a_n t^{w_n} x^n .\eqno (3.3) $$
We shall study the $n$ roots as an algebraic function of $t$.
For $t$ close to the origin the $n$ roots split into $r$ groups,
one for each segment $[i_{j-1},i_j]$, for $j=1,\ldots,r$.
(This is a special case of the multivariate
construction in [9, \S 3].)
The roots in the $j$-th group possess a Puiseux expansion of the form
$$ \psi_{j,\xi}(t) \quad = \quad
\xi \cdot a_{i_{j-1}}^{1 \over d_j} \cdot a_{i_j}^{- {1 \over d_j}} \cdot
t^{{1\over d_j} (w_{i_{j-1}}-w_{i_j})}
\,+\, \hbox{higher terms in $t$}, $$
where $\xi$ satisfies $\xi^5 = - 1$.
We shall prove that $\, \psi_{j,\xi}(1) \,= \,X_{j,\xi}\,$
for all $(a_0,\ldots,a_n) \in {\cal U}$.
To this end we first determine the second lowest term in the 
series $\psi_{j,\xi}(t)$.
After modifying the weight vector $w$ by
an affine transformation $\,w_i \mapsto \alpha w_i + \beta $,
which does not alter the structure of the series,
we may assume that $\,
w_{i_j} = w_{i_{j-1}} = 0 $, and there is a unique index $r$
with $w_r = 1$, and $w_l > 1 $ for $l \in \{0,1,\ldots,n \} \backslash
\{ i_j, i_{j-1}, r \}$. An explicit calculation reveals
$$ \eqalign{  \psi_{j,\xi}(t) & \quad = \quad
\xi  a_{i_{j-1}}^{1 \over d_j}  a_{i_j}^{- {1 \over d_j}}  \cr
\,& + \,\, {1 \over d_j} \cdot \xi^{ r + 1 - i_{j-1} } \cdot a_r \cdot 
a_{i_{j-1}}^{(r+1-i_j)/d_j} \cdot a_{i_j}^{(i_{j-1}-r-1)/d_j} \cdot t 
\,\,+\,\, \hbox{higher terms in $t$}. \cr} \eqno (3.4) $$
By varying $w$ within the cone $C$ we can arrange that the role of $r$
is played by any index in the set
$$ \bigl(\{0,1,\ldots,n\} \backslash \{i_1,i_2,\ldots,i_r\} \bigr)
\,\cup \,\{i_{j-1},i_{j+2}\}. \eqno (3.5) $$
Note that the cardinality of this set is at least $d_j-1$. 
Consider the linear map that extracts from a series in $a_0,\ldots,a_n$
all those terms which are lowest or second lowest with respect to
the grading defined by some $w \in C$. Call this map $T$.
Consider the image of a root $\psi_{j,\xi}(1)$ under $T$.
This is a polynomial (with fractional exponents) having at
least $d_j$ distinct terms, one for the lowest term in (3.4)
and  at least $d_j - 1$ for the distinct
values of $r$ coming from (3.5). The coefficients of
these terms are distinct powers of $\xi$ divided by $d_j$.
These considerations imply that the images of the roots
$\psi_{j,\xi}(1)$ under $T$ are linearly independent.
Therefore the roots themselves
are linearly independent over ${\bf C}$. Moreover, they
all satisfy the ${\cal A}$-hypergeometric system with parameter
${-1 \choose \phantom{-} 0 }$, by Proposition 2.1.
 We conclude that the space ${\cal H}$
is $n$-dimensional and the roots $\psi_{j,\xi}(1)$ 
form  a basis for ${\cal H}$.

We wish to prove $\,\psi_{j,\xi}(1) = X_{j,\xi}$. It suffices 
to show $\, T(\psi_{j,\xi}(1)) = T(X_{j,\xi})\,$ because the 
functional $T$ defined above separates the space ${\cal H}$.
Equivalently, given any generic $w$ in $C$, we must show that,
in the $w$-grading, $\,X_{j,\xi}\,$ has the same first two terms as
$\,\psi_{j,\xi}(1)$. This is clear for the first term, so we only need
to look at the second term
$ \, {1 \over d_j} \xi^{ r + 1 - i_{j-1} } \cdot
a_r \cdot a_{i_{j-1}}^{(r+1-i_j)/d_j} \cdot a_{i_j}^{(i_{j-1}-r-1)/d_j} $.
Let $a^v$ denote this monomial.
Let $k $ be the unique integer between $1$ and $d_j$
such that $r$ is congruent to $i_{j-1} - 1 + k$ modulo $d_j$.
If $k = 1$ then $a^v$ equals the second lowest term 
(with respect to $w$) of the 
series $\,\xi \cdot [ a_{i_{j-1}}^{1/d_j} a_{i_{j}}^{-1/d_j} ]$.
If $2 \leq k \leq d_j -1$ then $a^v$ equals
the $w$-lowest term of the series indexed by $k$
in the central sum in the definition of $X_{j,\xi}$.
For $k= d_j$ there are two subcases:
if $ r > i_{j-1} $  then $a^v$ equals
the $w$-lowest term of the series indexed by $k = d_j$
in the center sum; if $r < i_{j-1}$ then $a^v$ equals
the $w$-lowest term of the rightmost series 
$\,{1 \over d_j} [ a_{i_{j-1}-1} / a_{i_j-1}]$.
In each of these four cases  $a^v$ is the second lowest term
in $X_{j,\xi}$. This completes the proof of Theorem 3.2. \Box

\beginsection 4. Integrality Issues

To prove the identity $f(X_{j,\xi}) = 0$
in Theorem 3.2 we used the complex numbers.
But, a posteriori, there is no need to stick with
an algebraically closed field of characteristic zero.
If $K$ is any integral domain such that the equation
$\xi^{d_j}=-1$ has $d_j$ distinct solutions
and the coefficients of $X_{j,\xi}$ are defined in $K$,
then $f(X_{j,\xi}) = 0$ is a valid identity in a
suitable fractional power series ring over $K$.
The following result shows that $K$ can be
an algebraically closed field of characteristic $p$,
provided $p$ is relatively prime to $\,d_1 d_2 \cdots d_r $.
The proof of Theorem 4.1 using Hensel's Lemma
was suggested to me by Hendrik Lenstra.

\proclaim Theorem 4.1. All coefficients of the
series $\,X_{j,\xi} \,$ lie in the ring
$\,{\bf Z}[ {1 \over d_j} ][ \xi ]$.

\noindent {\sl Proof: } The series $\psi_{j,\xi}(t)$ 
in (3.4) lies in the fractional power series ring
$\,R_1 [[ t^{1 \over d_j }]] $, where
$$ R_1 \quad := \quad
{\bf Q}[\,\xi \,]\bigl[ a_s : s \not\in \{i_{j-1},i_j\} \bigr]
[ a^{\pm {1 \over d_j}}_{i_{j-1}}, a^{\pm {1 \over d_j}}_{i_{j}}]. $$
This holds because $\omega$ is integral and
each term of $X_{j,\xi}=\psi_{j,\xi}(1)$ lies in $R_1$.
To prove Theorem 4.1, it suffices to show that  $\psi_{j,\xi}(t)$  is 
an element of the subring $\,R_2 [[ t^{1 \over d_j }]] $, where
$$ R_2 \quad := \quad
{\bf Z}[{1 \over d_j}][\,\xi\,]\bigl[ a_s : s \not\in \{i_{j-1},i_j\} \bigr]
[ a^{\pm {1 \over d_j}}_{i_{j-1}}, a^{\pm {1 \over d_j}}_{i_{j}}]. $$
We shall apply Hensel's Lemma [5, Theorem 7.3] to
the integral domain $\,R_2 [[ t^{1 \over d_j }]] $.
This domain is complete with respect to the principal ideal
$\, {\bf m} := \langle t^{1 \over d_j } \rangle $, and it
contains all  coefficients of the polynomial  $f_t(x)$ in (3.3).
The constant term of $\psi_{j,\xi}(t)$ lies in $R_2$;
we denote it by
$\,A \,:= \,\xi a_{i_{j-1}}^{1 \over d_j}  a_{i_j}^{-{1 \over d_j}}$.
It is an ``approximate root'' in the sense that
$\,f_t(A) \in {\bf m}$ and $\,f_t'(A)\,$ is the sum
of a unit in $R_2$ and an element of ${\bf m}$.
By Hensel's Lemma there exists a unique element
$B$ in $\,R_2 [[ t^{1 \over d_j }]] \,$ such that
$f_t(B) = 0$ and $A - B \in  {\bf m}$.
By repeating the uniqueness part of this application of Hensel's Lemma for the
coefficient domain $R_1$ instead of $R_2$, we conclude
that $B$ must be equal to our Puiseux series $\psi_{j,\xi}(t)$. \Box

\vskip .1cm

A case of special interest is the finest triangulation,
where $\,I = \{1,2,\ldots,n-1\}$, $r=n $, $d_j = 1 $ for all $j$.
Theorem 4.1 implies that in this case our construction is
characteristic-free. In other words, the series
solutions $ X_{j,-1}$ have integer coefficients. 
These series are
$$ X_{j,-1} \quad = \quad
- \biggl[ {a_{j-1} \over a_j } \biggr]
+ \biggl[ {a_{j-2} \over a_{j-1} } \biggr] \qquad \qquad
\hbox{for} \quad j=1,2,\ldots,n \eqno (4.1) $$
where $\, \bigl[ {a_{j-1} \over a_j } \bigr]\,$
is the sum over all Laurent monomials
$$ { (-1)^{i_j} \over i_{j-1}+1} \cdot
\,{ i_j \choose i_0 \ldots i_{j-1} \, i_{j+1} \ldots i_n}\cdot
{a_0^{i_0} a_1^{i_1} \cdots a_{j-2}^{i_{j-2}} \,a_{j-1}^{i_{j-1}+1}  \,
   a_{j+1}^{i_{j+1}} \cdots a_n^{i_n} 
\over a_j^{i_j+1}} \eqno (4.2)
$$ where $i_0,i_1,\ldots,i_n$ are non-negative integers 
satisfying the relations
$$ \eqalign{
& i_0 + i_1 + i_2 + i_3 +  \cdots + i_{j-1}\,\, 
- i_j\,\, + i_{j+1} + \cdots + i_n 
\quad = \quad 0 \cr
   & i_1 + 2 i_2 + 3 i_3 \cdots + (j-1) i_{j-1} \, - j i_j \,
+ (j+1) i_{j+1} + \cdots + n i_n  \quad = \quad 0 .\cr}  \eqno (4.3) $$
For generating the series  $\, \bigl[ {a_{j-1} \over a_j } \bigr]\,$
on a computer it convenient to rewrite (4.3) as follows.
$$ \eqalign{
i_{j-1} \quad & = \quad
 - j i_0 - (j-1) i_1 - (j-2) i_2 - \cdots 
- 2 i_{j-2} \,+\, i_{j+1} + 2 i_{j+2} + \cdots + (n-j) i_n, \cr
i_{j} \quad & = \quad
 - (j-1) i_0 - (j-2) i_1  - \cdots 
 -  i_{j-2} \,\,\,+\, 2 i_{j+1} + 3 i_{j+2} + \cdots \,+ (n-j+1) i_n .\cr}$$
These equations ensure that the multinomial coefficient in (4.2)
is divisible by $i_{j-1}+1$.

\beginsection 5. Multivariate Outlook

Consider a system of $n$ polynomial equations in $n$ variables,
where the terms in the \break $i$-th equation have their
exponent vectors in a fixed set $\,{\cal A}_i \subset {\bf Z}^n $.
(This is the meaning of ``sparse'' in [9]).
By Bernstein's Theorem, the system has
mixed volume many roots $(X_1,\ldots,X_n)$.
Each coordinate $X_i$ is an algebraic function in
all the coefficients. It is
natural to wonder whether $X_i$ satisfies the ${\cal A}$-hypergeometric 
differential equations where ${\cal A}$ is the
configuration arising from the ``Cayley trick'' \
(cf.~[7, 2.5], [8, page 273]):
$$ {\cal A} \quad := \quad
{\cal A}_1 \times \{ e_1 \} \,\,\cup \,\,
{\cal A}_2 \times \{ e_2 \} \,\,\cup \,\,
\cdots \,\,\cup \,\,
{\cal A}_n \times \{ e_n \} \quad \subset \quad
{\bf Z}^{2n} .$$
The answer is {\bf no}. The coordinates of the roots
$(X_1,\ldots,X_n)$ are not ${\cal A}$-hypergeometric.
For example, let $(X_1,X_2)$ be the unique solution of the 
system of two linear equations
$$ a_0  \,+\, a_1 x_1 \,+\, a_2 x_2 \quad = \quad
   b_0  \,+\, b_1 x_1 \,+\, b_2 x_2 \quad = \quad 0 .$$
Then we find
$$ {\partial^2 X_2  \over \partial a_0 \partial b_1} \quad 
\not= \quad 
{\partial^2 X_2  \over \partial a_1 \partial b_0} .$$
Note also that Mayr's differential equations (40)
and (45) in [10] are no longer binomial.

\vskip .2cm

The correct generalization of Proposition 2.1 to 
higher dimensions is the following.
Let $J(x_1,\ldots,x_n)$ denote the
{\sl Jacobian} $\,det({ \partial f_i \over \partial x_j }) \,$
of the given equations $\,f_1 = \cdots = f_n = 0$.

\proclaim Proposition 5.1.
For any integers $\,u_1,u_2,\ldots,u_n \,$ the
algebraic function
$$ { X_1^{u_1} X_2^{u_2}  \cdots X_n^{u_n} \over
    J(X_1,X_2,\ldots, X_n) } \eqno (5.1) $$
satisfies the ${\cal A}$-hypergeometric 
differential equations  arising from the Cayley trick.

Over the field of complex numbers, we can
derive Proposition 5.1 from Theorem 2.7 in [7] using
Cauchy's formula in several variables.
An alternative algebraic proof follows
from the construction in [1, \S 2].
The quantity in (5.1) should be thought of as a
{\it local residue} on a toric variety [4]. The 
sum over all (mixed volume many) local residues
is the {\it global residue} associated with
the monomial $x_1^{u_1} \cdots x_n^{u_n}$ relative
to the given equations $\,f_1= \cdots = f_n = 0 $.
The global residue is a rational function. It is
of importance in elimination theory. 
We plan to extend the techniques in Section 3 to 
the setting of Proposition 5.1 in a subsequent joint work 
with E.~Cattani and A.~Dickenstein. The goal 
of that project is to develop new 
formulas and algorithms for computing local and global residues.
Another interesting question is whether explicit
Puiseux series expansions of (5.1) might be useful to 
improve the numerical component in the
homotopy algorithm proposed in [9].

\vskip .6cm

\centerline{\bf Acknowledgements}

\vskip .1cm

\noindent
I am grateful to Alicia Dickenstein, Hendrik Lenstra and  
Michael Trott for helpful comments on earlier drafts of this paper.
This project was partially supported by
grants from the National Science Foundation (NYI)
and the David and Lucile Packard Foundation.


\vskip .8cm


{\baselineskip=11pt
\centerline{\bf References}

\vskip .1cm
\item{[1]} A.~Adolphson and S.~Sperber: Differential modules
defined by systems of equations, 
{\sl Seminario Rendiconti di Universita di Padova}, to appear.

\vskip .1cm
\item{[2]} R.~Birkeland: \"Uber die Aufl\"osung algebraischer
Gleichungen durch hypergeometrische Funktionen,
{\sl Mathematische Zeitschrift} {\bf 26} (1927) 565--578.

\vskip .1cm
\item{[3]} J.-L.~Brylinski: Radon transform and functionals
on the space of curves, Manuscript, 1993.

\vskip .1cm
\item{[4]} E.~Cattani, D.~Cox and A.~Dickenstein:
Residues in toric varieties, {\sl Compositio Math.}, to appear.

\vskip .1cm
\item{[5]} D.~Eisenbud:
{\sl Commutative Algebra with a View Towards Algebraic Geometry},
Graduate Texts in Mathematics, Springer Verlag, New York, 1995.

\vskip .1cm
\item{[6]} I.M.~Gel'fand, A.V.~Zelevinsky and M.M.~Kapranov:
Hypergeometric functions and toral manifolds,
{\sl Functional Analysis and its Applications}
{\bf 23} (1989) 94--106.

\vskip .1cm
\item{[7]} I.M.~Gel'fand, A.V.~Zelevinsky and M.M.~Kapranov:
Generalized Euler integrals and ${\cal A}$-hypergeometric
functions, {\sl Advances in Mathematics} (1990) {\bf 84} 255--271.

\vskip .1cm
\item{[8]} I.M.~Gel'fand, A.V.~Zelevinsky and M.M.~Kapranov:
{\sl Discriminants, Resultants, and Multidimensional Determinants},
Birkh\"auser, Boston, 1994.

\vskip .1cm
\item{[9]} B.~Huber and B.~Sturmfels:
A polyhedral method for solving sparse polynomial systems,
{\sl Mathematics of Computation} {\bf 64} (1995) 1541--1555.

\vskip .1cm
\item{[10]} K.~Mayr: \"Uber die Aufl\"osung algebraischer
Gleichungssysteme durch hypergeometri\-sche Funktionen,
{\sl Monatshefte f\"ur Mathematik und Physik}
{\bf 45} (1937) 280--313.

\vskip .1cm
\item{[11]} J.~McDonald: Fiber polytopes and fractional power series,
{\sl Journal of Pure and Applied Algebra}, 
{\bf 104} (1995) 213--233.


\vskip .1cm
\item{[12]} {\sl Solving the Quintic with Mathematica}, poster
distributed by Wolfram Research, 1994.

}
\bye
