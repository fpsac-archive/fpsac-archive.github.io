%%%%%%%%%%%%%%%%%%%%%%%%%%%%%%%%%%%%%%%%%%%%%%%%%%
% Feb 24, 2004 
\documentclass{amsart}
\usepackage{amssymb,latexsym}
\begin{document}

\begin{center}
\Large {\bf CURRENT DIRECTIONS IN ALGEBRAIC COMBINATORICS 2003}
\medskip

\large FPSAC '03 WHITE PAPER 
\end{center}
\vspace{.6cm}

\begin{verse}
{\bf Content:}\\
1. Introduction\\
2. Enumeration\\
3. Statistical physics\\
4. Discrete probability\\
5. Bioinformatics\\
6. Quantum analogues\\
7. Non-commutative and non-symmetric functions\\
8. Face numbers and algebraic shifting\\
9. Cluster algebras and the Laurent phenomenon\\
10. Hopf algebras\\
11. Smoothed analysis
\end{verse}

\section{Introduction}
The field of algebraic combinatorics has developed rapidly in recent years
and is rich in connections with other areas,
within mathematics as well as with physics, computer science, bioinformatics and
other sciences. This activity and breadth is well represented at the annual FPSAC conferences, 
the most recent one taking place in Vadstena, Sweden, in June 2003.

Our aim here is to highlight some of the recent progress, capture some of the most important
current directions, and try to delineate some of the emerging trends in algebraic combinatorics.
This document was prepared, based on discussions at the FPSAC'03 conference, by an 
{\em ad hoc} committee consisting of Nantel Bergeron, Anders Bj\"orner, Mireille Bousquet-M\'elou, 
Art Duval, Kimmo Eriksson, Christian Krattenthaler, Alain Lascoux, Svante Linusson 
and Richard Stanley.


\section{Enumeration} Enumerative questions are at the core of algebraic combinatorics.
They are involved in almost all topics discussed in  this document.


The subject of enumeration of (symmetry classes of)
plane partitions and alternating sign matrices is a classical subject
by now. It got a new twist by the recent observations that plane
partitions
can be seen as rhombus tilings and that
alternating sign matrices are in bijection with configurations in
the intensively studied six vertex model of statistical mechanics.
While many of the originally conjectured results are
proved by now (by Andrews, Kuperberg,
MacMahon, Mills--Robbins--Rumsey,
Proctor, Stanley, Stembridge, Zeilberger), the understanding
of this circle of problems is still very incomplete, particularly for
the alternating sign matrices: the fact that there are
still several conjectures unproved
(such as the conjecture about the $q$-enumeration of totally symmetric
plane partitions, or the conjectured enumeration formulas
for three symmetry classes of
alternating sign matrices) implies that we are still missing some
important enumeration techniques which would lead to solutions of
these problems. This is particularly true for the alternating sign
matrices, where the solutions of the problems so far always contain
a substantial guessing part. Second, there is still no understanding
at all what plane partitions (rhombus tilings) and
alternating sign matrices (six vertex model configurations) have to
do with each other, although, empirically, the connection is
``obvious'' from comparison of the (proven or conjectured) enumeration
results. Here, it seems, a new, fundamental bijection has to be found.

Finally, new mysteries have arisen recently in work by mathematical
physicists (Razumov, Stroganov, de Gier, Batchelor, Nienhuis) on
the XXZ spin chain: the ground states of certain Hamiltonians
seem to be given by the numbers of configurations in the $O(1)$ loop
model (another disguise of six vertex model configurations), but there
is no understanding at all why.
It is however obvious that some very intriguing combinatorics must lie
behind these phenomena.


\section{Statistical physics}
The field of physics known as statistical physics aims at predicting
the behavior of macroscopic objects from the description of the
interactions between their particles at a microscopic scale. The {\em
  models\/} of statistical physics simplify these interactions by
assuming that the particles live on a regular grid and are only
subjected to nearest-neighbor interactions. {\em Solving\/} a model
means computing its partition function, which enumerates all possible
configurations of particles according to their energy: hence, models
of statistical physics are actually problems in enumerative
combinatorics.

During a few decades, the main consequence of this connection was
that many physicists were doing combinatorics. We can cite the names of
Hammersley, Fisher, Temperley, in the 50's, for their work on
self-avoiding walks (polymer models, in physics), or Kac and Ward for the first
combinatorial solution of the most famous magnetism model, the Ising
model. 

Nowadays, the border between combinatorics and physics has been
crossed many times by researchers from both fields. In particular,
combinatorialists have realized that statistical physics is an
extremely profound and motivating source of problems in
enumeration. Let us cite a few recent examples that illustrate this
fruitful interplay.

\subsection{Statistical physics in high dimensions.} For many models,
it is believed that there exists a {\em critical dimension\/}, above
which the system behaves like an unconstrained (or: mean-field, in
physics terms) model. For example, large self-avoiding walks (SAW) are
supposed to behave like large random walks in dimension 5 or more. It
might seem surprising to combinatorialists that high-dimensional models
are sometimes more tractable than low-dimensional ones, but this is
true: about 10 years ago, Hara and Slade proved the mean-field
behavior of SAW in dimension 5 or more -- whereas nothing similar
is known in smaller dimension. Their technique involves a mixture of
combinatorial and analytic arguments, and is gradually being
generalized to other models (lattice trees, percolation...)

\subsection{Quantum gravity and statistical physics on planar maps}
In 2D quantum gravity, the change in the geometry of the space is
modelled by putting particles, not on the vertices of a regular grid,
but on the vertices of a random planar map (a map is the embedding of
a graph in the plane). Since the early 70's, certain physicists have
developped very effective (but usually non-rigorous) techniques,
based on matrix integrals, to solve models on maps. Their results have
sometimes a remarkably simple structure, which begs for a combinatorial
understanding. This was done a year ago by Bousquet-M\'elou and
Schaeffer for the Ising model. A lot remains to be understood.

\subsection{Alternating sign matrices, the 6-vertex model, and the XXZ
  spin chain}
The intriguing story of these objects and related conjectures has
already been told in the section about enumerative combinatorics. Let
us simply underline that certain determinantal expressions found by
physicists for the 6-vertex model were the key step in the first
``simple'' proof of the ASM-conjecture, and that the more recent
conjectures relating ASM to a certain Markov chain are also due to
physicists...

More examples could be added, for instance a very nice collection of
results on the enumeration of perfect matchings of graphs (in
physics language, dimer coverings) by Propp, Kenyon, Wilson et al. Let us
 conclude by mentioning a few simple ``physics'' results that still
await a combinatorial explanation: the enumeration of 3D {\em directed
  animals\/} (Dhar) or the refined enumeration of 2D directed animals
(Bousquet-M\'elou) yield nice algebraic functions, but have only been
obtained via an equivalence with certain gas models; a determinant
related to meanders (Di Francesco et al.) admits a mysterious
expression in
terms of Chebychev polynomials; finally, why is the spontaneous
magnetization of the 2D Ising model so simple?


\section{Discrete Probability}

Combinatorics and discrete probability theory benefit from each other in a
variety of ways; some related to questions in physics,
others not. In extremal combinatorics, the so-called {\em
probabilistic method}, which goes back to Erd\H{o}s, continues to be
enormously useful.
In the other direction, combinatorial arguments often form the core part of
proofs in various parts of probability, such as in large deviations theory,
percolation theory, and the study of systems with spatial interaction. A
striking example is Smirnov's celebrated breakthrough for establishing
conformal invariance in two-dimensional critical percolation. The talk by
Olle H\"aggstr\"om at the FPSAC gave other interesting examples when the
combinatorial properties of graphs and lattices are crucial to what can be
said for problems in the above-mentioned fields.  One of the satellite
conferences to the European Congress of Mathematics 2004 has ``combinatorial
methods in statistical mechanics'' in the title. It is a topic of rising
interest.

During the last decade we have seen a growth in problems that have interested
researchers across disciplinary boundaries. A good example is the Random
Assignment Problem (a.k.a. random bipartite matching) studied by
researchers from optimization, statistical mechanics, probability and
combinatorics. In short it is the problem of deciding the size of a minimal
assignment when the entries in the matrix are random variables. A
fascinating conjecture from 1998 by the theoretical physicist Parisi giving
the exact answer $1+1/4+1/9+\dots+1/k^2$ in the case of exponentially
distributed random variables with mean 1 in a $k \times k$ matrix has
inspired much of the further work. The conjecture was proved simultaneously
by two independent groups in March 2003. The most general of these
solutions (due to Linusson and W\"astlund)
uses algebraic-combinatorial tools, such as the M\"obius
function of a partially ordered set, at the heart of its discovery.


\section{Bioinformatics}
Modern biology is being reshaped in the post-genomic era, and its
new basic building blocks are combinatorial in nature, such as
genomes (with their finer structure of chromosomes, genes and
DNA-sequences), the genetic code, protein structures,
protein-protein interactions, and phylogenetic trees. It is by now
well-known that there are interesting applications of
combinatorial methods within the rapidly growing field of
bioinformatics/computational biology. Although some very good
combinatorialists have entered the field, as yet it has attracted
many more statisticians and computer scientists. For this reason
FPSAC '03 devoted a special session to combinatorial problems in
bioinformatics.

There are two main directions of research in bioinformatics that
are captured by the words {\em evolution} and {\em function}. In
evolutionary bioinformatics we already see combinatorialists
helping answer fundamental questions about the tree of life and
the mechanisms of evolution. At FPSAC '03 there were reports of
recent work where classical concepts in algebraic combinatorics
(such as edge labelled trees or the characters of the symmetric
group) turned out to be important tools for making progress on
questions in this area. Without doubt this development will
continue for the next decade, as the increasing amount of data
makes the biologists refine or revise their questions and
hypotheses.

On the other hand, the area of functional genomics still very much
awaits the breakthrough of combinatorial methods, e.g.~to predict
function (via 3D-structure) of proteins directly from their
sequences of amino acids, to understand the interactions of
proteins in the cell, and to make sense of amassing micro-array data.


\section{Quantum analogues}

An important recent development in algebraic geometry has been the
development of ``quantum analogues'' of the cohomology rings of flag
varieties. In particular, the quantum cohomology ring
QH$(\mathrm{Gr}_{kn}; \mathbb{C})$ of the Grassmann variety Gr$_{kn}$
is related to a remarkable generalization of Schubert calculus due to
Gromov and Witten which allows the computation of ``3-point
Gromov-Witten invariants.'' These may be regarded as a $q$-analogue of
the Littlewood-Richardson coefficients arising as intersection numbers
in the classical case. A fundamental breakthrough in this area was
achieved in 2002 by Alexander Postnikov. Ordinary
Littlewood-Richardson coefficients appear in the expansion of skew
Schur functions in terms of Schur functions. Skew Schur functions are
defined combinatorially in terms of Young diagrams, which are certain
planar arrays of boxes. Postnikov's basic insight was that 3-point
Gromov-Witten invariants have a similar description if one replaces
planar Young diagrams with cylindrical and toroidal analogues. Using
this idea Postnikov settled some open problems on 3-point
Gromov-Witten invariants and discovered new symmetry properties that
they satisfy. The subject of cylindrical and toroidal Schur functions
has tremendous potential for further development.


\section{Non-commutative and non-symmetric functions}
\subsection{Non-commutativity}
Non-commutative methods have transformed our understanding of such
classical fields as symmetric functions, determinants and minors, etc.
For example, the classical Littlewood-Richardson rule for multiplying
Schur functions becomes a straightforward consequence of the monoid
structure that one can put on the set of Young
tableaux. Non-commutative symmetric functions have a wider scope of
applications than the commutative functions, e.g.  they furnish new
idempotents in Lie algebras.  One can also define non-commutative
extensions of quasi symmetric functions, and Schubert polynomials.
Recent developments lead to similar algebras whose elements are other
combinatorial objects like trees (Loday-Ronco algebra).


\subsection{Non-symmetric functions}
Symmetric functions theory is a topic in mathematics dating from the
beginning of algebra. It has found applications in many 
diverse domains,
including, most recently, Fock spaces in theoretical physics. 

The theory of non-symmetric functions is much
less developed, from the
point of view of algebraic identities extending those satisfied by the
different families of symmetric polynomials.  The situation has
changed during the last thirty years, motivated by algebraic geometry
(flag manifolds, Schubert cycles, degeneracy loci) and representation
theory (Demazure characters).  One now has several families of
polynomials extending Schur functions --- Schubert polynomials,
Grothendieck polynomials, key polynomials --- which can be studied
combinatorially independently of their geometrical interpretation.
Properties of Schubert or Grothendieck polynomials are regularly
presented at the FPSAC conferences since their start, this time by
Yong at FPSAC 03.

The above-mentioned polynomials are also of much interest from the point 
of view of computer algebra.
This is particularly true for algebraic computations of
polynomials in several variables, where combinatorial manipulations of
permutations can replace calculus with variables without the usual
limitations on the number of variables. Schubert polynomials also
occur as universal coefficients in Newton-type interpolation formulas.

Judging from the wide 
applicability of symmetric functions, one may
predict with certainty that this field will develop in the near future
and find new domains of application.


\section{Face numbers  and algebraic shifting}

The study of face numbers for various classes of complexes continues
to be a rich source of problems for algebraic combinatorics.
Among the most interesting results of recent years is the proof
by Ed Swartz that $g$-vectors of matroid complexes satisfy 
the same conditions as those of the $g$-theorem of simplicial
polytopes. What is noteworthy is that his method does not depend on the
hard Lefschetz theorem of a variety; this may open the door to
further progress for other classes of complexes, 
notably triangulated spheres.

Other decisive progress has been made using the method of 
{\em algebraic shifting}.
The first version of algebraic shifting to be studied was exterior
algebraic shifting, so-called because it takes place in exterior algebra.
Its most prominent success was in Bj\"orner and Kalai's characterization of
pairs of $f$-vectors and (homology) Betti numbers of simplicial complexes.
More recent attention has been paid to symmetric algebraic shifting, which
takes place in commutative algebra.  Symmetric algebraic shifting has
advanced quickly, helped by existing results on generic initial ideals,
since the key step is the construction of a certain generic initial ideal.

Recently, Babson, Novik, and Thomas continued this trend by adapting
``iterated Betti numbers'' from exterior to symmetric algebraic
shifting.  These iterated Betti numbers are easy to compute for an
arbitrary complex $\Gamma$ from purely combinatorial information of
the algebraically shifted complex $\Delta(\Gamma)$.  Most
intriguingly, certain ``extremal'' iterated Betti numbers determine
extremal (free resolution) Betti numbers, which, in turn, determine a
great deal of algebraic information, and which had been previously
shown by Bayer, Charalambous, and Popescu to be preserved by symmetric
algebraic shifting.  Symmetric iterated Betti numbers and algebraic
shifting thus reduce questions of the characterizations of many
algebraic invariants to purely combinatorial questions about shifted
complexes.

The differences and similarities between exterior and symmetric algebraic
shifting remain unclear.  Resolving this confusion will continue to be
worthwhile for the near future.

Isabella Novik's talk at FPSAC on characterizing face numbers of manifolds
with symmetry featured a potentially very useful wrinkle on algebraic
shifting.  She replaced the completely generic matrix normally used in
algebraic shifting by a matrix with certain entries specified, and others
left generic as usual.  Perhaps other specializations of the generic
matrix will help algebraic shifting solve characterization problems for
other special classes of simplicial complexes.


\section{Cluster algebras and the Laurent phenomenon} 
The theory of canonical bases, originated by Lusztig, plays a key role
in the representation theory and algebraic geometry related to
semisimple Lie algebras.  A completely novel combinatorial approach to
the theory of canonical bases was found recently by Sergey Fomin and
Andrei Zelevinsky and is currently under intense investigation. This
approach is based on the concept of \emph{cluster algebras}, a class
of commutative rings with distinguished generating sets related in a
combinatorial fashion. Closely related to cluster algebras is the
subject of \emph{tropical geometry}, which roughly speaking allows
piecewise linear geometry to be replaced with subtraction-free
manipulation of rational functions.  As a bonus, Fomin and Zelevinsky
realized that their techniques are applicable to some purely
combinatorial problems that have been a source of much puzzlement
since the 1980's. These problems concern certain sequences, the first
due to Michael Somos, whose terms are \emph{a priori} rational
numbers, but in fact turn out to be integers.  More generally, certain
sequences have terms that are \emph{a priori} rational functions of
several variables with complicated denominators, but whose
denominators turn out to be monomials, an example of what Fomin and
Zelevinsky call the \emph{Laurent phenomenon}.


\section{Hopf Algebras}

Recent developments have linked distinct subjects within combinatorics,
algebra, geometry, and theoretical physics thereby uncovering exciting new
avenues for research. An old theme in algebraic combinatorics, initiated by
Rota, is that many combinatorial objects possess natural multiplication and
comultiplication structures. Enumeration and classification of these
structures often give rise to an associated graded Hopf algebra. Many of
these Hopf algebras of combinatorial objects (combinatorial Hopf algebras)
possess important structures. The FPSAC talk by
A. Lascoux reaffirmed
the importance of comultiplication in algebraic combinatorics. As another
example, Connes and Kreimer show that a Hopf algebra of rooted trees encodes
renormalization in quantum field theory. Loday and Ronco show a Hopf algebra
of planar binary trees has interesting operadic structure. Recent work of
Ehrenborg, Bergeron, Sottile, Aguiar, and many others, give a natural Hopf
morphism from any combinatorial Hopf algebras to the Hopf algebra of
quasi-symmetric functions Qsym. This morphism arises from a universal
property of the Hopf algebra Qsym as a terminal object in the category of
graded Hopf algebras equipped with a zeta-function. In particular, many
enumerative combinatorial invariants (among them flag $f$-vectors,
Littlewood-Richardson coefficients, and chromatic symmetric polynomials) are
obtained from this universal property. In light of this, the Hopf algebra
Qsym, its generalization and its sub-structures are bound to play a very
important role in algebraic combinatorics.

The connections uncovered between combinatorial Hopf algebras suggest a
number of exciting possibilities. This includes a unified approach to
positivity questions in different realms of enumerative combinatorics or
using these results as a bridge to transfer ideas and techniques between
these different realms. For example, both the Schubert calculus and
polytope/poset theory have important open positivity questions and
well-developed techniques to study these.

The work of Sagan and Rosas at this FPSAC suggested a new generalization of
Qsym. As well, Ehrenborg suggested a new construction (which can be
rephrased in Hopf algebraic language) to generate inequalities of $g$-vectors.
Hugh Thomas presented maps between important lattices. These induce
morphisms between related Hopf algebras. Finally, related to some of the
combinatorial Hopf algebras is the study of their invariants. This gives
some interesting space related to geometry. At least two presentations gave
us further generalizations to consider. First, the poster of Aval presented
quasi-invariants and super covariants. Second, the talk of Biagioli and
Caselli gave the Hilbert series of some invariant algebras.


\section{Smoothed analysis}
The theory of algorithms has been challenged by the existence of
remarkable algorithms, in particular the simplex method, that work
well in practice but whose theoretical analysis is negative or
inconclusive. The root of the problem is that algorithms have
traditionally been measured in two ways: worst-case behavior or
average behavior. In both situations, the analysis deals with
instances of the algorithms that may differ widely from what actually
arises in practice. A spectacular breakthrough in the analysis of
algorithms was achieved by Daniel Spielman and Shang-Hua Teng 
when they introduced a third
method of analyzing algorithms, known as \emph{smoothed analysis}. In
smoothed analysis, the performance of an algorithm is measured under
slight random perturbations of arbitrary inputs. A major result of
Spielman and Teng is that the simplex algorithm has polynomial smoothed
complexity. Numerous other fundamental algorithms have subsequently
been shown to also have polynomial smoothed complexity, and the area
of smoothed analysis has become one of the most active subjects in the
theory of algorithms.

\end{document}




