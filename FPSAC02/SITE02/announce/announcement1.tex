

\documentstyle{article}

%\pagestyle{empty}

\textheight=255mm
\textwidth=170mm

\footskip 1cm

\voffset -30mm
\hoffset -25mm

\def\vtr#1{\vrule width 0mm depth 0mm height #1mm}
\def\wtr#1#2{\vrule width 0mm depth #1mm height #2mm}
\def\euro{{$\subset\hspace{-11pt}\vbox{\hbox{\scriptsize =}\vspace{0.6pt}}
\,$}}
\def\colitem#1{\bigskip\noindent{\large\bf #1} \medskip}

\begin{document}

\null
\vskip -7mm
\hbox to \textwidth{\hrulefill}


\vskip 4mm
\centerline{\Large\bf $\mathbf{14^{th}}$ International Conference on}
\vskip 2mm
\centerline{\Large\bf Formal Power Series and Algebraic Combinatorics}
\vskip 7mm
\centerline{\Large\bf FPSAC 2002}
\vskip 3mm
\centerline{\large\bf July 8 -- 12, 2002}
\vskip 3mm
\centerline{\large\bf The University of Melbourne (Australia)}
\vskip 3.5mm

\vskip 3.5mm

\hbox to \textwidth{\hrulefill}
\vskip 3mm
\centerline{\large First announcement -- Call for papers}
\vskip 2mm
\hbox to \textwidth{\hrulefill}

\colitem{Topics}

\noindent
All aspects of combinatorics and their relationship with other
parts of ma\-the\-matics, computer science and physics.

\colitem{Conference program}

\noindent
Invited lectures, contributed presentations, poster session,
problem session and software demonstrations.

\colitem{Official languages}

\noindent
The official languages of the conference are English and
French.

\colitem{Invited Speakers}

\noindent
The list of invited speakers is not complete yet.
However, the following
scientists have already accepted to give an invited talk at
FPSAC 2002:
H{\' e}l{\` e}ne Barcelo (USA), Alan Sokal (USA),
Philippe Di Francesco (France),
Christian Krattenthaler (Austria), Thomas Prellberg (Germany),
Peter Forrester
(Australia), Brendan McKay (Australia), Jan de Gier (Australia), Ole
Warnaar (Australia).

\colitem{Call for papers and posters}

\noindent
Authors are invited to submit {\bf either} extended abstracts of at most twelve
pages, or full articles of at most twenty-five pages by {\em December~15, 2001}.

Note that this constitutes an extension of the system used at previous 
conferences. The idea is that those who wish to submit full articles will
only have to undergo one cycle of refereeing. Their papers will appear both
in the conference proceedings and in the subsequent published journal volume. If
extended abstracts are submitted, they will, as previously, be refereed
for the conference. Subsequently, authors will be invited to submit full
articles of the extended abstracts for journal publication.
These will then be refereed by the relevant journal.

The aim of the current proposal is to simplify matters for those ready
to submit a full article. Please indicate, on your submission, whether
it is an extended abstract or a full article.


\smallskip
\noindent
To submit your papers, preferably use the submission server of the
conference, which  is  available through the Internet at the URL 
{\verb!http://www.fpsac.ms.unimelb.edu.au/!}.
If you are unable to use
the web for submission, please email {\verb!fpsac-submission@ms.unimelb.edu.au!} for
further instructions.

\smallskip
\noindent
The submitted papers should begin with a summary written in English
and French (translations will be provided if necessary).  Authors should
indicate the mode of presentation which they consider appropriate for
their paper, i.e. lecture or poster session. Notification of
acceptance or rejection is scheduled for late February 2002.

%\pagebreak
\colitem{Open problem session}

\noindent
Contributions to the problem session are invited in advance of
the conference dates. People interested in submitting a problem
should submit it as described above, before {\it June 1, 2002}.

\pagebreak
\colitem{Software demonstrations}

\noindent
Demonstrations of software relevant to the topics of the conference
are encouraged. People interested in giving a software demonstration
should submit before {\it February 15, 2002} a paper including the
hardware  requirements, as described above.


\colitem{Program committee}

\medskip



\begin{tabular}{ll}
Susumu Ariki (Tokyo)\qquad & Gilbert Labelle (UQAM)\\
Sara Billey (MIT) \qquad & Jean-Christophe Novelli (Lille)\\
Maylis Delest (France) \qquad & Renzo Pinzani (Florence)\\
Art Duval (Texas-El Paso)\qquad & Andrew Rechnitzer (Toronto \&
Melbourne)\\
Omar Foda (Melbourne, {\it Co-Chair})\qquad &
Frank Sottile  (Mass.)\\
Sergey Fomin (Michigan)\qquad & Itaru Terada (Tokyo)\\
Vesselin Gasharov (Cornell) \qquad & Jean-Yves Thibon (Marne-la-Vall{\'
e}e)\\
Anthony Guttmann (Melbourne, {\it Co-Chair})\qquad & Dominic Welsh
(Oxford)\\
Angele Hamel (Waterloo)\qquad & Trevor Welsh (Melbourne)\\
Ron King (Southampton) \qquad & Nicholas Wormald (Melbourne)
\end{tabular}


\colitem{Participant support}

\noindent
We have applied for grants to provide
partial support of participants---in  particular of students and junior
scientists.  The success or otherwise of these grant applications
will be posted on our website as soon as they become available.
If the applications are successful, requests for such support
should contain a letter of recommendation and
include the estimated transportation and living expenses for the duration
of the  conference as well as the amount of any support available from
other  sources. All requests should be sent {\em in duplicate by January
15,  2002\/} to the following address:
\noindent
Prof. Anthony Guttmann, FPSAC 2002,\\
Department of Mathematics and Statistics, The University of Melbourne,\\
VIC. 3010, AUSTRALIA.\\

\colitem{Location}

\noindent
The conference will take place on the campus of the University of
Melbourne, located in Parkville, Melbourne, Australia.
The first talk is scheduled for {\it July 8, 2002} at 9:00 a.m.

\colitem{Accommodation}

\noindent
On-campus single accommodation with shared bathroom and laundry
facilities is available at \$55 AUD per night.  A choice of
standard hotel accommodation, most within easy walking distance
of the campus, can be found on the conference website.

\colitem{Further information}

\noindent
All important information concerning FPSAC 2002 can be found on the
conference web site available through the Internet at the http address
{\verb!http://www.fpsac.ms.unimelb.edu.au/!}.
More details will be given in future announcements. For any further
question, write to {\tt fpsac@ms.unimelb.edu.au}.

\colitem{Organising committee}

\noindent
Nantel Bergeron (York),
Richard Brak (Melbourne), Catherine Greenhill (Melbourne), Anthony
Guttmann (Melbourne, {\it Chair}), and Aleks Owczarek (Melbourne).

\colitem{Registration}

\noindent
Until {\it April~1, 2002}, the regular registration fee is \$440 AUD.
A reduced fee of \$220 AUD is offered for students.
These fees will respectively be \$660 AUD and \$330 AUD
in case of payment after April 1, 2002. (These prices include the recently
introduced Goods and Services Tax, so $\frac{1}{11}$ of this fee goes
to the Australian Government.)

\colitem{Currency}

\noindent
All prices are quoted in Australian dollars (AUD).  At the time
of this announcement, \$1 AUD is worth approximately 0.585 Euro,
\$0.535 US and 64 Yen.
 
\end{document}





