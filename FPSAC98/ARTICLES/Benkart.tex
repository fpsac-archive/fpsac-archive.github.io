%!PS-Adobe-3.0
%%BoundingBox: 54 72 558 720
%%Creator: Mozilla (NetScape) HTML->PS
%%DocumentData: Clean7Bit
%%Orientation: Portrait
%%Pages: 15
%%PageOrder: Ascend
%%Title: http://msfac1.math.yorku.ca/Web/FPSAC98/Benkart/Benkart.tex
%%EndComments
%%BeginProlog
[ /.notdef /.notdef /.notdef /.notdef /.notdef /.notdef /.notdef /.notdef /.notdef /.notdef
 /.notdef /.notdef /.notdef /.notdef /.notdef /.notdef /.notdef /.notdef /.notdef /.notdef
 /.notdef /.notdef /.notdef /.notdef /.notdef /.notdef /.notdef /.notdef /.notdef /.notdef
 /.notdef /.notdef /space /exclam /quotedbl /numbersign /dollar /percent /ampersand /quoteright
 /parenleft /parenright /asterisk /plus /comma /hyphen /period /slash /zero /one
 /two /three /four /five /six /seven /eight /nine /colon /semicolon
 /less /equal /greater /question /at /A /B /C /D /E
 /F /G /H /I /J /K /L /M /N /O
 /P /Q /R /S /T /U /V /W /X /Y
 /Z /bracketleft /backslash /bracketright /asciicircum /underscore /quoteleft /a /b /c
 /d /e /f /g /h /i /j /k /l /m
 /n /o /p /q /r /s /t /u /v /w
 /x /y /z /braceleft /bar /braceright /asciitilde /.notdef /.notdef /.notdef
 /.notdef /.notdef /.notdef /.notdef /.notdef /.notdef /.notdef /.notdef /.notdef /.notdef
 /.notdef /.notdef /.notdef /.notdef /.notdef /.notdef /.notdef /.notdef /.notdef /.notdef
 /.notdef /.notdef /.notdef /.notdef /.notdef /.notdef /.notdef /.notdef /.notdef /.notdef
 /space /exclamdown /cent /sterling /currency /yen /brokenbar /section /dieresis /copyright
 /ordfeminine /guillemotleft /logicalnot /hyphen /registered /macron /degree /plusminus /twosuperior /threesuperior
 /acute /mu /paragraph /periodcentered /cedilla /onesuperior /ordmasculine /guillemotright /onequarter /onehalf
 /threequarters /questiondown /Agrave /Aacute /Acircumflex /Atilde /Adieresis /Aring /AE /Ccedilla
 /Egrave /Eacute /Ecircumflex /Edieresis /Igrave /Iacute /Icircumflex /Idieresis /Eth /Ntilde
 /Ograve /Oacute /Ocircumflex /Otilde /Odieresis /multiply /Oslash /Ugrave /Uacute /Ucircumflex
 /Udieresis /Yacute /Thorn /germandbls /agrave /aacute /acircumflex /atilde /adieresis /aring
 /ae /ccedilla /egrave /eacute /ecircumflex /edieresis /igrave /iacute /icircumflex /idieresis
 /eth /ntilde /ograve /oacute /ocircumflex /otilde /odieresis /divide /oslash /ugrave
 /uacute /ucircumflex /udieresis /yacute /thorn /ydieresis] /isolatin1encoding exch def
/c { matrix currentmatrix currentpoint translate
     3 1 roll scale newpath 0 0 1 0 360 arc setmatrix } bind def
/F0
    /Times-Roman findfont
    dup length dict begin
	{1 index /FID ne {def} {pop pop} ifelse} forall
	/Encoding isolatin1encoding def
    currentdict end
definefont pop
/f0 { /F0 findfont exch scalefont setfont } bind def
/F1
    /Times-Bold findfont
    dup length dict begin
	{1 index /FID ne {def} {pop pop} ifelse} forall
	/Encoding isolatin1encoding def
    currentdict end
definefont pop
/f1 { /F1 findfont exch scalefont setfont } bind def
/F2
    /Times-Italic findfont
    dup length dict begin
	{1 index /FID ne {def} {pop pop} ifelse} forall
	/Encoding isolatin1encoding def
    currentdict end
definefont pop
/f2 { /F2 findfont exch scalefont setfont } bind def
/F3
    /Times-BoldItalic findfont
    dup length dict begin
	{1 index /FID ne {def} {pop pop} ifelse} forall
	/Encoding isolatin1encoding def
    currentdict end
definefont pop
/f3 { /F3 findfont exch scalefont setfont } bind def
/F4
    /Courier findfont
    dup length dict begin
	{1 index /FID ne {def} {pop pop} ifelse} forall
	/Encoding isolatin1encoding def
    currentdict end
definefont pop
/f4 { /F4 findfont exch scalefont setfont } bind def
/F5
    /Courier-Bold findfont
    dup length dict begin
	{1 index /FID ne {def} {pop pop} ifelse} forall
	/Encoding isolatin1encoding def
    currentdict end
definefont pop
/f5 { /F5 findfont exch scalefont setfont } bind def
/F6
    /Courier-Oblique findfont
    dup length dict begin
	{1 index /FID ne {def} {pop pop} ifelse} forall
	/Encoding isolatin1encoding def
    currentdict end
definefont pop
/f6 { /F6 findfont exch scalefont setfont } bind def
/F7
    /Courier-BoldOblique findfont
    dup length dict begin
	{1 index /FID ne {def} {pop pop} ifelse} forall
	/Encoding isolatin1encoding def
    currentdict end
definefont pop
/f7 { /F7 findfont exch scalefont setfont } bind def
/rhc {
    {
        currentfile read {
	    dup 97 ge
		{ 87 sub true exit }
		{ dup 48 ge { 48 sub true exit } { pop } ifelse }
	    ifelse
	} {
	    false
	    exit
	} ifelse
    } loop
} bind def

/cvgray { % xtra_char npix cvgray - (string npix long)
    dup string
    0
    {
	rhc { cvr 4.784 mul } { exit } ifelse
	rhc { cvr 9.392 mul } { exit } ifelse
	rhc { cvr 1.824 mul } { exit } ifelse
	add add cvi 3 copy put pop
	1 add
	dup 3 index ge { exit } if
    } loop
    pop
    3 -1 roll 0 ne { rhc { pop } if } if
    exch pop
} bind def

/smartimage12rgb { % w h b [matrix] smartimage12rgb -
    /colorimage where {
	pop
	{ currentfile rowdata readhexstring pop }
	false 3
	colorimage
    } {
	exch pop 8 exch
	3 index 12 mul 8 mod 0 ne { 1 } { 0 } ifelse
	4 index
	6 2 roll
	{ 2 copy cvgray }
	image
	pop pop
    } ifelse
} def
/cshow { dup stringwidth pop 2 div neg 0 rmoveto show } bind def
/rshow { dup stringwidth pop neg 0 rmoveto show } bind def
%%EndProlog
%%Page: 1 1
%%BeginPageSetup
/pagelevel save def
54 0 translate
%%EndPageSetup
newpath 0 72 moveto 504 0 rlineto 0 648 rlineto -504 0 rlineto  closepath clip newpath
0 711.9 moveto
10 f4
(%\\input amstex) show
0 701.4 moveto
10 f4
(\\documentstyle{gen-j}) show
0 690.9 moveto
10 f4
(\\NoBlackBoxes) show
0 680.4 moveto
10 f4
(\\TagsOnLeft) show
0 669.9 moveto
10 f4
(%\\magnification = \\magstep1 ) show
0 659.4 moveto
10 f4
(\\overfullrule0pt) show
0 648.9 moveto
10 f4
(%\\hsize=35 true pc) show
0 638.4 moveto
10 f4
(%\\vsize=54 true pc) show
0 627.9 moveto
10 f4
(%\\voffset=0.1 true in) show
0 617.4 moveto
10 f4
(%\\hoffset=1.0 true cm  ) show
0 606.9 moveto
10 f4
(\\def \\n {\\noindent}) show
0 596.4 moveto
10 f4
(\\def \\s {\\smallskip}) show
0 585.9 moveto
10 f4
(\\def \\m {\\medskip}) show
0 575.4 moveto
10 f4
(\\def \\b {\\bigskip}) show
0 564.9 moveto
10 f4
(\\def \\a {\\alpha}) show
0 554.4 moveto
10 f4
(\\def \\ad {{\\text {ad}}} ) show
0 543.9 moveto
10 f4
(\\def \\be {\\beta} ) show
0 533.4 moveto
10 f4
(\\def \\d {\\Delta}) show
0 522.9 moveto
10 f4
(\\def \\del {\\delta} ) show
0 512.4 moveto
10 f4
(\\def \\e {\\epsilon} ) show
0 501.9 moveto
10 f4
(\\def \\End {{\\text {End}}} ) show
0 491.4 moveto
10 f4
(\\def\\fsl{\\frak s\\frak l_2} ) show
0 480.9 moveto
10 f4
(\\def\\fst{\\frak s\\frak l_3}) show
0 470.4 moveto
10 f4
(\\def\\fss{\\frak s\\frak l\(1,1\)}) show
0 459.9 moveto
10 f4
(\\def \\fb {{\\frak b}}) show
0 449.4 moveto
10 f4
(\\def \\ga {\\gamma}  ) show
0 438.9 moveto
10 f4
(\\def \\g {{\\frak g}} ) show
0 428.4 moveto
10 f4
(\\def \\h {{\\frak h}}) show
0 417.9 moveto
10 f4
(\\def \\k {\\kappa} ) show
0 407.4 moveto
10 f4
(\\def \\Hom {{\\text {Hom}}}) show
0 396.9 moveto
10 f4
(\\def \\l {\\lambda}) show
0 386.4 moveto
10 f4
(\\def \\la {\\langle}) show
0 375.9 moveto
10 f4
(\\def \\o {\\overline}) show
0 365.4 moveto
10 f4
(\\def \\oc {{\\Cal O}}) show
0 354.9 moveto
10 f4
(\\def \\osp{\\frak o\\frak s\\frak p} ) show
0 344.4 moveto
10 f4
(\\def \\ot {\\otimes}) show
0 333.9 moveto
10 f4
(\\def \\p {\\frak p}) show
0 323.4 moveto
10 f4
(\\def \\ra {\\rangle} ) show
0 312.9 moveto
10 f4
(\\def \\sgn {{\\text {sgn}}}) show
0 302.4 moveto
10 f4
(\\def \\t {\\theta}) show
0 291.9 moveto
10 f4
(\\def \\tr {{\\text {tr}}}) show
0 281.4 moveto
10 f4
(\\def \\u {\\underline}) show
0 270.9 moveto
10 f4
(\\def \\w {\\omega} ) show
0 260.4 moveto
10 f4
(\\def \\z {\\zeta}) show
0 249.9 moveto
10 f4
(\\def \\C {{\\text {\\bf C}}}) show
0 239.4 moveto
10 f4
(\\def \\K {{\\text {\\bf K}}} ) show
0 228.9 moveto
10 f4
(\\def \\Z {{\\text {\\bf Z}}} ) show
0 207.9 moveto
10 f4
(\\topmatter) show
0 197.4 moveto
10 f4
(\\title Differential Posets and Down-up Algebras ) show
0 186.9 moveto
10 f4
(\\endtitle) show
0 176.4 moveto
10 f4
(\\author Georgia Benkart \\endauthor) show
0 165.9 moveto
10 f4
(\\thanks  The author gratefully acknowledges ) show
0 155.4 moveto
10 f4
(support from  National Science ) show
0 144.9 moveto
10 f4
(Foundation Grant \\#{}DMS--9622447.  \\endthanks  ) show
0 134.4 moveto
10 f4
(\\subjclass ) show
0 123.9 moveto
10 f4
(Primary 16S15, 16S30, 17B35, 17B10  ) show
0 113.4 moveto
10 f4
(Secondary  81R50, 06A07 \\endsubjclass  ) show
0 92.4 moveto
10 f4
(\\abstract) show
0 81.9 moveto
10 f4
(Down-up algebras originated in the study of differential posets. ) show
pagelevel restore
showpage
%%Page: 2 2
%%BeginPageSetup
/pagelevel save def
54 0 translate
%%EndPageSetup
newpath 0 72 moveto 504 0 rlineto 0 648 rlineto -504 0 rlineto  closepath clip newpath
0 711.9 moveto
10 f4
(In this paper we discuss their combinatorial origins,  ) show
0 701.4 moveto
10 f4
(representations, and structure.  Down-up algebras) show
0 690.9 moveto
10 f4
(exhibit many of the important features of the universal enveloping) show
0 680.4 moveto
10 f4
(algebra  $U\(\\fsl\)$ of the Lie algebra $\\fsl$ including a) show
0 669.9 moveto
10 f4
(Poincar\\'e-Birkhoff-Witt type basis and a well-behaved representation theory.) show
0 659.4 moveto
10 f4
(They have many interesting connections with Weyl algebras, quantum groups, ) show
0 648.9 moveto
10 f4
(and Witten's ) show
0 638.4 moveto
10 f4
(deformations of $U\(\\fsl\)$. ) show
0 627.9 moveto
10 f4
(\\endabstract) show
0 606.9 moveto
10 f4
(\\endtopmatter) show
0 585.9 moveto
10 f4
(\\document) show
0 564.9 moveto
10 f4
(\\subhead {Differential posets} \\endsubhead ) show
0 554.4 moveto
10 f4
(\\m) show
0 543.9 moveto
10 f4
(In [St2], Stanley introduced a class of partially ordered sets,) show
0 533.4 moveto
10 f4
(which he termed {\\it differential posets}.  Many of the) show
0 522.9 moveto
10 f4
(remarkable enumerative and combinatorial properties of these posets ) show
0 512.4 moveto
10 f4
(involve counting saturated chains $y_1 \\prec  y_2 \\prec  \\dots) show
0 501.9 moveto
10 f4
(\\prec y_k$  or Hasse walks $y_1, y_2, \\dots, y_k$, \(where either $y_{i+1}$) show
0 491.4 moveto
10 f4
(covers $y_i$ or $y_i$ covers $y_{i+1}$\).  ) show
0 480.9 moveto
10 f4
(Essential in the computations are two operators,) show
0 470.4 moveto
10 f4
($d$ \(down\) and  $u$ \(up\), which are defined on the complex vector space ) show
0 459.9 moveto
10 f4
($\\C P$ having basis the elements of the poset $P$.  ) show
0 449.4 moveto
10 f4
(If $y \\in P$, then $d\(y\)$ is the sum of all elements that $y$ covers ) show
0 438.9 moveto
10 f4
(and $u\(y\)$ is the) show
0 428.4 moveto
10 f4
(sum of all elements that cover) show
0 417.9 moveto
10 f4
($y$.  For many posets the down and up operators give well-defined) show
0 407.4 moveto
10 f4
(linear transformations of $\\C P$.  Precursors) show
0 396.9 moveto
10 f4
(of the operators $d$ and $u$ appeared in [St1] and [P], where) show
0 386.4 moveto
10 f4
(they were used to show posets are Sperner or rank unimodal. ) show
0 375.9 moveto
10 f4
(\\m) show
0 365.4 moveto
10 f4
(The characterizing property of an $r$-differential poset is) show
0 354.9 moveto
10 f4
(that the down and up) show
0 344.4 moveto
10 f4
(operators satisfy) show
0 333.9 moveto
10 f4
($du - u d = r I$  for some positive integer $r$ \(see [St2, Thm. 2.2]\),) show
0 323.4 moveto
10 f4
(where $I$ is the identity transformation on $\\C P$. ) show
0 312.9 moveto
10 f4
(Thus, the poset affords a representation of the  Weyl algebra, \(the associative) show
0 302.4 moveto
10 f4
(algebra with generators) show
0 291.9 moveto
10 f4
($y,x$ subject to the relation $y x - x y = 1$\),  via the mapping $y \\mapsto d/r$,) show
0 281.4 moveto
10 f4
(and $x \\mapsto u$.  Since the Weyl algebra also can be realized) show
0 270.9 moveto
10 f4
(as differential operators $y \\mapsto d/dx$ and $x \\mapsto x$ \(multiplication) show
0 260.4 moveto
10 f4
(by $x$\)  on $\\C[x]$, Stanley referred to the ) show
0 249.9 moveto
10 f4
(posets satisfying $du-ud = rI$) show
0 239.4 moveto
10 f4
(as $r$-{\\it differential}) show
0 228.9 moveto
10 f4
(or simply {\\it differential} when $r = 1$.  Fomin [F] studied) show
0 218.4 moveto
10 f4
(essentially the same class of posets for $r = 1$, calling them) show
0 207.9 moveto
10 f4
(``$Y$-graphs''.  This terminology comes from the fact that Young's) show
0 197.4 moveto
10 f4
(lattice $Y$ of all partitions of all nonnegative integers ) show
0 186.9 moveto
10 f4
(is the prototypical example.    ) show
0 176.4 moveto
10 f4
(\\m) show
0 155.4 moveto
10 f4
(A partition $\\mu$ of a nonnegative integer $m$ can be) show
0 144.9 moveto
10 f4
(regarded as a descending sequence ) show
0 134.4 moveto
10 f4
($\\mu = \(\\mu_1 \\geq \\mu_2 \\geq \\dots\)$ of parts whose sum $|\\mu| = \\sum_i) show
0 123.9 moveto
10 f4
(\\mu_i$ equals $m$.  If  $\\nu = \(\\nu_1 \\geq \\nu_2 \\geq) show
0 113.4 moveto
10 f4
(\\dots\)$ is a second partition, then) show
0 102.9 moveto
10 f4
($\\mu \\leq \\nu$ when $\\mu_i \\leq \\nu_i$ for all $i$.   The partition) show
0 92.4 moveto
10 f4
($\\nu$ covers $\\mu$ \(written here as $\\mu \\prec \\nu$\) if $\\mu < \\nu$) show
0 81.9 moveto
10 f4
(and $|\\nu| = 1 + |\\mu|$.  Thus, $\\mu \\prec \\nu$ if the partition $\\mu$ is) show
pagelevel restore
showpage
%%Page: 3 3
%%BeginPageSetup
/pagelevel save def
54 0 translate
%%EndPageSetup
newpath 0 72 moveto 504 0 rlineto 0 648 rlineto -504 0 rlineto  closepath clip newpath
0 711.9 moveto
10 f4
(obtained from $\\nu$ by subtracting 1 from exactly one of the parts of $\\nu$,) show
0 701.4 moveto
10 f4
(and $d\(\\nu\)$ is the sum of all such $\\mu$. Analogously,) show
0 690.9 moveto
10 f4
($u\(\\nu\)$ is the sum of all partitions $\\pi$ obtained from $\\nu$ by) show
0 680.4 moveto
10 f4
(adding 1 to one part of $\\nu$.  Young's lattice $Y$ is) show
0 669.9 moveto
10 f4
(a 1-differential poset, and $Y^r$ is $r$-differential ) show
0 659.4 moveto
10 f4
(\([St2, Cor. 1.4]\). ) show
0 648.9 moveto
10 f4
(\\m) show
0 638.4 moveto
10 f4
(The down and up operators on Young's lattice  have a representation) show
0 627.9 moveto
10 f4
(theoretic significance.  The simple modules of the symmetric group) show
0 617.4 moveto
10 f4
($S_n$ are indexed by the partitions $\\nu$ of $n$.  Upon restriction to) show
0 606.9 moveto
10 f4
($S_{n-1}$, the representation labelled by $\\nu$) show
0 596.4 moveto
10 f4
(decomposes into a direct sum of simple $S_{n-1}$-modules) show
0 585.9 moveto
10 f4
(indexed by the) show
0 575.4 moveto
10 f4
(partitions) show
0 564.9 moveto
10 f4
($\\mu \\prec \\nu$, so it is given by $d\(\\nu\)$.  When the) show
0 554.4 moveto
10 f4
(simple module labelled by $\\nu$ is induced to a representation) show
0 543.9 moveto
10 f4
(of $S_{n+1}$, it decomposes into a sum of simple $S_{n+1}$-modules indexed by) show
0 533.4 moveto
10 f4
(the partitions $\\pi$ of $n+1$ such that $\\nu \\prec \\pi$,  which is just) show
0 522.9 moveto
10 f4
($u\(\\nu\)$.    ) show
0 512.4 moveto
10 f4
(\\m ) show
0 501.9 moveto
10 f4
(In his study [T1] of) show
0 491.4 moveto
10 f4
(uniform posets, Terwilliger considered finite ranked posets $P$) show
0 480.9 moveto
10 f4
(whose down and up operators satisfy the) show
0 470.4 moveto
10 f4
(following relation ) show
0 449.4 moveto
10 f4
($$ d_{i}d_{i+1}u_{i}=\\alpha _{i}d_{i}u_{i-1}d_{i}+\\beta) show
0 438.9 moveto
10 f4
(_{i}u_{i-2}d_{i-1}d_{i} +\\gamma _{i}d_{i}, $$) show
0 417.9 moveto
10 f4
(\\n where $d_{i}$ and $u_{i}$ denote the restriction of $d$ and $u$ to the) show
0 407.4 moveto
10 f4
(elements of rank $i$. \(There is an analogous second relation,) show
0 386.4 moveto
10 f4
($$ d_{i+1}u_{i}u_{i-1}=\\alpha _{i}u_{i-1}d_{i}u_{i-1}+\\beta) show
0 375.9 moveto
10 f4
(_{i}u_{i-1}u_{i-2}d_{i-1} +\\gamma _{i}u_{i-1}, $$) show
0 365.4 moveto
10 f4
( ) show
0 354.9 moveto
10 f4
(\\n which holds automatically in this case because $d_{i+1}$ and $u_i$ are) show
0 344.4 moveto
10 f4
(adjoint operators relative to a certain bilinear form.\)) show
0 333.9 moveto
10 f4
(In many examples the constants in these relations do not) show
0 323.4 moveto
10 f4
(depend on the rank $i$.   In particular, a poset whose down and up) show
0 312.9 moveto
10 f4
(operators satisfy) show
0 291.9 moveto
10 f4
($$) show
0 281.4 moveto
10 f4
(\\aligned) show
0 270.9 moveto
10 f4
(d^{2}u & = q\(q+1\)dud -q^{3}ud^{2}+rd \\\\) show
0 260.4 moveto
10 f4
(du^2 & = q\(q+1\)udu - q^3 u^2 d + r u \\\\) show
0 249.9 moveto
10 f4
(\\endaligned $$) show
0 228.9 moveto
10 f4
(\\n where $q$ and $r$ are fixed complex numbers is ) show
0 218.4 moveto
10 f4
(said to be  {\\it $\(q,r\)$-differential}.   Many interesting) show
0 207.9 moveto
10 f4
(examples of $\(q,r\)$-differential posets in [T1] arise) show
0 197.4 moveto
10 f4
(from considering certain subspaces of a vector space over the) show
0 186.9 moveto
10 f4
(field $GF\(q\)$ of $q$ elements:) show
0 176.4 moveto
10 f4
(\\m) show
0 165.9 moveto
10 f4
(\\item{\(1\)} Assume $W$ is an $n$-dimensional vector space over) show
0 155.4 moveto
10 f4
($GF\(q\)$ and consider the set of pairs $P = \\{\(U,f\) \\mid U$ is) show
0 144.9 moveto
10 f4
(a subspace of $W$ and $f$ is an alternating bilinear form on $U\\}$) show
0 134.4 moveto
10 f4
(with the ordering:  $\(U,f\) \\leq \(V,g\)$  if $U$ is a subspace of $V$) show
0 123.9 moveto
10 f4
(and $g|_U = f$.  Then $P$ is a $\(q,r\)$-differential poset) show
0 113.4 moveto
10 f4
(with $r = -q^n\(q+1\)$. ) show
0 102.9 moveto
10 f4
(\\m) show
0 92.4 moveto
10 f4
(\\item{\(2\)} In example \(1\),  replace ``an alternating bilinear form'' with) show
0 81.9 moveto
10 f4
(``a quadratic form''.  The resulting poset $P$ is $\(q,) show
pagelevel restore
showpage
%%Page: 4 4
%%BeginPageSetup
/pagelevel save def
54 0 translate
%%EndPageSetup
newpath 0 72 moveto 504 0 rlineto 0 648 rlineto -504 0 rlineto  closepath clip newpath
0 711.9 moveto
10 f4
(-q^{n+1}\(q+1\)\)$-differential.) show
0 701.4 moveto
10 f4
(\\m) show
0 690.9 moveto
10 f4
(\\item{\(3\)} In this example assume $W$ is an $n$-dimensional space over) show
0 680.4 moveto
10 f4
($GF\(q^2\)$ and the bilinear forms are Hermitian. The poset) show
0 669.9 moveto
10 f4
($P$ is $\(q^{2},-q^{2n+1}\(q^{2}+1\)\)$-differential in this case.) show
0 648.9 moveto
10 f4
(\\b ) show
0 638.4 moveto
10 f4
(\\subhead {Down-up algebras} \\endsubhead ) show
0 627.9 moveto
10 f4
(\\m) show
0 617.4 moveto
10 f4
( ) show
0 606.9 moveto
10 f4
(To better understand the algebra generated by the down and up operators of a) show
0 596.4 moveto
10 f4
(poset and its action on the poset,  we introduced the notion of) show
0 585.9 moveto
10 f4
(a down-up algebra in our joint work with Roby \(see [BR]\).  Although the initial) show
0 575.4 moveto
10 f4
(motivation for our investigations came from posets, we made no assumptions) show
0 564.9 moveto
10 f4
(about the existence of posets whose down and up operators satisfy our) show
0 554.4 moveto
10 f4
(relations.  However, when such a poset exists, it affords a representation) show
0 543.9 moveto
10 f4
(of the down-up algebra, so our primary focus in [BR]  was on determining) show
0 533.4 moveto
10 f4
(explicit information about the) show
0 522.9 moveto
10 f4
(representations of down-up algebras.    ) show
0 512.4 moveto
10 f4
(\\m ) show
0 501.9 moveto
10 f4
(\\proclaim {Definition 1} Let $\\a,\\be,\\ga$ be fixed but arbitrary complex numbers. ) show
0 491.4 moveto
10 f4
(The unital associative algebra $A\(\\a,\\be,\\ga\)$ over $\\C$ with generators) show
0 480.9 moveto
10 f4
($d,u$ and defining relations) show
0 470.4 moveto
10 f4
(\\s) show
0 459.9 moveto
10 f4
(\\item {}{\(R1\)} $d^2u = \\a dud + \\be ud^2 + \\ga d,$) show
0 449.4 moveto
10 f4
(\\s ) show
0 438.9 moveto
10 f4
(\\item {}{\(R2\)} $du^2 = \\a udu + \\be u^2 d + \\ga u,$ ) show
0 428.4 moveto
10 f4
(\\m) show
0 417.9 moveto
10 f4
(\\n is a down-up algebra. \\endproclaim   ) show
0 407.4 moveto
10 f4
(\\m) show
0 396.9 moveto
10 f4
(It is easy to see that when $\\ga \\neq 0$ the down-up algebra $A\(\\a,\\be,\\ga\)$) show
0 386.4 moveto
10 f4
(is isomorphic to $A\(\\a,\\be,1\)$ by the map, $d \\mapsto d'$, $u \\mapsto \\ga u'$.) show
0 375.9 moveto
10 f4
(Therefore, it would suffice to treat just two cases $\\ga = 0,1$, but to) show
0 365.4 moveto
10 f4
(avoid dividing considerations into these two cases, we retain) show
0 354.9 moveto
10 f4
(the notation $\\ga$. ) show
0 344.4 moveto
10 f4
(\\b) show
0 333.9 moveto
10 f4
(\\subhead Examples of down-up algebras  \\endsubhead) show
0 323.4 moveto
10 f4
(\\m) show
0 312.9 moveto
10 f4
(\\n {\\bf Example \(i\)}. \\   If $B$ is the associative algebra generated by the down) show
0 302.4 moveto
10 f4
(and up operators $d,u$  of) show
0 291.9 moveto
10 f4
(a $\(q,r\)$-differential poset,  ) show
0 281.4 moveto
10 f4
(then relations \(R1\) and \(R2\) hold with ) show
0 270.9 moveto
10 f4
($\\a = q\(q+1\)$, $\\be = -q^{3}$, and $\\ga =  r$.  Thus, $B$ is a homomorphic image) show
0 260.4 moveto
10 f4
(of the algebra $A\(\\a,\\be,\\ga\)$ with these parameters, and the action of) show
0 249.9 moveto
10 f4
($B$ on the poset gives a representation of $A\(\\a,\\be,\\ga\)$. ) show
0 239.4 moveto
10 f4
(\\m ) show
0 218.4 moveto
10 f4
(\\n {\\bf Example \(ii\)}. \\  The relation $du - ud = rI$ of an $r$-differential poset) show
0 207.9 moveto
10 f4
(can be multiplied on the left by $d$ and on the right by $d$ and) show
0 197.4 moveto
10 f4
(the resulting equations can be added to get the relation) show
0 186.9 moveto
10 f4
($d^2 u - ud^2 = 2rd$ of a $\(-1,2r\)$-differential poset.) show
0 176.4 moveto
10 f4
(Thus, the Weyl algebra is a homomorphic image \(by the ideal generated) show
0 165.9 moveto
10 f4
(by $du-ud - r1$\)  of the algebra $A\(0,1,2r\)$.  ) show
0 155.4 moveto
10 f4
(Similarly, the $q$-Weyl algebra is a homomorphic image of the algebra ) show
0 144.9 moveto
10 f4
($A\(0,q^2,\(q+1\)\)$ by the ideal generated by $du - qud - 1$.  ) show
0 134.4 moveto
10 f4
(The skew polynomial) show
0 123.9 moveto
10 f4
(algebra $\\C_q[d,u]$, or quantum plane \(see [M]\), is the associative algebra with) show
0 113.4 moveto
10 f4
(generators $d,u$ which satisfy the relation  $du = qud$.  Therefore,) show
0 102.9 moveto
10 f4
($\\C_q[d,u]$ is a homomorphic image \(by the ideal generated by $du-qud$\)) show
0 92.4 moveto
10 f4
(of the algebra $A\(2q,-q^2,0\)$. ) show
0 81.9 moveto
10 f4
(\\m) show
pagelevel restore
showpage
%%Page: 5 5
%%BeginPageSetup
/pagelevel save def
54 0 translate
%%EndPageSetup
newpath 0 72 moveto 504 0 rlineto 0 648 rlineto -504 0 rlineto  closepath clip newpath
0 711.9 moveto
10 f4
(\\n) show
0 701.4 moveto
10 f4
({\\bf Example \(iii\)}. \\   Consider the poset ${\\Cal L}\(2,2\) = \\{ {\\u a} = \(a_1,a_2\)) show
0 690.9 moveto
10 f4
(\\mid  2 \\geq a_1 \\geq a_2$ and $a_1,a_2 \\in \\Z_{\\geq 0}\\}$ with the order relation) show
0 680.4 moveto
10 f4
($\\u a \\leq \\u b$ if $a_i \\leq b_i$ for $i = 1,2$.  This is just) show
0 669.9 moveto
10 f4
(the set of partitions which fit into a $2 \\times 2$ box.  By direct calculation) show
0 659.4 moveto
10 f4
(it is easy to verify that the down and up operators on this poset satisfy) show
0 648.9 moveto
10 f4
($d^2u = dud - ud^2 +d$ and  $du^2 = udu - u^2 d + u$, so the algebra they  ) show
0 638.4 moveto
10 f4
(generate is a homomorphic image of the down-up algebra $A\(1,-1,1\)$.  ) show
0 627.9 moveto
10 f4
(\\m ) show
0 617.4 moveto
10 f4
(\\n {\\bf Example \(iv\)}. \\   Suppose $\\g$ is a 3-dimensional Lie algebra over $\\C$) show
0 606.9 moveto
10 f4
(with basis) show
0 596.4 moveto
10 f4
($x,y,[x,y]$  such that $[x[x,y]] = \\ga x$ and  $[[x,y],y] = \\ga y$.  ) show
0 585.9 moveto
10 f4
(In the universal enveloping) show
0 575.4 moveto
10 f4
(algebra $U\(\\g\)$ of $\\g$ where $[x,y] = xy-yx$, these relations become) show
0 554.4 moveto
10 f4
($$\\aligned) show
0 543.9 moveto
10 f4
(& x^2 y -2 xyx + yx^2 = \\ga x \\\\) show
0 533.4 moveto
10 f4
(& xy^2 -2yxy +  y^2x = \\ga y. \\\\) show
0 522.9 moveto
10 f4
(\\endaligned$$) show
0 501.9 moveto
10 f4
(\\n Thus, $U\(\\g\)$ is a homomorphic image of the down-up algebra $A\(2,-1,\\ga\)$) show
0 491.4 moveto
10 f4
(via the mapping $\\phi: A\(2,-1,\\ga\) \\rightarrow U\(\\g\)$ with $\\phi: d \\mapsto x$,) show
0 480.9 moveto
10 f4
($\\phi: u) show
0 470.4 moveto
10 f4
(\\mapsto y$.  The) show
0 459.9 moveto
10 f4
(mapping $\\psi: \\g \\rightarrow A\(2,-1,\\ga\)$ with $\\psi: x \\mapsto d$, $) show
0 449.4 moveto
10 f4
(\\psi: y \\mapsto u$,) show
0 438.9 moveto
10 f4
(and $\\psi: [x,y] \\mapsto du - ud$ extends, by the universal property of $U\(\\g\)$,) show
0 428.4 moveto
10 f4
(to an algebra homomorphism $\\psi: U\(\\g\) \\rightarrow A\(2,-1,\\ga\)$ which) show
0 417.9 moveto
10 f4
(is the inverse of $\\phi$. ) show
0 407.4 moveto
10 f4
(Consequently, $U\(\\g\)$ is isomorphic to $A\(2,-1,\\ga\)$.  ) show
0 396.9 moveto
10 f4
(\\m ) show
0 375.9 moveto
10 f4
(The Lie algebra) show
0 365.4 moveto
10 f4
($\\fsl$ of $2 \\times 2$ complex matrices of trace zero has a standard) show
0 354.9 moveto
10 f4
(basis $e = E_{1,2}, f = E_{2,1},$ and $h = E_{1,1}-E_{2,2}$ of) show
0 344.4 moveto
10 f4
(matrix units, which satisfies $[e,f] = h, \\; [h,e] = 2e$, and $[h,f] = -2 f$. From) show
0 333.9 moveto
10 f4
(this we see that $U\(\\fsl\) \\cong A\(2,-1,-2\)$.  The Heisenberg Lie algebra $\\frak H$ has) show
0 323.4 moveto
10 f4
(a basis $x,y,z$ where $[x,y] = z$, and $[z,{\\frak H}] = 0$,) show
0 312.9 moveto
10 f4
(so $U\({\\frak H}\) \\cong) show
0 302.4 moveto
10 f4
(A\(2,-1,0\)$.  ) show
0 291.9 moveto
10 f4
(\\m) show
0 281.4 moveto
10 f4
(\\n {\\bf Example \(v\)}. \\   The $2 \\times 2$ complex matrices  ) show
0 270.9 moveto
10 f4
($y = \\left \(\\matrix  y_1 & y_2 \\\\) show
0 260.4 moveto
10 f4
(                     y_3 & y_4  \\endmatrix \\right \)$) show
0 249.9 moveto
10 f4
(with supertrace $y_1-y_4 = 0$  is the special linear) show
0 239.4 moveto
10 f4
(Lie superalgebra  $L = \\fss = L_{\\o 0} \\oplus L_{\\o 1}$ ) show
0 228.9 moveto
10 f4
(under the supercommutator $[x,y] =) show
0 218.4 moveto
10 f4
(xy-\(-1\)^{ab}yx$ for $x\\in L_{\\o a},\\;y\\in L_{\\o b}$.   It has a ) show
0 207.9 moveto
10 f4
(presentation by generators $e,f$ \(which belong to $L_{\\o 1}$ and can be identified) show
0 197.4 moveto
10 f4
(with the matrix units $e = E_{1,2}$, $f = E_{2,1}$\) and relations) show
0 186.9 moveto
10 f4
($[e,[e,f]] = 0$, $[[e,f],f] = 0$, $[e,e] = 0$, $[f,f] = 0$.) show
0 176.4 moveto
10 f4
(The universal enveloping algebra $U\(\\fss\)$ of $\\fss$  has generators $e,f$ and) show
0 165.9 moveto
10 f4
(relations $e^2 f - f e^2 = 0$, $e f^2 - f^2 e = 0$, $e^2 = 0$, $f^2 = 0$.) show
0 155.4 moveto
10 f4
(Thus, $U\(\\fss\)$ is a homomorphic image of the down-up algebra) show
0 144.9 moveto
10 f4
($A\(0,1,0\)$ by the ideal generated by the elements $e^2$ and $f^2$, which are) show
0 134.4 moveto
10 f4
(central in $A\(0,1,0\)$.   ) show
0 123.9 moveto
10 f4
(\\m) show
0 113.4 moveto
10 f4
(\\n {\\bf Example \(vi\)}. \\   The orthosymplectic Lie superalgebra $\\osp\(1,2\) = L_{\\o 0}) show
0 102.9 moveto
10 f4
(\\oplus L_{\\o 1}$ has generators $x,y \\in L_{\\o 1}$ which satisfy) show
0 81.9 moveto
10 f4
($$xy + yx = t \\in L_{\\o 0}  \\quad \\quad tx - xt = x \\quad \\quad yt -ty = y. $$) show
pagelevel restore
showpage
%%Page: 6 6
%%BeginPageSetup
/pagelevel save def
54 0 translate
%%EndPageSetup
newpath 0 72 moveto 504 0 rlineto 0 648 rlineto -504 0 rlineto  closepath clip newpath
0 708.9 moveto
10 f4
(\\n By combining these relations, we see that its universal enveloping algebra) show
0 698.4 moveto
10 f4
(\\break  $U\(\\osp\(1,2\)\)$ is a homomorphic image of $A\(0,1,1\)$.  ) show
0 687.9 moveto
10 f4
(\\m) show
0 677.4 moveto
10 f4
(\\n {\\bf Example \(vii\)}. \\   Consider the field $\\C\(q\)$ of rational functions in the) show
0 666.9 moveto
10 f4
(indeterminate $q$ over the complex numbers, and let) show
0 656.4 moveto
10 f4
($U_q\(\\g\)$ be the quantized enveloping algebra \(quantum group\) of) show
0 645.9 moveto
10 f4
(a finite-dimensional simple complex Lie algebra $\\g$  corresponding) show
0 635.4 moveto
10 f4
(to the Cartan matrix $\\frak A = \(a_{i,j}\)_{i,j= 1}) show
0 624.9 moveto
10 f4
(^n$. There are relatively prime integers $\\ell_i$ so that) show
0 614.4 moveto
10 f4
(the matrix $\(\\ell_i a_{i,j}\)$ is symmetric.  Let ) show
0 593.4 moveto
10 f4
($$q_i = q^{\\ell_i},\\quad \\quad \\text {and} \\quad \\quad  ) show
0 582.9 moveto
10 f4
([m]_i = \\frac {q_i^{m} - q_i^{-m}} {q_i -) show
0 572.4 moveto
10 f4
(q_i^{-1}}$$) show
0 551.4 moveto
10 f4
(\\n for all $m \\in \\Z_{\\geq 0}$.  When $m \\geq 1$,  let ) show
0 540.9 moveto
10 f4
($[m]_i! = \\prod_{j = 1}^m [j]_i.$  Set $[0]_i! = 1$ and define) show
0 519.9 moveto
10 f4
($$\\left [\\matrix m \\\\ n \\\\ \\endmatrix) show
0 509.4 moveto
10 f4
(\\right]_i = \\frac{[m]_i!}) show
0 498.9 moveto
10 f4
({[n]_i! [m-n]_i!}.$$ ) show
0 477.9 moveto
10 f4
(\\n Then $U = U_q\(\\g\)$ is the unital associative algebra over $\\C\(q\)$ with generators ) show
0 467.4 moveto
10 f4
($E_i, F_i, K_{i}, K_{i}^{-1}$ \($i = 1, \\dots, n$\) subject to the relations) show
0 456.9 moveto
10 f4
(\\b) show
0 446.4 moveto
10 f4
(\\item {}{\(Q1\)} $K_i K_i^{-1} = K_i^{-1} K_i,$) show
0 435.9 moveto
10 f4
(\\quad \\quad $K_i K_j = K_jK_i$) show
0 425.4 moveto
10 f4
(\\m) show
0 414.9 moveto
10 f4
(\\item {}{\(Q2\)} $K_i E_j K_i^{-1} =) show
0 404.4 moveto
10 f4
(q_i^{a_{i,j}}E_j$ \\quad \\quad  $K_i F_j K_i^{-1} =) show
0 393.9 moveto
10 f4
(q_i^{-a_{i,j}}F_j$ ) show
0 383.4 moveto
10 f4
(\\m) show
0 372.9 moveto
10 f4
(\\item {}{\(Q3\)} $E_i F_j - F_j E_i) show
0 362.4 moveto
10 f4
(= \\displaystyle \\delta_{i,j} \\frac {K_i - K_i^{-1}}) show
0 351.9 moveto
10 f4
({q_i - q_i^{-1}}$ ) show
0 341.4 moveto
10 f4
(\\m) show
0 330.9 moveto
10 f4
(\\item {}{\(Q4\)}$\\displaystyle \\sum_{k = 0}^{1-a_{i,j}} \(-1\)^k) show
0 320.4 moveto
10 f4
(\\left [\\matrix 1-a_{i,j}\\\\ k  \\\\ \\endmatrix) show
0 309.9 moveto
10 f4
(\\right]_i E_i^{1-a_{i,j}-k} E_j E_i^k = 0$ \\quad \\quad  for  \\quad $i \\neq j$) show
0 299.4 moveto
10 f4
(\\m) show
0 288.9 moveto
10 f4
(\\item {}{\(Q5\)}$\\displaystyle \\sum_{k = 0}^{1-a_{i,j}}) show
0 278.4 moveto
10 f4
(\(-1\)^k\\left [\\matrix 1-a_{i,j}\\\\ k  \\\\ \\endmatrix) show
0 267.9 moveto
10 f4
(\\right]_i F_i^{1-a_{i,j}-k} F_j F_i^k = 0$ \\quad \\quad for \\quad $i \\neq j.$) show
0 257.4 moveto
10 f4
(\\b) show
0 246.9 moveto
10 f4
(Suppose $a_{i,j} = -1 = a_{j,i}$ for some $i \\neq j$, and consider the subalgebra) show
0 236.4 moveto
10 f4
($U_{i,j}$ generated by $E_i, E_j$. In this special case, the quantum Serre) show
0 225.9 moveto
10 f4
(relation \(Q4\) reduces to) show
0 204.9 moveto
10 f4
($$\\aligned & E_i^2 E_j - [2]_i E_i E_j E_i +) show
0 194.4 moveto
10 f4
(E_j E_i^2 = 0\\quad \\quad \\text {and}  \\\\) show
0 183.9 moveto
10 f4
(& E_j^2 E_i - [2]_j E_j E_i E_j +) show
0 173.4 moveto
10 f4
(E_i E_j^2 = 0.\\\\) show
0 162.9 moveto
10 f4
(\\endaligned $$) show
0 141.9 moveto
10 f4
(\\n Since $-\\ell_i = \\ell_ia_{i,j} = \\ell_ja_{j,i} = -\\ell_j$, the coefficients) show
0 131.4 moveto
10 f4
($[2]_i$ and $[2]_j$ are equal. ) show
0 120.9 moveto
10 f4
(The algebra  $U_{i,j}$ \(with $q$ specialized to) show
0 110.4 moveto
10 f4
(a complex number which is not a root of unity\) is isomorphic to) show
0 99.9 moveto
10 f4
($A\([2]_i,-1,0\)$ by the mapping $E_i \\mapsto d$,) show
0 89.4 moveto
10 f4
($E_j \\mapsto u$.   The same result is true if) show
0 78.9 moveto
10 f4
(the corresponding $F$'s are used in place of the $E$'s. ) show
pagelevel restore
showpage
%%Page: 7 7
%%BeginPageSetup
/pagelevel save def
54 0 translate
%%EndPageSetup
newpath 0 72 moveto 504 0 rlineto 0 648 rlineto -504 0 rlineto  closepath clip newpath
0 711.9 moveto
10 f4
(In particular, when $\\g = \\fst$ \($3 \\times 3$ matrices of trace 0\), the ) show
0 701.4 moveto
10 f4
(algebra $U_{i,j}$  is just the subalgebra of $U_q\(\\fst\)$) show
0 690.9 moveto
10 f4
(generated by the $E$'s. ) show
0 680.4 moveto
10 f4
(\\m ) show
0 669.9 moveto
10 f4
(\\n {\\bf Example \(viii\)}. \\   To provide an explanation of the existence of quantum) show
0 659.4 moveto
10 f4
(groups, Witten \([W1], [W2]\) introduced a 7-parameter deformation of the) show
0 648.9 moveto
10 f4
(universal enveloping algebra $U\(\\fsl\)$.) show
0 638.4 moveto
10 f4
(Witten's deformation is a unital associative algebra) show
0 627.9 moveto
10 f4
(over a field $\\K$ \(which is algebraically closed of) show
0 617.4 moveto
10 f4
(characteristic zero and which could be) show
0 606.9 moveto
10 f4
(assumed to be $\\C$\) and depends on a 7-tuple $\\u \\xi =) show
0 596.4 moveto
10 f4
(\(\\xi_1, \\dots, \\xi_7\)$ of elements of $\\K$.  It has) show
0 585.9 moveto
10 f4
(a presentation by generators $x,y,z$ and defining) show
0 575.4 moveto
10 f4
(relations) show
0 564.9 moveto
10 f4
($$\\align ) show
0 554.4 moveto
10 f4
(& xz - \\xi_1 z x = \\xi_2 x \\tag 2\\\\) show
0 543.9 moveto
10 f4
(& zy - \\xi_3 yz = \\xi_4 y \\tag 3\\\\) show
0 533.4 moveto
10 f4
(& yx - \\xi_5 xy = \\xi_6 z^2 + \\xi_7 z. \\tag 4 \\\\) show
0 522.9 moveto
10 f4
(\\endalign$$) show
0 501.9 moveto
10 f4
(\\n We denote the resulting algebra by $ {\\frak W}\(\\u \\xi\) $.  ) show
0 491.4 moveto
10 f4
(In applications of these deformation algebras, the parameters ) show
0 480.9 moveto
10 f4
(depend on the coupling constant of the particular physical theory,) show
0 470.4 moveto
10 f4
(and Witten [W2] gives an evaluation of them in the special) show
0 459.9 moveto
10 f4
(case of the three-dimensional Chern-Simons gauge theory. ) show
0 449.4 moveto
10 f4
(\\m ) show
0 438.9 moveto
10 f4
(Let us assume $\\xi_6 = 0$ and $\\xi_7 \\neq 0$.   Then) show
0 428.4 moveto
10 f4
(substituting expression \(4\) into \(2\) and \(3\)  and rearranging shows that) show
0 417.9 moveto
10 f4
( ) show
0 407.4 moveto
10 f4
($$\\aligned & -\\xi_5 x^2y + \(1  + \\xi_1\\xi_5\)xyx - \\xi_1yx^2 = \\xi_2 \\xi_7 x) show
0 396.9 moveto
10 f4
(\\\\) show
0 386.4 moveto
10 f4
(& -\\xi_5 xy^2 + \(1+\\xi_3\\xi_5\)yxy -\\xi_3 y^2 x = \\xi_4 \\xi_7 y. \\\\) show
0 375.9 moveto
10 f4
(\\endaligned $$) show
0 365.4 moveto
10 f4
(\\n In particular, when $\\xi_5 \\neq 0$, $\\xi_1 = \\xi_3$,) show
0 354.9 moveto
10 f4
(and $\\xi_2 = \\xi_4$ we obtain) show
0 344.4 moveto
10 f4
($$) show
0 333.9 moveto
10 f4
(\\aligned) show
0 323.4 moveto
10 f4
(& x^2 y = \\frac{1+\\xi_1 \\xi_5}{\\xi_5} xyx - \\frac{\\xi_1}{\\xi_5}yx^2 - ) show
0 312.9 moveto
10 f4
(\\frac{\\xi_2 \\xi_7}{\\xi_5} x \\\\) show
0 302.4 moveto
10 f4
(& x y^2 = \\frac{1+\\xi_1 \\xi_5}{\\xi_5} yxy - \\frac{\\xi_1}{\\xi_5}y^2x - ) show
0 291.9 moveto
10 f4
(\\frac{\\xi_2 \\xi_7}{\\xi_5}y. \\\\) show
0 281.4 moveto
10 f4
(\\endaligned$$) show
0 260.4 moveto
10 f4
(\\n From this it is easy to see that ) show
0 249.9 moveto
10 f4
(a Witten deformation algebra ${\\frak W}\(\\u \\xi\)  $ with) show
0 239.4 moveto
10 f4
($\\xi_6 = 0$, $\\xi_5\\xi_7 \\neq 0$, $\\xi_1 = \\xi_3$,) show
0 228.9 moveto
10 f4
(and $\\xi_2 = \\xi_4$ is a homomorphic image) show
0 218.4 moveto
10 f4
(of the down-up algebra $A\(\\a,\\be,\\gamma\)$ ) show
0 207.9 moveto
10 f4
(with) show
0 197.4 moveto
10 f4
($$\\a = \\frac{1+\\xi_1 \\xi_5}{\\xi_5}, \\quad \\quad ) show
0 186.9 moveto
10 f4
(\\be = - \\frac{\\xi_1}{\\xi_5}, \\quad \\quad \\gamma = -\\frac{\\xi_2) show
0 176.4 moveto
10 f4
(\\xi_7}{\\xi_5}. \\tag 5$$) show
0 155.4 moveto
10 f4
(\\n In fact, in [B, Thm. 2.6] we proved the following) show
0 144.9 moveto
10 f4
(\\m) show
0 134.4 moveto
10 f4
(\\proclaim{Proposition 6} A Witten deformation algebra ${\\frak W}\(\\u \\xi\)  $) show
0 123.9 moveto
10 f4
(with) show
0 113.4 moveto
10 f4
($$\\xi_6 = 0, \\;\\; \\xi_5\\xi_7 \\neq 0, \\; \\xi_1 = \\xi_3, \\; \\;\\text{and}) show
0 102.9 moveto
10 f4
(\\;\\;\\xi_2 = \\xi_4 \\tag 7$$) show
0 81.9 moveto
10 f4
(\\n is isomorphic to the down-up algebra $A\(\\a,\\be,\\ga\)$) show
pagelevel restore
showpage
%%Page: 8 8
%%BeginPageSetup
/pagelevel save def
54 0 translate
%%EndPageSetup
newpath 0 72 moveto 504 0 rlineto 0 648 rlineto -504 0 rlineto  closepath clip newpath
0 711.9 moveto
10 f4
(with $\\a,\\be,\\ga$ given by \(5\). Conversely, any) show
0 701.4 moveto
10 f4
(down-up algebra $A\(\\a,\\be,\\ga\)$ with not both $\\a$ and $\\be$) show
0 690.9 moveto
10 f4
(equal to 0 is isomorphic to a Witten deformation algebra ${\\frak W}\(\\u \\xi\)  $) show
0 680.4 moveto
10 f4
(whose parameters satisfy \(7\). \\endproclaim ) show
0 669.9 moveto
10 f4
(\\m) show
0 659.4 moveto
10 f4
(A deformation algebra ${\\frak W}\(\\u \\xi\)  $) show
0 648.9 moveto
10 f4
(has a filtration,  and ) show
0 638.4 moveto
10 f4
(Le Bruyn \([L1], [L2]\) investigated the algebras ${\\frak W}\(\\u \\xi\)  $ ) show
0 627.9 moveto
10 f4
(whose associated graded algebras are Auslander regular. ) show
0 617.4 moveto
10 f4
(They determine a 3-parameter family of deformation algebras which) show
0 606.9 moveto
10 f4
(are called {\\it conformal $\\fsl$ algebras} and whose defining relations) show
0 596.4 moveto
10 f4
(are ) show
0 585.9 moveto
10 f4
($$xz - a zx = x, \\quad \\quad ) show
0 575.4 moveto
10 f4
(zy - a yz = y, \\quad \\quad yx - c xy = b z^2 + z  \\tag 8$$) show
0 554.4 moveto
10 f4
(\\n When $c \\neq 0$ and $b = 0$, the conformal $\\fsl$ algebra) show
0 543.9 moveto
10 f4
(with defining relations given by \(8\) is isomorphic to) show
0 533.4 moveto
10 f4
(the down-up algebra $A\(\\a,\\be,\\ga\)$ with) show
0 522.9 moveto
10 f4
($\\a = c^{-1}\(1+ac\), \\be = -ac^{-1}$ and $\\ga = -c^{-1}$.) show
0 512.4 moveto
10 f4
(If $b= c = 0$ and $a \\neq 0$, then the conformal $\\fsl$ algebra) show
0 501.9 moveto
10 f4
(is isomorphic to the down-up algebra $A\(\\a,\\be,\\ga\)$) show
0 491.4 moveto
10 f4
(with $\\a = a^{-1}$, $\\be = 0$ and $\\ga = -a^{-1}$.    ) show
0 480.9 moveto
10 f4
(\\m) show
0 470.4 moveto
10 f4
(In a recent paper [K1], Kulkarni has shown that under certain) show
0 459.9 moveto
10 f4
(assumptions on the parameters a Witten deformation algebra is isomorphic to) show
0 449.4 moveto
10 f4
(a conformal $\\fsl$ algebra or to a double skew polynomial extension.) show
0 438.9 moveto
10 f4
(Kulkarni studies the simple modules of) show
0 428.4 moveto
10 f4
(the conformal $sl_2$ algebras and of the skew polynomial) show
0 417.9 moveto
10 f4
(algebras.  Critical to the investigations in [K1] is the) show
0 407.4 moveto
10 f4
(observation that the conformal $\\fsl$ algebra of \(8\) can be realized as a ) show
0 396.9 moveto
10 f4
({\\it hyperbolic) show
0 386.4 moveto
10 f4
(ring}.  Kulkarni then applies results of Rosenberg [R]) show
0 375.9 moveto
10 f4
(on noncommutative algebraic geometry ) show
0 365.4 moveto
10 f4
(to describe the left ideals in the left spectrum of the algebra and to) show
0 354.9 moveto
10 f4
(determine the maximal left ideals for the conformal $sl_2$ algebras.) show
0 344.4 moveto
10 f4
(\\m ) show
0 333.9 moveto
10 f4
(\\n {\\bf Example \(ix\)}. \\ The quadratic Askey-Wilson algebras studied in) show
0 323.4 moveto
10 f4
([GLZ] can be regarded as having generators $a,b$ and defining relations) show
0 312.9 moveto
10 f4
(which depend on fixed parameters $\(\\a,\\gamma,\\delta,\\epsilon, \\zeta,\\eta,\\mu,\\nu\)$) show
0 302.4 moveto
10 f4
(according to:) show
0 281.4 moveto
10 f4
($$\\aligned a^2b & = \\a aba - ba^2 + \\zeta\(ab+ba\) +  \\eta a^2 + \\gamma a + \\delta b) show
0 270.9 moveto
10 f4
(+ \\mu 1 \\\\) show
0 260.4 moveto
10 f4
(ab^2 & = \\a bab - b^2a + \\eta\(ab+ba\) +\\zeta b^2 + \\gamma b + \\epsilon a) show
0 249.9 moveto
10 f4
(+ \\nu 1. \\\\ \\endaligned$$) show
0 228.9 moveto
10 f4
(\\n It is apparent that when $\\delta=\\epsilon=\\zeta=\\eta=\\mu=\\nu = 0$, this) show
0 218.4 moveto
10 f4
(algebra is just the down-up algebra $A\(\\a,-1,\\ga\)$. Askey-Wilson algebras) show
0 207.9 moveto
10 f4
(are related to the Leonard systems introduced in [T2] as abstract ) show
0 197.4 moveto
10 f4
(algebraic generalizations) show
0 186.9 moveto
10 f4
(of $q$-Racah polynomials and of families of orthogonal polynomials that) show
0 176.4 moveto
10 f4
(include the quantum $q$-Krawtchouk, Racah, Hahn, dual Hahn, and) show
0 165.9 moveto
10 f4
(Krawtchouk polynomials.   ) show
0 144.9 moveto
10 f4
(  ) show
0 134.4 moveto
10 f4
(\\b ) show
0 123.9 moveto
10 f4
(\\subhead {Highest weight modules} \\endsubhead) show
0 113.4 moveto
10 f4
(\\m ) show
0 102.9 moveto
10 f4
(Down-up algebras have a rich representation theory \(see [BR, Sec. 2]\). In particular,) show
0 92.4 moveto
10 f4
(they have highest weight modules and weight modules ) show
0 81.9 moveto
10 f4
(which mimic those of $\\fsl$. ) show
pagelevel restore
showpage
%%Page: 9 9
%%BeginPageSetup
/pagelevel save def
54 0 translate
%%EndPageSetup
newpath 0 72 moveto 504 0 rlineto 0 648 rlineto -504 0 rlineto  closepath clip newpath
0 711.9 moveto
10 f4
(\\m ) show
0 701.4 moveto
10 f4
(A module $V$ for $A = A\(\\a,\\be,\\ga\)$) show
0 690.9 moveto
10 f4
(is said to be a {\\it highest weight module of weight $\\l$}) show
0 680.4 moveto
10 f4
(if $V$ has a vector $y_0$ such that $d \\cdot y_0 = 0$,) show
0 669.9 moveto
10 f4
($\(d u\) \\cdot y_0 = \\l y_0$, and $V = Ay_0$.   The vector) show
0 659.4 moveto
10 f4
($y_0$ is a {\\it maximal vector} or {\\it highest weight vector} of) show
0 648.9 moveto
10 f4
($V$.  ) show
0 638.4 moveto
10 f4
(\\m ) show
0 627.9 moveto
10 f4
(\\proclaim {Proposition 9} \(See [BR, Sec. 2]\)  Set $\\l_{-1} = 0$) show
0 617.4 moveto
10 f4
(and let) show
0 606.9 moveto
10 f4
($\\l_0 = \\l \\in \\C$ be arbitrary.    For $n \\geq 1$,  ) show
0 596.4 moveto
10 f4
( define $\\l_n$ inductively by the  ) show
0 585.9 moveto
10 f4
(recurrence relation, ) show
0 575.4 moveto
10 f4
($$\\lambda_n = \\a \\lambda_{n-1} + \\be \\l_{n-2} + \\ga. \\tag 10$$) show
0 554.4 moveto
10 f4
(\\n The $\\C$-vector space $V\(\\l\)$ with basis $\\{v_n \\mid n = 0,1,2, \\dots\\}$) show
0 543.9 moveto
10 f4
(and with $A\(\\a,\\be,\\ga\)$-action given by) show
0 533.4 moveto
10 f4
($$\\aligned) show
0 522.9 moveto
10 f4
(d \\cdot v_n & = \\l_{n-1} v_{n-1}, \\quad n \\geq 1, \\quad \\text {and}) show
0 512.4 moveto
10 f4
(\\quad d) show
0 501.9 moveto
10 f4
(\\cdot v_0 = 0 \\\\) show
0 491.4 moveto
10 f4
(u \\cdot v_n & = v_{n+1}. \\\\) show
0 480.9 moveto
10 f4
(\\endaligned \\tag 11$$) show
0 459.9 moveto
10 f4
(\\n is a highest weight) show
0 449.4 moveto
10 f4
(module for $A\(\\a,\\be,\\ga\)$.  Every $A\(\\a,\\be,\\ga\)$-module of) show
0 438.9 moveto
10 f4
(highest weight $\\l$ is a homomorphic image of $V\(\\l\)$.  The module) show
0 428.4 moveto
10 f4
($V\(\\l\)$ is simple) show
0 417.9 moveto
10 f4
(if and only if $\\l_n \\neq 0$ for any $n$. ) show
0 407.4 moveto
10 f4
(\\endproclaim) show
0 396.9 moveto
10 f4
(\\m ) show
0 386.4 moveto
10 f4
(Because it shares the) show
0 375.9 moveto
10 f4
(same universal property and) show
0 365.4 moveto
10 f4
(many of the same features as Verma modules for finite-dimensional) show
0 354.9 moveto
10 f4
(semisimple complex Lie algebras, the module $V\(\\l\)$ is) show
0 344.4 moveto
10 f4
(said to be a {\\it Verma module} for $A\(\\a,\\be,\\ga\)$. ) show
0 333.9 moveto
10 f4
(\\b  ) show
0 323.4 moveto
10 f4
(\\subhead {Weight modules} \\endsubhead ) show
0 312.9 moveto
10 f4
(\\m) show
0 302.4 moveto
10 f4
(If we multiply the relation $d^2 u - \\a dud - \\be u d^2 = \\ga d$) show
0 291.9 moveto
10 f4
(on the left by $u$ and the relation $d u^2 - \\a udu - \\be u^2 d = \\ga u$) show
0 281.4 moveto
10 f4
(on the right by $d$ and subtract the second from the first, the resulting) show
0 270.9 moveto
10 f4
(equation is ) show
0 249.9 moveto
10 f4
($$0 = ud^2u - du^2d \\quad \\quad \\text {or} \\quad \\quad \(du\)\(ud\) = \(ud\)\(du\).$$) show
0 228.9 moveto
10 f4
(\\n Therefore, the  elements $du$ and $ud$ commute in $A = A\(\\a,\\be,\\ga\)$.  For) show
0 218.4 moveto
10 f4
(any basis element $v_n \\in V\(\\l\)$, we have $du \\cdot v_n = \\l_n v_n$ and $ud \\cdot v_n) show
0 207.9 moveto
10 f4
(= \\l_{n-1}v_n$.  Using that with $n = 0$ and $\\l \\neq 0$, it is easy to see that $du$ and $ud$) show
0 197.4 moveto
10 f4
(are linearly independent.   Let $\\h = \\C du \\oplus \\C ud$. ) show
0 186.9 moveto
10 f4
(\\b) show
0 176.4 moveto
10 f4
(We say an $A$-module $V$ is a {\\it weight module} if $V = \\bigoplus_{\\nu \\in \\h^*}) show
0 165.9 moveto
10 f4
(V_\\nu$, where $V_\\nu = \\{v \\in V \\mid h \\cdot v = \\nu\(h\)v$ for all $h \\in \\h\\}$, and) show
0 155.4 moveto
10 f4
(the sum is over elements in the dual space $\\h^*$ of $\\h$.  ) show
0 144.9 moveto
10 f4
(Any submodule of a weight module is a weight module.    ) show
0 134.4 moveto
10 f4
(If $V_\\nu \\neq \(0\)$, then $\\nu$ is a {\\it weight} and $V_\\nu$ is the) show
0 123.9 moveto
10 f4
(corresponding {\\it weight space}. Each weight $\\nu$) show
0 113.4 moveto
10 f4
(is determined by the pair $\(\\nu',\\nu''\)$ of complex numbers, $\\nu' = \\nu\(du\)$) show
0 102.9 moveto
10 f4
(and $\\nu'' = \\nu\(ud\)$. In particular, highest weight modules are weight modules in) show
0 92.4 moveto
10 f4
(this sense. The basis vector $v_n$ of $V\(\\l\)$ is a weight vector whose weight) show
0 81.9 moveto
10 f4
(is given by the pair $\(\\l_n,\\l_{n-1}\)$.  Finding these weights explicitly involves) show
pagelevel restore
showpage
%%Page: 10 10
%%BeginPageSetup
/pagelevel save def
54 0 translate
%%EndPageSetup
newpath 0 72 moveto 504 0 rlineto 0 648 rlineto -504 0 rlineto  closepath clip newpath
0 711.9 moveto
10 f4
(solving the linear recurrence relation in \(10\), which can be done) show
0 701.4 moveto
10 f4
(by standard methods as in [Br, Chap.7] for example.  ) show
0 690.9 moveto
10 f4
( ) show
0 680.4 moveto
10 f4
(\\proclaim { Proposition 12} Assume  $\\l_{-1} = 0$, $\\l_0 = \\l\\in \\C$,) show
0 669.9 moveto
10 f4
(and $\\l_{n}$ for) show
0 659.4 moveto
10 f4
($n \\geq 1$ is given by the recurrence relation) show
0 648.9 moveto
10 f4
($\\l_{n}-\\a\\l_{n-1} -\\be \\l_{n-2} = \\ga$.  Fix $t \\in \\C$ such) show
0 638.4 moveto
10 f4
(that) show
0 627.9 moveto
10 f4
($$t^2 = \\frac {\\a^2 + 4\\be}{4}.$$) show
0 617.4 moveto
10 f4
(\\m) show
0 606.9 moveto
10 f4
(\\item {}{\(i\)} If $\\a^2 + 4 \\be\\neq 0$, then ) show
0 585.9 moveto
10 f4
($$\\l_n = c_1 r_1^n) show
0 575.4 moveto
10 f4
(+ c_2 r_2^n  + x_n, \\quad \\quad \\text{where}$$ ) show
0 554.4 moveto
10 f4
($$) show
0 543.9 moveto
10 f4
(\\aligned & r_1 = \\frac {\\a} {2} + t, \\quad) show
0 533.4 moveto
10 f4
( \\quad ) show
0 522.9 moveto
10 f4
( r_2 = \\frac {\\a} {2} - t, \\\\) show
0 512.4 moveto
10 f4
(& x_n = \\cases ) show
0 501.9 moveto
10 f4
(\(1-\\a-\\be\)^{-1}\\ga \\ \\ \\text {\\quad if \\quad  $\\a+\\be \\neq 1$} \\\\) show
0 491.4 moveto
10 f4
(\(2-\\a\)^{-1}\\ga n \\ \\ \\text {\\quad if \\quad $\\a+\\be = 1$ \\quad \(necessarily \\quad $\\a) show
0 480.9 moveto
10 f4
(\\neq 2\),$} ) show
0 470.4 moveto
10 f4
(\\\\) show
0 459.9 moveto
10 f4
(\\endcases \\\\) show
0 449.4 moveto
10 f4
(& \\text {and} \\quad \\quad \\left \(\\matrix c_1 \\\\ c_2 \\\\ \\endmatrix \\right \)) show
0 438.9 moveto
10 f4
( = \\frac {1}{r_2-r_1}) show
0 428.4 moveto
10 f4
(\\left \(\\matrix r_2 & -1 \\\\ -r_1 & 1 \\\\ \\endmatrix \\right \)) show
0 417.9 moveto
10 f4
(\\left \(\\matrix \\l-x_0 \\\\ \\a\\l+\\ga-x_1 \\\\ \\endmatrix \\right \). ) show
0 407.4 moveto
10 f4
(\\endaligned $$) show
0 396.9 moveto
10 f4
( \\m) show
0 386.4 moveto
10 f4
(\\item {}{\(ii\)} If $\\a^2 + 4 \\be = 0$ and $\\a \\neq 0$, then ) show
0 365.4 moveto
10 f4
($$\\l_n = c_1 s^n + c_2 n s^n + x_n \\quad \\text {where}$$) show
0 354.9 moveto
10 f4
( ) show
0 333.9 moveto
10 f4
($$\\aligned & s = \\displaystyle \\frac {\\a}{2} \\\\) show
0 323.4 moveto
10 f4
(& x_n = \\cases ) show
0 312.9 moveto
10 f4
(\(1-\\a-\\be\)^{-1}\\ga \\ \\  \\text {\\quad if \\quad $\\a+\\be \\neq 1$} \\\\) show
0 302.4 moveto
10 f4
( 2^{-1} n^2 \\ga \\ \\  \\text {\\quad if \\quad $\\a + \\be = 1$ ) show
0 291.9 moveto
10 f4
(\\quad i.e. \\quad if \\quad  $\\a = 2,\\; \\be = -1$,}  ) show
0 281.4 moveto
10 f4
(\\\\ \\endcases \\\\ ) show
0 270.9 moveto
10 f4
(& \\text {and}  \\quad \\quad \\left \(\\matrix c_1 \\\\ c_2 \\\\ \\endmatrix \\right\)) show
0 260.4 moveto
10 f4
( =  ) show
0 249.9 moveto
10 f4
(\\left \(\\matrix 1 & 0 \\\\ -1 & 2\\a^{-1}  \\\\ \\endmatrix \\right \)) show
0 239.4 moveto
10 f4
(\\left \(\\matrix \\l-x_0 \\\\ \\a\\l+\\ga-x_1 \\\\ \\endmatrix \\right \). ) show
0 228.9 moveto
10 f4
(\\endaligned $$) show
0 218.4 moveto
10 f4
(\\m) show
0 207.9 moveto
10 f4
(\\item {}{\(iii\)} If $\\a^2 + 4 \\be = 0$  and) show
0 197.4 moveto
10 f4
($\\a = 0$,  then $\\be = 0$ and $\\l_n = \\ga$ for all $n \\geq 1$.) show
0 186.9 moveto
10 f4
(\\endproclaim) show
0 176.4 moveto
10 f4
(\\m) show
0 165.9 moveto
10 f4
(If $\\a,\\be$ are) show
0 155.4 moveto
10 f4
(real, then it is natural to take  $t  = \\displaystyle\\frac{\\sqrt{\\a^2+4\\be}}{2}$) show
0 144.9 moveto
10 f4
(in the above calculations.) show
0 134.4 moveto
10 f4
(\\m ) show
0 123.9 moveto
10 f4
(Let us consider several special cases. ) show
0 113.4 moveto
10 f4
(\\m) show
0 102.9 moveto
10 f4
(\\n {\\bf Example \(a\)}. \\   Recall that ) show
0 92.4 moveto
10 f4
(the universal enveloping algebra $U\(\\fsl\)$ of $\\fsl$ is) show
0 81.9 moveto
10 f4
(isomorphic to the algebra $A\(2,-1,-2\)$, and the universal enveloping) show
pagelevel restore
showpage
%%Page: 11 11
%%BeginPageSetup
/pagelevel save def
54 0 translate
%%EndPageSetup
newpath 0 72 moveto 504 0 rlineto 0 648 rlineto -504 0 rlineto  closepath clip newpath
0 711.9 moveto
10 f4
(algebra $U\({\\frak H}\)$ of the Heisenberg Lie algebra $\\frak H$ is isomorphic) show
0 701.4 moveto
10 f4
(to $A\(2,-1,0\)$.  ) show
0 690.9 moveto
10 f4
(Applying \(ii\) with $s = \\a/2 = 1$ and $x_n = n^2 \\ga/2$ for any) show
0 680.4 moveto
10 f4
(algebra) show
0 669.9 moveto
10 f4
($A\(2,-1,\\ga\)$,  we have that ) show
0 659.4 moveto
10 f4
( ) show
0 648.9 moveto
10 f4
($$\\l_n  = \\l + \(\\l+\\frac{\\ga}{2}\)n + \\frac{\\ga n^2}{2} =  ) show
0 638.4 moveto
10 f4
(\(n+1\)\(\\l+\\frac {\\ga n}{2}\).$$) show
0 627.9 moveto
10 f4
( ) show
0 617.4 moveto
10 f4
(In the $\\fsl$-case, it is customary to use the operator $h = du-ud$  rather) show
0 606.9 moveto
10 f4
(than $du$.  The eigenvalues of $h$ are $\\l_n -\\l_{n-1} = \\l+n\\ga = \\l-2n$,) show
0 596.4 moveto
10 f4
($n = 0,1, \\dots$.  The analogous) show
0 585.9 moveto
10 f4
(computation in the Heisenberg Lie algebra shows that the central element $z = du-ud$) show
0 575.4 moveto
10 f4
(has  constant eigenvalue $\\l_n = \\l$. ) show
0 564.9 moveto
10 f4
(\\m) show
0 554.4 moveto
10 f4
(\\n {\\bf Example \(b\)}. \\   Recall that the quantum case discussed earlier involves) show
0 543.9 moveto
10 f4
(the down-up algebra $A\([2]_i,-1,0\)$.   ) show
0 533.4 moveto
10 f4
(In the particular case of $U_q\(\\fst\)$, the subalgebra generated by) show
0 522.9 moveto
10 f4
(the $E_i$'s is isomorphic to $A\([2],-1,0\)$ where $[2] = ) show
0 512.4 moveto
10 f4
(\\displaystyle{\\frac{q^2-q^{-2}}{q-q^{-1}}}$, and ) show
0 501.9 moveto
10 f4
($\\l_n = [n+1]\\l = \\Big\(\\displaystyle{\\frac{q^{n+1}-q^{-\(n+1\)}}{q-q^{-1}}}) show
0 491.4 moveto
10 f4
(\\Big\)\\l$ \\ for all $n \\geq 0$ ) show
0 480.9 moveto
10 f4
(in that case.  ) show
0 470.4 moveto
10 f4
(\\m) show
0 459.9 moveto
10 f4
(\\n {\\bf Example \(c\)}. \\   For the algebra $A\(1,1,0\)$, ) show
0 449.4 moveto
10 f4
( the solutions to the associated) show
0 438.9 moveto
10 f4
(linear recurrence $\\l_n = \\l_{n-1} + \\l_{n-2}$, $\\l_0 = \\l$, $\\l_{-1} = 0$,) show
0 428.4 moveto
10 f4
(\(hence the eigenvalues of $du$ and $ud$ on $V\(\\l\)$\)   ) show
0 417.9 moveto
10 f4
(are given by the Fibonacci sequence) show
0 407.4 moveto
10 f4
($\\l_0 = \\l$, $\\l_1 = \\l$, $\\l_2 = 2 \\l$, $\\l_3 = 3\\l$, $\\l_4 = 5 \\l$,) show
0 396.9 moveto
10 f4
($\\dots$.  In this case, the equations in Proposition 12 reduce to ) show
0 386.4 moveto
10 f4
($\\displaystyle{\\l_n = \\l\\frac {\\sqrt 5}{5}\\left\( \\left \(\\frac {1 + \\sqrt 5}) show
0 375.9 moveto
10 f4
({2}\\right\)^{n+1} -\\left\(\\frac {1 - \\sqrt 5} {2}\\right\)^{n+1}\\right\).}$ ) show
0 365.4 moveto
10 f4
( ) show
0 354.9 moveto
10 f4
(\\m ) show
0 344.4 moveto
10 f4
(In [BR] we investigated in detail the weight space and submodule structure of the) show
0 333.9 moveto
10 f4
(Verma module $V\(\\l\)$.  Roots of unity play a critical role in determining) show
0 323.4 moveto
10 f4
(the dimension of a weight space.  We introduced ``category $\\Cal O$'' modules) show
0 312.9 moveto
10 f4
(in the spirit of [BGG] ) show
0 302.4 moveto
10 f4
(and showed the simple objects were highest weight modules,  and we explored) show
0 291.9 moveto
10 f4
(a more general category $\\Cal O'$ of modules for down-up algebras.) show
0 281.4 moveto
10 f4
(We briefly summarize some of the main results.) show
0 270.9 moveto
10 f4
(\\b ) show
0 260.4 moveto
10 f4
(\\proclaim{Proposition 13}\([BR, Secs. 2, 4, 5]\) ) show
0 249.9 moveto
10 f4
(\\item {\(a\)} In $V\(\\l\)$ each weight space is either) show
0 239.4 moveto
10 f4
(one-dimensional or infinite-dimensional.  If an infinite-dimensional) show
0 228.9 moveto
10 f4
(weight space occurs, there are only finitely many weights.) show
0 218.4 moveto
10 f4
(\\m) show
0 207.9 moveto
10 f4
(\\item{\(b\)} If each weight space of $V\(\\l\)$ is) show
0 197.4 moveto
10 f4
(one-dimensional, then the proper submodules of $V\(\\l\)$ have the) show
0 186.9 moveto
10 f4
(form $N =$ span$_\\C\\{v_j \\mid j \\geq n+1\\}$ for some $n \\geq  0$ with $\\l_n = 0$.) show
0 176.4 moveto
10 f4
(Hence they are contained in $M\(\\l\) =$ span$_\\C\\{v_j \\mid j \\geq m+1\\}$,) show
0 165.9 moveto
10 f4
(where $\\l_m = 0$ and $m$ is minimal with that property.   ) show
0 155.4 moveto
10 f4
(\\m) show
0 144.9 moveto
10 f4
(\\item{\(c\)} If $\\gamma = 0 = \\lambda$, then) show
0 134.4 moveto
10 f4
($V\(\\l\)$ has infinitely many maximal proper submodules, each) show
0 123.9 moveto
10 f4
(of the form ) show
0 113.4 moveto
10 f4
($N^{\(\\tau\)} = \\text{span}_\\C\\{v_n -\\tau v_{n-1}\\mid n = 1,2, \\dots\\}$) show
0 102.9 moveto
10 f4
(for some $\\tau \\in \\C$, and infinitely many one-dimensional) show
0 92.4 moveto
10 f4
(simple modules,  $L\(0,\\tau\) = V\(0\)/N^{\(\\tau\)}$.  In) show
0 81.9 moveto
10 f4
(all other cases, $M\(\\l\)$ is the unique maximal submodule of $V\(\\l\)$,) show
pagelevel restore
showpage
%%Page: 12 12
%%BeginPageSetup
/pagelevel save def
54 0 translate
%%EndPageSetup
newpath 0 72 moveto 504 0 rlineto 0 648 rlineto -504 0 rlineto  closepath clip newpath
0 711.9 moveto
10 f4
(and there is a unique simple highest weight module,) show
0 701.4 moveto
10 f4
($L\(\\l\)= V\(\\l\)/M\(\\l\)$, of weight $\\l$  up to) show
0 690.9 moveto
10 f4
(isomorphism. \\endproclaim) show
0 680.4 moveto
10 f4
(\\m) show
0 669.9 moveto
10 f4
(\\n {\\bf Example}. \\   Recall that the poset $\\Cal L\(2,2\)$ affords a) show
0 659.4 moveto
10 f4
(representation of the down-up algebra $A\(1,-1,1\)$. It is easy) show
0 648.9 moveto
10 f4
(to see that the down and up operators satisfy $d\(\\u 0\) = 0$) show
0 638.4 moveto
10 f4
(and $du\(\\u 0\) = \\u 0$, where $\\u 0 = \(0,0\)$. Thus, the) show
0 627.9 moveto
10 f4
(element $\\u 0 \\in \\Cal L\(2,2\)$ generates a highest weight module) show
0 617.4 moveto
10 f4
(with $\\l = 1$.     If we solve) show
0 606.9 moveto
10 f4
(the corresponding recurrence relation in Proposition 12, we get from) show
0 596.4 moveto
10 f4
(\(i\) that $r_1 = 1/2\(1 + \\sqrt{-3}\)$, $r_2 = 1/2\(1 - \\sqrt{-3}\)$,) show
0 585.9 moveto
10 f4
(and $\\l_n = 1+ \(r_2^n - r_1^n\)/\(r_2-r_1\)$.  Since $r_1^3 = -1 = r_2^3$,) show
0 575.4 moveto
10 f4
(\(and hence $r_1,r_2$ are 6th roots of unity\), ) show
0 564.9 moveto
10 f4
(we see that the sequence $\\l_0 = \\l,\\l_1,\\l_2, \\dots,$ is given) show
0 554.4 moveto
10 f4
(by $1,2,2,1,0,0,1,2,2,1,0,0,\\dots$.  Thus, in the Verma module) show
0 543.9 moveto
10 f4
($V\(1\)$, the maximal submodule $M\(1\) =$ span$_\\C\\{v_j \\mid j \\geq 5\\}$.) show
0 533.4 moveto
10 f4
(The irreducible quotient $L\(1\) = V\(1\)/M\(1\)$ is 5-dimensional, and it is) show
0 522.9 moveto
10 f4
(spanned modulo $M\(1\)$ by $v_0,v_1,v_2,v_3, v_4$.  As an) show
0 512.4 moveto
10 f4
($A\(1,-1,1\)$-module, $\\Cal L\(2,2\)$ decomposes as $L\(1\) \\oplus L\(0\)$,) show
0 501.9 moveto
10 f4
(where we identify the copy of $L\(1\)$ with the span of the vectors) show
0 491.4 moveto
10 f4
($v_0 = \(0,0\)$, $v_1 =) show
0 480.9 moveto
10 f4
(\(1,0\)$, $v_2 = \(2,0\) + \(1,1\)$,) show
0 470.4 moveto
10 f4
($v_3 = 2 \\cdot \(2,1\)$, $v_4 = 2 \\cdot \(2,2\)$, and $L\(0\)$ with) show
0 459.9 moveto
10 f4
(the span of $\(2,0\) - \(1,1\)$.  ) show
0 449.4 moveto
10 f4
(\\b) show
0 438.9 moveto
10 f4
(\\subhead  The structure of down-up algebras \\endsubhead  ) show
0 428.4 moveto
10 f4
(\\m) show
0 417.9 moveto
10 f4
(>From a ring theoretic viewpoint, down-up algebras exhibit many interesting) show
0 407.4 moveto
10 f4
(features. For example, it is apparent from the) show
0 396.9 moveto
10 f4
(defining relations that) show
0 386.4 moveto
10 f4
(the monomials $u^i\(du\)^jd^k,$ $i,j,k = 0,1, \\dots$ in) show
0 375.9 moveto
10 f4
(a down-up algebra $A = A\(\\a,\\be,\\ga\)$   ) show
0 365.4 moveto
10 f4
(determine a spanning set.  ) show
0 354.9 moveto
10 f4
(In [BR, Thm. 3.1] we applied the Diamond Lemma \(see [Be]\) to prove a ) show
0 344.4 moveto
10 f4
(Poincar\\'e-Birkhoff-Witt type result for down-up algebras.  There) show
0 333.9 moveto
10 f4
(is one essential ambiguity,  $\(d^2u\)u  = d\(du^2\)$,  and the result of) show
0 323.4 moveto
10 f4
(resolving the ambiguity in the two possible ways is the same.  ) show
0 312.9 moveto
10 f4
(\\m) show
0 302.4 moveto
10 f4
(\\proclaim {Theorem 14} \(Poincar\\'e-Birkhoff-Witt Theorem\) Assume $A =) show
0 291.9 moveto
10 f4
(A\(\\a,\\be,\\ga\)$ is a down-up algebra over $\\C$. Then) show
0 281.4 moveto
10 f4
($\\{u^i\(du\)^jd^k \\mid i,j,k = 0,1,\\dots \\}$ is a basis of $A$.) show
0 270.9 moveto
10 f4
(\\endproclaim) show
0 260.4 moveto
10 f4
(\\m) show
0 249.9 moveto
10 f4
(The Gelfand-Kirillov dimension is a natural dimension to assign to an algebra $A$,) show
0 239.4 moveto
10 f4
(and in many cases \(such as when $A$ is  a domain\), ) show
0 228.9 moveto
10 f4
(it provides important structural information. ) show
0 218.4 moveto
10 f4
(Theorem 14 enables us to compute the GK-dimension of any down-up algebra $A =) show
0 207.9 moveto
10 f4
(A\(\\a,\\be,\\ga\)$.   The spaces  $A^{\(n\)} =$ span$_\\C\\{u^i \(du\)^j d^k \\mid) show
0 197.4 moveto
10 f4
(i+2j+k \\leq n\\}$ afford a filtration) show
0 186.9 moveto
10 f4
($\(0\) \\subset A^{\(0\)} \\subset A^{\(1\)} \\subset \\dots \\subset \\cup_n A^{\(n\)} =) show
0 176.4 moveto
10 f4
(A\(\\a,\\be,\\ga\)$ of the down-up algebra, and $A^{\(m\)} A^{\(n\)} \\subseteq A^{\(m+n\)}$ since the) show
0 165.9 moveto
10 f4
(defining relations replace the words $d^2 u$ and $d u^2$ by words of the) show
0 155.4 moveto
10 f4
(same or lower total degree.  The number of monomials $u^i \(du\)^j d^k$) show
0 144.9 moveto
10 f4
(with $i+2j+k = \\ell$ is $\(m+1\)\(m+1\)$ if $\\ell = 2m$ and is $\(m+1\)\(m+2\)$) show
0 134.4 moveto
10 f4
(if $\\ell = 2m+1$.    Thus, dim$A^{\(n\)}$ is a polynomial in $n$ with) show
0 123.9 moveto
10 f4
(positive coefficients of degree 3, and ) show
0 113.4 moveto
10 f4
(the Gelfand-Kirillov dimension is given by) show
0 102.9 moveto
10 f4
($$ ) show
0 92.4 moveto
10 f4
(\\text{GKdim}\\big\(A\(\\a,\\be,\\ga\)\\big\) = \\limsup_{n \\rightarrow \\infty} \\log_n\(\\dim) show
0 81.9 moveto
10 f4
(A^{\(n\)}\)) show
pagelevel restore
showpage
%%Page: 13 13
%%BeginPageSetup
/pagelevel save def
54 0 translate
%%EndPageSetup
newpath 0 72 moveto 504 0 rlineto 0 648 rlineto -504 0 rlineto  closepath clip newpath
0 711.9 moveto
10 f4
( = \\lim_{n \\rightarrow \\infty}{\\frac {\\ln\\Big\(\\dim) show
0 701.4 moveto
10 f4
(A^{\(n\)}\\Big\)}{\\ln n}} = 3.$$) show
0 690.9 moveto
10 f4
(\\m) show
0 680.4 moveto
10 f4
(\\proclaim {Proposition 15}\([BR, Sec. 3]\) If $A\(\\a,\\be,\\ga\)$) show
0 669.9 moveto
10 f4
(has infinitely many simple Verma modules) show
0 659.4 moveto
10 f4
($V\(\\l\)$, then the intersection  of the) show
0 648.9 moveto
10 f4
(annihilators of the simple Verma modules is zero.  \\endproclaim   ) show
0 638.4 moveto
10 f4
(\\m) show
0 627.9 moveto
10 f4
(It follows immediately that for such a down-up algebra $A\(\\a,\\be,\\ga\)$) show
0 617.4 moveto
10 f4
(the Jacobson radical, which is the intersection) show
0 606.9 moveto
10 f4
(of the annihilators of all the simple modules, is zero.) show
0 596.4 moveto
10 f4
(\\m) show
0 585.9 moveto
10 f4
(When $\\be = 0$, then $d\(du - \\a ud -\\ga\) = 0$) show
0 575.4 moveto
10 f4
(so that  $A\(\\a,\\be,\\ga\)$ has zero divisors for any choice of $\\a,\\ga \\in \\C$.) show
0 564.9 moveto
10 f4
(Necessary and sufficient conditions for $A\(\\a,\\be,\\ga\)$ to be a domain or to be) show
0 554.4 moveto
10 f4
(Noetherian have been proven recently ) show
0 543.9 moveto
10 f4
(using Theorem 14:) show
0 533.4 moveto
10 f4
(\\m) show
0 522.9 moveto
10 f4
(\\proclaim {Proposition 16}\([KMP], and compare also [K2].\) For a down-up algebra $A =) show
0 512.4 moveto
10 f4
(A\(\\a,\\be,\\ga\)$, the following are equivalent:) show
0 501.9 moveto
10 f4
(\\s) show
0 491.4 moveto
10 f4
(\\item{}{\(i\)} $\\beta \\neq 0$.) show
0 480.9 moveto
10 f4
(\\s) show
0 470.4 moveto
10 f4
(\\item{}{\(ii\)} $A$ is a domain. ) show
0 459.9 moveto
10 f4
(\\s) show
0 449.4 moveto
10 f4
(\\item{}{\(iii\)} $A$ is right and left Noetherian.) show
0 438.9 moveto
10 f4
(\\s) show
0 428.4 moveto
10 f4
(\\item{}{\(iv\)} $\\C[du,ud]$ is a polynomial ring in the two variables,) show
0 417.9 moveto
10 f4
(\(i.e. $du$ and $ud$ are algebraically independent\). \\endproclaim) show
0 407.4 moveto
10 f4
(\\m) show
0 396.9 moveto
10 f4
( A down-up algebra $A = A\(\\a,\\be,\\ga\)$ is $\\Z$-graded by assigning deg$\(d\) = -1$ and) show
0 386.4 moveto
10 f4
(deg$\(u\) = 1$ and extending to all of $A$ by setting deg$\(ab\)$ = deg$\(a\)$+deg$\(b\)$.) show
0 375.9 moveto
10 f4
(Then $A = \\bigoplus_{n \\in \\Z} A_n$ where $A_n = \\{a \\in A \\mid$ deg$\(a\) = n\\}$,) show
0 365.4 moveto
10 f4
(and $A_0$ is a commutative subalgebra.) show
0 354.9 moveto
10 f4
(It is shown in [BR] that if $A$ has infinitely many) show
0 344.4 moveto
10 f4
(simple Verma modules, the center lies in $A_0$. The center of $A$ has) show
0 333.9 moveto
10 f4
(been completely described in recent work \([K2] and [Z]\). ) show
0 323.4 moveto
10 f4
(   ) show
0 302.4 moveto
10 f4
(\\Refs  ) show
0 291.9 moveto
10 f4
(\\widestnumber\\key{BGG}) show
0 281.4 moveto
10 f4
(\\b) show
0 270.9 moveto
10 f4
(\\ref \\key B \\by G. Benkart, \\paper Down-up algebras and Witten's ) show
0 260.4 moveto
10 f4
(deformations of) show
0 249.9 moveto
10 f4
(the universal enveloping algebra of $\\fsl$, \\jour Contemp. Math.) show
0 239.4 moveto
10 f4
(Amer. Math. Soc. \\toappear \\endref  ) show
0 228.9 moveto
10 f4
(\\m) show
0 218.4 moveto
10 f4
(\\ref \\key BR \\by G. Benkart and T. Roby \\paper Down-up algebras) show
0 207.9 moveto
10 f4
(\\jour J. Algebra \\toappear \\endref ) show
0 197.4 moveto
10 f4
(\\m) show
0 186.9 moveto
10 f4
(\\ref\\key Be \\by G.M. Bergman \\paper The diamond lemma for ring theory \\jour) show
0 176.4 moveto
10 f4
(Adv. in Math. \\vol 29 \\yr 1978 \\pages 178--218\\endref) show
0 165.9 moveto
10 f4
(\\m) show
0 155.4 moveto
10 f4
( ) show
0 144.9 moveto
10 f4
(\\ref\\key BGG\\by J. Bernstein, I.M. Gelfand, and S.I. Gelfand \\paper A category) show
0 134.4 moveto
10 f4
(of $\\g$-modules \\jour) show
0 123.9 moveto
10 f4
(Func. Anal. Appl. \\vol 10 \\yr 1976 \\pages 87--92\\endref) show
0 113.4 moveto
10 f4
(\\m) show
0 102.9 moveto
10 f4
(\\ref \\key Br \\by R.A. Brualdi \\book Introductory Combinatorics   ) show
0 92.4 moveto
10 f4
(\\bookinfo Second Edition \\publ North Holland, New York\\yr 1992 \\endref) show
0 81.9 moveto
10 f4
(\\m ) show
pagelevel restore
showpage
%%Page: 14 14
%%BeginPageSetup
/pagelevel save def
54 0 translate
%%EndPageSetup
newpath 0 72 moveto 504 0 rlineto 0 648 rlineto -504 0 rlineto  closepath clip newpath
0 711.9 moveto
10 f4
(\\ref \\key F \\by S.V. Fomin \\paper Duality of graded graphs \\jour) show
0 701.4 moveto
10 f4
(J. Alg. Comb. \\vol 3 \\yr 1994 \\pages 357--404\\endref ) show
0 690.9 moveto
10 f4
(\\m) show
0 680.4 moveto
10 f4
(\\ref \\key GLZ \\by Ya. I. Granovski$\\check{\\text i}$, I.M. Lutzenko, and A.S.) show
0 669.9 moveto
10 f4
(Zhedanov \\paper  Quadratic algebras and dynamical symmetry of the) show
0 659.4 moveto
10 f4
(Schr\\"odinger equation \\jour Soviet Phys. JETP \\vol 72 \(2\) \\yr 1991 ) show
0 648.9 moveto
10 f4
(\\pages 205-209 \\endref ) show
0 638.4 moveto
10 f4
(\\m) show
0 627.9 moveto
10 f4
(\\ref \\key K1 \\by R. Kulkarni \\paper Irreducible) show
0 617.4 moveto
10 f4
(representations of Witten's deformations of $U\(sl_2\)$ \\toappear  \\endref) show
0 606.9 moveto
10 f4
(\\m ) show
0 596.4 moveto
10 f4
(\\ref \\key K2 \\by R. Kulkarni \\paper Down-up algebras) show
0 585.9 moveto
10 f4
(and their representations \\toappear \\endref) show
0 575.4 moveto
10 f4
(\\m) show
0 564.9 moveto
10 f4
(\\ref \\key KMP \\by E. Kirkman, I. Musson, and D. Passman  ) show
0 554.4 moveto
10 f4
(\\paper Noetherian down-up algebras \\toappear \\endref) show
0 543.9 moveto
10 f4
(\\m ) show
0 533.4 moveto
10 f4
(\\ref \\key L1 \\by L. Le Bruyn \\paper) show
0 522.9 moveto
10 f4
(Two remarks on Witten's quantum enveloping algebra \\jour) show
0 512.4 moveto
10 f4
(Comm. Algebra \\vol 22 \\yr 1994 \\pages 865--876 \\endref) show
0 501.9 moveto
10 f4
(\\m) show
0 491.4 moveto
10 f4
(\\ref \\key L2 \\by L. Le Bruyn ) show
0 480.9 moveto
10 f4
(\\paper Conformal $\\fsl$ enveloping algebras ) show
0 470.4 moveto
10 f4
(\\jour Comm. Algebra \\vol 23 \\yr 1995 \\pages 1325--1362 \\endref ) show
0 459.9 moveto
10 f4
( ) show
0 449.4 moveto
10 f4
(\\m) show
0 438.9 moveto
10 f4
(\\ref \\key M \\by Yu. Manin \\paper Some remarks on Koszul algebras and quantum) show
0 428.4 moveto
10 f4
(groups \\jour Ann. Inst. Fourier \\vol 37 \\yr 1987 \\pages 191--205 \\endref) show
0 417.9 moveto
10 f4
(\\m) show
0 407.4 moveto
10 f4
(\\ref \\key P \\by R.A. Proctor \\paper Representations of $sl\(2,\\C\)$ on posets) show
0 396.9 moveto
10 f4
(and the Sperner property, SIAM J. Algebraic and Discrete Methods) show
0 386.4 moveto
10 f4
(\\vol 3 \\yr 1982 \\pages 275--280 \\endref ) show
0 375.9 moveto
10 f4
(\\m) show
0 365.4 moveto
10 f4
(\\ref \\key R \\by A.L. Rosenberg \\book Noncommutative Algebraic Geometry and) show
0 354.9 moveto
10 f4
(Representations of Quantized Algebras \\publ Kluwer \\yr 1995  \\endref) show
0 344.4 moveto
10 f4
(\\m) show
0 333.9 moveto
10 f4
(\\ref\\key St1 \\by R.P. Stanley  \\paper Weyl groups, the hard Lefschetz theorem,) show
0 323.4 moveto
10 f4
(and the Sperner property \\jour SIAM J. Algebraic and Discrete Methods \\vol) show
0 312.9 moveto
10 f4
(1 \\yr 1980 \\pages 168--184 \\endref) show
0 302.4 moveto
10 f4
(\\m) show
0 291.9 moveto
10 f4
(\\ref\\key St2 \\by R.P. Stanley  \\paper  ) show
0 281.4 moveto
10 f4
(Differential posets \\jour) show
0 270.9 moveto
10 f4
(J. of Amer. Math. Soc.  \\vol 4 \\yr 1988 \\pages 919--961 \\endref) show
0 260.4 moveto
10 f4
( ) show
0 249.9 moveto
10 f4
(\\m) show
0 239.4 moveto
10 f4
(\\ref\\key T1 \\by P. Terwilliger \\paper The incidence algebra of a uniform poset) show
0 228.9 moveto
10 f4
(\\pages 193-212   \\moreref \\book Coding Theory and Design Theory, Part I ) show
0 218.4 moveto
10 f4
(\\bookinfo D. Ray-Chaudhuri, ed., IMA Series \\vol 20 \\publ) show
0 207.9 moveto
10 f4
(Springer-Verlag, New York \\yr 1990 \\endref) show
0 197.4 moveto
10 f4
(\\m ) show
0 186.9 moveto
10 f4
(\\ref\\key T2 \\by P. Terwilliger \\book Leonard Systems,  An Algebraic Approach) show
0 176.4 moveto
10 f4
(to the $q$-Racah Polynomials  \\toappear \\endref) show
0 165.9 moveto
10 f4
(\\m) show
0 155.4 moveto
10 f4
(\\ref \\key W1 \\by E. Witten ) show
0 144.9 moveto
10 f4
(\\paper Gauge theories, vertex models, and quantum groups \\jour) show
0 134.4 moveto
10 f4
(Nuclear Phys. B \\vol 330 \\yr 1990  \\pages  285--346 ) show
0 123.9 moveto
10 f4
(\\endref) show
0 113.4 moveto
10 f4
(\\m) show
0 102.9 moveto
10 f4
(\\ref ) show
0 92.4 moveto
10 f4
(\\key W2 \\by E. Witten \\paper) show
0 81.9 moveto
10 f4
(Quantization of Chern-Simons gauge theory with complex gauge group ) show
pagelevel restore
showpage
%%Page: 15 15
%%BeginPageSetup
/pagelevel save def
54 0 translate
%%EndPageSetup
newpath 0 72 moveto 504 0 rlineto 0 648 rlineto -504 0 rlineto  closepath clip newpath
0 711.9 moveto
10 f4
(\\jour Comm. Math. Phys. \\vol 137 \\yr 1991  \\pages  29--66 \\endref) show
0 701.4 moveto
10 f4
(\\m ) show
0 690.9 moveto
10 f4
(\\ref \\key Z \\by K. Zhao \\paper Centers of down-up algebras  ) show
0 680.4 moveto
10 f4
(\\toappear \\endref) show
0 659.4 moveto
10 f4
(\\endRefs) show
0 638.4 moveto
10 f4
(\\vskip 3.5 mm) show
0 627.9 moveto
10 f4
( ) show
0 617.4 moveto
10 f4
(\\address ) show
0 606.9 moveto
10 f4
(\\newline ) show
0 596.4 moveto
10 f4
(Department of Mathematics, University of Wisconsin, Madison,) show
0 585.9 moveto
10 f4
(Wisconsin 53706 USA  \\newline) show
0 575.4 moveto
10 f4
(benkart\\@math.wisc.edu \\endaddress) show
0 554.4 moveto
10 f4
(\\enddocument \\end  ) show
pagelevel restore
showpage
%%EOF
