      \input amstex
      %\documentstyle{gen-j}
      \documentstyle{amsppt}
      \NoBlackBoxes
      \TagsOnLeft
      %\magnification = \magstep1 
      \overfullrule0pt
      %\hsize=35 true pc
      %\vsize=54 true pc
      %\voffset=0.1 true in
      %\hoffset=1.0 true cm  
      \def \n {\noindent}
      \def \s {\smallskip}
      \def \m {\medskip}
      \def \b {\bigskip}
      \def \a {\alpha}
      \def \ad {{\text {ad}}} 
      \def \be {\beta} 
      \def \d {\Delta}
      \def \del {\delta} 
      \def \e {\epsilon} 
      \def \End {{\text {End}}} 
      \def\fsl{\frak s\frak l_2} 
      \def\fst{\frak s\frak l_3}
      \def\fss{\frak s\frak l(1,1)}
      \def \fb {{\frak b}}
      \def \ga {\gamma}  
      \def \g {{\frak g}} 
      \def \h {{\frak h}}
      \def \k {\kappa} 
      \def \Hom {{\text {Hom}}}
      \def \l {\lambda}
      \def \la {\langle}
      \def \o {\overline}
      \def \oc {{\Cal O}}
      \def \osp{\frak o\frak s\frak p} 
      \def \ot {\otimes}
      \def \p {\frak p}
      \def \ra {\rangle} 
      \def \sgn {{\text {sgn}}}
      \def \t {\theta}
      \def \tr {{\text {tr}}}
      \def \u {\underline}
      \def \w {\omega} 
      \def \z {\zeta}
      \def \C {{\text {\bf C}}}
      \def \K {{\text {\bf K}}} 
      \def \Z {{\text {\bf Z}}} 
      \topmatter
      \title Differential Posets and Down-up Algebras 
      \endtitle
      \author Georgia Benkart \endauthor
      \thanks  The author gratefully acknowledges 
      support from  National Science 
      Foundation Grant \#{}DMS--9622447.  \endthanks  
      \subjclass 
      Primary 16S15, 16S30, 17B35, 17B10  
      Secondary  81R50, 06A07 \endsubjclass  
      \abstract
      Down-up algebras originated in the study of differential posets. 
      In this paper we discuss their combinatorial origins,  
      representations, and structure.  Down-up algebras
      exhibit many of the important features of the universal enveloping
      algebra  $U(\fsl)$ of the Lie algebra $\fsl$ including a
      Poincar\'e-Birkhoff-Witt type basis and a well-behaved representation theory.
      They have many interesting connections with Weyl algebras, quantum groups, 
      and Witten's 
      deformations of $U(\fsl)$. 
      \endabstract
      \endtopmatter
      \document
      \subhead {Differential posets} \endsubhead 
      \m
      In [St2], Stanley introduced a class of partially ordered sets,
      which he termed {\it differential posets}.  Many of the
      remarkable enumerative and combinatorial properties of these posets 
      involve counting saturated chains $y_1 \prec  y_2 \prec  \dots
      \prec y_k$  or Hasse walks $y_1, y_2, \dots, y_k$, (where either $y_{i+1}$
      covers $y_i$ or $y_i$ covers $y_{i+1}$).  
      Essential in the computations are two operators,
      $d$ (down) and  $u$ (up), which are defined on the complex vector space 
      $\C P$ having basis the elements of the poset $P$.  
      If $y \in P$, then $d(y)$ is the sum of all elements that $y$ covers 
      and $u(y)$ is the
      sum of all elements that cover
      $y$.  For many posets the down and up operators give well-defined
      linear transformations of $\C P$.  Precursors
      of the operators $d$ and $u$ appeared in [St1] and [P], where
      they were used to show posets are Sperner or rank unimodal. 
      \m
      The characterizing property of an $r$-differential poset is
      that the down and up
      operators satisfy
      $du - u d = r I$  for some positive integer $r$ (see [St2, Thm. 2.2]),
      where $I$ is the identity transformation on $\C P$. 
      Thus, the poset affords a representation of the  Weyl algebra, (the associative
      algebra with generators
      $y,x$ subject to the relation $y x - x y = 1$),  via the mapping $y \mapsto d/r$,
      and $x \mapsto u$.  Since the Weyl algebra also can be realized
      as differential operators $y \mapsto d/dx$ and $x \mapsto x$ (multiplication
      by $x$)  on $\C[x]$, Stanley referred to the 
      posets satisfying $du-ud = rI$
      as $r$-{\it differential}
      or simply {\it differential} when $r = 1$.  Fomin [F] studied
      essentially the same class of posets for $r = 1$, calling them
      ‘‘$Y$-graphs''.  This terminology comes from the fact that Young's
      lattice $Y$ of all partitions of all nonnegative integers 
      is the prototypical example.    
      \m
      A partition $\mu$ of a nonnegative integer $m$ can be
      regarded as a descending sequence 
      $\mu = (\mu_1 \geq \mu_2 \geq \dots)$ of parts whose sum $|\mu| = \sum_i
      \mu_i$ equals $m$.  If  $\nu = (\nu_1 \geq \nu_2 \geq
      \dots)$ is a second partition, then
      $\mu \leq \nu$ when $\mu_i \leq \nu_i$ for all $i$.   The partition
      $\nu$ covers $\mu$ (written here as $\mu \prec \nu$) if $\mu < \nu$
      and $|\nu| = 1 + |\mu|$.  Thus, $\mu \prec \nu$ if the partition $\mu$ is
      obtained from $\nu$ by subtracting 1 from exactly one of the parts of $\nu$,
      and $d(\nu)$ is the sum of all such $\mu$. Analogously,
      $u(\nu)$ is the sum of all partitions $\pi$ obtained from $\nu$ by
      adding 1 to one part of $\nu$.  Young's lattice $Y$ is
      a 1-differential poset, and $Y^r$ is $r$-differential 
      ([St2, Cor. 1.4]). 
      \m
      The down and up operators on Young's lattice  have a representation
      theoretic significance.  The simple modules of the symmetric group
      $S_n$ are indexed by the partitions $\nu$ of $n$.  Upon restriction to
      $S_{n-1}$, the representation labelled by $\nu$
      decomposes into a direct sum of simple $S_{n-1}$-modules
      indexed by the
      partitions
      $\mu \prec \nu$, so it is given by $d(\nu)$.  When the
      simple module labelled by $\nu$ is induced to a representation
      of $S_{n+1}$, it decomposes into a sum of simple $S_{n+1}$-modules indexed by
      the partitions $\pi$ of $n+1$ such that $\nu \prec \pi$,  which is just
      $u(\nu)$.    
      \m 
      In his study [T1] of
      uniform posets, Terwilliger considered finite ranked posets $P$
      whose down and up operators satisfy the
      following relation 
      $$ d_{i}d_{i+1}u_{i}=\alpha _{i}d_{i}u_{i-1}d_{i}+\beta
      _{i}u_{i-2}d_{i-1}d_{i} +\gamma _{i}d_{i}, $$
      \n where $d_{i}$ and $u_{i}$ denote the restriction of $d$ and $u$ to the
      elements of rank $i$. (There is an analogous second relation,
      $$ d_{i+1}u_{i}u_{i-1}=\alpha _{i}u_{i-1}d_{i}u_{i-1}+\beta
      _{i}u_{i-1}u_{i-2}d_{i-1} +\gamma _{i}u_{i-1}, $$
       
      \n which holds automatically in this case because $d_{i+1}$ and $u_i$ are
      adjoint operators relative to a certain bilinear form.)
      In many examples the constants in these relations do not
      depend on the rank $i$.   In particular, a poset whose down and up
      operators satisfy
      $$
      \aligned
      d^{2}u & = q(q+1)dud -q^{3}ud^{2}+rd \\
      du^2 & = q(q+1)udu - q^3 u^2 d + r u \\
      \endaligned $$
      \n where $q$ and $r$ are fixed complex numbers is 
      said to be  {\it $(q,r)$-differential}.   Many interesting
      examples of $(q,r)$-differential posets in [T1] arise
      from considering certain subspaces of a vector space over the
      field $GF(q)$ of $q$ elements:
      \m
      \item{(1)} Assume $W$ is an $n$-dimensional vector space over
      $GF(q)$ and consider the set of pairs $P = \{(U,f) \mid U$ is
      a subspace of $W$ and $f$ is an alternating bilinear form on $U\}$
      with the ordering:  $(U,f) \leq (V,g)$  if $U$ is a subspace of $V$
      and $g|_U = f$.  Then $P$ is a $(q,r)$-differential poset
      with $r = -q^n(q+1)$. 
      \m
      \item{(2)} In example (1),  replace ‘‘an alternating bilinear form'' with
      ‘‘a quadratic form''.  The resulting poset $P$ is $(q,
      -q^{n+1}(q+1))$-differential.
      \m
      \item{(3)} In this example assume $W$ is an $n$-dimensional space over
      $GF(q^2)$ and the bilinear forms are Hermitian. The poset
      $P$ is $(q^{2},-q^{2n+1}(q^{2}+1))$-differential in this case.
      \b 
      \subhead {Down-up algebras} \endsubhead 
      \m
       
      To better understand the algebra generated by the down and up operators of a
      poset and its action on the poset,  we introduced the notion of
      a down-up algebra in our joint work with Roby (see [BR]).  Although the initial
      motivation for our investigations came from posets, we made no assumptions
      about the existence of posets whose down and up operators satisfy our
      relations.  However, when such a poset exists, it affords a representation
      of the down-up algebra, so our primary focus in [BR]  was on determining
      explicit information about the
      representations of down-up algebras.    
      \m 
      \proclaim {Definition 1} Let $\a,\be,\ga$ be fixed but arbitrary complex numbers. 
      The unital associative algebra $A(\a,\be,\ga)$ over $\C$ with generators
      $d,u$ and defining relations
      \s
      \item {}{(R1)} $d^2u = \a dud + \be ud^2 + \ga d,$
      \s 
      \item {}{(R2)} $du^2 = \a udu + \be u^2 d + \ga u,$ 
      \m
      \n is a down-up algebra. \endproclaim   
      \m
      It is easy to see that when $\ga \neq 0$ the down-up algebra $A(\a,\be,\ga)$
      is isomorphic to $A(\a,\be,1)$ by the map, $d \mapsto d'$, $u \mapsto \ga u'$.
      Therefore, it would suffice to treat just two cases $\ga = 0,1$, but to
      avoid dividing considerations into these two cases, we retain
      the notation $\ga$. 
      \b
      \subhead Examples of down-up algebras  \endsubhead
      \m
      \n {\bf Example (i)}. \   If $B$ is the associative algebra generated by the down
      and up operators $d,u$  of
      a $(q,r)$-differential poset,  
      then relations (R1) and (R2) hold with 
      $\a = q(q+1)$, $\be = -q^{3}$, and $\ga =  r$.  Thus, $B$ is a homomorphic image
      of the algebra $A(\a,\be,\ga)$ with these parameters, and the action of
      $B$ on the poset gives a representation of $A(\a,\be,\ga)$. 
      \m 
      \n {\bf Example (ii)}. \  The relation $du - ud = rI$ of an $r$-differential poset
      can be multiplied on the left by $d$ and on the right by $d$ and
      the resulting equations can be added to get the relation
      $d^2 u - ud^2 = 2rd$ of a $(-1,2r)$-differential poset.
      Thus, the Weyl algebra is a homomorphic image (by the ideal generated
      by $du-ud - r1$)  of the algebra $A(0,1,2r)$.  
      Similarly, the $q$-Weyl algebra is a homomorphic image of the algebra 
      $A(0,q^2,(q+1))$ by the ideal generated by $du - qud - 1$.  
      The skew polynomial
      algebra $\C_q[d,u]$, or quantum plane (see [M]), is the associative algebra with
      generators $d,u$ which satisfy the relation  $du = qud$.  Therefore,
      $\C_q[d,u]$ is a homomorphic image (by the ideal generated by $du-qud$)
      of the algebra $A(2q,-q^2,0)$. 
      \m
      \n
      {\bf Example (iii)}. \   Consider the poset ${\Cal L}(2,2) = \{ {\u a} = (a_1,a_2)
      \mid  2 \geq a_1 \geq a_2$ and $a_1,a_2 \in \Z_{\geq 0}\}$ with the order relation
      $\u a \leq \u b$ if $a_i \leq b_i$ for $i = 1,2$.  This is just
      the set of partitions which fit into a $2 \times 2$ box.  By direct calculation
      it is easy to verify that the down and up operators on this poset satisfy
      $d^2u = dud - ud^2 +d$ and  $du^2 = udu - u^2 d + u$, so the algebra they  
      generate is a homomorphic image of the down-up algebra $A(1,-1,1)$.  
      \m 
      \n {\bf Example (iv)}. \   Suppose $\g$ is a 3-dimensional Lie algebra over $\C$
      with basis
      $x,y,[x,y]$  such that $[x[x,y]] = \ga x$ and  $[[x,y],y] = \ga y$.  
      In the universal enveloping
      algebra $U(\g)$ of $\g$ where $[x,y] = xy-yx$, these relations become
      $$\aligned
      & x^2 y -2 xyx + yx^2 = \ga x \\
      & xy^2 -2yxy +  y^2x = \ga y. \\
      \endaligned$$
      \n Thus, $U(\g)$ is a homomorphic image of the down-up algebra $A(2,-1,\ga)$
      via the mapping $\phi: A(2,-1,\ga) \rightarrow U(\g)$ with $\phi: d \mapsto x$,
      $\phi: u
      \mapsto y$.  The
      mapping $\psi: \g \rightarrow A(2,-1,\ga)$ with $\psi: x \mapsto d$, $
      \psi: y \mapsto u$,
      and $\psi: [x,y] \mapsto du - ud$ extends, by the universal property of $U(\g)$,
      to an algebra homomorphism $\psi: U(\g) \rightarrow A(2,-1,\ga)$ which
      is the inverse of $\phi$. 
      Consequently, $U(\g)$ is isomorphic to $A(2,-1,\ga)$.  
      \m 
      The Lie algebra
      $\fsl$ of $2 \times 2$ complex matrices of trace zero has a standard
      basis $e = E_{1,2}, f = E_{2,1},$ and $h = E_{1,1}-E_{2,2}$ of
      matrix units, which satisfies $[e,f] = h, \; [h,e] = 2e$, and $[h,f] = -2 f$. From
      this we see that $U(\fsl) \cong A(2,-1,-2)$.  The Heisenberg Lie algebra $\frak H$ has
      a basis $x,y,z$ where $[x,y] = z$, and $[z,{\frak H}] = 0$,
      so $U({\frak H}) \cong
      A(2,-1,0)$.  
      \m
      \n {\bf Example (v)}. \   The $2 \times 2$ complex matrices  
      $y = \left (\matrix  y_1 & y_2 \\
                           y_3 & y_4  \endmatrix \right )$
      with supertrace $y_1-y_4 = 0$  is the special linear
      Lie superalgebra  $L = \fss = L_{\o 0} \oplus L_{\o 1}$ 
      under the supercommutator $[x,y] =
      xy-(-1)^{ab}yx$ for $x\in L_{\o a},\;y\in L_{\o b}$.   It has a 
      presentation by generators $e,f$ (which belong to $L_{\o 1}$ and can be identified
      with the matrix units $e = E_{1,2}$, $f = E_{2,1}$) and relations
      $[e,[e,f]] = 0$, $[[e,f],f] = 0$, $[e,e] = 0$, $[f,f] = 0$.
      The universal enveloping algebra $U(\fss)$ of $\fss$  has generators $e,f$ and
      relations $e^2 f - f e^2 = 0$, $e f^2 - f^2 e = 0$, $e^2 = 0$, $f^2 = 0$.
      Thus, $U(\fss)$ is a homomorphic image of the down-up algebra
      $A(0,1,0)$ by the ideal generated by the elements $e^2$ and $f^2$, which are
      central in $A(0,1,0)$.   
      \m
      \n {\bf Example (vi)}. \   The orthosymplectic Lie superalgebra $\osp(1,2) = L_{\o 0}
      \oplus L_{\o 1}$ has generators $x,y \in L_{\o 1}$ which satisfy
      $$xy + yx = t \in L_{\o 0}  \quad \quad tx - xt = x \quad \quad yt -ty = y. $$
      \n By combining these relations, we see that its universal enveloping algebra
      \break  $U(\osp(1,2))$ is a homomorphic image of $A(0,1,1)$.  
      \m
      \n {\bf Example (vii)}. \   Consider the field $\C(q)$ of rational functions in the
      indeterminate $q$ over the complex numbers, and let
      $U_q(\g)$ be the quantized enveloping algebra (quantum group) of
      a finite-dimensional simple complex Lie algebra $\g$  corresponding
      to the Cartan matrix $\frak A = (a_{i,j})_{i,j= 1}
      ^n$. There are relatively prime integers $\ell_i$ so that
      the matrix $(\ell_i a_{i,j})$ is symmetric.  Let 
      $$q_i = q^{\ell_i},\quad \quad \text {and} \quad \quad  
      [m]_i = \frac {q_i^{m} - q_i^{-m}} {q_i -
      q_i^{-1}}$$
      \n for all $m \in \Z_{\geq 0}$.  When $m \geq 1$,  let 
      $[m]_i! = \prod_{j = 1}^m [j]_i.$  Set $[0]_i! = 1$ and define
      $$\left [\matrix m \\ n \\ \endmatrix
      \right]_i = \frac{[m]_i!}
      {[n]_i! [m-n]_i!}.$$ 
      \n Then $U = U_q(\g)$ is the unital associative algebra over $\C(q)$ with generators 
      $E_i, F_i, K_{i}, K_{i}^{-1}$ ($i = 1, \dots, n$) subject to the relations
      \b
      \item {}{(Q1)} $K_i K_i^{-1} = K_i^{-1} K_i,$
      \quad \quad $K_i K_j = K_jK_i$
      \m
      \item {}{(Q2)} $K_i E_j K_i^{-1} =
      q_i^{a_{i,j}}E_j$ \quad \quad  $K_i F_j K_i^{-1} =
      q_i^{-a_{i,j}}F_j$ 
      \m
      \item {}{(Q3)} $E_i F_j - F_j E_i
      = \displaystyle \delta_{i,j} \frac {K_i - K_i^{-1}}
      {q_i - q_i^{-1}}$ 
      \m
      \item {}{(Q4)}$\displaystyle \sum_{k = 0}^{1-a_{i,j}} (-1)^k
      \left [\matrix 1-a_{i,j}\\ k  \\ \endmatrix
      \right]_i E_i^{1-a_{i,j}-k} E_j E_i^k = 0$ \quad \quad  for  \quad $i \neq j$
      \m
      \item {}{(Q5)}$\displaystyle \sum_{k = 0}^{1-a_{i,j}}
      (-1)^k\left [\matrix 1-a_{i,j}\\ k  \\ \endmatrix
      \right]_i F_i^{1-a_{i,j}-k} F_j F_i^k = 0$ \quad \quad for \quad $i \neq j.$
      \b
      Suppose $a_{i,j} = -1 = a_{j,i}$ for some $i \neq j$, and consider the subalgebra
      $U_{i,j}$ generated by $E_i, E_j$. In this special case, the quantum Serre
      relation (Q4) reduces to
      $$\aligned & E_i^2 E_j - [2]_i E_i E_j E_i +
      E_j E_i^2 = 0\quad \quad \text {and}  \\
      & E_j^2 E_i - [2]_j E_j E_i E_j +
      E_i E_j^2 = 0.\\
      \endaligned $$
      \n Since $-\ell_i = \ell_ia_{i,j} = \ell_ja_{j,i} = -\ell_j$, the coefficients
      $[2]_i$ and $[2]_j$ are equal. 
      The algebra  $U_{i,j}$ (with $q$ specialized to
      a complex number which is not a root of unity) is isomorphic to
      $A([2]_i,-1,0)$ by the mapping $E_i \mapsto d$,
      $E_j \mapsto u$.   The same result is true if
      the corresponding $F$'s are used in place of the $E$'s. 
      In particular, when $\g = \fst$ ($3 \times 3$ matrices of trace 0), the 
      algebra $U_{i,j}$  is just the subalgebra of $U_q(\fst)$
      generated by the $E$'s. 
      \m 
      \n {\bf Example (viii)}. \   To provide an explanation of the existence of quantum
      groups, Witten ([W1], [W2]) introduced a 7-parameter deformation of the
      universal enveloping algebra $U(\fsl)$.
      Witten's deformation is a unital associative algebra
      over a field $\K$ (which is algebraically closed of
      characteristic zero and which could be
      assumed to be $\C$) and depends on a 7-tuple $\u \xi =
      (\xi_1, \dots, \xi_7)$ of elements of $\K$.  It has
      a presentation by generators $x,y,z$ and defining
      relations
      $$\align 
      & xz - \xi_1 z x = \xi_2 x \tag 2\\
      & zy - \xi_3 yz = \xi_4 y \tag 3\\
      & yx - \xi_5 xy = \xi_6 z^2 + \xi_7 z. \tag 4 \\
      \endalign$$
      \n We denote the resulting algebra by $ {\frak W}(\u \xi) $.  
      In applications of these deformation algebras, the parameters 
      depend on the coupling constant of the particular physical theory,
      and Witten [W2] gives an evaluation of them in the special
      case of the three-dimensional Chern-Simons gauge theory. 
      \m 
      Let us assume $\xi_6 = 0$ and $\xi_7 \neq 0$.   Then
      substituting expression (4) into (2) and (3)  and rearranging shows that
       
      $$\aligned & -\xi_5 x^2y + (1  + \xi_1\xi_5)xyx - \xi_1yx^2 = \xi_2 \xi_7 x
      \\
      & -\xi_5 xy^2 + (1+\xi_3\xi_5)yxy -\xi_3 y^2 x = \xi_4 \xi_7 y. \\
      \endaligned $$
      \n In particular, when $\xi_5 \neq 0$, $\xi_1 = \xi_3$,
      and $\xi_2 = \xi_4$ we obtain
      $$
      \aligned
      & x^2 y = \frac{1+\xi_1 \xi_5}{\xi_5} xyx - \frac{\xi_1}{\xi_5}yx^2 - 
      \frac{\xi_2 \xi_7}{\xi_5} x \\
      & x y^2 = \frac{1+\xi_1 \xi_5}{\xi_5} yxy - \frac{\xi_1}{\xi_5}y^2x - 
      \frac{\xi_2 \xi_7}{\xi_5}y. \\
      \endaligned$$
      \n From this it is easy to see that 
      a Witten deformation algebra ${\frak W}(\u \xi)  $ with
      $\xi_6 = 0$, $\xi_5\xi_7 \neq 0$, $\xi_1 = \xi_3$,
      and $\xi_2 = \xi_4$ is a homomorphic image
      of the down-up algebra $A(\a,\be,\gamma)$ 
      with
      $$\a = \frac{1+\xi_1 \xi_5}{\xi_5}, \quad \quad 
      \be = - \frac{\xi_1}{\xi_5}, \quad \quad \gamma = -\frac{\xi_2
      \xi_7}{\xi_5}. \tag 5$$
      \n In fact, in [B, Thm. 2.6] we proved the following
      \m
      \proclaim{Proposition 6} A Witten deformation algebra ${\frak W}(\u \xi)  $
      with
      $$\xi_6 = 0, \;\; \xi_5\xi_7 \neq 0, \; \xi_1 = \xi_3, \; \;\text{and}
      \;\;\xi_2 = \xi_4 \tag 7$$
      \n is isomorphic to the down-up algebra $A(\a,\be,\ga)$
      with $\a,\be,\ga$ given by (5). Conversely, any
      down-up algebra $A(\a,\be,\ga)$ with not both $\a$ and $\be$
      equal to 0 is isomorphic to a Witten deformation algebra ${\frak W}(\u \xi)  $
      whose parameters satisfy (7). \endproclaim 
      \m
      A deformation algebra ${\frak W}(\u \xi)  $
      has a filtration,  and 
      Le Bruyn ([L1], [L2]) investigated the algebras ${\frak W}(\u \xi)  $ 
      whose associated graded algebras are Auslander regular. 
      They determine a 3-parameter family of deformation algebras which
      are called {\it conformal $\fsl$ algebras} and whose defining relations
      are 
      $$xz - a zx = x, \quad \quad 
      zy - a yz = y, \quad \quad yx - c xy = b z^2 + z  \tag 8$$
      \n When $c \neq 0$ and $b = 0$, the conformal $\fsl$ algebra
      with defining relations given by (8) is isomorphic to
      the down-up algebra $A(\a,\be,\ga)$ with
      $\a = c^{-1}(1+ac), \be = -ac^{-1}$ and $\ga = -c^{-1}$.
      If $b= c = 0$ and $a \neq 0$, then the conformal $\fsl$ algebra
      is isomorphic to the down-up algebra $A(\a,\be,\ga)$
      with $\a = a^{-1}$, $\be = 0$ and $\ga = -a^{-1}$.    
      \m
      In a recent paper [K1], Kulkarni has shown that under certain
      assumptions on the parameters a Witten deformation algebra is isomorphic to
      a conformal $\fsl$ algebra or to a double skew polynomial extension.
      Kulkarni studies the simple modules of
      the conformal $sl_2$ algebras and of the skew polynomial
      algebras.  Critical to the investigations in [K1] is the
      observation that the conformal $\fsl$ algebra of (8) can be realized as a 
      {\it hyperbolic
      ring}.  Kulkarni then applies results of Rosenberg [R]
      on noncommutative algebraic geometry 
      to describe the left ideals in the left spectrum of the algebra and to
      determine the maximal left ideals for the conformal $sl_2$ algebras.
      \m 
      \n {\bf Example (ix)}. \ The quadratic Askey-Wilson algebras studied in
      [GLZ] can be regarded as having generators $a,b$ and defining relations
      which depend on fixed parameters $(\a,\gamma,\delta,\epsilon, \zeta,\eta,\mu,\nu)$
      according to:
      $$\aligned a^2b & = \a aba - ba^2 + \zeta(ab+ba) +  \eta a^2 + \gamma a + \delta b
      + \mu 1 \\
      ab^2 & = \a bab - b^2a + \eta(ab+ba) +\zeta b^2 + \gamma b + \epsilon a
      + \nu 1. \\ \endaligned$$
      \n It is apparent that when $\delta=\epsilon=\zeta=\eta=\mu=\nu = 0$, this
      algebra is just the down-up algebra $A(\a,-1,\ga)$. Askey-Wilson algebras
      are related to the Leonard systems introduced in [T2] as abstract 
      algebraic generalizations
      of $q$-Racah polynomials and of families of orthogonal polynomials that
      include the quantum $q$-Krawtchouk, Racah, Hahn, dual Hahn, and
      Krawtchouk polynomials.   
        
      \b 
      \subhead {Highest weight modules} \endsubhead
      \m 
      Down-up algebras have a rich representation theory (see [BR, Sec. 2]). In particular,
      they have highest weight modules and weight modules 
      which mimic those of $\fsl$. 
      \m 
      A module $V$ for $A = A(\a,\be,\ga)$
      is said to be a {\it highest weight module of weight $\l$}
      if $V$ has a vector $y_0$ such that $d \cdot y_0 = 0$,
      $(d u) \cdot y_0 = \l y_0$, and $V = Ay_0$.   The vector
      $y_0$ is a {\it maximal vector} or {\it highest weight vector} of
      $V$.  
      \m 
      \proclaim {Proposition 9} (See [BR, Sec. 2])  Set $\l_{-1} = 0$
      and let
      $\l_0 = \l \in \C$ be arbitrary.    For $n \geq 1$,  
       define $\l_n$ inductively by the  
      recurrence relation, 
      $$\lambda_n = \a \lambda_{n-1} + \be \l_{n-2} + \ga. \tag 10$$
      \n The $\C$-vector space $V(\l)$ with basis $\{v_n \mid n = 0,1,2, \dots\}$
      and with $A(\a,\be,\ga)$-action given by
      $$\aligned
      d \cdot v_n & = \l_{n-1} v_{n-1}, \quad n \geq 1, \quad \text {and}
      \quad d
      \cdot v_0 = 0 \\
      u \cdot v_n & = v_{n+1}. \\
      \endaligned \tag 11$$
      \n is a highest weight
      module for $A(\a,\be,\ga)$.  Every $A(\a,\be,\ga)$-module of
      highest weight $\l$ is a homomorphic image of $V(\l)$.  The module
      $V(\l)$ is simple
      if and only if $\l_n \neq 0$ for any $n$. 
      \endproclaim
      \m 
      Because it shares the
      same universal property and
      many of the same features as Verma modules for finite-dimensional
      semisimple complex Lie algebras, the module $V(\l)$ is
      said to be a {\it Verma module} for $A(\a,\be,\ga)$. 
      \b  
      \subhead {Weight modules} \endsubhead 
      \m
      If we multiply the relation $d^2 u - \a dud - \be u d^2 = \ga d$
      on the left by $u$ and the relation $d u^2 - \a udu - \be u^2 d = \ga u$
      on the right by $d$ and subtract the second from the first, the resulting
      equation is 
      $$0 = ud^2u - du^2d \quad \quad \text {or} \quad \quad (du)(ud) = (ud)(du).$$
      \n Therefore, the  elements $du$ and $ud$ commute in $A = A(\a,\be,\ga)$.  For
      any basis element $v_n \in V(\l)$, we have $du \cdot v_n = \l_n v_n$ and $ud \cdot v_n
      = \l_{n-1}v_n$.  Using that with $n = 0$ and $\l \neq 0$, it is easy to see that $du$ and $ud$
      are linearly independent.   Let $\h = \C du \oplus \C ud$. 
      \b
      We say an $A$-module $V$ is a {\it weight module} if $V = \bigoplus_{\nu \in \h^*}
      V_\nu$, where $V_\nu = \{v \in V \mid h \cdot v = \nu(h)v$ for all $h \in \h\}$, and
      the sum is over elements in the dual space $\h^*$ of $\h$.  
      Any submodule of a weight module is a weight module.    
      If $V_\nu \neq (0)$, then $\nu$ is a {\it weight} and $V_\nu$ is the
      corresponding {\it weight space}. Each weight $\nu$
      is determined by the pair $(\nu',\nu'')$ of complex numbers, $\nu' = \nu(du)$
      and $\nu'' = \nu(ud)$. In particular, highest weight modules are weight modules in
      this sense. The basis vector $v_n$ of $V(\l)$ is a weight vector whose weight
      is given by the pair $(\l_n,\l_{n-1})$.  Finding these weights explicitly involves
      solving the linear recurrence relation in (10), which can be done
      by standard methods as in [Br, Chap.7] for example.  
       
      \proclaim { Proposition 12} Assume  $\l_{-1} = 0$, $\l_0 = \l\in \C$,
      and $\l_{n}$ for
      $n \geq 1$ is given by the recurrence relation
      $\l_{n}-\a\l_{n-1} -\be \l_{n-2} = \ga$.  Fix $t \in \C$ such
      that
      $$t^2 = \frac {\a^2 + 4\be}{4}.$$
      \m
      \item {}{(i)} If $\a^2 + 4 \be\neq 0$, then 
      $$\l_n = c_1 r_1^n
      + c_2 r_2^n  + x_n, \quad \quad \text{where}$$ 
      $$
      \aligned & r_1 = \frac {\a} {2} + t, \quad
       \quad 
       r_2 = \frac {\a} {2} - t, \\
      & x_n = \cases 
      (1-\a-\be)^{-1}\ga \ \ \text {\quad if \quad  $\a+\be \neq 1$} \\
      (2-\a)^{-1}\ga n \ \ \text {\quad if \quad $\a+\be = 1$ \quad (necessarily \quad $\a
      \neq 2),$} 
      \\
      \endcases \\
      & \text {and} \quad \quad \left (\matrix c_1 \\ c_2 \\ \endmatrix \right )
       = \frac {1}{r_2-r_1}
      \left (\matrix r_2 & -1 \\ -r_1 & 1 \\ \endmatrix \right )
      \left (\matrix \l-x_0 \\ \a\l+\ga-x_1 \\ \endmatrix \right ). 
      \endaligned $$
       \m
      \item {}{(ii)} If $\a^2 + 4 \be = 0$ and $\a \neq 0$, then 
      $$\l_n = c_1 s^n + c_2 n s^n + x_n \quad \text {where}$$
       
      $$\aligned & s = \displaystyle \frac {\a}{2} \\
      & x_n = \cases 
      (1-\a-\be)^{-1}\ga \ \  \text {\quad if \quad $\a+\be \neq 1$} \\
       2^{-1} n^2 \ga \ \  \text {\quad if \quad $\a + \be = 1$ 
      \quad i.e. \quad if \quad  $\a = 2,\; \be = -1$,}  
      \\ \endcases \\ 
      & \text {and}  \quad \quad \left (\matrix c_1 \\ c_2 \\ \endmatrix \right)
       =  
      \left (\matrix 1 & 0 \\ -1 & 2\a^{-1}  \\ \endmatrix \right )
      \left (\matrix \l-x_0 \\ \a\l+\ga-x_1 \\ \endmatrix \right ). 
      \endaligned $$
      \m
      \item {}{(iii)} If $\a^2 + 4 \be = 0$  and
      $\a = 0$,  then $\be = 0$ and $\l_n = \ga$ for all $n \geq 1$.
      \endproclaim
      \m
      If $\a,\be$ are
      real, then it is natural to take  $t  = \displaystyle\frac{\sqrt{\a^2+4\be}}{2}$
      in the above calculations.
      \m 
      Let us consider several special cases. 
      \m
      \n {\bf Example (a)}. \   Recall that 
      the universal enveloping algebra $U(\fsl)$ of $\fsl$ is
      isomorphic to the algebra $A(2,-1,-2)$, and the universal enveloping
      algebra $U({\frak H})$ of the Heisenberg Lie algebra $\frak H$ is isomorphic
      to $A(2,-1,0)$.  
      Applying (ii) with $s = \a/2 = 1$ and $x_n = n^2 \ga/2$ for any
      algebra
      $A(2,-1,\ga)$,  we have that 
       
      $$\l_n  = \l + (\l+\frac{\ga}{2})n + \frac{\ga n^2}{2} =  
      (n+1)(\l+\frac {\ga n}{2}).$$
       
      In the $\fsl$-case, it is customary to use the operator $h = du-ud$  rather
      than $du$.  The eigenvalues of $h$ are $\l_n -\l_{n-1} = \l+n\ga = \l-2n$,
      $n = 0,1, \dots$.  The analogous
      computation in the Heisenberg Lie algebra shows that the central element $z = du-ud$
      has  constant eigenvalue $\l_n = \l$. 
      \m
      \n {\bf Example (b)}. \   Recall that the quantum case discussed earlier involves
      the down-up algebra $A([2]_i,-1,0)$.   
      In the particular case of $U_q(\fst)$, the subalgebra generated by
      the $E_i$'s is isomorphic to $A([2],-1,0)$ where $[2] = 
      \displaystyle{\frac{q^2-q^{-2}}{q-q^{-1}}}$, and 
      $\l_n = [n+1]\l = \Big(\displaystyle{\frac{q^{n+1}-q^{-(n+1)}}{q-q^{-1}}}
      \Big)\l$ \ for all $n \geq 0$ 
      in that case.  
      \m
      \n {\bf Example (c)}. \   For the algebra $A(1,1,0)$, 
       the solutions to the associated
      linear recurrence $\l_n = \l_{n-1} + \l_{n-2}$, $\l_0 = \l$, $\l_{-1} = 0$,
      (hence the eigenvalues of $du$ and $ud$ on $V(\l)$)   
      are given by the Fibonacci sequence
      $\l_0 = \l$, $\l_1 = \l$, $\l_2 = 2 \l$, $\l_3 = 3\l$, $\l_4 = 5 \l$,
      $\dots$.  In this case, the equations in Proposition 12 reduce to 
      $\displaystyle{\l_n = \l\frac {\sqrt 5}{5}\left( \left (\frac {1 + \sqrt 5}
      {2}\right)^{n+1} -\left(\frac {1 - \sqrt 5} {2}\right)^{n+1}\right).}$ 
       
      \m 
      In [BR] we investigated in detail the weight space and submodule structure of the
      Verma module $V(\l)$.  Roots of unity play a critical role in determining
      the dimension of a weight space.  We introduced ‘‘category $\Cal O$'' modules
      in the spirit of [BGG] 
      and showed the simple objects were highest weight modules,  and we explored
      a more general category $\Cal O'$ of modules for down-up algebras.
      We briefly summarize some of the main results.
      \b 
      \proclaim{Proposition 13}([BR, Secs. 2, 4, 5]) 
      \item {(a)} In $V(\l)$ each weight space is either
      one-dimensional or infinite-dimensional.  If an infinite-dimensional
      weight space occurs, there are only finitely many weights.
      \m
      \item{(b)} If each weight space of $V(\l)$ is
      one-dimensional, then the proper submodules of $V(\l)$ have the
      form $N =$ span$_\C\{v_j \mid j \geq n+1\}$ for some $n \geq  0$ with $\l_n = 0$.
      Hence they are contained in $M(\l) =$ span$_\C\{v_j \mid j \geq m+1\}$,
      where $\l_m = 0$ and $m$ is minimal with that property.   
      \m
      \item{(c)} If $\gamma = 0 = \lambda$, then
      $V(\l)$ has infinitely many maximal proper submodules, each
      of the form 
      $N^{(\tau)} = \text{span}_\C\{v_n -\tau v_{n-1}\mid n = 1,2, \dots\}$
      for some $\tau \in \C$, and infinitely many one-dimensional
      simple modules,  $L(0,\tau) = V(0)/N^{(\tau)}$.  In
      all other cases, $M(\l)$ is the unique maximal submodule of $V(\l)$,
      and there is a unique simple highest weight module,
      $L(\l)= V(\l)/M(\l)$, of weight $\l$  up to
      isomorphism. \endproclaim
      \m
      \n {\bf Example}. \   Recall that the poset $\Cal L(2,2)$ affords a
      representation of the down-up algebra $A(1,-1,1)$. It is easy
      to see that the down and up operators satisfy $d(\u 0) = 0$
      and $du(\u 0) = \u 0$, where $\u 0 = (0,0)$. Thus, the
      element $\u 0 \in \Cal L(2,2)$ generates a highest weight module
      with $\l = 1$.     If we solve
      the corresponding recurrence relation in Proposition 12, we get from
      (i) that $r_1 = 1/2(1 + \sqrt{-3})$, $r_2 = 1/2(1 - \sqrt{-3})$,
      and $\l_n = 1+ (r_2^n - r_1^n)/(r_2-r_1)$.  Since $r_1^3 = -1 = r_2^3$,
      (and hence $r_1,r_2$ are 6th roots of unity), 
      we see that the sequence $\l_0 = \l,\l_1,\l_2, \dots,$ is given
      by $1,2,2,1,0,0,1,2,2,1,0,0,\dots$.  Thus, in the Verma module
      $V(1)$, the maximal submodule $M(1) =$ span$_\C\{v_j \mid j \geq 5\}$.
      The irreducible quotient $L(1) = V(1)/M(1)$ is 5-dimensional, and it is
      spanned modulo $M(1)$ by $v_0,v_1,v_2,v_3, v_4$.  As an
      $A(1,-1,1)$-module, $\Cal L(2,2)$ decomposes as $L(1) \oplus L(0)$,
      where we identify the copy of $L(1)$ with the span of the vectors
      $v_0 = (0,0)$, $v_1 =
      (1,0)$, $v_2 = (2,0) + (1,1)$,
      $v_3 = 2 \cdot (2,1)$, $v_4 = 2 \cdot (2,2)$, and $L(0)$ with
      the span of $(2,0) - (1,1)$.  
      \b
      \subhead  The structure of down-up algebras \endsubhead  
      \m
      >From a ring theoretic viewpoint, down-up algebras exhibit many interesting
      features. For example, it is apparent from the
      defining relations that
      the monomials $u^i(du)^jd^k,$ $i,j,k = 0,1, \dots$ in
      a down-up algebra $A = A(\a,\be,\ga)$   
      determine a spanning set.  
      In [BR, Thm. 3.1] we applied the Diamond Lemma (see [Be]) to prove a 
      Poincar\'e-Birkhoff-Witt type result for down-up algebras.  There
      is one essential ambiguity,  $(d^2u)u  = d(du^2)$,  and the result of
      resolving the ambiguity in the two possible ways is the same.  
      \m
      \proclaim {Theorem 14} (Poincar\'e-Birkhoff-Witt Theorem) Assume $A =
      A(\a,\be,\ga)$ is a down-up algebra over $\C$. Then
      $\{u^i(du)^jd^k \mid i,j,k = 0,1,\dots \}$ is a basis of $A$.
      \endproclaim
      \m
      The Gelfand-Kirillov dimension is a natural dimension to assign to an algebra $A$,
      and in many cases (such as when $A$ is  a domain), 
      it provides important structural information. 
      Theorem 14 enables us to compute the GK-dimension of any down-up algebra $A =
      A(\a,\be,\ga)$.   The spaces  $A^{(n)} =$ span$_\C\{u^i (du)^j d^k \mid
      i+2j+k \leq n\}$ afford a filtration
      $(0) \subset A^{(0)} \subset A^{(1)} \subset \dots \subset \cup_n A^{(n)} =
      A(\a,\be,\ga)$ of the down-up algebra, and $A^{(m)} A^{(n)} \subseteq A^{(m+n)}$ since the
      defining relations replace the words $d^2 u$ and $d u^2$ by words of the
      same or lower total degree.  The number of monomials $u^i (du)^j d^k$
      with $i+2j+k = \ell$ is $(m+1)(m+1)$ if $\ell = 2m$ and is $(m+1)(m+2)$
      if $\ell = 2m+1$.    Thus, dim$A^{(n)}$ is a polynomial in $n$ with
      positive coefficients of degree 3, and 
      the Gelfand-Kirillov dimension is given by
      $$ 
      \text{GKdim}\big(A(\a,\be,\ga)\big) = \limsup_{n \rightarrow \infty} \log_n(\dim
      A^{(n)})
       = \lim_{n \rightarrow \infty}{\frac {\ln\Big(\dim
      A^{(n)}\Big)}{\ln n}} = 3.$$
      \m
      \proclaim {Proposition 15}([BR, Sec. 3]) If $A(\a,\be,\ga)$
      has infinitely many simple Verma modules
      $V(\l)$, then the intersection  of the
      annihilators of the simple Verma modules is zero.  \endproclaim   
      \m
      It follows immediately that for such a down-up algebra $A(\a,\be,\ga)$
      the Jacobson radical, which is the intersection
      of the annihilators of all the simple modules, is zero.
      \m
      When $\be = 0$, then $d(du - \a ud -\ga) = 0$
      so that  $A(\a,\be,\ga)$ has zero divisors for any choice of $\a,\ga \in \C$.
      Necessary and sufficient conditions for $A(\a,\be,\ga)$ to be a domain or to be
      Noetherian have been proven recently 
      using Theorem 14:
      \m
      \proclaim {Proposition 16}([KMP], and compare also [K2].) For a down-up algebra $A =
      A(\a,\be,\ga)$, the following are equivalent:
      \s
      \item{}{(i)} $\beta \neq 0$.
      \s
      \item{}{(ii)} $A$ is a domain. 
      \s
      \item{}{(iii)} $A$ is right and left Noetherian.
      \s
      \item{}{(iv)} $\C[du,ud]$ is a polynomial ring in the two variables,
      (i.e. $du$ and $ud$ are algebraically independent). \endproclaim
      \m
       A down-up algebra $A = A(\a,\be,\ga)$ is $\Z$-graded by assigning deg$(d) = -1$ and
      deg$(u) = 1$ and extending to all of $A$ by setting deg$(ab)$ = deg$(a)$+deg$(b)$.
      Then $A = \bigoplus_{n \in \Z} A_n$ where $A_n = \{a \in A \mid$ deg$(a) = n\}$,
      and $A_0$ is a commutative subalgebra.
      It is shown in [BR] that if $A$ has infinitely many
      simple Verma modules, the center lies in $A_0$. The center of $A$ has
      been completely described in recent work ([K2] and [Z]). 
         
      \Refs  
      \widestnumber\key{BGG}
      \b
      \ref \key B \by G. Benkart, \paper Down-up algebras and Witten's 
      deformations of
      the universal enveloping algebra of $\fsl$, \jour Contemp. Math.
      Amer. Math. Soc. \toappear \endref  
      \m
      \ref \key BR \by G. Benkart and T. Roby \paper Down-up algebras
      \jour J. Algebra \toappear \endref 
      \m
      \ref\key Be \by G.M. Bergman \paper The diamond lemma for ring theory \jour
      Adv. in Math. \vol 29 \yr 1978 \pages 178--218\endref
      \m
       
      \ref\key BGG\by J. Bernstein, I.M. Gelfand, and S.I. Gelfand \paper A category
      of $\g$-modules \jour
      Func. Anal. Appl. \vol 10 \yr 1976 \pages 87--92\endref
      \m
      \ref \key Br \by R.A. Brualdi \book Introductory Combinatorics   
      \bookinfo Second Edition \publ North Holland, New York\yr 1992 \endref
      \m 
      \ref \key F \by S.V. Fomin \paper Duality of graded graphs \jour
      J. Alg. Comb. \vol 3 \yr 1994 \pages 357--404\endref 
      \m
      \ref \key GLZ \by Ya. I. Granovski$\check{\text i}$, I.M. Lutzenko, and A.S.
      Zhedanov \paper  Quadratic algebras and dynamical symmetry of the
      Schr\"odinger equation \jour Soviet Phys. JETP \vol 72 (2) \yr 1991 
      \pages 205-209 \endref 
      \m
      \ref \key K1 \by R. Kulkarni \paper Irreducible
      representations of Witten's deformations of $U(sl_2)$ \toappear  \endref
      \m 
      \ref \key K2 \by R. Kulkarni \paper Down-up algebras
      and their representations \toappear \endref
      \m
      \ref \key KMP \by E. Kirkman, I. Musson, and D. Passman  
      \paper Noetherian down-up algebras \toappear \endref
      \m 
      \ref \key L1 \by L. Le Bruyn \paper
      Two remarks on Witten's quantum enveloping algebra \jour
      Comm. Algebra \vol 22 \yr 1994 \pages 865--876 \endref
      \m
      \ref \key L2 \by L. Le Bruyn 
      \paper Conformal $\fsl$ enveloping algebras 
      \jour Comm. Algebra \vol 23 \yr 1995 \pages 1325--1362 \endref 
       
      \m
      \ref \key M \by Yu. Manin \paper Some remarks on Koszul algebras and quantum
      groups \jour Ann. Inst. Fourier \vol 37 \yr 1987 \pages 191--205 \endref
      \m
      \ref \key P \by R.A. Proctor \paper Representations of $sl(2,\C)$ on posets
      and the Sperner property, SIAM J. Algebraic and Discrete Methods
      \vol 3 \yr 1982 \pages 275--280 \endref 
      \m
      \ref \key R \by A.L. Rosenberg \book Noncommutative Algebraic Geometry and
      Representations of Quantized Algebras \publ Kluwer \yr 1995  \endref
      \m
      \ref\key St1 \by R.P. Stanley  \paper Weyl groups, the hard Lefschetz theorem,
      and the Sperner property \jour SIAM J. Algebraic and Discrete Methods \vol
      1 \yr 1980 \pages 168--184 \endref
      \m
      \ref\key St2 \by R.P. Stanley  \paper  
      Differential posets \jour
      J. of Amer. Math. Soc.  \vol 4 \yr 1988 \pages 919--961 \endref
       
      \m
      \ref\key T1 \by P. Terwilliger \paper The incidence algebra of a uniform poset
      \pages 193-212   \moreref \book Coding Theory and Design Theory, Part I 
      \bookinfo D. Ray-Chaudhuri, ed., IMA Series \vol 20 \publ
      Springer-Verlag, New York \yr 1990 \endref
      \m 
      \ref\key T2 \by P. Terwilliger \book Leonard Systems,  An Algebraic Approach
      to the $q$-Racah Polynomials  \toappear \endref
      \m
      \ref \key W1 \by E. Witten 
      \paper Gauge theories, vertex models, and quantum groups \jour
      Nuclear Phys. B \vol 330 \yr 1990  \pages  285--346 
      \endref
      \m
      \ref 
      \key W2 \by E. Witten \paper
      Quantization of Chern-Simons gauge theory with complex gauge group 
      \jour Comm. Math. Phys. \vol 137 \yr 1991  \pages  29--66 \endref
      \m 
      \ref \key Z \by K. Zhao \paper Centers of down-up algebras  
      \toappear \endref
      \endRefs
      \vskip 3.5 mm
       
      \address 
      \newline 
      Department of Mathematics, University of Wisconsin, Madison,
      Wisconsin 53706 USA  \newline
      benkart\@math.wisc.edu \endaddress
      \enddocument \end  
