
\documentstyle[fullpage,12pt]{article}
\newcommand {\cRp} {{\cal R}_q}
\newcommand {\ord}{\mathop{\rm ord}}
\newcommand {\lcm}{\mathop{\rm lcm}}
\newcommand {\lc}{\mathop{\rm lc}}
\newcommand {\qed}{\hfill$\Box$\par\medskip}
\newtheorem{Theorem}{Theorem}
%\newtheorem{Proposition}{Proposition}
\newtheorem{Lemma}{Lemma}
\def\o{\circ}
\def\wt{\widetilde}
\title{\Large\bf
Rational Solutions of
First Order Linear $q$-Difference Systems
}

\author{\large\sl S.~A.~Abramov
\thanks{Supported in part by
the RFBR and INTAS under Grant 95-IN-RU-412
and by French-Russian Lyapunov institute
under Grant 98-03.}
\\
\normalsize Computer Center of\\
\normalsize the Russian Academy of Science, \\
\normalsize Vavilova 40, Moscow 117967, Russia \\
\normalsize {\tt abramov@ccas.ru}\\
\normalsize {\tt sabramov@cs.msu.su}
}

\begin{document}

\date{}

\maketitle

\begin{abstract}
We propose an algorithm to compute rational
function solutions for a first order system
of linear $q$-difference equations
with rational coefficients.
We make use of the fact
that $q$-difference equations bear similarity with
differential equations at the point 0 and with difference equations
at other points. This allows combining known algorithms for the
differential and the difference cases.
This algorithm does not require
preliminary uncoupling of the given system.
\begin{center}
\bf R\'esum\'e
\end{center}

Nous proposons un algorithme calculant les fonctions rationnelles
solutions d'un syst\`eme  du premier ordre d'\'equations aux
$q$-diff\'erences lin\'eaires \`a coefficients rationnels. Nous
utilisons le fait que les \'equations aux $q$-diff\'erences ont des
similarit\'es avec les \'equations diff\'erentielles au point 0 et
les \'equations aux diff\'erences aux autres points. Cela permet
de combiner les algorithmes connus pour ces deux types d'\'equations.
Cet algorithme ne requiert pas un d\'ecouplage
 pr\'ealable du
syst\`eme donn\'e.
\end{abstract}
\section{Introduction}\label{introd}
Let $K$ be a computable field of characteristic zero,
$q\in K$ a nonzero element which is not a root of unity,
and $x$ transcendental over $K$.
%Denote by $Q$
%the unique authomorphism of $K(x)$ which fix $K$ and
%satisfies $Qx=qx$.

A system of first order linear $q$-difference equations
with rational coefficients over the field $K$ is a system of the form~:
\newpage
$$
p_1(x)y_1(qx)=a_{11}(x)y_1(x)+\cdots
+a_{1m}(x)y_m(x)+b_1(x)
$$
\begin{eqnarray}\label{sys}
\cdots\end{eqnarray}
$$
p_m(x)y_m(qx)=a_{m1}(x)y_1(x)+\cdots
+a_{mm}(x)y_m(x)+b_m(x).
$$
%with polynomial coefficients $p_i(x),a_{ij}(x),b_j(x)$ over
%the field $K$.
We will assume
the coefficients $p_i(x),a_{ij}(x),b_j(x)$ to be
polynomials
over
the field $K$.


$q$-Calculus and the theory and algorithms
for linear $q$-difference equations are of interest in
combinatorics, especially in the theory of partitions
(\cite{And,q-series}).
In this paper we solve the problem
of computing all the rational solutions
$y(x) =(y_1(x),\dots ,y_m(x)) \in K(x)^m$
of (\ref{sys}).
Algorithms for solving this problem in the scalar case (that is
the case of a single scalar linear
$q$-difference equation of arbitrary order)
have been proposed in \cite{ABP,Aprog}.
%abramovp, A89, Amon,Aprog, ABP}.
The algorithmic study of systems is, generally, less well-developed.


The traditional computer algebra approach to solve
linear functional systems
is via the cyclic vector method, or other similar elimination
methods~\cite{BP96,AZ}, that convert
the systems to scalar equations (such a procedure is called
the {\em uncoupling}).
Gr\"obner bases technique also can be used to reduce a recurrent
system to the uncoupled form \cite{ChyzGB}.
The major, and well-known, problem of this approach
is the increasing size of the coefficients of
equations, that makes those
approaches applicable only to systems of very small dimension.
In this paper we propose an alternative
approach (a direct method) to solve the problem for the
$q$-difference case.

It should be noted that there is some progress in solving
the analogous problem in the differential  and the
difference cases: direct methods have been
proposed
in \cite{Bar97} and in
\cite{AB98,vh98}.
The methods \cite{AB98,vh98} are applicable to the
$q$-difference case, excepting the situation where the
denominators of some of $y_i(x)$ are divisible by $x$.

We will show below that by combining
both
differential and difference approaches
it is possible
to solve completely the problem in
the $q$-difference case.
A characteristic feature of
$q$-difference equations is that they are similar to differential
equations near the point 0 and are similar to difference equations
near other points. It was used in the scalar case in \cite{Aprog}.

Similarly to the differential and difference cases the construction
proceeds in two steps. In the first step we construct a
{\em universal denominator}. We mean a polynomial $U(x) \in K[x]$
such that: for all $y(x) \in K(x)^m$, if $y(x)$ is a solution of
(\ref{sys}), then $U(x)y(x)$ is a polynomial vector. Then the
substitution
\begin{equation}
\label{subs}
y(x)=\frac 1 {U(x)}z(x)
\end{equation}
into (\ref{sys}) reduces
the problem to finding polynomial solutions of a system in $z(x)$ of
the same type as (\ref{sys}).  The second step of our method deals
with this last problem.

{}From time to time we will need to find the largest
non-negative $n$
such that $q^n$ is a root of a given polynomial with
coefficients in $K$.  Therefore we assume that $K$ is a {\em
$q$-suitable field}, meaning that there exists an algorithm which
given $p\in K[x]$ finds all
non-negative integer $n$ such that $p(q^n)=0$.  For
instance, if $K=k(q)$ where $q$ is transcendental over $k$ we can
proceed as follows: Let $p(x)=\sum_{i=0}^d c_i x^i$ where $c_i\in
k[q]$. Compute $s = \min\{i;\ c_i\ne 0\}$ and $t = \max\{j;\ q^j
\,|\, c_s\}$. Then $p(q^n)=0$ only if $n\le t$, and the set of all
such $n$ can be found by testing the values $n=t,t-1,\ldots,0$
(\cite{fpsac}).

\section{Universal Denominators}
\label{unden}

%Let a system of $q$-difference equations (\ref{sys})
%be given.

\subsection{Factors other than $x$}
\label{othx}
Here we demonstrate
the $q$-modification of the algorithm \cite{AB98}.
The algorithm \cite{vh98} also allows such a modification.
In this paper we do not compare  the algorithms \cite{AB98} and
\cite{vh98}, and take the former because its description
is shorter.

We can assume the matrix
\begin{equation}
\label{matr}
 \left(
 \matrix{
  a_{11}       & \ldots    & a_{1m}      \cr
 \vdots     &            & \vdots \cr
  a_{m1}   & \ldots   & a_{mm} }
 \right),
\end{equation}
which corresponds to (\ref{sys}) and whose elements
are polynomials over $K$, to be invertible
(otherwise either the system is incompatible
or the number of unknowns can be reduced).

We name the
modification ${\rm qUD}$.  First of all we set
$A(x)$ to be equal to
$$\lcm (p_1(q^{-1}x),\dots ,p_m(q^{-1}x))$$
and compute $B(x)$ as $\lcm$ of denominators
of the elements of the matrix inverse of (\ref{matr}). Then compute
${\rm qUD}(A,B)$, considering the polynomial $V(x)$ as the result
of the computation:\\
\verb|    |$V(x):=1$;\\
\verb|    |$R(n):=Res_x(A(x),B(q^nx))$;\\
\verb|    if |$R(n)$ has some nonnegative integer roots \verb|then|\\
\verb|       |$N:=$ the largest nonnegative integer root of $R(n);$\\
\verb|       for |$i=N,N-1,\dots ,0$ \verb| do|\\
\verb|         |$d(x):=gcd(A(x),B(q^ix));$\\
\verb|         |$A(x):=A(x)/d(x);$\\
\verb|         |$B(x):=B(x)/d(q^{-i}x);$\\
\verb|         |$V(x):=V(x)d(x)d(q^{-1}x)\cdots d(q^{-i}x)$\\
\verb|       od|\\
\verb|    fi.|\\

If rational functions $F_1(x),\dots ,F_m(x)$ are such that
$$p_1(x)F_1(qx)-a_{11}(x)F_1(x)-\cdots -a_{1m}(x)F_m(x)$$
\begin{equation}
\label{exprs}
\dots
\end{equation}
$$p_m(x)F_m(qx)-a_{m1}(x)F_1(x)-\cdots -a_{mm}(x)F_m(x)$$
are polynomials and $A(x),B(x)$ are not divisible by $x$ then
the denominators of
$F_1(x)$, $\dots $, $F_m(x)$ (taken in the reduced form) divide
$V(x)$.

The proof is similar to the proof of Theorems 1, 2 in \cite{AB98}.
All arguments that
were given in \cite{AB98} will hold if we replace the shifts
$$x\rightarrow x+i,\; x\rightarrow x-i,$$
where $i$ is a nonnegative integer, by
$$x\rightarrow q^ix,\; x\rightarrow q^{-i}x,$$
and if we ignore the factor $x$ when considering irreducible
factors of polynomials. The factor $x$ has to be considered
separately since the polynomial $x$ and $qx$ are not relatively prime
over $K$, though any other irreducible polynomial $r(x)$ is
relatively prime with $r(qx)$.

\subsection{A bound for the exponent of $x$}
\label{x}

In the general case the components of
any rational solution $y(x)=(y_1(x),\dots ,y_m(x))$ of
(\ref{sys}) can be represented in the form
\begin{equation}
\label{ratss}
y_i(x)=\frac{f_i(x)}{g_i(x)}
+\frac{l_{i1}}{x}
+\frac{l_{i2}}{x^2}
+\cdots
+\frac{l_{ih_{i}}}{x^{h_i}},\;g_i(0)\neq 0,
\end{equation}
$i=1,\dots ,m$. The substitutions of
$$F_i^{(1)}(x)=\frac{f_i(x)}{g_i(x)}$$
and
$$F_i^{(2)}(x)=
\frac{l_{i1}}{x}
+\frac{l_{i2}}{x^2}
+\cdots
+\frac{l_{ih_{i}}}{x^{h_i}}$$
into expressions (\ref{exprs}) for $F_i,\,i=1,\dots m$,
give for each of those expressions two rational functions with
the relatively primes denominators. Thereby the rational functions
are polynomials. Freeing  $A(x)$ and $B(x)$
from the factor $x$ (denote the result by ${\wt A}(x),{\wt B}(x)$)
we compute
${\rm qUD}({\wt A}(x),{\wt B}(x))$
and get $V(x)$ which is
divisible by all $g_i(x),\,i=1,\dots ,m$.

Now it is sufficient
to find an upper bound $H$ for all $h_1,\dots ,h_m$ and then it will
be possible to use $U(x)=x^HV(x)$ as a universal denominator for all
rational solutions of (\ref{sys}).
To obtain such a bound one can use the technique of indicial
equations. (This technique is well known in the theory of linear
ordinary differential equations.) The main computational problem
connected with the construction of the indicial equation
in the differential case is
reducing the given system to  the super-irreducible form
(\cite{HW87,Bar89}). To the author's knowledge,
the super-irreducible form
for  $q$-difference systems
has not been considered yet. But in any case we can use a universal
approach which called $EG$-{\em eliminations} (\cite{EG}). This
approach allows one, in particular, to construct the indicial
equations for differential and $q$-difference equations.

In other words, any rational solution
(\ref{ratss}) of system (\ref{sys}) can be considered as a
solution in the class of Laurent series:  $F_i^{(1)}(x)$ and
$F_i^{(2)}(x)$, $i=1,\dots ,m$, are, resp., regular and singular
parts of the Laurent solution. Thus an upper bound for
its pole order
at $x=0$ can be taken as $H$.  $EG$-eliminations allow one to find
such a bound.

The general scheme of using $EG$-method is the following
(see \cite{EG} for details). Rewrite
system (\ref{sys}) in the operator form:
\begin{equation}
\label{opform}
 \left(
 \matrix{
  p_1Q-a_{11} & -a_{12} & \ldots & -a_{1m} \cr
  -a_{21}& p_2Q-a_{22} & \ldots & -a_{2m} \cr
 \vdots     & \vdots & & \vdots  \cr
  -a_{m1} &-a_{m2}&\ldots &p_mQ-a_{mm} }
 \right)y(x)=b(x),
\end{equation}
where the operator $Q$ is such that $Qf(x)=f(qx)$ for any
function $f(x)$.
Consider $y_i(x),b_i(x)$, $i=1,\dots ,m$, as Laurent series.
Let $z_i(n),c_i(n)$, $i=1,\dots, m$, be the  sequences of
coefficients of
these series, $-\infty <n< \infty $ .
Consider the mapping $\cRp :
K[x,Q] \rightarrow K[q^n,E^{-1}]$ defined by
\begin{equation}
\label{e37'}
\cRp Q=q^n,\;\;\cRp x= E^{-1},
\end{equation}
where the operator $E$ is such that $Ef(x)=f(x+1)$ for any
function $f(x)$.
(In \cite{APR} it has been shown that $\cRp$ is an isomorphism from $K[x,Q]$
onto $K[q^n,E^{-1}]$.) Applying $\cRp$ to the elements of the
operator matrix of (\ref{opform}) we get the operator matrix of the
recurrent system for the column of sequences $z(n)=(z_1(n),\dots
,z_m(n))^T$.  This system can be rewritten in the form
%a recurrent system $S$
\begin{equation} \label{e38}
P_lz(n+l)+P_{l-1}z(n+l-1)+\cdots +P_tz(n+t)=c(n),
\end{equation}
where $l,t$ are integer, $l\geq t$;
$c(n)=(c_1(n),\dots ,c_m(n))^T$;
$P_l,\dots,P_t$ are $m\times m$-matrices over $K[n]$
with $P_l$ and $P_t$ (the
{\em leading} and {\em trailing}  matrices of
the system) being nonzero but the determinants of $P_l,P_t$
can vanish for all $n$.
The process of $EG$-eliminations
in the {\em explicit} matrix $P=(P_l|P_{l-1}|\dots |P_t)$
allows one to transform
(\ref{e38})
into an equivalent system $S$ with nonsingular leading (or
analogously, trailing) matrix.
Suppose that using $EG$-eliminations
we constructed the corresponding system $S$
of the form (\ref{e38}) with non-singular $P_l(n)$. Then $\det
P_l(n)=0$ is the indicial equation  for (\ref{sys}) at the point 0.
The following theorem can be easily proven
in the usual way.
\begin{Theorem}
\label{l5}
Let $p(n)=\det P_l(n)$ be a nonzero polynomial in $q^n$.
Let $n_0$ be the smallest integer root of $p(n)=0$ if
such roots exist and $n_0=1$ otherwise.
Then
the pole orders of a Laurent series solution $y(x)=(y_1(x),\dots ,
y_m(x))$  of (\ref{sys}) do not exceed
$\vert \min \{ n_0+l, 0 \} \vert .$
\end{Theorem}
{\bf Proof \/}: All $z_i(n)$ are equal to 0 for all negative integer
$n$ with large enough $\vert n \vert$. Besides, $c_i(n)=0$ for
all negative $n$, $i=1,\dots ,m$. The matrix $P_l(n)$ is invertible
for all $n<n_0$ and we can use the recurrence (\ref{e38}) to
get $z_i(n)=0$, $i=1,\dots ,m$, for all $n<n_0+l$. \qed

\section{Polynomial solutions}
\label{psol}
After substitution (\ref{subs}) and
cleaning denominators we have to solve
the problem of finding polynomial solutions of a system of the form
(\ref{sys}). It is sufficient to find an upper bound for degrees
of all $y_1(x),\dots ,y_m(x)$. Then all polynomial solutions of
(\ref{sys}) can be found by the method of undetermined coefficients
(the problem can be reduced to a system of linear algebraic
equations). Less costly approaches such as those proposed
for difference system
in  \cite{AB98}
are also possible.

We can construct the corresponding recurrent system for (\ref{sys}).
Suppose that using $EG$-eliminations
we transformed the recurrent system to the  system $S$
of the form (\ref{e38}) with non-singular $P_t(n)$. Then $\det
P_t(n)=0$ is the indicial equation  for (\ref{sys}) at $\infty $.
The following theorem can be easily proven
in the usual way.
\begin{Theorem}
\label{l6}
Let $p(n)=\det P_t(n)$ be a nonzero polynomial in $q^n$.
Let $n_1$ be the largest integer root of $p(n)=0$ if
such roots exist and $n_1=-1$ otherwise. Let
$d=\max _{i=1}^m \deg b_i$ for (\ref{sys}).
Then the
degrees of
the components of polynomial solution $y(x)=(y_1(x),\dots ,
y_m(x))$  of (\ref{sys}) do not exceed
$ \max \{ n_1+t, d+t \} .$
\end{Theorem}
{\bf Proof \/}: For polynomial solutions all
$z_i(n)$ are equal to 0 for all large enough
$n$. Besides, $c_i(n)=0$ for
all $n>d$, $i=1,\dots ,m$. The matrix $P_t(n)$ is invertible
for all $n>n_1$ and we can use the recurrence (\ref{e38}) to
get $z_i(n)=0$, $i=1,\dots ,m$, for all
$n> \max \{ n_1+t, d+t \}$. \qed
\section{Example}

Let's consider the following system of $q$-difference equations:
$$
\begin{array}{ll}
( -qx + q^{3}x)\,y_1(qx) + (x- q^{4}x)y_1(x) +
(q^{4}x + 100q^{4} - q^{2}x - 100q^{2})y_2(x)=0 \\
(qx + 100)y_2(qx) - xy_1(x)=0.\\
\end{array}
$$
Here
$$p_1(x)=- qx + q^{3}x,\;p_2(x)= x\,q + 100$$
and corresponing matrix (\ref{matr}) is
$$
\left(
{\begin{array}{cc}
x - q^{4}x & q^{4}x + 100q^{4} - q^{2}x - 100q^{2} \\
 - x & 0
\end{array}}
 \right).
$$
The inverse matrix is
$$
\left(
{\begin{array}{cc}
0 &  - {\frac {1}{x}}  \\ [2ex]
{ \frac {1}{q^{2}(q^{2}x + 100q^{2} - x - 100)
}}  &  - { \frac {q^{2} + 1}{(x + 100)q^{2}}}
\end{array}}
 \right).
$$
Thus
$$
A(x) = q^{4}x^{2} + 100q^{4}x - q^{2}x^{2} - 100q^{
2}x,\;
B(x) =  - q^{2}x^{2} - 100xq + q^{4}x^{2} + 100q^{3
}x.
$$
Freeing from the factor $x$:
$$
{\wt A}(x) = q^{2}(q^{2}x + 100q^{2} - x - 100),\;
{\wt B}(x) = q(q^{3}x - xq - 100 + 100q^{2})
$$
we get
$V={\rm qUD}({\wt A}(x),{\wt B}(x))=x + 100$.

Now we should find the
exponent of $x$ in the universal denominatior.
The corresponding recurrent system has the explicit matrix
$$
\left(
{\begin{array}{rccc}
0 & q^{2}(100q^{2} - 100) & q^{n +2} - q^n
- q^{4} +1& q^{2}(q^{2} - 1) \\
0 & 100\,q^{n} & -1 & q^n
\end{array}}
 \right)$$
with
$l=0$, $t=-1$. $EG$-eliminations lead to the system with
the following leading matrix:
$$
\left(
{\begin{array}{cccc}
  100(q^{2n + 4} - q^{2n + 2} + q^{n + 1} - q^{n + 5} +
q^{4} - q^{2}) &
  0  \\
 0 & 100(q^{4}-q^{2})
%& q^{(n + 2)} -
%q^{n} + 1 - q^{4} & q^{4}-q^{2}
\end{array}}
 \right).$$
The equation
$10000(q^{2n + 4} - q^{2n + 2} + q^{n + 1} - q^{n + 5} +
q^{4} - q^{2})(q^{4}-q^{2})=0$
has the integer roots
$-1, \,1$.
So the degree of the pole is
$\leq \vert -1+l\vert =1$
and the universal denominator is $$ U(x) =xV(x)=
x(x + 100). $$ After the substitution of the found
denominator the system is $$ \begin{array}{ll}
((q^{2}- 1)x^2 + 100(q^{2} - 1)x)z_1(qx)
- (q(q^{4} + 1)x^2 + 100(1 -q^4)x)z_1(x)
- \\  (q^{3}(1 + q^{2})x^2- 100q^{2}((q^3+q^2-1)x
+ 100q^2 -
100)z_2(x)=0  \\
(x+100)z_2(qx) - qxz_1(x) =0.
\end{array}
$$
The corresponding recurrent system has the explicit matrix
$$\left(\matrix {0&0\cr
q^2(10000q^2 - 10000)&100q^n\cr
100q^{n + 1} - 100q^{n - 1} - 100q^{4}+100& -q\cr
q^2(100q^{3}
+100q^2 -100q-100
)&q^{n - 1}\cr
- q^{5} + q +  q^n - q^{n - 2}&0\cr
q^2(q^3 - q)&0}\right )^T.
$$
with
$l=0$, $t=-2$.
$EG$-eliminations lead to the system with
the following trailing matrix:
$$
 \left(\matrix
  {
0&q^{2n-2}-q^{2n-4}-q^{n+3}+q^{n-1}+q^6-q^4 \cr
- q& q^{n - 2} } \right)
.
$$
The polynomial
$-q(q^{2n-2}-q^{2n-4}-q^{n+3}+q^{n-1}+q^6-q^4)$
has the integer roots
$3,\; 5$.
So the polynomial solutions of the system has the degree
$\leq 5+t=3$.
It leads us to the following solution of the system
for finding numerator

$$
 \left[  100\,
{\_c_{1}} + x\,\_c_{1} + x^{2}\,\_c_{2}
   + { \frac {x^{3}\,\_c_{2}}{100}} ,
\\
   x\, \_c_{1}  + {\frac {1}{100}}
\,{ \frac {x^{3}\,\_c_{2}}{q^{2}}}
\right]
$$
and correspondingly to the solution of the given system
$$
[{\frac {x^{2}\,{
{\it \_c}_{2}} + 100\,{{\it \_c}_{1}}}{x}} , \,
{\frac {100\,{{\it \_c}_{1}}\,q^{
2} + x^{2}\,{{\it \_c}_{2}}}{(x + 100)\,q^{2}}} ].
$$
\section{Implementation}
This algorithm is implemented in Maple V by D.Khmelnov.
\section*{Acknowledgement} I would like to thank D.Khmelnov
and H.Q.Le who provided me with useful comments.
\begin{thebibliography}{99}

\bibitem{Aprog}
S.A.~Abramov (1995):
Rational solutions of li\-near dif\-fe\-ren\-ce and
$q$-dif\-fe\-ren\-ce  eq\-ua\-ti\-ons with polynomial coefficients.
{\em Programming and Comput.\ Software\/} {\bf 21}, No 6, 273--278.
Transl.\
from {\em Programmirovanie}, No 6, 3--11.

\bibitem{EG}
S.A.~Abramov (1998):
$EG$-eliminations. {\em Journal of
Difference Equations and Applications}, to appear.

\bibitem{AB98}
S.A.~Abramov, M.A.~Barkatou (1998):
Rational solutions of first order difference systems.
{\em Proc.~ISSAC'98}, 124--131.

\bibitem{ABP} S.A. Abramov, M. Bronstein and M. Petkovsek (1995):
 On polynomial solutions of linear
operators equations.
{\em Proc.\ ISSAC'95}, 290--295.

\bibitem{fpsac} S.A. Abramov, M. Petkov\v sek (1995): Finding all
$q$-hypergeometric solutions of $q$-difference equations.
{\em Proc. FPSAC'95}, Universite de Marne - la  Valle, 1--10.

\bibitem{APR}
S.~Abramov, M.~Petkov\v sek, A.~Ryabenko (1996):
Special formal series solutions of linear operator equations.
{\em Discret Math.}, to appear.


\bibitem{AZ}
S.A.~Abramov, E.V.~Zima (1997):
A Universal Program to Uncouple Linear Systems.
{\em Proc.~of CMCP'96 (International Conference on Computational Modeling
and Computing in Physics, Dubna, Russia, Sept.~16-21, 1996)}, 16--26.

\bibitem{And}
G.E.~Andrews (1976): {The Theory of Partitions}. {\em Encyclopedia of
Mathematics and its Applications}, Vol.\ 2, Addison-Wesley, Reading,
Mass.

\bibitem{q-series}
G.E.~Andrews (1986): {
$q$-Series: Their Development and Application in Analysis, Number
Theory, Combinatorics, Physics, and Computer Algebra}.
{\em CBMS Regional Conference Series}, No. 66, AMS, R.I.

\bibitem{Bar89}
M.A. Barkatou (1989):
{Contribution \`a l'\'etude des
\'equations diff\'erentielles et aux diff\'erences dans le champ
complexe}.
{\em Th\`ese soutenue le 6 juin 1989  \`a l'institut
national polytechnique de Grenoble (France).}

\bibitem{Bar97}
M.A. Barkatou (1998):
{On rational solutions of systems
of linear differential equations}.
{\em Journal of Symbolic Computation}, to appear.

\bibitem{BP96}
M.~Bronstein, M.~Petkov\v sek (1996):
An introduction to pseudo--linear algebra.
{\em Theoretical Computer Science~}{\bf 157}, 3--33.

\bibitem{ChyzGB}
F.~Chyzak (1998):
Gr\"obner bases, symbolic summation and symbolic integration.
In: {\em Gr\"obner Bases and Applications}, B.~Buchberger and
F.~Winkler, Eds., London Mathematical Society Lecture Note Series
251, Cambridge University Press, Cambridge UK, 1998, 32--60.

\bibitem{HW87}
A. Hilali, A. Wazner (1987):
{Formes super-irr\'eductibles
des syst\`emes diff\'erentiels lin\'eaires}.
{\em Num. Math.\/} {\bf 50}, 429-449.

\bibitem{vh98}
M.~van Hoeij (1998):
Rational solutions of linear difference equations.
{\em Proc.~ISSAC'98},120--123.

\end{thebibliography}

\end{document}

