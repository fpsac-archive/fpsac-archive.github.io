\documentclass[12pt,leqno,bbold]{article}

\topmargin=-0.2in
\textheight=9in
\oddsidemargin=0.1in
\textwidth=6in

\usepackage{amssymb}
\newtheorem{condition}{Condition}
\newtheorem{theorem}{Theorem}
\newtheorem{corollary}{Corollary}
\newtheorem{proposition}{Proposition}
\newcommand{\proofstart}{{\bf Proof\hspace{2em}}}
\newcommand{\proofend}{\hspace*{\fill}\mbox{$\Box$}}
\newcommand{\react}[2]{\mbox{ \raisebox{-2ex} %reversible reaction arrows
{$\stackrel   %3 on top of 1
{\stackrel   %2 on top of 1
{ \stackrel{  \mbox{\raisebox{0.4ex}{ ${\scriptstyle #1}$ }} }
{\textstyle \rightleftharpoons } }{#2 } }{}$} }}
%\newcommand{\<>}{\hspace{0.1in}^{>}\!\!\!\!_{<}\hspace{0.1in}}
\newcommand{\be}{\begin{equation}}
\newcommand{\ee}{\end{equation}}
\newcommand{\bm}[1]{\mbox{\boldmath$ #1 $}}
\newcommand{\no}{\noindent}
\def\Z{\mathbb{Z}}
\def\Ker{\rm Ker}
\def\A{\mathcal A}
\def\F{\mathcal F}
\def\L{\mathcal L}
\def\>~{ \mbox{\raisebox{-.9ex}{$\stackrel{\textstyle >}{\sim}$}} }
\def\<~{ \mbox{\raisebox{-.9ex}{$\stackrel{\textstyle <}{\sim}$}} }

\title{Counting Polynomials for Channel Assignments, 
Hypergraph Colouring and Lattice Point Enumeration}
\author{Dominic Welsh\\ University of Oxford}
\date{22 March 1999}

\begin{document}

\maketitle

\begin{center} {\bf Abstract} \end{center}

I shall present the results of some recent work with Geoffrey Whittle on a
problem which originated in trying to extend classical theory of the Tutte
polynomial of a graph to wider classes of structures.  This we do by focusing
on an equivalent version of the Tutte polynomial --- what we call the {\it bad
colouring polynomial} of a graph, and denote by $B(G;\lambda, s)$.  It is
defined by
        \[B(G;\lambda,s)=\sum^{|E|}_{k=0}b_k(G,\lambda)s^k\]
where $b_k(G,\lambda)$ denotes the number of ways of colouring $G$ with
$\lambda$-colours so that exactly $k$ edges are {\it bad}, that is have their
end points the same colour.  Equivalently, $B(G;\lambda,s)$ is recognisable as
a mildly distorted version of the partition function $Z$ of the standard
$Q$-state Potts
model on $G$.  The correspondence being obtained by the substitution
$\lambda\rightarrow Q$,
$s$ to inverse temperature.

Instead of the usual delete/contract recursion of the Tutte polynomial, the bad
colouring polynomials $b_k(G;\lambda)$ satisfy for all $k\geq 1$, the recursion
     \[b_k(G;\lambda)=b_k(G^\prime_e;\lambda)-b_k(G^{\prime\prime}_e;\lambda)+b_{k-1}(G^{\prime\prime}_e,\lambda).\]
Here $G^\prime_e,G^{\prime\prime}_e$ have their usual meaning of $G$ delete
(respectively contract) $e$.  We realised that this relation, which we
call the {\it 3-term recurrence} is satisfied by several other naturally occurring
counting functions.  We develop such a theory for the very general concept of a
{\it configuration} $Q$.  This is just a pair $(E,f)$ with $E$ a finite set and
$f:2^E\rightarrow{\mathbb Z}$ satisfying
\begin{enumerate}
\item[(i)] $f(\emptyset)=0$,
\item[(ii)] $f(A)\leq f(B)$ whenever $A\subseteq B$.
\end{enumerate}

Much of the standard theory goes through and applications include:
\begin{enumerate}
\item[a)] Counting the number of lattice points belonging to exactly $j$
members of an arrangement of subspaces in $AG(n,q)$.
\item[b)] Extending the theory of the Tutte polynomial from graphs to general hypergraphs.
\item[c)] Developing further the theory of the classical critical problem developed
by Crapo and Rota 1970.
\end{enumerate}

At the same time as we were working on these purely abstract
problems we were also considering the highly applicable problem of assigning
radio channel frequencies.  This is a problem of huge commercial significance,
which has received a great deal of interest over the last few years.  
One of the nicer
applications of the theory that we have developed is to a
counting problem in this area.  

\begin{center} {\bf Reference} \end{center}

\begin{enumerate}
\item[[1.]]  Dominic Welsh and Geoffrey Whittle, Arrangements, Channel
Assignments and Associated Polynomials, {\it Advances in Applied Mathematics},
1999 (to appear).
\end{enumerate}


\end{document}





