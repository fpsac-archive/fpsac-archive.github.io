\magnification=\magstep1
\input amstex
\input pictex
\documentstyle{amsppt}

\hsize=5in
\vsize=7.5in
\leftheadtext{George E. Andrews}
\rightheadtext{Stacked Lattice Boxes and Theorems}
\def\c{\cite}
\def\fr{\frac}
\def\call{\Cal L}

\topmatter
\title
        Stacked Lattice Boxes and Theorems of Liouville,
        Bell and MacMahon
\endtitle
\author
by \\
\ \ \\
George E. Andrews
\endauthor
\endtopmatter

\document
\baselineskip 20pt

This is a brief summary of the work presented in \c1 together
with some comments about more recent work.

Arithmetically we are considering $L_k(n)$, the number of
solutions of
$$
        n = x_1 x_2 + x_2 x_3 + x_3 x_4 + \cdots + x_k x_{k+1}
\tag1
$$
in positive integers $x_i$.  In the mid-nineteenth century,
Liouville developed a mysterious method for studying such
problems.  I say ``mysterious'' because he stated all his
central theorems without proof.  However numerous mathematicians
(most notably Humbert, Pepin and Bell) managed to find proofs
of the Liouville Theorems.  The theorems apparently have their
genesis in the theory of elliptic functions (as Humbert \c6
observed); however, Pepin \c9 was able to develop elementary
methods to prove many of Liouville's results.

According to Bell \c4, the last result of Liouville \c7 to
be proved was equivalent to the assertion that
$$
        L_4(n) = D_2(n) - n \,D_0(n) - D_{0,0}(n)\,,
\tag2
$$
where
$$
        D_{i_1 i_2\dots i_r}(n) = \sum_{\Sb j_1 + j_2 +\cdots +
        j_r = n  \\ j_1,\dots,j_r \geqq 1 \endSb}
        D_{i_1}(j_1) D_{i_2}(j_2) \cdots D_{i_r} (j_r)\,,
\tag3
$$
and
$$
        D_i(n) = \sum_{j|n} j^i\,.
\tag4
$$

The phrase ``stacked lattice boxes'' derives from the fact
that a solution of (1) may be viewed as yielding $n$ as the area
of the following figure
$$
\hbox{{\hskip -2in}
\vbox{\vskip .3in{Figure 1}}
\hbox{{\hskip -3.5in} 
\putrule from -.3 0 to 3 0
\setcoordinatesystem units <.75in,.75in>
\putrule from 0 -.3 to 0 2.4
%\linethickness=1pt 
\putrule from -.3 0 to 3 0
\putrectangle corners at 0 .3 and .75 0
\putrectangle corners at 0 .3 and .75 1
\putrectangle corners at .75 1 and 1.75 .3
\putrectangle corners at .75 1.3 and 1.75 .3
\putrectangle corners at 1.75 1.3 and 2.5 1
\put {$x_1\Big\{$} at -.15 .15 
\put {$x_3\Bigg\{$} at -.15 .65
\put {$x_5\Big\{$} at .6 1.15  
\put {$\undersetbrace{x_2}\to{\ \ \ \ \ \ \ \ \ \ \ }$} at .4 -.15  
\put {$\undersetbrace{x_6}\to{\ \ \ \ \ \ \ \ \ \ \ \; }$} at 2.15 .85
\put {$\undersetbrace{x_4}\to{\ \ \ \ \ \ \ \ \ \ \ \ \ \ \ }$} at 1.25 .15  }}
$$

In addition to (2), Liouville's ``last'' theorem, one can derive
fairly easily (with a little help from MacMahon \c8), that
$$
        L_1 (n) = D_0(n)
\tag5
$$
$$
        L_2 (n) = D_1 (n) - D_0 (n)
\tag6
$$
$$
        L_3(n) = \fr12 D_{0,0}(n) + \fr12 D_0(n) - \fr12 
        D_1(n)\,.
\tag7
$$

Our approach to these questions is to use $q$-series extensively
following the ideas of W. N. Bailey \c3.  This allows us to prove
$$
\align
        L_5(n) & = \fr16 D_{0,0,0}(n) + D_{0,0}(n) + \fr12
        D_{0,1}(n)
        \\
        & \qquad \left(2n - \fr16\right) D_0(n) - \fr{11}{6} D_2(n)\,.
        \tag8
\endalign
$$

These discoveries dovetail interestingly with the study 
(begin in \c2) of the $q$-series:
$$
        M_k(q) = \sum_{n=1}^{\infty}
        \fr{(-1)^{n-1} q^{\binom{n+1}{2}}}{(1 - q)(1 - q^2)
        \cdots (1 - q^{n-1})(1 - q^n)^{k+1}}\;.
\tag9
$$

Let us denote the generating function for $L_k(n)$ by $\call_k(q)$.

Using the methods of \c2, we can show that
$$
\align
        M_1(q) & = \call_1(q)    \tag10  \\
        M_2(q) & = \call_1(q) + \call_2(q) + \call_3(q)\,,
        \tag11
\endalign
$$
and
$$
        M_3 (q) = \call_1(q) + 2\call_2(q) + 3\call_3(q) + 2\call_4(q)
        + \call_5(q)\,.
\tag12
$$

The table of coefficients in these identities, namely
$$
\gathered
        1  \\
        1\quad 1\quad 1  \\
        1 \quad 2\quad 3\quad 2\quad 1
\endgathered
$$

suggests the trinomial coefficients \c{5; p. 78}.  However
if we examine the next suggested identity we find instead
$$
\align
        M_4(q) = & \call_1(q) + 3\call_2(q) + 6 \call_3(q)
        + 7 \call_4(q)
        \\
        & + 6 \call_5(q) + 3\call_6(q) + \call_7(q)   \\
        & - (q^7 + 2q^8 + 6q^9 + 11q^{10} + 22q^{11} 
        + 33q^{12} + 57 q^{13} + \dots )  \tag13
\endalign
$$

So having surveyed the world of the mysterious Liouville 
identities, we are led to a new ``Liouville mystery.''

\Refs
\ref
  \no 1
  \by G. E. Andrews
  \paper Stacked lattice boxes
  \jour Annals of combinatorics
  \finalinfo (to appear)
\endref
\ref
  \no 2
  \by G. E. Andrews, D. Crippa, K. Simon
  \paper $q$-Series arising from the study of random graphs
  \jour S.I.A.M. J. Discr. Math.
  \vol 10
  \yr 1997
  \pages 41--56
\endref

\ref
  \no 3
  \by W. N. Bailey
  \paper An algebraic identity
  \jour J. London Math. Soc.
  \vol 11
  \yr 1936
  \pages 156--160
\endref


\ref
  \no 4
  \by E. T. Bell
  \paper The form $wx + xy + yz + zu$
  \jour Bull. Amer. Math. Soc.
  \vol 42
  \yr 1936
  \pages 377--380
\endref

\ref
  \no 5
  \by L. Comtet
  \paper Advanced Combinatorics
  \paperinfo Reidel, Dordrecht, 1974
\endref

\ref
  \no 6
  \by G. Humbert
  \paper D\'emonstration analytique d'une formule de Liouville
  \jour Bull. des Sc. Math (2)
  \vol 34
  \yr 1910
  \pages 29--31
\endref


\ref
  \no 7
  \by J. Liouville
  \paper $x_1 x_2 + x_2 x_3 + x_3 x_4 + x_4 x_5$
  \paperinfo Comptes Rendus, 62 (1866), 714 and Journal de
                Math\;ematiques (2), 12 (1867), 47--48
\endref


\ref
  \no 8
  \by P. A. MacMahon
  \paper Divisors of numbers and their continuations in the theory
        of partitions
  \jour Proc. London Math. Soc.
  \finalinfo (1919), 75--113 (Reprinted: The Collected Papers
        of P. A. MacMahon, Vol. II, G. E. Andrews ed., M.I.T.
        Press, Cambridge, 198  
\endref

\ref
  \no 9
  \by T. Pepin
  \paper Sur quelques formules d'Analyse utiles dans la The'orie
        des nombres
  \jour J. de Math. Pures et Appl.
  \vol 4
  \yr 1888
  \pages 83--127
\endref

\endRefs
\enddocument


