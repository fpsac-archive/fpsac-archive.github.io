\documentstyle[12pt]{article}

\setlength{\textheight}{200mm}
\setlength{\textwidth}{145mm}
\setlength{\unitlength}{2mm}
\setcounter{footnote}{1}
%\newfont{\msa}{msam10}
\newsavebox{\squarebar}
\newsavebox{\blacksquarebar}
\savebox{\squarebar}{\framebox(1,1){}}
\savebox{\blacksquarebar}{\rule{\unitlength}{1.1\unitlength}}

\def\phi{\varphi}
\def\rank{{\rm rank}\,}
\def\cha{{\rm char}\,}
\def\Ad{{\rm Ad}\,}
\def\ad{{\rm ad}\,}
\def\Ker{{\rm Ker}}
\def\Orb{{\rm Orb}}
\def\Aut{{\rm Aut}}
\def\id{{\rm id}}
\def\thepage{}

\makeatletter


\def\Let@{\relax\iffalse{\fi\let\\=\cr\iffalse}\fi}
\def\vspace@{\def\vspace##1{\noalign{\vskip##1\relax}}}
\def\multilimits@{\bgroup\vspace@\Let@
 \baselineskip\fontdimen10 \scriptfont\tw@
 \advance\baselineskip\fontdimen12 \scriptfont\tw@
 \lineskip\thr@@\fontdimen8 \scriptfont\thr@@
 \lineskiplimit\thr@@\fontdimen8 \scriptfont\thr@@
 \vbox\bgroup\ialign\bgroup$\scriptstyle{##}$\hfil\cr}
\def\Sb{_\multilimits@}
\def\endSb{\cr\egroup\egroup\egroup}




%\def\seq#1#2{#1_1,\ldots,#1_{#2}}
\def\seq#1#2{#1_1,\ldots,#1_{#2}}
\def\pa#1#2{{\textstyle\partial #1\over\textstyle\partial #2}}
\def\subst#1#2{\hbox{$#1$}
               \hbox{\vphantom{$\hbox{$#1$}\Sb #2 \endSb$}\vline\,}
               \hbox{\vphantom{$\hbox{$#1$}$}}\Sb #2 \endSb}
\def\pp#1{\mbox{$#1$-}}
\def\asprod#1#2{\mathop{\prod\nolimits^*_{#2}}\limits_{#1}}


\def\ut{\widetilde{u}}

\def\epsilon{\varepsilon}

\def\seq#1#2{#1_1,\ldots,#1_{#2}}
\def\pp#1{\mbox{$#1$-}}

\makeatother
\def\blacksquare{\usebox{\blacksquarebar}}

\def\finishsymb#1{{\unskip\nobreak\hfil\penalty50
    \hskip2em\hbox{}\nobreak\hfil{#1}
    \parfillskip=0pt \finalhyphendemerits=0 \par}}

\newenvironment{proof}{\rm \trivlist
    \item[\hskip \labelsep{\bf Proof}\hskip 2mm]}{
    \finishsymb\blacksquare\endtrivlist}



\newtheorem{lemma}{Lemma}
\newtheorem{problem}{Problem}
\newtheorem{theorem}{Theorem}
\newtheorem{question}{Question}
\newtheorem{cor}{Corollary}
\newtheorem{proposition}{Proposition}
\newtheorem{remark}{Remark}
\newtheorem{example}{Example}
\def\thecor{}
\def\thetheorem{}
\def\theexample{}
\def\theproposition{}


  \begin{document}
  
  \newcommand{\msachoice}[1]{{\mathchoice
    {\mbox {\tenmsa \char #1}}{\mbox {\tenmsa \char #1}}%
    {\mbox {\sevenmsa \char #1}}{\mbox {\sevenmsa \char #1}}}}
  \newcommand{\kd}{\mbox{$\scriptstyle\blacksquare$}}
  \newcommand {\df}{\mbox{-}\nolinebreak\hspace{0pt}}

  %\setcounter{footnote}{1}
  
  \author{\mbox{\begin{tabular}{c}
  G\'erard Duchamp ~and~ Alexander A.~Mikhalev
   \end{tabular}}}

  \title{Graded shuffle algebras\protect\\
  over fields of prime characteristic}
  \date{}

\maketitle


\vspace{1cm}

{\bf Abstract}.~ We describe
the structure of
the free associative algebra over a field of prime characteristic
with the new multiplication given by the super shuffle product.

 \vspace{1cm}

{\bf R\'esum\'e}.~ Nous d\'ecrivons
la structure de l'alg\`ebre libre
sur un corps de caract\'eristique premi\`ere quand elle est munie 
de la nouvelle multiplication donn\'ee par le super produit de shuffle.


\vspace{2cm}

\section{Preliminaries}


Let $G$ be an abelian monoid, $K$ a commutative associative ring
with identity element, $char K\neq 2$, $U(K)$ the group of  invertible
elements of $K$,
$\epsilon : G\times G \rightarrow  U(K)$ a skew symmetric bilinear form
(a bicharacter), that is
$$\epsilon (g_{1}+g_{2},h)=\epsilon (g_{1},h) \epsilon (g_{2},h), \, \,
\epsilon (g,h_{1}+h_{2})=\epsilon (g,h_{1}) \epsilon (g,h_{2}),$$
$$\epsilon (g,h) \epsilon (h,g) =1, \, \, \epsilon (g,g)=\pm 1$$ for all
$g, g_{1}, g_{2}, h, h_{1}, h_{2} \in G$,
$$G_{-} = \{ g \in G \mid \epsilon (g,g)=-1 \}, \, \,
G_{+} = \{ g \in G \mid \epsilon (g,g)=+1 \}.$$

Let $X=\cup_{g\in G}X_g$ be a  $G$-graded  set,  i.e.  $X_g\cap  X_f=
\emptyset$ for $g\not=f,\ d(x)=g$ for $x\in X_g$; let also
$S(X)$  be  the  free monoid
of associative words on $X$.
For $u=x_1\dots x_n\in  S(X)$,  $x_i\in  X$,  we
consider the word length $l(u)=n$, and set $d(u)=\sum_{i=1}^nd(x_i)\in G$,
$S(X)_g=\{u\in S(X) \mid d(u)=g\}$.
Let $A(X)_g$ ($g\in G$) be  the  $K$-linear  spans  of  the  subsets
$S(X)_g$ in the free associative algebra $A(X)$. Then
$A(X)=\oplus_{g\in  G}A(X)_g$ is the free $G$-graded associative $K$-algebra
on $X$.

Suppose that the set $X=\cup_{g\in G}X_g$ is totally ordered  and  the
set $S(X)$ is ordered lexicographically, i.e.  for  $u=x_1\dots  x_r$
and $v=y_1\dots y_m$ where $x_i,y_j\in X$  we  have  $u<v$  if  either
$x_i=y_i$  for  $i=0,1,\dots,t-1$  and  $x_t<y_t$  or  $x_i=y_i$   for
$i=1,2,\dots,m$ and $r>m$.

A word $u \in S(X)$ is said to be {\it regular\/} if $u \ne 1$ and it follows
from $u=ab, a,b \in S(X), a,b \ne 1$, that $u=ab>ba$ (this condition
is equivalent to the condition $u=ab>b$). A word $w \in S(X)$ is said to be
$s$-{\it regular\/} if either $w$ is a regular word, or $w=uu$ with $u$ a
regular word, $d(u) \in G_-.$
Let $p$ be a prime number not equal to $2$. A word $w\in S(X)$ is said to be
$ps$-regular if it is either an $s$-regular word, or $w=u^{p^t}$ with
$t\in {\bf N}$, $u$ an $s$-regular word, $d(u)\in G_+$.

Shuffle algebras were introduced by R.~Ree in \cite{Ree1,Ree2}. Details of
applications of shuffle algebras to free Lie algebras may be found in
\cite{Reut}.

Let $V$ abd $Z$ be $G$-graded sets, $V\cap Z= \emptyset$, $v_1, \ldots , v_k$
pairwise distinct elements of $V$, $z_1 \ldots , z_l$ pairwise distinct
elements
of $Z$. We say that a word $w\in S(V\cup Z) \setminus 1$ is a
shuffle word of the words
$v_1 \cdots v_k$ and $z_1 \cdots z_l$ if   $w$ has the multidegree
$$m(w)= v_1 + \cdots  + v_k  + z_1 \cdots z_l$$ and
$$\subst{w}{z_1=1,  \ldots , z_l=1}= v_1 \cdots  v_k;
\, \, \subst{w}{v_1=1, \ldots , v_k=1}= z_1 \cdots z_l.$$
The parity $\sigma (w)$ of a shuffle word $w$ of the words
$v_1 \cdots v_k$ and $z_1 \cdots z_l$ is the sum of all
$\varepsilon (z_i , v_j)$
such that $z_i$ is situated before $v_j$ in $w$.

Let $X$ be a $G$-graded set.
For
$x_{i_1}, \ldots x_{i_k}, x_{j_1} , \ldots , x_{j_l} \in X$
we define the shuffle product $(x_{i_1} \cdots  x_{i_k}) * (x_{j_1}
\cdots x_{j_l})$
as the following linear combination of words of multidegree
$x_{i_1} + \ldots + x_{i_k} + x_{j_1}  \ldots + x_{j_l}$:
$$(x_{i_1} \cdots  x_{i_k}) * (x_{j_1} \cdots x_{j_l})=
\subst{\sum_w \sigma (w) w}{v_s = x_{i_s}, s=1, \ldots , k;
z_t = x_{j_t}, t=1, \ldots , l}$$
where $w$ is running through all shuffle words of the words
$v_1 \cdots v_k$
and $z_1 \cdots z_l$ with $d(v_s)=d(x_{i_s}), d(z_t)=d(x_{j_t})$.
Taking $1 * u = u * 1 = u$ for all
$u\in S(X)\setminus 1$ and
extending $*$ on  the free $G$-graded associative algebra $A(X)$
on $X$ over
a commutative associative ring $K$ with the identity element
by linearity, we define the shuffle product $*$ on
$A(X)$. Then $A(X)$ with this product is an $\varepsilon$-commutative and
associative algebra (see \cite{Ree2}).


One can
define the shuffle product $*$ on $A(X)$ in the following way:
$$1*u=u*1=u; \, \, \, \, \, (xu)*(yv) = x(u*(yv)) + \varepsilon(y,x)
\varepsilon(y,u)y((xu)*v)$$
for all $x,y \in X, \, u,v \in S(X)\setminus 1$
(with the extension $*$ on $A(X)$ by linearity).

If we consider $A(X)$ as the universal enveloping algebra of the free
color Lie superalgebra $L(X)$, then $*$ is the adjoint
of coproduct $\delta$ of $A(X)$.

In fact, for this definition (and associativity of this law),
$\varepsilon$ need only be bilinear and (see \cite{NCSF3}) is the
unique law for which $1$ is neutral and the operators
$(x^{-1})_{x\in X}$ (i.e. the adjoints of the multiplication by
letters) are superderivations.

Let $Y$ be a $G$-graded set,
and let $J$ be the two-sided ideal of the free $G$-graded
associative algebra $A (Y)$
generated by the $G-$homogeneous elements
$$ab-\varepsilon (d(a),d(b))ba,$$ where $a,b$ are
elements of $ S ( Y )$, and
$K_{\varepsilon}[Y]=A(Y)/J$. Then the algebra $K_{\varepsilon}[Y]$ is the
free $\varepsilon$-commutative associative $K$-algebra with the set
$Y$ of free generators.
If $G_- =\emptyset$, then  $K_{\varepsilon}[Y]$   is the algebra of quantum
polynomials. If $\varepsilon\equiv 1$, then  $K_{\varepsilon}[Y]$  is
the usual polynomial algebra. In general case the algebra
$K_{\varepsilon}[Y]$  is the universal enveloping algebra of a
Abelian color Lie superalgebra
(see \cite{BMPZ}, \cite{MZBook}).


\section{Main result}

D. Radford in \cite{Radford} proved that in the case of trivial grading group
the free associative algebra as a shuffle algebra is the free commutative
associative algebra with a set of free generators consisting of Lyndon words
(see also \cite{Reut}).
A.~A.~Mikhalev and A.~A.~Zolotykh showed in \cite{MiZo1,MiZo2}
that if $K$ is a ${\bf Q}$-algebra, then $A(X)$ with the shuffle product
$*$ is the free $\varepsilon$-commutative algebra with a set of free
generators consisting of $s$-regular words.

We consider the case where $K$ is a field, $char K = p>2$.
Let $R(X)$ be the set of $ps$-regular words of $S(X)$, $R(X)=R_+ \cup R_-$,
where
$$R_+=\{ r\in R(X) \mid d(r)\in G_+ \}, \, \, \,
R_-=\{ r\in R(X) \mid d(r)\in G_- \}.$$

By $K_{\varepsilon}[R(X)]$ we denote the free $\varepsilon$-commutative
$K$-algebra generated by the set $R(X)$.

\begin{theorem}
Let
$K$ be a field, $char K = p>2$. Then the free $G$-graded associative
algebra $A(X)$ with the new
multiplication given by the shuffle product $*$ is isomorphic to the
factor algebra $K_{\varepsilon}[R(X)]/I$, where
$I$ is the ideal of $K_{\varepsilon}[R(X)]$ generated by the set
$\{ u^p \mid u\in R_+ \}$.
In particular, if $G=\{ e \}$, then the algebra
$A(X)$ with the shuffle multiplication is isomorphic to the algebra of
$p$-reduced polynomials on regular (or on Lyndon) words.
\end{theorem}


\begin{thebibliography}{99999}

  \bibitem[BMPZ]{BMPZ}
  Yu.~A.~Bahturin, A.~A.~Mikhalev, V.~M.~Petrogradsky, and M.~V.~Zaicev,
  {\it Infinite Dimensional Lie Superalgebras\/}. Walter de Gruyter Publ.,
  Berlin--New York, 1992.

\bibitem[DKKT]{NCSF3}  G.~Duchamp, A.~Klyachko, D.~Krob, and J.-Y.~Thibon,
{\it Noncommutative symmetric functions III: Deformation of Cauchy and
convolution algebras\/}.
Disc. Math. and Th. Comp. Sci. {\bf 1} (1997), \hbox{159--216}.

\bibitem[MZ1]{MiZo1}
A.~A.~Mikhalev and A.~A.~Zolotykh,
{\it Natural basis for $\epsilon$-shuffle algebras\/}.
7th Conf. {\sl Formal Power Series and
Algebraic Combinatorics\/},
Univ. Marne-la-Vall\'ee, 1995,
\hbox{423--426}.

\bibitem[MZ2]{MiZo2}
A.~A.~Mikhalev and A.~A.~Zolotykh,
{\it Bases of free super shuffle algebras\/}.
Uspekhi  Matem. Nauk {\bf 50} (1995), no.~1, \hbox{199--200}.
English translation: Russian Math. Surveys {\bf 50} (1995), no.~1,
\hbox{225--226}.

 \bibitem[MZ3]{MZBook}
  A.~A.~Mikhalev and A.~A.~Zolotykh, {\it Combinatorial Aspects
  of  Lie Superalgebras\/}. CRC Press, Boca Raton, New York, 1995.

\bibitem[Rad]{Radford}
D.~E.~Radford, {\it A natural ring basis for the shuffle algebra and an
application to group schemes\/}. J.~Algebra {\bf 58} (1979),
\hbox{432--454}.

\bibitem[Ree1]{Ree1}
R.~Ree, {\it Lie elements and an algebra associated with shuffles\/}.
Annals of Math. {\bf 68} (1958), \hbox{210--220}.

\bibitem[Ree2]{Ree2}
R.~Ree, {\it Generalized Lie elements\/}. Canadian J. Math. {\bf 12}
(1960), \hbox{493--502}.


\bibitem[Reu]{Reut}
Ch.~Reutenauer, {\it Free Lie Algebras\/}. Clarendon Press, Oxford,
1993.

\end{thebibliography}

\end{document}

