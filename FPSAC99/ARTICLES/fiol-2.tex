




%% A LaTeX2e file for a 6 page document%%
\documentclass[11pt]{article}

\oddsidemargin .5cm
\evensidemargin .5cm
\textwidth=15.2truecm
\textheight=21truecm
\unitlength=1cm
\parskip=3pt
\baselineskip=15pt 

\newtheorem{theo}{Theorem}[section]
\newtheorem{propo}[theo]{Proposition} 
\newtheorem{lema}[theo]{Lemma}
\newtheorem{defi}[theo]{Definition}
\newtheorem{example}[theo]{Example}  
\newtheorem{coro}[theo]{Corollary}
\newtheorem{guess}[theo]{Conjecture}
\def\proof{{$\boldmath Proof.$}\hskip 0.3truecm}
\def\endproof{\ \  $\Box$\vskip .2cm} 

\def\A{\mbox{\boldmath $A$}}
\def\B{\mbox{\boldmath $B$}} \def\Bo{{\cal B}}
\def\C{\mbox{\boldmath $C$}}
\def\D{\mbox{\boldmath $D$}}
\def\E{\mbox{\boldmath $E$}}
\def\G{\Gamma}
\def\I{\mbox{\boldmath $I$}}
\def\J{\mbox{\boldmath $J$}}
\def\K{\mbox{\boldmath $K$}}
\def\M{{\cal M}}
\def\O{\mbox{\boldmath $O$}}
\def\U{\mbox{\boldmath $U$}}
\def\d{\partial}
\def\dist{\mathop{\rm dist}\nolimits}
\def\ecc{\mathop{\rm ecc}\nolimits}
\def\exc{\mathop{\rm exc}\nolimits}
\def\e{\mbox{\boldmath $e$}}
\def\f{\mbox{\boldmath $f$}}
\def\j{\mbox{\boldmath $j$}}        
\def\u{{\mbox {\boldmath $u$}}}
\def\v{{\mbox {\boldmath $v$}}}
\def\w{{\mbox {\boldmath $w$}}}
\def\x{\mbox{\boldmath $x$}}
\def\z{\mbox{\boldmath $z$}}
\def\vecphi{\mbox{\boldmath $\phi$}}
\def\vecnu{\mbox{\boldmath $\nu$}}
\def\vecrho{\mbox{\boldmath $\rho$}}
\def\vec0{\mbox{\bf 0}}
\def\ev{\mathop{\rm ev}\nolimits}
\def\Ker{\mathop{\rm Ker}\nolimits}
\def\sp{\mathop{\rm sp}\nolimits}
\def\tr{\mathop{\rm tr}\nolimits}

\begin{document}
\title{Algebraic Characterizations  \\  
of  Distance-Regular Graphs
\thanks{ 
{\it 11th Inter. Conf. on Formal Power Series and Algebraic
Combinatorics, Barcelona, 1999.} 
\newline
 Work supported in part by the Spanish Research Council
(Comisi\'on Interministerial de Ciencia y Tecnolog\'\i a, CICYT)
     under projects  TIC 94-0592 and TIC 97-0963. 
}}     
\author{M.A. Fiol 
\\ {\small Departament de Matem\`atica Aplicada i Telem\`atica} \\
{\small Universitat Polit\`ecnica de Catalunya} \\ {\small Jordi
Girona, 1--3 , M\`odul C3, Campus Nord }\\ {\small 08034 Barcelona,
Spain; email: {\tt fiol@mat.upc.es}}   }
\date{}
\maketitle

\begin{abstract}
We survey some old and some new characterizations of
distance-regular graphs, which depend on information retrieved
from their adjacency matrix. 
\end{abstract}
Distance-regular graphs were introduced by Biggs around 1970,  by
changing a symme\-try-type requirement, that of
distance-transitivity, by a regularity-type condition concerning the
cardinality of some vertex subsets. Thus, one  common
way of looking at distance-regularity is to ``hang" the graph from
a given vertex and  observe the resulting different layers;
that is, the subsets of vertices at a given distance from the root.
Then, if vertices in the same layer are ``indistinguishable" from
each other, and the whole configuration does not depend on  the
chosen vertex, the graph is said to be distance-regular.  More
precisely, a (connected) graph
$G=(V,E)$ with diameter $D$ is {\it distance-regular} if and only if,
for any two vertices $u,v\in V$ at distance
$\dist(u,v)=k$, $0\le k\le D$, the numbers $c_k$, $a_k$, $b_k$, of
vertices which are adjacent to
$v$, and at distance $k-1$, $k$, $k+1$ respectively  from $u$, do 
not depend on the chosen vertices $u$ and $v$, but only on their
distance $k$.

Since their introduction, distance-regular graphs and
their main  generalization, the association schemes, 
have proved to be a key concept in algebraic combinatorics, having
important connections with other branches of mathematics, such as
geometry, coding theory, group theory, design theory, as well as
to other areas of graph theory. As stated in the preface of the
comprehensive textbook of Brouwer, Cohen and Neumaier
\cite{bcn89}, this is because most finite objects bearing ``enough
regularity"  are closely related to certain
distance-regular graphs.
Apart from the above definition, there are other well-known
combinatorial characterizations of distance-regular, such as the
following:
\begin{itemize}
\item[(a)]
A graph $G$ with diameter $D$ is distance-regular if and only if, for 
any pair of vertices $u,v\in V$ and integers $0\le i,j\le D$, the
number of vertices which are at distance $i$ from $u$ and at
distance $j$ from
$v$ only depends on 
$\dist(u,v)$. 
\item[(b)]
A graph $G$ is distance-regular if and only if for any integer $k\ge 0$, the
number of walks of length $k$ between two vertices $u,v\in V$ only
depends on $\dist(u,v)$. 
\end{itemize}

In this work, we aim to survey some characterizations  of
distance-regular graphs which are of an algebraic nature. Such
characterizations rely mainly on the adjacency matrix $\A$ of the
graph and/or some of its invariants, such as its spectrum
$$
\sp G:=\sp \A=\{\lambda_0^{m_0},\lambda_1^{m_1},\ldots,
\lambda_d^{m_d}\}
$$
where the eigenvalues $\lambda_l$, $0\le l\le d$ are in decreasing
order and the superscripts denote multiplicities, or their
corresponding eigenspaces
$$
{\cal E}_l:=\Ker(\A-\lambda_l\I)\qquad (0\le l\le d).
$$

Thus, as the number $a_{uv}^k$ of walks of length $k$ between
vertices $u,v$ is no more than the $uv$-entry of the $k$-th
power of $\A$, we can see (b) as a simple characterization in terms
of the adjacency matrix of $G$. In fact,   it can be
easily proved that we do not  need to impose the invariance
condition on all such numbers of walks. For instance, if $G$ is
regular we have the following result:
\begin{itemize}
\item[(c)]
A  regular graph $G$ with diameter $D$ is distance-regular if and
only if for any two vertices $u,v\in V$ at distance $\dist(u,v)=k$,
$1\le k\le D$, the numbers of walks $a_{uv}^k$ and $a_{uv}^{k+1}$
only depend on
$k$.
\end{itemize}

Another characterization involving adjacency matrices states that
a graph  $G$ with  diameter
$D$ is   distance-regular if and only if, for any $k$, the {\it
distance-$k$ matrix}
$\A_k$ ---whose $uv$-entry is $1$ if $\dist(u,v)=k$ and $0$
otherwise--- is a polynomial of degree $k$ in $\A$; that is:
\begin{equation}
\label{all-dist-pol}
\A_k=p_k(\A)\qquad (0\le k\le D).
\end{equation}
Notice that the existence of such $p_0$ and $p_1$ is always
guaranteed since $\A_0=\I$ and $\A_1=\A$. 
In general, the polynomial
$p_k$ is referred to as the {\it distance-$k$ polynomial} of the
graph. See,  for instance, Biggs \cite{b93} or Brouwer et.al.
\cite{bcn89}. In fact, if  every vertex $u\in V$ has the maximum
possible eccentricity ``allowed by the spectrum"; that is, the number
of distinct eigenvalues minus one: $\ecc(u)=d$, the existence of the
highest degree distance polynomial suffices, and hence we 
need only to require that
\begin{equation}
\label{D-dist-pol}
 \A_D=p_D(\A).
\end{equation}
This was proved in the context of pseudo-distance-regularity
---a generalization of distance-regularity that makes sense even for
non-regular graphs--- by Garriga, Yebra and the author in
\cite{fgy96-2}. 

For each eigenvalue $\lambda_l$, $0\le l\le d$, let $\U_l$ be the
matrix whose columns form an orthonormal basis of its eigenspace
${\cal E}_l$. The {\it (principal) idempotents} of $\A$ are
the matrices $\E_l:=\U_l\U_l^\top$  representing the orthogonal
projections onto  ${\cal E}_l$. Accordingly, such matrices
satisfy  $\E_0+\E_1+\cdots +\E_d=\I$ (as expected, since the sum of
all orthogonal projections gives the original vector), and  the
so-called {\it spectral decomposition theorem}
$$
\sum_{l=0}^d \lambda_l\E_l=\A
$$
(see for instance, Godsil \cite{g93}).
Since both $\{\I,\A,\ldots,\A^d\}$ and 
$\{\E_0,\E_1,\ldots,\E_d\}$ are basis of ${\cal A}(\A)$, the {\it
adjacency\/} or {\it Bose-Mesner algebra} of  matrices
which are polynomials in $\A$, it is not strange to have
characterizations of distance-regularity in terms of the entries of
the above idempotents. These numbers are called the {\it
crossed $uv$-local multiplicities} of  $\lambda_l$, and  they
are denoted by $m_{uv}(\lambda_l)$. Notice that, if $\z_{ul}$
represents the orthogonal projection of the $u$-canonical vector
$\e_u$ on ${\cal E}_l$, the crossed local multiplicities correspond
to the scalar products:
$$
m_{uv}(\lambda_l):=(\E_l)_{uv}
=\langle \E_l\e_u,\e_v\rangle=\langle
\E_l\e_u,\E_l\e_v\rangle=\langle\z_{ul},\z_{vl}\rangle
\qquad (u,v\in V).
$$
For instance, if the graph is regular, then the eigenvector
of $\lambda_0$ is the all-$1$ vector $\j$, and the above gives
$m_{uv}(\lambda_0)=\langle\frac{1}{n}\j,
\frac{1}{n}\j\rangle =1/n$ for any $u,v\in V$.
Moreover,  Godsil \cite{g88} proved that, if $G$ is distance-regular,
then for any given eigenvalue $\lambda_l$, $0\le l\le d$, the crossed
$uv$-local multiplicity $m_{uv}(\lambda_l)$  depends only on the
distance $\dist(u,v)$, and it is not difficult to realize that the
converse is also true. In fact, in the spirit of characterization (c),
we can prove the following result (see \cite{f99}):
\begin{itemize}
\item[(d)]
A regular graph $G$, with eigenvalues
$\lambda_0>\lambda_1>\cdots>\lambda_d$ and diameter
$D$, is distance-regular if and only if the crossed $uv$-local
multiplicities 
$m_{uv}(\lambda_1)$ and $m_{uv}(\lambda_d)$  depend only on the
distance
$\dist(u,v)=k$, for any $0\le k\le D$.
\end{itemize}

Of course, we may also ask for some characterizations involving
only the spectrum. Then, the question would now be: {\it Can we see
from the spectrum of a graph  whether it is distance-regular?}  In
this context, it  has been known for a long time that the answer  is
`yes' when $D\le 2$ and `not' if $D\ge 4$, whereas the case $D=3$
was  undecided until recently, when Haemers \cite{ha96} gave also a
negative answer.  Thus,  in general the spectrum is not sufficient to
assure distance-regularity and, if we  want to go further,  we must
require the graph to satisfy some additional conditions. In this
direction, Van Dam and Haemers 
\cite{vdh96} showed that, in the case $D=3$, such a condition could
be the number $n_d(u)$ of vertices at ``extremal distance" $D=d$
from each vertex $u\in V$. Then,  Garriga and the author
\cite{fg97} solved the general case, characterizing distance-regular
graphs as those regular graphs whose number of vertices at distance
$d$ from each vertex is what it should be (a number that depends
only on the spectrum of the graph).  
To be more precise, they proved that a (connected) regular graph $G$
with $n$ vertices and spectrum 
$\sp G =\{\lambda_0^{m_0},\lambda_1^{m_1},\dots,
\lambda_d^{m_d}\}$ is distance-regular if and only if  the 
number of  vertices at distance $d$ from each vertex $u$ is
\begin{equation}
\label{numeric}
n_{d}(u)
=n\left(\sum_{l=0}^{d}\frac{\pi_0^2}{m_l\pi_l^2}\right)^{-1}
\end{equation}
where the $\pi_l's$ are moment-like parameters defined by
$\pi_l :=\prod_{k=0,k\neq l}^{d} |\lambda_l-\lambda_k|$, $0\le l\le
d$. In fact, since the  right-hand expression in (\ref{numeric}) is the
value at $\lambda_0$ of a degree-$d$ polynomial,
the characterization (\ref{D-dist-pol}) can now be seen as a
corollary.

Notice that the cases $d=1,2$  of the above result are trivial, in the
sense that every (connected) regular graph $G$ with two or three 
different eigenvalues is distance-regular ($G=K_n$ if $d+1=2$ and
$G$ is strongly regular when $d+1=3$---see, for instance, Godsil
\cite{g93}). As we have already mentioned, the first
``non-trivial" case $d=3$ is due to Haemers and Van Dam
\cite{vdh96},  while the  case $n_d(u)=1$ (that is,  $2$-antipodal
graphs) was also studied  by Garriga, Yebra and the author in a
previous paper \cite{fgy98}.
In fact, in this case only the distinct
eigenvalues matter, as their multiplicities can be deduced from
them by the formulae $m_l=\pi_0/\pi_l$, $0\le l\le d$.
Furthermore, in this case we do not need to require regularity  
since it is  inferred by  condition (\ref{numeric}). Then it turns out
that a graph $G$ with eigenvalues
$\lambda_0>\lambda_1>\cdots>\lambda_d$ is a $2$-antipodal
distance-regular graph if and only if
\begin{equation}

\label{2-antipodal}
n_{d}(u)=n\left(\sum_{l=0}^{d}\frac{\pi_0}{\pi_l}\right)^{-1}=1
\end{equation}
for each  $u\in V$.
The graphs satisfying the second equality of (\ref{2-antipodal}),
that is $\sum_{l=0}^{d}(\pi_0/\pi_l)=n$, are called 
{\it boundary graphs} since they  satisfy  an extremal property that
arises from a bound for the diameter of a graph in terms of its 
distinct eigenvalues. Namely, it was proved in \cite{fgy96-1} that, 
if $G$ is regular and $\sum_{l=0}^{d}(\pi_0/\pi_l)>n$, then  
$D\le d-1$. 

From the result in (\ref{numeric}), some other spectral
characterizations have been given for special classes of
distance-regular graphs.  Thus, as a generalization of the above
$2$-antipodal case, it was proved in
\cite{f97} that  a regular graph $G$, with 
eigenvalues $\lambda_0>\lambda_1>\cdots>\lambda_d$, is
an $r$-antipodal distance-regular graph if and only if the distance
graph $G_d$ (that is, the graph whose adjacency matrix is $\A_d$) is
constituted by disjoint copies of the complete graph
$K_r$, with $r$ satisfying an expression in terms of
$n$ and the distinct eigenvalues. Namely, 
\begin{equation} 
\label{antipodal}
r=2n\left(\sum_{l=0}^d\frac{\pi_0}{\pi_l}\right)^{-1}.
\end{equation}
Note that the case $r=2$ corresponds to (\ref{2-antipodal}). 

Recall that a $\delta$-regular graph $G$ on $n$ vertices  is called
{\it $(n,\delta;a,c)$-strongly regular} if every pair of adjacent
(respectively, nonadjacent) vertices  $u,v$ have $a$ (respectively
$c$) common neighbours. Thus, if connected, a strongly regular graph
$G$ is the same as a distance-regular graph (with diameter
two).  Otherwise, it is known that $G$ is constituted by several
copies of $K_r$. Furthermore, a graph
$G$ with diameter $D=d$ is called {\it $(n,\delta;a,c)$-strongly 
distance-regular} if
$G$ is distance-regular and its distance-$d$ graph $G_d$ is strongly
regular with the indicated parameters.  Some known examples of
such graphs are the connected strongly regular graphs, with
$G_d=\overline G$ (the complement of $G$), and the $r$-antipodal
distance-regular graphs with $G_d=mK_r$ so that they are
$(n,\delta;r-1,0)$-strongly distance-regular graphs.  Hence, some
spectral conditions for a (regular or distance-regular) graph to be
strongly distance-regular have been already given above. In 
particular, notice that $2$-antipodal distance-regular graphs
characterized in (\ref{2-antipodal}) correspond to the case $a=c=0$.
In this context, the more general case $a=c$ was dealt with in 
\cite{fg99}, where one can find the following result. Let  $G$ be a
regular graph on $n$ vertices, with distinct eigenvalues
$\lambda_0>\lambda_1>\cdots>\lambda_d$, $d>1$.  Then $G$ is
$(n,\delta; c,c)$-strongly distance-regular  if and only if
\begin{equation}
\label{a=c}
n_{d}(u)=\frac{n(n-1)}{\left(\sum_{l=1}^{d}\frac{\pi_0}{\pi_l}
\right)^2+n-1}
\end{equation}
for every vertex $u\in V$. Moreover, in such a case, the above
parameters are $\delta=n_d:=n_{d}(u)$, $c=n_d(n_d-1)/(n-1)$, and
the multiplicity of eigenvalue $\lambda_i$ is 
\begin{equation}
\label{mult-extrem-bound-graph}
 m_i=\frac{\pi_0}{\pi_i}\sqrt{\frac{(n-1)n_d}{n-n_d}}
 =(n-1)\frac{\frac{1}{\pi_i}}{\sum_{l=1}^d \frac{1}{\pi_l}}
 \qquad (1\le i\le d). 
\end{equation}

The necessity of condition (\ref{a=c}) was proved by Van Dam
\cite{vd98} using the Laplace matrix of $G$ and  Haemers' method
of eigenvalue interlacing
\cite{ha95}. He also proved that (\ref{a=c}) is sufficient to assure
the strong regularity of
$G_d$ and that, in the case $d=3$, it also implies the
distance-regularity of $G$. In such a case, Van Dam also offered
examples of graphs satisfying the result. Namely,  the odd graph
$O_4$ (4-regular,
$n=35$, $n_3=18$), and the generalized hexagons $GH(q,q)$, with $q$
a prime power, ($(q+1)$-regular, $n=2(q+1)(q^4+q^2+1)$, $n_3=q^5$);
for a description of these graphs, see for instance
\cite{b93,bcn89}. On the other hand, notice that for the case
$n_{d}(u)=1$, studied in \cite{fgy98}, the values in
(\ref{numeric}) and (\ref{a=c}) coincide since, in
both cases,  $G$ must be a 2-antipodal distance-regular graph. 

Finally, for general values of $a$ and $c$, and $d=3$,
the following characterization has been recently given in
\cite{f98}.   A regular (connected) graph $G$, with $n$ vertices and
distinct eigenvalues
$\lambda_0>\lambda_1>\lambda_2>\lambda_3$, is strongly
distance-regular if and only if
$\lambda_2=-1$, and 
\begin{equation}
\label{main-theo}
n_3(u)=\frac{(n-\lambda_0-1)[\pi_0/(\lambda_0+1)-n(\lambda_0
+\lambda_1\lambda_3)]}
{\pi_0-n(\lambda_0+\lambda_1\lambda_3)}.
\end{equation}
for every vertex $u\in V$. (In this case, $a$ and $c$ satisfy also
expressions in terms of the eigenvalues.)

Although, up to now, we are not aware of any generalization of the
above theorem for $d> 3$,  we neither know, if fact, of
any example of strongly distance-regular graph with diameter
greater  than three (apart from the $r$-antipodal ones). This
suggests to end with the following conjecture: A  (connected)
regular graph $G$, with $n$ vertices and distinct eigenvalues
$\lambda_0>\lambda_1>\cdots>\lambda_d$, is strongly 
distance-regular if and only if one of the three following conditions
holds.
\begin{enumerate}
\item
$G$ is strongly regular ($d=2$);  
\item
$d=3$, $\lambda_2=-1$, and $G_3$ is $k$-regular with
degree 
$k$ satisfying  (\ref{main-theo}); 
\item
$G_d$ is constituted by disjoint copies of  
$K_r$ with $r$ satisfying (\ref{antipodal})  (that is, $G$ is an 
antipodal distance-regular graph).
\end{enumerate}

\begin{thebibliography}{99}

\bibitem{b93} 
N. Biggs, {\it Algebraic Graph Theory.}  Cambridge
University Press,  Cambridge, 1974; second edition: 1993.
 
\bibitem{bcn89}  
A.E. Brouwer, A.M. Cohen and A. Neumaier, {\it Distance-Regular
Graphs.}  Springer-Verlag, Berlin, 1989.

\bibitem{vd98} 
E.R. van Dam,  Bounds on special subsets in graphs, eigenvalues and 
association schemes, {\it J. Algebraic Combin.} {\bf 7} (1998),  no. 3, 
321--332.

\bibitem{vdh96} 
E.R. van Dam and W.H. Haemers, A characterization
of distance-regular  graphs with diameter three,  {\it J. Algebraic
Combin.} {\bf 6} (1977), no. 3, 299--303.

\bibitem{f97}
M.A. Fiol, An eigenvalue characterization of antipodal
distance-regular graphs.   {\it Electron. J. Combin.} {\bf 4} (1997), 
no. 1, \#R30.

\bibitem{f98}
M.A. Fiol, Some spectral characterizations of strongly
distance-regular graphs, submitted.

\bibitem{f99}
M.A. Fiol, 
On pseudo-distance-regularity, submitted.

\bibitem{fg97}  
M.A. Fiol and E. Garriga, From local adjacency 
polynomials to locally pseudo-distance-regular graphs, 
{\it J.  Combin. Theory Ser. B} {\bf 71} (1997), 162--183.

\bibitem{fg99}  
M.A. Fiol and E. Garriga,
Pseudo-strong regularity around a set, submitted. 

\bibitem{fgy96-1} 
M.A. Fiol,  E. Garriga, and J.L.A. Yebra,  On a class of polynomials and
its relation with the  spectra and diameters of graphs,  {\it J. 
Combin. Theory Ser. B} {\bf 67} (1996), 48--61.

\bibitem{fgy96-2} 
M.A. Fiol,  E. Garriga, and J.L.A. Yebra, Locally
pseudo-distance-regular graphs,    {\it J.  Combin. Theory Ser. B} 
{\bf 68} (1996), 179--205.

\bibitem{fgy98} 
M.A. Fiol, E. Garriga and J.L.A. Yebra, From regular
boundary graphs to antipodal distance-regular graphs, 
{\it J. Graph Theory} {\bf 27} (1998), no. 3, 123--140. 

\bibitem{g88} 
C.D. Godsil, Bounding the diameter of distance regular graphs, 
{\it Combinatorica} {\bf 8} (1988),  333--343.

\bibitem{g93}  
C.D. Godsil, {\it Algebraic Combinatorics.}  Chapman and Hall, 
New York, 1993.

\bibitem{ha96}  
W.H. Haemers, Distance-regularity and the spectrum
of graphs, {\it Linear Algebra Appl.} {\bf 236} (1996), 265--278.

\bibitem{ha95} 
W.H. Haemers, Interlacing eigenvalues and graphs, 
{\it Linear Algebra  Appl.} {\bf 227--228} (1995), 593--616.  

\end{thebibliography} 
  
\end{document}

