\documentclass[12pt]{article}
\pagestyle{empty}
\usepackage{amsmath}
\usepackage{amssymb}
\usepackage{latexsym}
%\usepackage{amstex}
%\usepackage{amssymb}
\textwidth15.6truecm
\textheight22.8truecm
\hoffset-1.73truecm
\voffset-1.3truecm
%\usepackage{amscd,amssymb}
\usepackage{amsthm}
% Final version commands--no notes!
\newcommand{\note}[1]{}
%\reversemarginpar
% Other commands
\newcommand{\Z}{\mathbb{Z}}
\newcommand{\N}{\mathbb{N}}
\newcommand{\Q}{\mathbb{Q}}
\newcommand{\C}{\mathbb{C}}
\newcommand{\set}[1]{\{#1\}}
\newcommand{\isom}{\cong}
\newcommand{\la}{{\lambda}}
\newcommand{\al}{\alpha}
\newcommand{\ptn}{\vdash}
\newcommand{\size}[1]{\left|#1\right|}
\newcommand{\Klm}{K_{\la \mu}}
\newcommand{\Klmt}{\tilde{K}_{\la \mu}}
\newcommand{\Klt}[1]{\tilde{K}_{\la {#1}}}
\newcommand{\KK}{\tilde{K}}
\newcommand{\x}{\mathbf{x}}
\newcommand{\X}{\mathbf{X}}
\newcommand{\var}[2]{{#1}_{1},\dots,{#1}_{#2}}
\newcommand{\Ind}{\text{Ind}}
\newcommand{\Ch}{\text{Ch}}
\newcommand{\charge}{c}
\newcommand{\cocharge}{\hat{c}}
\newcommand{\chg}{\charge}
\newcommand{\ccg}{\cocharge}
\newcommand{\meet}{\wedge}
\newcommand{\join}{\vee}
\newcommand{\bigmeet}{\bigwedge}
\newcommand{\bigjoin}{\bigvee}
\newcommand{\inject}{\hookrightarrow}
\newcommand{\compose}{\circ}
\newcommand{\stdzn}[2][{}]{\theta_{#2}^{#1}}
%\newcommand{\n}{n}
%\newcommand{\l}{l}
\newcommand{\nmu}{n(\mu)}
\newcommand{\CSlm}{\text{CS}(\la,\mu)}
\newcommand{\CSl}[1]{\text{CS}(\la,#1)}
\newcommand{\CSm}{\text{CS}(\mu)}
\newcommand{\CS}[1]{\text{CS}(#1)}
\newcommand{\schur}[1]{s_{#1}(\x)}
\newcommand{\mon}[1]{m_{#1}(\x)}
\newcommand{\homog}[1]{h_{#1}(\x)}
\newcommand{\HL}[2][q]{P_{#2}(\x;#1)}
\newcommand{\Symm}[2][n]{\Lambda_{#1}(#2)}
\newcommand{\dsum}{\oplus}
\newcommand{\bigdsum}{\bigoplus}
\newcommand{\LS}{Lascoux and Sch\"{u}tzenberger}
\newcommand{\I}{{\mathcal{I}}}
\newcommand{\A}{{\mathcal{A}}}
\renewcommand{\L}{{\mathcal{L}}}
\newcommand{\Sn}{S_{n}}
\newcommand{\GLn}{{GL}_{n}}
\newcommand{\gln}{{\mathfrak{gl}}_{n}}
\newcommand{\n}{{\mathcal{N}}}
\newcommand{\Cmu}{C_{\mu}}
\newcommand{\closure}[1]{\overline{#1}}
\newcommand{\restrict}[1]{\overline{#1}}
\newcommand{\Cbar}[1]{\closure{C_{#1}}}
\newcommand{\Cmubar}{\Cbar{\mu}}
%\newcommand{\nmu}{\Cbar{\mu}}
\newcommand{\Rmu}{R_{\mu}}
\newcommand{\Rmuhat}{\hat{R}_{\mu}}
\newcommand{\Jmu}{J_{\mu}}
\newcommand{\Imu}{I_{\mu}}
\renewcommand{\t}{{\mathfrak{t}}}
\newcommand{\ideal}{\I}
\newcommand{\Spec}{\operatorname{Spec}}
\newcommand{\rank}{\operatorname{rank}}
\newcommand{\Mor}{\operatorname{Mor}}
\newcommand{\Hom}{\operatorname{Hom}}
\newcommand{\id}{\operatorname{id}}
\newcommand{\Span}{\operatorname{span}}
\newcommand{\mult}{\operatorname{mult}}
\newcommand{\qmult}{\operatorname{mult}_{q}}
\newcommand{\blank}{\underline{\quad}}
\newcommand{\tensor}{\otimes}
\newcommand{\opp}[1]{{#1}^{*}}
\newcommand{\Tor}{\operatorname{Tor}}
\newcommand{\Char}{\operatorname{char}}
\newcommand{\isomto}{\stackrel{\isom}{\to}}
% New environments
% \newenvironment{proof}{{\em Proof.\/}}{\quad\rule{1ex}{1ex}}
% Equation numbering
\numberwithin{equation}{section}
% Theorem-like environments 
\theoremstyle{plain}
\newtheorem{theorem}{Theorem}[section]
\newtheorem{lemma}[theorem]{Lemma}
\newtheorem{prop}[theorem]{Proposition}
\newtheorem{cor}[theorem]{Corollary}
\theoremstyle{definition}
%{\theorembodyfont{\rmfamily}
\newtheorem{definition}[theorem]{Definition}
\theoremstyle{remark}
\newtheorem{prob}[theorem]{Problem}
\newtheorem{rmk}[theorem]{Remark}
\newtheorem{example}[theorem]{Example}
% Title info
\title{Nilpotent orbit varieties and the atomic decomposition of the $q$-Kostka polynomials}
%\subjclass{Primary: 05E15. Secondary: 05E10, 14L35, 20G05}
%\keywords{$q$-Kostka polynomials.  Atomic decomposition.  Nilpotent conjugacy classes.  Nilpotent orbit varieties.}
\author{William Brockman\thanks{Supported by the U. S. Department of Education GAANN Fellowship.}\and 
Mark Haiman\thanks{Supported in part by N.S.F. Mathematical Sciences grant
DMS-9400934.}}
%\thanks{Produced on \today}
\begin{document}
\maketitle
\thispagestyle{empty}
%\vspace{-0.4in}
\begin{abstract}
We study the coordinate rings~$R_{\mu}$ of scheme-theoretic
intersections of nilpotent orbit closures with the diagonal matrices.
de Concini and Procesi \cite{deConcini&Procesi} proved a conjecture of
Kraft \cite{Kraft} that these rings are isomorphic to the cohomology
rings of the varieties constructed by Springer
\cite{Springer76,Springer78}.  The famous $q$-Kostka
polynomial~$K_{\lambda\mu}(q)$ is the Hilbert series for the
multiplicity of the irreducible symmetric group representation indexed
by~$\lambda$ in the ring $R_\mu$.  \LS~\cite{L&S:Plaxique,Lascoux} gave
combinatorially a decomposition of~$K_{\lambda \mu}(q)$ as a sum of
``atomic'' polynomials with non-negative integer coefficients, and
Lascoux proposed a corresponding decomposition in the cohomology model.

Our work provides a geometric interpretation of the atomic
decomposition.  The Frobenius-splitting results of Mehta and van der
Kallen \cite{Mehta&vanderKallen} imply a direct-sum decomposition of
the ideals of nilpotent orbit closures, arising from the inclusions of
the corresponding sets.  We carry out the restriction to the diagonal
using a recent theorem of Broer \cite{Broer}.  This gives a direct-sum
decomposition of the ideals yielding the~$R_{\mu}$, and a new proof of
the atomic decomposition of the $q$-Kostka polynomials.

\begin{center}
{\bf R\'esum\'e}
\end{center}

Nous \'etudions l'anneau de coordonn\'ees
$R_{\mu}$ d'intersection (au sens des faisceaux) des cl\^otures
d'orbites nilpotents avec les matrices diagonales.  
de Concini et Procesi \cite{deConcini&Procesi} ont demontr\'e une
conjecture de Kraft \cite{Kraft} disant que ces anneaux sont
isomorphes aux anneaus de cohomologie des vari\'et\'es construites par
Springer \cite{Springer76,Springer78}.  Les c\'el\`ebres
$q$-polyn\^omes de Kostka $K_{\lambda\mu}(q)$ sont les s\'eries de
Hilbert pour la multiplicit\'e de la repr\'esentation irr\'eductible
du groupe sym\'etrique index\'ee par $\lambda$ dans l'anneau $R_\mu$.
Lascoux et Sch\"utzenberger \cite{L&S:Plaxique,Lascoux} ont donn\'e
une d\'ecomposition de $K_{\lambda\mu}(q)$ comme somme de polyn\^omes
``atomiques'' \`a coefficients entiers non-negatifs et Lascoux avait
propos\'e une d\'ecomposition correspondante dans le mod\`ele
cohomologique.

Notre travail fournit une interpr\'etation g\'eom\'etrique
de cette  d\'ecom\-po\-si\-tion atomique.
Les r\'esultats de Mehta et van der Kallen \cite{Mehta&vanderKallen} sur la
d\'ecomposition de Frobenius impliquent une d\'ecomposition en
somme directe des id\'eaux de cl\^otures d'orbites nilpotents qui
viennent de l'inclusion des ensembles correspondants.
Nous explicitons la restriction \`a la diagonale en utilisant
un th\'eor\`eme r\'ecent de Broer \cite{Broer}.
On obtient ainsi une d\'ecomposition en somme directe des
id\'eaux constituant les $R_\mu$, ainsi qu'une nouvelle
d\'emonstration de la d\'ecomposition atomique des $q$-polyn\^omes
de Kostka.
\end{abstract}
%\tableofcontents
\section{Introduction and Overview}

The {\it $q$-Kostka polynomials} $K_{\lambda \mu }(q)$ have been
central to numerous developments over the last two decades at the
crossroads of combinatorics, algebra, and geometry.  
Defined through the expansion
\begin{equation}\label{e:definition}
s_{\lambda }(\x) = \sum _{\mu } K_{\lambda \mu }(q) P_{\mu }(\x;q)
\end{equation}
expressing the Schur function $s_{\lambda }(\x)$ as a linear
combination of Hall-Littlewood polynomials $P_{\mu }(\x;q)$, they are
polynomials in $q$ which specialize upon setting $q=1$ to the ordinary
Kostka numbers $K_{\lambda \mu }$. 

Foulkes \cite{Foulkes}, remarking that computations suggested that the
polynomials $K_{\lambda \mu }(q)$ have non-negative integer
coefficients, conjectured that they could be interpreted
combinatorially.  Lascoux and Sch\"utzenberger \cite{Schutzenberger78}
proved the Foulkes conjecture by expressing $K_{\lambda \mu }(q)$ as a
sum
\begin{equation}\label{e:chargeformula}
K_{\lambda \mu }(q) = \sum _{T\in CS(\lambda ,\mu )} q^{\chg(T)},
\end{equation}
in which $T$ varies over the set $CS(\lambda ,\mu )$ of column-strict
tableaux of shape $\lambda $ containing $\mu _{1}$ ones, $\mu _{2}$
twos, etc., and $\chg(T)$, the {\it charge} of $T$, is a certain
intricate combinatorial statistic.  Their result and subsequent work
\cite{L&S:Croissance,L&S:Plaxique,Lascoux} reveal the remarkable
combinatorial structures underlying the $q$-Kostka polynomials.
 
Several other algebraic and geometric interpretations of $K_{\lambda
\mu }(q)$ have been given.  Hotta and Springer \cite{Hotta&Springer}
identified $K_{\lambda \mu }(q)$ as the Poincar\'e series for the
multiplicities of the irreducible symmetric group representation
$V_{\lambda }$ in the cohomology groups of a certain algebraic variety
$X_{\mu }$.  Lusztig \cite{Lusztig81, Lusztig83} described $K_{\lambda
\mu }(q)$ as the local intersection homology Poincar\'e series of a
nilpotent orbit variety, as an affine Kazhdan-Lusztig polynomial, and
as a $q$-analog of weight multiplicities.  Our concern here will be
with one further interpretation, introduced by Kraft and de
Concini-Procesi \cite{Kraft,deConcini&Procesi}, and recently given a
simple and purely elementary treatment by Garsia and Procesi
\cite{Garsia&Procesi}, which we now pause to describe.

In brief, Garsia and Procesi take as their starting point a family of
rings described by \cite{Kraft,deConcini&Procesi},
\begin{equation}\label{e:R1n}
R_{(1^{n})} = Q[x_{1},\ldots,x_{n}]/(e_{1}(\x),\ldots,e_{n}(\x)),
\end{equation}
where $e_{k}(\x)$ is the $k$-th elementary symmetric function, and 
\begin{equation}\label{e:Rmu}
R_{\mu } = R_{(1^{n})}/I_{\mu } \qquad \text{$\mu$ a partition of $n$.}
\end{equation}
Here $I_{\mu }$ is an $S_{n}$-invariant ideal, generated by certain
completely explicit {\it Tanisaki generators} \cite{Tanisaki}.  Then
setting
\begin{equation}\label{e:Ktilde}
\KK_{\lambda \mu }(q) = q^{n(\mu )} K_{\lambda \mu }(1/q),
\end{equation}
with $n(\mu ) = \sum _{i}(i-1)\mu _{i}$, they prove that
\begin{equation}\label{e:Kinterpretation}
\KK_{\lambda \mu }(q) = \sum _{d} \mult (\chi ^{\lambda }, (R_{\mu
})_{d}) q^{d},
\end{equation}
where $(R_{\mu })_{d}$ denotes the homogeneous component of degree
$d$, and $\mult(\chi ^{\lambda }, V)$ denotes the multiplicity of the
irreducible character $\chi ^{\lambda }$ in the character of an
$S_{n}$ module $V$.  It happens that $R_{\mu }$ is isomorphic to the
cohomology ring of the Hotta--Springer variety $X_{\mu }$, but in the
Garsia--Procesi treatment there is no need for recourse to this
information.


The ideals $I_{\mu }$ satisfy
%\begin{equation}\label{e:Idominance}
$I_{\mu }\subseteq I_{\nu } %\qquad
\text{whenever }\mu \leq \nu$,
%\end{equation}
with $\leq$ the dominance partial order on partitions.  Thus
the Garsia--Procesi interpretation immediately yields the
coefficientwise {\it monotonicity}
\begin{equation}\label{e:monotonicity}
\KK _{\lambda \nu }(q) \leq \KK _{\lambda \mu }(q) \qquad
\text{whenever }\mu \leq \nu 
\end{equation}
first described by \LS\ \cite{L&S:Plaxique}.
In fact, much more holds: there is an {\it atomic decomposition}
\begin{equation}\label{e:atomic}
\KK _{\lambda \mu }(q) = \sum _{\nu \geq \mu } R_{\lambda \nu }(q)
\end{equation}
in which the {\it atom polynomials} $R_{\lambda \nu }(q)$ themselves
have non-negative coefficients.  Lascoux and Sch\"utzenberger
\cite{L&S:Plaxique,Lascoux} originally obtained this atomic
decomposition combinatorially, using certain co-charge preserving
embeddings
\begin{equation}\label{e:embedding}
\theta \colon CS(\lambda ,\mu )\hookrightarrow ST(\lambda ),
\end{equation}
where $ST(\lambda ) = CS(\lambda ,1^{n})$ is the set of standard
Young tableaux of shape $\lambda $, and the {\it co-charge}
$\ccg(T)$ of a tableau $T$ of weight $\mu$ is merely $n(\mu )-\chg(T)$.  

In this paper we interpret the atomic decomposition \eqref{e:atomic}
in the Garsia--Procesi setting, thus proving anew the non-negativity
of the atom polynomials $R_{\lambda \nu }$.  More precisely, we
show (Theorem~\ref{thm:Main}) that there is a direct-sum decomposition
\begin{equation}\label{e:directsum}
R_{(1^{n})} = \bigoplus_{\nu} A_{\nu }
\end{equation}
of graded $S_{n}$-modules, such that for all $\mu $,
\begin{equation}\label{e:Imusum}
I_{\mu } = \bigoplus_{\nu \not \geq \mu } A_{\nu }.
\end{equation}
From this it is immediate that \eqref{e:atomic} holds with
\begin{equation}\label{e:Rpolinterp}
R_{\lambda \nu }(q) = \sum _{d} \mult (\chi ^{\lambda }, (A_{\nu
})_{d}) q^{d}.
\end{equation}
Lascoux \cite[Thm 6.5]{Lascoux} anticipated our Theorem~\ref{thm:Main}
by stating, without proof, a similar algebraic version of the atomic
decomposition, in the setting of the cohomology ring of the flag
variety.

Our proof departs from the purely elementary spirit of Garsia and
Procesi, relying fairly heavily upon algebraic geometry.  For us the
relevant geometric interpretation is that of Kraft, de Concini and
Procesi, wherein $\operatorname{Spec} R_{\mu }$ is recognized as the
{\em scheme-theoretic intersection} of the diagonal matrices with a
{\em nilpotent orbit variety} (the Zariski closure of a conjugacy
class of nilpotent matrices).  We describe this setting in detail in
sections \ref{sec:nilpotentorbits}~and~\ref{sec:intersectdiagonal},
while section~\ref{sec:mainthm} outlines the proof of our main
results.  Using a Frobenius-splitting result of Mehta and van der
Kallen we deduce (Theorem~\ref{thm:MainUpstairs}) that there is a
decomposition analogous to \eqref{e:directsum}--\eqref{e:Imusum}
for the nilpotent orbit varieties $\Cmubar$ in the nullcone, $\n$:
\begin{equation}
k[\n] = \bigdsum_{\nu\ptn n} \hat{A}_{\nu}
\qquad\text{with }
\I(\Cmu)=\bigdsum_{\nu\not\geq\mu}\hat{A}_{\nu}\qquad\forall\ \mu\ptn n.
\end{equation}
This decomposition holds in prime characteristics and hence also in
characteristic zero.  Then, a recent result of Broer (intended by him
for quite different purposes) enables us to carry the decomposition
down to the diagonal matrices and complete the proof of
Theorem~\ref{thm:Main}.

\subsection{Acknowledgments} We thank Bram Broer, Adriano Garsia, and Mark
Shimozono for several valuable conversations, and the referee for very
helpful comments.
 
\section{The $q$-Kostka polynomials}
\label{combinatorialsection}
Our notation for partitions, tableaux, symmetric functions, and
related objects will generally be compatible with
Macdonald~\cite{Macdonald}.  In particular, we write ``$\lambda$ is a
partition of $n$'' as $\la\ptn n$, and put our partitions in
decreasing order, $\la=(\la_1\geq\dots\geq\la_n\geq0)$.  The only
order on $P_{n}$, the set of partitions of $n$, that we will use is
the ``natural'' or {\em dominance} partial order
\cite[I.1]{Macdonald}, defined by $\mu \leq \la$ if and only if
\begin{equation}
\mu_1 + \mu_2 + \dots + \mu_i \leq \la_1+\la_2+\dots+\la_i \qquad 
1\leq i \leq n.
\end{equation}
Thus the smallest partition of $n$ in dominance is
$(1,1,\dots,1)=(1^n)$ while the largest is $(n)$.  In fact $P_{n}$ is
a lattice under the dominance order \cite[p. 11]{Macdonald}.

Tableaux always have entries drawn from $\set{1,\dots,n}$ and are {\em
column-strict}, i.e.\ weakly increasing in the rows, strictly in the
columns.  The {\em weight} of a tableau is the vector giving the
number of occurrences of each integer in the tableau; a {\em standard}
tableau is one of weight $(1,1,\dots,1)$.  The set of column-strict
tableaux of shape $\la$ and weight $\mu$ is written $\CSlm$, for
$\la\ptn n$ and $\mu$ a composition of $n$.  We will also need the set
$\CSm$ of column-strict tableaux of weight $\mu$ and any shape,
i.e. $\CSm = \bigcup_{\la\ptn n} \CSlm$.  Combinatorially, the {\em
Kostka number} $\Klm$ is the cardinality of $\CSlm$;
see~\cite[I.6]{Macdonald}.  

The {\em Hall-Littlewood polynomials} $\set{\HL{\mu}}_{\mu\ptn n}$ are
a remarkable basis for the symmetric polynomials in $\x=\set{\var
xn}$, over $\Q(q)$; see~\cite[Chapter~III]{Macdonald} for their
definition.  The structure constants of this basis give the {\em Hall
polynomials} \cite[III.3]{Macdonald}, while, more important for our
purposes, the {\em $q$-Kostka polynomials} $\Klm(q)$ are the entries
of the transition matrix from the Schur polynomials
$\set{\schur{\la}}$ to this basis \cite[III.6]{Macdonald}:
\begin{equation}
\schur{\la} = \sum_{\mu\ptn \size{\la}}\Klm(q) \HL{\mu}.
\end{equation}
The specialization $q\to1$ gives~\cite[III.2]{Macdonald}
$\schur{\la}=\sum_{\mu\ptn n}\Klm(1)\mon{\mu}$,
the defining relation for the Kostka numbers, 
so $\Klm(q)$ is indeed a $q$-analog of $\Klm$.

>From the definition above it follows that $\Klm(q)$ is a polynomial
with integer coefficients.  Computing $\Klm(q)$ for small $n$, one
finds the coefficients are non-negative. Several different proofs of
this fact appear in the literature.  We will be especially concerned
with those of \LS~\cite{Schutzenberger78,L&S:Plaxique} and of Garsia and
Procesi \cite{Garsia&Procesi}.

\section{Cocharge and atoms}
\label{KlmtSubSec}
\LS\ \cite{Schutzenberger78,L&S:Plaxique, L&S:Croissance} obtain
$\Klm(q)$ combinatorially, by $q$-counting the column-strict tableaux
of shape $\la$ and content $\mu$:
\begin{equation}
\Klm(q) = \sum_{T\in \CSlm}q^{\chg(T)}.
\end{equation}
The {\em charge} statistic $\chg(T)$ is computed directly from the
tableau $T$ by a somewhat complicated combinatorial process
\cite[p.242]{Macdonald}.  One sees that $\chg(T)\leq \nmu$
for $T\in \CSm$, where $\nmu=\sum_{i=1}^{n}(i-1)\mu_{i}$; and 
in many ways the {\em
cocharge} $\ccg(T) = \nmu-\chg(T)$ is in fact more natural than
charge.  For instance, in~\cite{L&S:Plaxique}, \LS\ define a ranked
poset on the column-strict tableaux of weight $\mu$ in which the rank
of a tableau is its cocharge.  

Using this poset and other tools, \LS~\cite{L&S:Croissance,
Lascoux} have explored the remarkable structure of the
$q$-Kostka polynomials, or more precisely of their ``cocharge''
variants
\begin{equation}
\Klmt(q) = q^{n(\mu)} \Klm(q^{-1}) = \sum_{T\in \CSlm}q^{\ccg(T)}.
\end{equation}
First, $\Klmt(q)$ is a monotonically decreasing function of $\mu$:
\begin{theorem}[\LS]  
$\Klmt(q)-\Klt{\nu}(q)$ has non-negative integer coefficients for all
partitions $\nu\geq\mu$.
\end{theorem}
The proof is combinatorial, by construction of a shape- and
cocharge-preserving injection $\stdzn[\mu]{\nu}:\CS{\nu}\inject\CSm$
for every $\nu\geq\mu$.  In particular, there is such a map into the
standard tableaux for every $\nu\ptn n$.  We will abbreviate this {\em
standardization} map as $\stdzn{\nu}:\CS{\nu}\inject\CS{(1^n)}$.
(Note that this map is not the same as the standardization map of
Schensted \cite{Schensted61}.)

\LS\ \cite{L&S:Plaxique,Lascoux} further strengthen the monotonicity result 
by studying the details of the standardization more closely.  
By construction the injections are compatible: if $\la\geq\mu\geq\nu$
then $\stdzn[\nu]{\mu}\compose \stdzn[\mu]{\la} =
\stdzn[\nu]{\la}$.  Thus if $\la\geq\mu$, then 
$\stdzn{\la}\CS{\la}\subseteq\stdzn{\mu}\CSm$. 
Furthermore, 
\begin{equation}
(\stdzn{\mu} \CSm) \cap (\stdzn{\nu}\CS{\nu})
= \stdzn{\mu\join\nu}\CS{\mu\join\nu}.
\end{equation}
So we may think of a ``contribution from $\mu$'' which is disjoint
from the contribution from any $\nu\neq\mu$.  Formally, the {\em atom}
associated with $\mu$ is
\begin{equation}
	A(\mu)=\stdzn{\mu}\CSm-\bigcup_{\nu>\mu}\stdzn{\nu}\CS{\nu}.
	\label{Latom}
\end{equation}
These results allow an ``atomic decomposition'' of $\CS{(1^n)}$ as a
disjoint union of the atoms.  The standardization maps are
shape-preserving, so we may define 
\begin{equation}
A(\la,\mu)=\stdzn{\mu}\CSlm-\bigcup_{\nu>\mu}\stdzn{\nu}\CSl{\nu}.
\end{equation}
Since the standardization maps also preserve cocharge, we now have:
\begin{theorem}[\LS]
\begin{equation}\label{eq:AtomDecompKlmt}
\Klmt(q) = \sum_{\nu\geq\mu} R_{\la\mu}(q), \qquad\text{where }
R_{\la\mu}(q) = \sum_{T\in A(\la,\mu)}q^{\ccg(T)}.
\end{equation}
Note $R_{\la\mu}(q)$ is also a polynomial with non-negative integer
coefficients.
\end{theorem}

\begin{example}\label{ex:Atompolys321}
The lattice of partitions $P_n$ is a chain for $n\leq 5$, in which
case the atomic decomposition reduces to the monotonicity result.
Consider, therefore, partitions of 6; Table~\ref{table:Atompolys321}
gives $\Klmt(q)$ and $R_{\la\mu}(q)$ for the partitions
$\la,\mu\geq(321)$.  Note that the only incomparable pair in this set
is $(411,33)$.
\end{example}

\begin{table}
\begin{tabular}{r@{}r|cccccc}
&&\multicolumn{6}{c}{$\Klmt(q)$}\\
& & 6 & 51 & 42 & 411 & 33 & 321 \\
\hline
&6 & 1 & 1 & 1 & 1 & 1 & 1 \\
&51 & & $q$ & $q$ & $q+q^2$ & $q$ & $q+q^2$ \\
&42 & & & $q^2$ & $q^2$ & $q^2$ & $q^2+q^3$ \\
\raisebox{1.5ex}[0cm][0cm]{\makebox[0cm][r]{$\lambda\ $}}
&411 & & & & $q^3$ & 0 & $q^3$ \\
&33 & & & & & $q^3$ & $q^3$ \\
&321 & & & & & & $q^4$ \\
\end{tabular}
\hfill
\begin{tabular}{r@{}r|cccccc}
&&\multicolumn{6}{c}{$R_{\la\mu}(q)$}\\
& & 6 & 51 & 42 & 411 & 33 & 321 \\
\hline
&6 & 1 & 0 & 0 & 0 & 0 & 0 \\
&51 & & $q$ & 0 & $q^2$ & 0 & 0 \\
&42 & & & $q^2$ & 0 & 0 & $q^3$ \\
\raisebox{1.5ex}[0cm][0cm]{\makebox[0cm][r]{$\lambda\ $}}
&411 & & & & $q^3$ & 0 & 0 \\
&33 & & & & & $q^3$ & 0 \\
&321 & & & & & & $q^4$ \\
\end{tabular}
\caption{$\Klmt(q)$ and $R_{\la\mu}(q)$, for $\la,\mu\geq(321)$.}
\label{table:Atompolys321}
\end{table}

\section{Nilpotent orbit varieties}
\label{sec:nilpotentorbits}
Our approach to the atomic decomposition of the $q$-Kostka polynomials
is based on the nilpotent orbit varieties, one of several geometric
constructions of these polynomials.  Aspects of this construction are
discussed by a number of authors, including Kostant
\cite{Kostant}, Steinberg \cite{Steinberg},
Spaltenstein \cite{Spaltenstein}, Springer \cite{Springer76,
Springer78}, Hotta and Springer
\cite{Hotta&Springer}, Kraft \cite{Kraft}, de
Concini and Procesi \cite{deConcini&Procesi}, Garsia and Procesi
\cite{Garsia&Procesi}, Bergeron and Garsia \cite{Bergeron&Garsia},
and Carrell \cite{Carrell86}. 

Let $k$ be an algebraically closed field.  We consider the variety of
$n$ by $n$ matrices $\gln(k)$ and its associated ring of polynomial
functions $k[\gln] = k[x_{11},x_{12},\dots,x_{nn}]$.  Write $\X$ for
the $n$ by $n$ matrix of indeterminates ${x_{ij}}$.  Denote by $\n$
the variety of nilpotent matrices in $\gln(k)$, i.e.\ those matrices
with sole eigenvalue 0.  The ideal of $\n$ in $\gln(k)$ is generated
\cite{Kostant} by the coefficients of the characteristic polynomial
$\det(tI-\X)$, i.e.
\begin{equation}
k[\n]=k[\gln]/\ideal(\n)=k[\X]/(E_{1}(\X),E_{2}(\X),\dots,E_{n}(\X)).
\end{equation}

\label{CmuSubSec}
$\GLn(k)$ acts on $\gln(k)$ by conjugation (the {\em adjoint action}),
and $\n$ is stable under this action.  Thus $\n$ is a union of
$\GLn$-conjugacy classes.  Since the only eigenvalue appearing is 0,
each conjugacy class is determined by
its Jordan block structure: Let $\Cmu\subset\n$ be the set of
matrices with Jordan block form $\mu'_{1}$, $\mu'_{2}$, \dots,
$\mu'_{n}$, where $\mu'$ is the {\em conjugate partition} to $\mu$.
Equivalently,
\begin{equation}
\Cmu=\set{N\in\n:\rank N^{i}=\mu_{i+1}+\dots+\mu_{n}}.
\end{equation}

We will be particularly concerned with the ideal of $\Cmu$, denoted 
$\I(\Cmu) \subseteq k[\gln]$.  
Since $\Cmu$ is a $\GLn$-conjugacy class, $\I(\Cmu)$ is $\GLn$-stable
as well, under the natural extension of the action to $k[\gln]$.  The
ideals $\I(\Cmu)$ were studied by Kraft \cite{Kraft}, de Concini and
Procesi \cite{deConcini&Procesi}, and Tanisaki \cite{Tanisaki}; Weyman
\cite{Weyman} found an explicit generating set for $\I(\Cmu)$, and also
confirmed the conjectured generators of \cite{deConcini&Procesi} and
\cite{Tanisaki}.

It is a standard fact that the Zariski closure $\Cmubar$, the variety
cut out by $\I(\Cmu)$, is exactly
\begin{equation}\label{eq:containCmu}
\Cmubar = \bigcup_{\nu\geq\mu}C_{\nu}.
\end{equation}
Thus we have 
\begin{equation}\label{eq:domCmu}
\nu\geq\mu \iff \Cbar{\nu}\subseteq \Cmubar \iff
\I(C_{\nu})\supseteq \I(\Cmu).
\end{equation}
The smallest of these {\em nilpotent orbit varieties} is
${C_{(n)}}$, which is closed and contains only the zero matrix, while
the largest is
%\begin{equation}
$\Cbar{(1^{n})} = \bigcup_{\nu\ptn n}C_{\nu} = \n$.
%\end{equation}
Since all the $\Cmubar$ are contained in $\n$, the ideals $\I(\Cmu)$
are essentially interchangeable with the ideals $\Jmu=\I(\Cmu)/\I(\n)
\subset k[\n]$.


\section{Intersecting with the diagonal}
\label{sec:intersectdiagonal}
Set-theoretically, the intersection of the nilpotent orbit variety
$\Cmubar$ with the diagonal matrices $\t\subset\gln$ contains only the
zero matrix.  However, the {\em scheme-theoretic intersection} of
$\Cmubar$ with $\t$ forms an interesting zero-dimensional scheme,
$\Spec k[\Cmubar\cap\t]$, where 
\begin{equation}
k[\Cmubar\cap\t]= k[\gln]/(\ideal(\Cmu)+\ideal(\t)).
\end{equation}
The coordinate ring $k[\Cmubar\cap\t]$ has finite $k$-dimension, since
$\Spec k[\Cmubar\cap\t]$ has only one underlying point.  Furthermore,
it carries an action of the {\em Weyl group} of $\GLn$, the symmetric
group $S_n$, as follows.  We view $\Sn\subset\GLn$ as the subgroup of
permutation matrices.  While the $\GLn$ action by conjugation does not
preserve $\t$, the restriction of this action to $\Sn$ does.
Likewise, $\I(\t)$ is $\Sn$-stable, as are the $\GLn$-stable sets
$\Cmu$ and $\I(\Cmu)$; so the ideal $\I(\Cmu)+\I(\t)$, the quotient
ring $k[\Cmubar\cap\t]$, and the scheme $\Spec k[\Cmubar\cap\t]$ all
carry $\Sn$-actions.

Since each ideal $\I(\Cmu)+\I(\t)$ contains the ideal
$\ideal(\t)=(x_{ij}: i\neq j)$ and the ideal
$\ideal(\n)=(E_{1}(\X),E_{2}(\X),\dots,E_{n}(\X))$, we lose nothing by
working in the quotient ring $k[\n\cap\t]=k[\X]/(\I(\n)+\I(\t))$.  Of course,
$k[\X]/\I(\t)\isom k[x_{11},x_{22},\dots,x_{nn}]$, and we may identify
the surviving variables $x_{ii}$ with $x_i$; as in
section~\ref{combinatorialsection}, we use $\x$ to represent the
variables $\set{\var xn}$.  In $k[\x]$, the polynomial $E_i(\X)$ becomes the
elementary symmetric function $e_i(\x)$.  Define the ring $R$ by: 
\begin{equation}
k[\X]/(\I(\n)+\I(\t)) \isom k[\x]/(e_{1}(\x), \dots, e_{n}(\x)) = R,
\end{equation}
and let the ideal $\Imu$ be the image of $\I(\Cmu)+\I(\t)$ in $R$.
(Explicit sets of generators for $I_{\mu}$ appear in
\cite{deConcini&Procesi, Tanisaki, Garsia&Procesi}.)  As described in
\eqref{e:Rmu}, we define $\Rmu=R/\Imu$; we will tend to consider the
isomorphism $\Rmu\isom k[\Cmubar\cap\t]$ an equality.  Note
also that $I_{(1^n)}=(0)$ and $R_{(1^n)}=R$.

At this point we should remark that all of the ideals $\I(\Cmu)$ and 
$\ideal(\t)$ are {\em homogeneous} ideals---generated by homogeneous 
polynomials---and thus the quotients $k[\Cmubar]$, $k[\t]$, 
and $\Rmu$ are {\em graded} rings.  (Equivalently, we are noting that 
the $k^{*}$-action on $\gln$ stabilizes $\Cmu$ and $\t$.)  So we 
have, in particular, $\Rmu=\dsum_{d\geq0}(\Rmu)_{d}$ for all $\mu\ptn
n$, where $(R_{\mu })_{d}$ denotes the homogeneous component of degree
$d$.

Now de Concini and Procesi \cite{deConcini&Procesi} proved that
$R_{\mu}$ is isomorphic as an $\Sn$ module to $\Ind_{S_{\mu}}^{\Sn}
1$, and by Young's rule the character of ${\Ind_{S_{\mu}}^{\Sn}1}$ is
$\sum_{\la\ptn n}\Klm \chi^{\la}$.  So, recording the multiplicity of
the irreducible $\chi^{\la}$ in each degree $(R_{\mu})_{d}$ as the
coefficient of $q^d$, one has
\begin{equation}
\sum _{d} \mult (\chi ^{\lambda }, (\Rmu)_{d}) q^{d} = \Klmt(q),
\end{equation}
wherein the $\Klmt(q)$ must be some $q$-analogs of the Kostka numbers
$\Klm$.   In
fact, the polynomials $\Klmt(q)$ are the same polynomials introduced
in section~\ref{KlmtSubSec}: 
%\begin{equation}\label{eq:CharacRmu}
$\Klmt(q) = q^{n(\mu)} \Klm(q^{-1})$.
%\end{equation}
This identification follows from the work of Kraft, de
Concini, Procesi, Hotta, and Springer 
\cite{Kraft, deConcini&Procesi, Springer76, Springer78, Hotta&Springer}.

Garsia and Procesi give an elementary proof of this connection between
the ideals $\Imu$ and the $q$-Kostka polynomials in
\cite{Garsia&Procesi}, which makes completely explicit several
combinatorial properties of the $q$-Kostka polynomials.  In
particular, their monotonicity is immediate
\cite[p. 103]{Garsia&Procesi}: Let $\mu$ and $\nu$ be partitions with
$\nu\geq\mu$.  As noted in section~\ref{CmuSubSec}, we have
$\I(C_{\nu})\supseteq \I(\Cmu)$, and hence $I_{\nu}\supseteq \Imu$.  (One
can also compute directly from the Tanisaki generators for $\Imu$ that
$I_{\nu}\supseteq \Imu$.)  Thus $R_{\mu}/I_{\nu}\isom R_{\nu}$, and
the surjection of $\Sn$-modules gives $\Klmt(q)\geq\Klt{\nu}(q)$.  This
surjection is an algebraic analog of Lascoux's injection
$\CS{\nu}\inject\CSm$. 

\subsection{Atoms}
One important property which is not seen immediately from the
Garsia-Procesi point of view is the atomic decomposition
$\Klmt=\sum_{\nu} R_{\la\nu}$ with $R_{\la\nu}$ a polynomial with
non-negative integer coefficients.  Since $\Klmt(q)$ is obtained from
the graded character of the $\Sn$-module $R_{\mu}$, we seek an
isomorphism of graded $\Sn$-modules
\begin{equation}\label{eq:RmuDecomp}
R_{\mu}\isom\bigdsum_{\nu\geq\mu}A_{\nu}.
\end{equation}  
Where can we find these modules $A_{\nu}$?  They most naturally live in $R$.
We must decompose $R$ as 
\begin{equation} \label{decompR}
R = \bigdsum_{\nu\ptn n} A_{\nu}			    
\qquad\text{so that }
I_{\mu} = \bigdsum_{\nu\not\geq\mu} A_{\nu}\quad \text{for all }
\mu\ptn n.
%\label{decompImu}
\end{equation}
Then we shall have immediately from \eqref{eq:RmuDecomp} and
\eqref{eq:AtomDecompKlmt} that 
\begin{equation}
\sum _{d} \mult (\chi ^{\lambda },
(A_{\nu })_{d}) q^{d} = R_{\la\nu}(q).
\end{equation}

\begin{example}
Using Macaulay 2 \cite{Macaulay2}, we found bases of the spaces
$A_\nu\subset R$ whose characters are described by the atomic
polynomials $R_{\la\nu}$ of example~\ref{ex:Atompolys321}.  These
bases are too large to present here; instead in
Table~\ref{table:Atoms321} we give one vector in each irreducible
$S_n$-representation.  Note that this example is somewhat trivial,
since there are only two incomparable partitions, and thus the
corresponding lattice is automatically distributive; in addition, the
multiplicity of representations in the graded components of the atoms
never exceeds 1.  The example would have to be significantly larger
($n=10$) to avoid these failings.
\end{example}

\begin{table}
\begin{center}
\begin{tabular}{rr|cccccc}
&&\multicolumn{6}{c}{$A_{\nu}(q)$}\\
& & 6 & 51 & 42 & 411 & 33 & 321 \\
\hline
&6 & 1 & 0 & 0 & 0 & 0 & 0 \\
&51 & & $x_6$ & 0 & $x_6^2$ & 0 & 0 \\
&42 & & & $x_5x_6$ & 0 & 0 & $e_3(x_3,x_4,x_5,x_6)$ \\
\raisebox{1.5ex}[0cm][0cm]{$\lambda$}
&411 & & & & $x_5x_6^2$ & 0 & 0 \\
&33 & & & & & $x_4x_5x_6$ & 0 \\
&321 & & & & & & $x_4x_5x_6^2$ \\
\end{tabular}
\end{center}
\caption{Generators of irreducibles of type $\la$ in $A_{\nu}(q)$, for $\nu\geq(321)$.}
\label{table:Atoms321}
\end{table}

\section{Main theorems}
\label{sec:mainthm}
\begin{theorem}\label{thm:MainUpstairs}
Let $k$ be an algebraically closed field.  There exists a direct-sum
decomposition of the coordinate ring of $\gln$,
\begin{equation}
k[\gln] = \bigdsum_{\nu\ptn n} \hat{A}_{\nu}
\end{equation}
compatible with the ideals of the conjugacy classes $\Cmu$:
\begin{equation}
\I(\Cmu)=\bigdsum_{\nu\not\geq\mu}\hat{A}_{\nu}\qquad\forall\ \mu\ptn n.
\end{equation}
\end{theorem}
\begin{theorem}[Atomic decomposition]\label{thm:Main}
There exists a direct-sum decomposition of $k[\n\cap\t]\isom R$,
\begin{equation}
R = \bigdsum_{\nu\ptn n} A_{\nu} \qquad\text{such that }
I_{\mu} = \bigdsum_{\nu\not\geq\mu} A_{\nu} \quad \text{for all }\mu\ptn n.
\end{equation}
%such that 
%\begin{equation}
%\end{equation}  
Thus we obtain the ``atom polynomial,''
\begin{equation}
R_{\la\nu}(q) = \sum _{d} \mult (\chi ^{\lambda }, (A_{\nu})_{d}) q^{d}.
\end{equation}
\end{theorem}
We only sketch the proofs in this extended abstract.  First we show
that the decomposition exists if and only if the ideals
$\set{\Imu:\mu\ptn n}$ generate a distributive sublattice of the
lattice of all ideals in $R$.  Similarly, there is such a
decomposition of $k[\n]$ if and only if the lattice generated by the
ideals of orbit varieties $\set{\I(\Cmu):\mu\ptn n}$ is distributive.
Geometrically the latter property is natural: we expect that the
lattice generated by the $\I(\Cmu)$ is the dual of the lattice
generated by the orbit varieties $\Cmubar$ under the set operations.

We show that the lattice generated by $\set{\I(\Cmu):\mu\ptn n}$ is 
distributive by showing that every 
element of the lattice is radical.  This fact is a consequence 
of a recent geometric result:
\begin{theorem}[Mehta and van der Kallen \cite{Mehta&vanderKallen}]
\label{MvdKthm}
Let $k$ be an algebraically closed field of positive characteristic.
There is a Frobenius splitting $\phi$ of $\gln$ such that every
nilpotent orbit variety $\Cmubar$ is compatibly Frobenius split.
\end{theorem}

Since the distributivity is proved in positive characteristic, we must
make a standard technical argument to extend the result to
characteristic~0.  This proves
Theorem~\ref{thm:MainUpstairs}.  Finally, we
address the intersection with the diagonal.  Essentially, we show that
the lattice generated by the $\I(\Cmu)$ together with $\I(\t)$ is
still distributive, even though its elements are no longer radical.

This argument relies on a result of Broer, who in
\cite{Broer} extends Chevalley's restriction theorem to modules of
covariants.  His theorem holds in great generality, but we will need
it only for the Lie group $\GLn$ and Lie algebra $\gln$.  (We view
$\gln$ as the algebraic variety of all $n$ by $n$ matrices over $k$.)
Let $T\subset\GLn$ be the subgroup of invertible diagonal matrices,
and $\t\subset\gln$ its Lie algebra.  Any $\GLn$-module $M$ has a
representation of the Weyl group $S_n$ on the fixed-point set $M^T$ of
the $T$-action.  Broer's theorem applies to {\em small}
$\GLn$-modules: those which do not have the $T$-weight $2\phi$, where
$\phi$ is the highest root of $\GLn$.
\begin{theorem}[Broer \cite{Broer}] \label{Broerthm}
Let $k$ be an algebraically closed field of characteristic zero, and
let $M$ be a small $\GLn$-module.  For any nilpotent orbit variety
$\Cmubar$, the map
\begin{equation}
\Hom_{\GLn}(M,k[\Cmubar])\to \Hom_{\Sn}(M^T,k[\Cmubar\cap\t])
\end{equation}
induced from the restriction $k[\Cmubar]\to k[\Cmubar\cap\t]$ is an
isomorphism of graded vector spaces.
\end{theorem}

\bibliographystyle{abbrv}
\bibliography{newmain}
{\sc Dept.\ of Mathematics, Michigan State University, East Lansing,
MI, 48824-1027}

{\sc Dept.\ of Mathematics, UCSD, La Jolla, CA, 92093-0112}

{\em E-mail address: {\tt brockman@math.msu.edu, mhaiman@macaulay.ucsd.edu}}

\end{document}


