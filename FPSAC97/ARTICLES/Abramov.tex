\documentstyle[fullpage,12pt]{article}
\pagestyle{empty}

\textwidth15.6truecm
\textheight22.8truecm
\hoffset-1.43truecm
\voffset-2truecm
%\newcommand {\complex} {{\mathbb C}}
\newcommand {\ord}{\mathop{\rm ord}}
\newcommand {\lc}{\mathop{\rm lc}}
%\newcommand {\gcd}{\mathop{\rm gcd}}
\newcommand {\rGCD}{\mathop{\rm GCD}}
\newcommand {\qed}{\hfill$\Box$\par\medskip}
\newcommand {\cS} {{\cal S}}
%\newcommand {\cA} {{\cal A}}
\newcommand {\cR} {{\cal R}}
%\newcommand {\cCO} {{\cal C}_{0}}
\newcommand {\cC} {{\cal C}}
%\newcommand {\cM} {{\cal M}}
%\newcommand {\cI} {{\cal I}}
%\newcommand {\cI} {{\rm In}}
%\newcommand {\cU} {{\cal U}}
%\newcommand {\cV} {{\cal V}}
\newcommand {\rodtp} {\complex [D,x]}
\newcommand {\rodcp} {\complex [E,E^{-1},n]}
\newtheorem{Theorem}{Theorem}
\newtheorem{Proposition}{Proposition}
\newtheorem{Lemma}{Lemma}
\newtheorem{Example}{Example}
\newtheorem{Corollary}{Corollary}
\def\o{\circ}
\def\complex{\hbox{C\kern -.58em {\raise .54ex \hbox{$\scriptscriptstyle |$}}
  \kern-.55em {\raise .53ex \hbox{$\scriptscriptstyle |$}} }}


\title{\Large\bf Solutions of linear differential equations
in the class of sparse power series}
\author{\large\sl Sergei A. Abramov\thanks{Supported in part by
the RFBR and INTAS under Grant 95-IN-RU-412. } \\
\normalsize Computer Center of the Russian Academy of Science\\
\normalsize  Vavilova 40, Moscow 117967, Russia\\
\normalsize abramov@ccas.ru,
\normalsize sabramov@cs.msu.su
}

\date{}
\begin{document}
\maketitle

\begin{center}
\large\bf Abstract
\end{center}

We introduce the notion of the $m$-sparse power series
(e.g. expanding $\sin x$ and $\cos x$ at $x=0$ gives
2-sparse power series: a coefficient $a_n$ of the series
can be nonzero only if $remainder(n,2)$ is equal to a
fixed number). Then we consider the problem of finding all
$m$-points of a linear ordinary differential equation
$Ly=0$ with polynomial coefficients (i.e. the points
at which the equation has a solution in the form of an
$m$-sparse series). It is easy to find an upper bound for $m$.
We prove that if $m$ is fixed then either there exists a finite
number of $m$-points and all of them can be found or all points
are $m$-points and $L$ can be factored $L=L_1 \circ C$ where $C$
is an operator of a special kind with constant coefficients.
Additionally we formulate simple necessary and sufficient
conditions for the existence of $m$-points for an irreducible $L$.

Combining the algorithms described in the paper with
Petkov\v sek-Salvy's algorithm we find all $m$-sparse
$m$-hypergeometric solutions of the equation $Ly=0$.

\begin{center}
\large\bf R\'esum\'e
\end{center}

On introduit la notion de s\'erie de puissances
$m$-creuse. (Les d\'ev\'elop\-pe\-ments
de $\sin x$ et de $\cos x$ autour de $x=0$ sont des exemples
de s\'eries $2$-creuses:
on demande que le coefficient $a_n$ de la s\'erie
soit non-nul seulement si $n$ appartient \`a une classe
fix\'ee de residus modulo 2).
On consid\`ere le probl\`eme de d\'eterminer
tous les $m$-points d'une \'equation diff\'erentielle
lin\'eaire $L y = 0$ \`a coefficients polynomiaux
(i.e. les points o\`u l'\'equation admet une solution
sous forme $m$-creuse).
Il est facile de trouver une borne sup\'erieure pour $m$.
Pour $m$ fix\'e on d\'emontre qu'ou bien il existe un
nombre fini de $m$-points et on peut les d\'eterminer,
ou bien tous les points sont des $m$-points et $L$ peut
se factoriser en $L = L_1 \circ C$ o\`u $C$ est un
op\'erateur d'un type particulier \`a coefficients
constants.
En plus, on donne des crit\`eres n\'ecessaires
et suffisants  simplespour l'existence de $m$-points lorsque
l'op\'erateur $L$ est irr\'eductible.

En combinaisant les algorithmes de cet article avec
l'algorithme de Petkov\v sek-Salvy on peut obtenir
toutes les solutions $m$-creuses et $m$-hyperg\'eom\'etriques
de l'\'equation $L y = 0$.


\section{$m$-Sparse sequences and series}
Let ${\cal C}$ be the set of infinite sequences $(c_0,c_1,\dots) \in
\complex^\infty$,
${\cal S}$ be the set of formal power series $c_0+c_1x+\dots$ with
$(c_0,c_1,\dots) \in {\cal C}$ and $m$ be an integer $\ge 2$.
Call $c=(c_0,c_1,\dots)\in {\cal C}$ an $m$-{\em sparse}
sequence if  there exists an integer $N$ such that
\begin{equation}
\label{one}
(c_n \neq 0) \Rightarrow (n = N \pmod{m}).
\end{equation}

Denote by ${\cal C}^{(m)}$ (resp. ${\cal S}^{(m)}$) the set of all $m$-sparse
elements of ${\cal C}$  (resp. the set of all elements of ${\cal S}$,
whose sequences of coefficients are in ${\cal C}^{(m)}$).

\noindent {\bf Example 1}
The series $x+x^4+x^7+\dots+x^{3n+1}+\dots$ is 3-sparse, i.e. belongs
to ${\cal S}^{(3)}$ with $N=1$.

It is obvious that
$$
(m_1 | m_2 )\Rightarrow ( {\cal C}^{(m_2)} \subset {\cal C}^{(m_1)},
{\cal S}^{(m_2)} \subset {\cal S}^{(m_1)} ).
$$

Consider a linear ordinary differential equation $Ly=0$ with
\begin{equation}
\label{two}
L=p_r(x) D^r + \dots + p_1(x) D + p_0(x),
\end{equation}
$p_0(x),\dots,p_r(x) \in \complex[x]$, $p_r(x) \ne 0$. Our goal is to find
all  $a \in \complex$ and formal power series solutions
\begin{equation}
\label{three}
y_a(x)=\sum_{n=0}^\infty c_n(x-a)^n
\end{equation}
such that
\begin{equation}
\label{four}
Ly_a(x) = 0
\end{equation}
and $(c_0,c_1,\dots)\in {\cal C}^{(m)}$. Observe that $y_a(x)$
satisfies~(\ref{four}) iff
$$
y(x)=\sum_{n=0}^\infty c_n x^n
$$
satisfies $L^ay(x)=0$ where
\begin{equation}
\label{five}
L^a=p_r(x+a)D^r+\dots+p_1(x+a)D+p_0(x+a).
\end{equation}

We will consider the problem of finding $m$ and $a$ such that
the equation $L^ay=0$ has a solution in ${\cal S}^{(m)}$.

\section{m-Points}
It is well known that the power series coefficients of a solution
of a linear differential equation with polynomial coefficients
satisfy a linear recurrence (a difference equation) $Rc=0$
with polynomial coefficients:
\begin{equation}
\label{six}
q_d(n)c_{n+d}+\dots+q_l(n)c_{n+l}=0,
\end{equation}
$$
d>...>l;\; q_l(n),\dots,q_d(n) \in \complex[n];\; q_d(n),q_l(n)\ne 0.
$$
The operator $R$ which is equal to
\begin{equation}
\label{seven}
q_d(n)E^d+\cdots+q_l(n)E^l
\end{equation}
is the ${\cal R}$-image of $L$ where ${\cal R}$ is the isomorphism of
$\complex[D,x,x^{-1}]$ onto \complex$[E,E^{-1},n]$:
\begin{equation}
\label{eight}
{\cal R}D=(n+1)E,\; {\cal R}x = E^{-1},\; {\cal R}x^{-1} = E ;
\end{equation}
resp.
$$
{\cal R}^{-1}E=x^{-1},\; {\cal R}^{-1}E^{-1}=x, \;{\cal R}^{-1}n = xD
$$
(see \cite{APR}). Note that it is possible
$l<0$ in~(\ref{six}),(\ref{seven}).
For $R$ of the form~(\ref{seven}) we denote $\omega (R)=d-l$.
If the coefficient of $x^i$ in the polynomial $p_j(x)$
is not equal to zero then we write $x^iD^j \in L$.
It is easy to check that if $L$ is of the form~(\ref{two})
and $R={\cal R}L$ then
\begin{equation}
\label{23}
d=\max_{x^iD^j\in L} \{j-i\},\; l=\min_{x^iD^j\in L} \{j-i\} ;
\end{equation}
and therefore $\omega (R)$ is equal to
\begin{equation}
\label{nine}
\max_{x^iD^j\in L} \{j-i\}- \min_{x^iD^j\in L} \{j-i\}.
\end{equation}

Let us define finite $m$-sparse objects as well.
Let $(d_s,d_{s+1},\dots,d_t)$ be an ordered finite set of elements of a ring.
Call this set $m$-sparse if there exists $N$ such that
$$
(d_j \neq 0) \Rightarrow (j = N \pmod{m})
$$
for $j=s,s+1,\dots,t$. Call operators of the form~(\ref{two})
and~(\ref{seven}) as well as the corresponding equations  $m$-sparse
if the sets
$$
(p_0(x),\dots,p_r(x))
$$
and
$$
(q_l(n),\dots,q_d(n))
$$
are $m$-sparse.

Let $c=(c_0,c_1,\dots)\in{\cal C}$. Denote by $(c,x)$ the formal series
$c_0+c_1x+\dots$ and by $(c)_{\geq k}$ the sequence
$(c_k,c_{k+1},...)\in \cC$ with $c_k=c_{k+1}=\cdots =c_{-1}=0$ if
$k<0$. It can be shown that if $R=\cR L$ and $R$ has the form
(\ref{seven}) then
%extended by the elements \begin{equation} \label{ten} c_n=0,
%n=-1,-2,\dots \end{equation} It is well known that
\begin{equation}
\label{eleven}
L(c,x)=0 \Leftrightarrow R(c)_{\geq l}=0.
\end{equation}
Let $R$ be of the form (\ref{seven}) and
$r_0$ be the maximal nonnegative integer root of $q_d(n)$ if such
roots exist, and $-1$ otherwise. Set
$$\iota ^* (R)= d+r_0.$$
If $L \in \rodtp$ and $R= \cR L$, then we set
$\iota ^*(L)= \iota ^*(R)$.
For any $(c_0,c_1,...)\in \cC$ such that $L(c,x)=0$ the values
$c_0,...,c_{\iota ^*(L)}$ let one compute (by means of $\cR L$) the
values $c_{\iota ^*(L)+1}, c_{\iota ^*(L)+2},...$
(these
$c_{\iota ^*(L)+1}, c_{\iota ^*(L)+2},...$
are uniquely determined  because the leading
coefficient of the recurrence $\cR L$ does not vanish when we
compute $c_n$ with $n>\iota ^*(L)$).

We can prove the following
lemma on the possible values of $m$.

\begin{Lemma}
\label{valm}
Let $L$ be of the form (\ref{two}). Let $R= \cR L$ and
$Ly=0$ have a non-polynomial solution $f(x)=c_0+c_1x+\cdots \in
\cS ^{(m)}$.  Then $m\leq \omega (R)$.
\end{Lemma}
{\em Proof:\/} If $m> \omega (R)$ then
there is  $k>\max \{\omega (R), \iota ^* (R)\}$ such
that $c_k=\cdots=c_{k+m-1}=0$. But then $c_n=0$ for all $n\ge k$, i.e.
$f(x)\in \complex [x]$. Contradiction. \qed


{}From now on we will deal only with non-polynomial solutions.
Polynomial solutions
can be found by the algorithm described in \cite{ABP}. Furthermore
we will suppose that $L$ is of the form~(\ref{two}), $R={\cal R}L$
is of the form~(\ref{seven}) and $m$ is a fixed integer such that
\begin{equation}
\label{twelve}
2\le m\le \omega (R).
\end{equation}

{}First we discuss the existence in ${\cal S}^{(m)}$
solutions of $Ly=0$
(i.e. $L^0y=0$). In Section 3 we will consider the search for all $a$
such that the equation $L^ay=0$ has solutions in ${\cal S}^{(m)}$.
%First we prove a preliminary lemma connected with the simplest case.
%
%\begin{Lemma}
%\label{nonsing}
%Let $R$ be an $m$-sparse recurrence and 0 be a non-singular point of
%$Ly=0$ (i.e. $p_r(0) \ne 0$). Then the equation $Ly=0$ has $r=\ord
%L$ linearly independent solutions in ${\cal S}^{(m)}$.  \end{Lemma}
%{\em Proof:\/}
%At a non-singular point, any $r$ initial
%coefficients $c_0,\dots,c_{r-1}$ determine
%a series which satisfies the equation $Ly=0$.
%Obviously, it is possible to construct such a basis for the space
%of vectors $(c_0,\dots,c_{r-1})\in \complex^r$ which contains only
%$m$-sparse vectors. For example, we can take the vectors
%$$(1,0,...,0),(0,1,...,0),...,(0,0,...,1).$$
%If $l<0$ in $R$ then extend
%every vector of the basis by elements $c_{-1}=0,\dots,c_{l}=0$. All
%the extended vectors are $m$-sparse.  Applying $m$-sparse recurrence
%to an $m$-sparse vector of initial elements, we obtain an infinite
%$m$-sparse sequence. \qed

Going back to $R$ of the form~(\ref{seven}), denote
$$
q_{ij}(n) = \left\{
           \begin{array}{ll}
                   q_{j}(n) & \mbox{if } j = i+l \pmod{m}\\
                   0        & \mbox{otherwise},
           \end{array}
              \right.
$$
$i=0,\dots,m-1$; $j=l,\dots,d$ and
$$
R_j = q_{id}(n)E^d+q_{i,d-1}(n)E^{d-1}+\dots+q_{il}(n)E^l,
$$
$i=0,\dots,m-1$ (it is not guaranteed that $q_{id}(n)$ and
$q_{il}(n)$ are nonzero polynomials). Call the set of operators
$$
R_0,\dots,R_{m-1}
$$
the $m$-{\em splitting} of the operator $R$. It is easy to see that
$R_0+\dots+R_{m-1}=R$.

\begin{Lemma}
\label{mspl}
Let $R$ be of the form (\ref{seven}) and
$R_0,...,R_{m-1}$ be the $m$-splitting of $R$. Let
$c\in {\cal C}^{(m)}$. Then
\begin{equation}
\label{13}
(R(c)_{\geq l}=0) \Leftrightarrow (R_i(c)_{\geq l}=0, i=0,\dots,m-1).
\end{equation}
\end{Lemma}
{\em Proof:\/}
a direct check. \qed

The lemma lets one write down a necessary condition for the existence
in ${\cal S}^{(m)}$ of solutions of $Ly=0$.

\begin{Theorem}
Let $R_0,...,R_{m-1}$ be the $m$-splitting of $R$. Let $Ly=0$ have
a solution in ${\cal S}^{(m)}$. Then the right greatest common divisor
($\rGCD$) of the operators $R_0,...,R_{m-1}$ has the positive
$\omega$:
\begin{equation}
\label{14}
\omega ( \rGCD(R_0,...,R_{m-1}))\ge 1.
\end{equation}
(We suppose as usual that $R$ has the form
(\ref{seven}) and that $l$ is the lowest exponent of $E$ in
$\rGCD(R_0,...,R_{m-1})$.)
\end{Theorem}
{\em Proof:\/}
Due to~(\ref{eleven}) and Lemma \ref{mspl}.\qed

The operator $V=\rGCD(R_0,...,R_{m-1})$ can be found by the (right)
Euclidean algorithm. If we apply the algorithm to $m$-sparse
operators then we obviously obtain again an $m$-sparse operator.
Hence, $V$ is an $m$-sparse operator. Coefficients of $V$ are in $\complex(n)$.
Let $\omega (V)\geq 1$.
For some $w(n)\in \complex [n]$ holds
$$w(n)R=Q\o W,$$
where $Q,W\in \complex [n,E],W$ is $m$-sparse and $\omega (W)=\omega
(V) \geq 1$.

It is useful to define  $\iota _*$ which will work together
with $\iota ^*$.
Let $R$ be of the form (\ref{seven}).
Let $r_1$ be the maximal nonnegative integer root of $q_l(n)$ if
such roots exist, and $-1$ otherwise. Set
$$\iota _* (R)= \max \{l+r_1,-1\}.$$
Let $L \in \rodtp$ and $R= \cR L$, then we set
$\iota _*(L)= \iota _*(R)$.
For any $(c_0,c_1,...)$ such that $L(c,x)=0$ the values
$c_k,c_{k+1},...$ with $k>\iota _*(L)+1$
let one compute (by means of $\cR L$) the
values $c_{\iota _*(L)+1}, c_{\iota _*(L)+2},...,c_{k-1}$
(these
$c_{\iota _*(L)+1}, c_{\iota _*(L)+2},...,c_{k-1}$
are uniquely determined  because the lowest
coefficient of the recurrence $\cR L$ does not vanish when we
compute $c_n$ with $n>\iota _*(L)$).

Set
$$u=\max \{\iota _* (W),\; \iota _*(w(n)R),\;
\iota ^*(W),\;\iota ^*(w(n)R)\},$$
$$v=u+\omega (R)-1.$$
%Let non-negative integer $v$ be greater than
%\begin{itemize}
%\item $\ord R$,
%\item all integer roots of the leading coefficient of $R$,
%\item all integer roots of the numerator of the leading coefficient
%of $V$, \item all integer poles of coefficients of $V$.
%\end{itemize}

Using an algorithm proposed in~\cite{ABP} we can find a basis for the
space ${\cal B}$ of vectors $(c_0,...,c_v)\in \complex^{v+1}$ such that
an infinite sequence $c=(c_0,c_1,...)\in {\cal C}$ satisfies the equality
$R(c)_{\geq l}=0$ iff $(c_0,...,c_v)\in {\cal B}$
and all elements $c_n$ with $n>v$ are calculated using the recurrence
$Rc=0$. After a basis $d_0,...,d_w$, $w\le v$ for ${\cal B}$ is found
one can check (a linear problem) whether there exist
$\alpha_0,...,\alpha_w \in \complex$, such that $\alpha_0d_0+...+\alpha_w d_w$
is an $m$-sparse vector whose the last $\omega (R)-1$ components
satisfy recurrence $Wc=0$. If such $\alpha_0,...,\alpha_w$
exist then we can extend the corresponding initial values using the
recurrence $Wc=0$. It will give us an infinite
$m$-sparse sequence $c$ which satisfies $R(c)_{\geq l}=0$.

Later we will need the following theorem
\begin{Theorem}
Let L have the form (\ref{two}) and $R=\cR L$ have the form
(\ref{seven}).
Let $R_0,...,R_{m-1}$ be the $m$-splitting of $R$. Let
$L_i={\cal R}^{-1}R_i$, $i=0,...,m-1$.
Let the equation $Ly=0$ have a solution
in ${\cal S}^{(m)}$. Then
\begin{equation}
\label{15}
\ord \rGCD(L_0,...,L_{m-1}) \ge 1.
\end{equation}
\end{Theorem}
{\em Proof:\/}
Let $f(x)=c_0+c_1x+... \in {\cal S}^{(m)}$, $Lf=0$. Then
$R(c)_{\geq l}=0$ where $c=(c_0,c_1,...)$. By Lemma 3 we have
$R_i(c)_{\geq l}=0$, $i=0,...,m-1$.  By~(\ref{eleven}) we get
$L_if=0$, $i=0,...,m-1$.\qed

%Observe that this Theorem is true in the both cases: 0 is
%an ordinary point and 0 is a singularity of $L$.

Now the last remark of this Section. Suppose we know that
for a fixed $m$ the equation $Ly=0$
has a solution in ${\cal S}^{(m)}$. Then the next step could be,
for example, the attempt to find an $m$-sparse series solution which is
at the same time $m$-{\em hypergeometric}~\cite{PS}
(a power series is $m$-hypergeometric if
its sequence of coefficients $(c_0,c_1,...)$ is $m$-hypergeometric, i.e.
$c_{n+m}=r(n)c_n$, $n=0,1,...$, for a rational function $r(n)$).
Using an algorithm proposed in~\cite{PS} we can find all $m$-hypergeometric
solutions of the recurrence $Vc=0$ and then answer the question about
$m$-hypergeometric $m$-sparse solutions of the original differential
equation.

Note that the mentioned algorithm from \cite{PS} lets one
find only primitive $m$-hy\-per\-ge\-omet\-ric solution of a
recurrence (an $m$-hypergeometric sequence $(c_k,c_{k+1},...)$ is
{\em primitive} if it satisfies no linear homogeneous recurrence with
rational coefficients of order $<m$). But it is obvious that an
$m$-sparse $m$-hypergeometric solution having $c_i\neq 0$ with
arbitrary large $i$ is a primitive $m$-hypergeometric. Thus the
algorithm from \cite{PS} is sufficient for our goal.

If we are interested only in $m$-hypergeometric $m$-sparse
series solutions then there is no necessity to compute
$\rGCD(R_0,...,R_{m-1})$. We can find solutions in the form of
$m$-hypergeometric elements of ${\cal S}$ and then select $m$-sparse
ones from them.
However the usage of~(\ref{14}),(\ref{15}) is convenient when we solve
the problem of searching for the points $a$ at which there exist solutions of
the form~(\ref{three}) with $(c_0,c_1,...)\in {\cal C}^{(m)}$. We will call
these points the $m$-{\em points}.

\section{The search for $m$-points}
Let again $Ly=0$ be an equation with operator of the form~(\ref{two})
and $m$ be a non-negative
integer such that~(\ref{twelve}) is satisfied. We formulate the problem of
the search
for $m$-points as follows: to find all complex values of $a$ such that
the equation
$L^ay=0$, with $L^a$ of the form~(\ref{five}), has a solution in
${\cal S}^{(m)}$.
Consider the operator $L^a$ and the equation $L^ay=0$ regarding $a$ as an
indeterminated parameter. Call them the {\em parametrized}
operator and equation.
We can also construct $R^a={\cal R}L^a$ and $R_i^a$, $i=0,...,m-1$
(the $m$-splitting
of $R^a$). Coefficients of the operators
\begin{equation}
\label{16}
R^a,R_0^a,...,R_{m-1}^a
\end{equation}
are polynomials in $n$ over $\complex[a]$. We can construct the operators
\begin{equation}
\label{17}
L_0^a={\cal R}^{-1}R_0^a,\;...,\;L_{m-1}^a={\cal R}^{-1}R_{m-1}^a.
\end{equation}
If $a_0$ is a value of the parameter $a$ then we can consider, on the one hand,
$L^{a_0}$ and
\begin{equation}
\label{18}
R^{a_0},R_0^{a_0},...,R_{m-1}^{a_0},
\end{equation}
and, on the other hand, the {\em specialization} of the operators~(\ref{16})
for $a=a_0$:
\begin{equation}
\label{19}
R^{a=a_0},R_0^{a=a_0},...,R_{m-1}^{a=a_0}
\end{equation}
as the result of the substitution $a_0$ in~(\ref{16}) for $a$. It will
be shown below that operators~(\ref{18}) and~(\ref{19}) are equal.
This equivalence leads to constructing a set including all $m$-points.
We can gather together all $a_0$ such that
\begin{equation}
\label{20}
\ord \rGCD(R_0^{a_0},...,R_{m-1}^{a_0}) \ge 1.
\end{equation}
Following similar ideas we can gather all $a_0$ such that
\begin{equation}
\label{21}
\ord \rGCD(L_0^{a_0},...,L_{m-1}^{a_0}) \ge 1,
\end{equation}
where $L_i^{a_0}={\cal R}^{-1}R_i^{a_0}$, $i=0,...,m-1$. Observe
that~(\ref{14}),
(\ref{15}) are particular versions of~(\ref{20}),(\ref{21}) with $a_0=0$.

It will be conveniently to
consider (\ref{21}).
Before proceeding further, we
investigate the structure of the operators
$L_0^a,...,L_{m-1}^a$. We consider initialy a specific property of
the operators such that it is possible to give very similar proofs
for the case $L,L_0,...,L_{m-1}$ and for the case
$L^a,L_0^a,...,L_{m-1}^a$. We consider the first case for simplicity.
Recall that $L$ has the form~(\ref{two}), $R={\cal R}L$ has the
form~(\ref{six}).

We denote by $p_{ji}$ the coefficient of $x^i$ in the
polynomial $p_j(x)$.


\begin{Theorem}
Let $R_0,...,R_{m-1}$ be the
$m$-splitting of $R$ and $L_t={\cal R}^{-1}R_t$, $t=0,...,m-1$. Then
\begin{equation}
\label{24}
L_t=\sum_{{x^iD^j\in L}\atop{j-i-l = t\pmod{m}}} p_{ji}x^iD^j,
\end{equation}
$t=0,...,m-1$.
\end{Theorem}
{\em Proof:\/}
Let $x^iD^j\in L$ then ${\cal R}(p_{ji}x^iD^j)$ gives a contribution
only to one of operators $R_0,...,R_{m-1}$, say to $R_t$,
\begin{equation}
\label{25}
j-i-l=t \pmod{m}.
\end{equation}
Since $L_t=\cR ^{-1}R_t$, all $p_{ji}x^iD^j$, $x^iD^j\in L$,
are distributed among $L_0,...,L_{m-1}$ without any transformation
(as among boxes). By~(\ref{25}) formula~(\ref{24}) is satisfied. \qed

Remark that in the analogous proof of this Theorem for
$L^a,L_0^a,...,L_{m-1}^a$ the corresponding $p_{ji}$ would be
polynomial in $a$.

\begin{Theorem}
\label{trco}
Let $R^a={\cal R}L^a$. Let as usual
$L$ and $R$ have the form (\ref{two}) and, resp., (\ref{seven}). Let
$R^a$ be
\begin{equation}
\label{26}
g_{d'}(n,a)E^{d'}+\cdots +g_{l'}(n,a)E^{l'}.
\end{equation}
Then
\begin{equation}
\label{27}
r=d' \ge d,\; l'=l,\; \deg_a g_{l'}(n,a)=0,
\end{equation}
i.e.~(\ref{26}) can be written as
\begin{equation}
\label{28}
g_r(n,a)E^r+\sdots +g_{l+1}(n,a)E^{l+1}+g_{l}(n)E^l.
\end{equation}
For any value $a_0$ of the parameter $a$ we have
\begin{equation}
\label{29}
{\cal
R}L^{a_0}=g_r(n,a_0)E^r+\cdots +g_{l+1}(n,a_0)E^{l+1}+g_{l}(n)E^l.
\end{equation}
\end{Theorem}

{\em Proof:\/} We prove $r=d'$ using (\ref{23}) and
$x^0D^r\in L^a$.
If $x^0D^r\in L$ then $r=d$ else $r>d$.
The equality $\deg_a g_{l'}(n,a)=0$ (i.e. $\deg_a g_l(n,a)=0$) is a
consequence of $\deg_a \lc_x p_j(x+a)=0$. The equality $l=l'$ is
obvious. Finaly, (\ref{29}) is a consequence of the fact that
$\cR$ is a ring-homomorphism. \qed
\\
\noindent {\bf Corollary} {\em (on the specialization).
Operators ({\ref{18}) and ({\ref{19}) are identical}.
\\

Now we investigate $\rGCD (L_0^a,\ldots, L_{m-1}^a)$.

\begin{Theorem}
\label {vw}
As usual let $L$ have the form~(\ref{two}) and for some $k$,
$0\le k\le r$, $\deg p_k(x)$ be a positive integer $s$.
Let $v$, $0\le v\le m-1$, be such that $x^sD^k \in L_v$. Then\\
$(i)\;\; x^sD^k \in L_v^a$,\\
%$(ii)\; L_v^{a_0} \ne 0$ for any $a_0$,\\
$(ii)$ there exists $L_w^a$, $w\ne v$, such that $x^{s-1}D^k \in L_w^a$.

\end{Theorem}
{\em Proof:\/}
Let $p_k(x)=p_{ks}x^s+p_{k,s-1}x^{s-1}+\cdots +p_{k0}$, $s>0$,
$p_{k,s}\ne 0$.  Then
$p_k(x+a)=p_{ks}x^s+(p_{ks}sa+p_{k,s-1})x^{s-1}+\cdots$ and
({\em i}) is obvious. The polynomial in $a$
$$p_{ks}sa+p_{k,s-1} $$
is nonzero. Together with
$$ k-s \ne k-s+1 \pmod m $$ is gives ({\em ii}).\qed

Call a differential operator $L$ of the form (\ref{two}) {\em
primitive} if
\begin{equation}
\label{gcd}
\gcd (p_0(x),...,p_r(x))=1,
\end{equation}
where $\gcd $
denotes the polynomial greatest common divisor. It easy to see that
$L^a$ and $L^{a_0}$ are primitive iff $L$ is primitive.


\begin{Theorem}
\label{ncc}
Let $L$ be a primitive operator of the form (\ref{two}).
Let there exist at least one polynomial of the degree $\ge 1$
among $p_0(x),...,p_r(x)$.
Then
$$
\ord \rGCD(L_0^a,...,L_{m-1}^a)<r
$$
(we mean $\rGCD$ of the operators over
{\rm $\complex (a)(x)$}).
\end{Theorem}
{\em Proof:\/}
Since
\begin{equation}
\label{suma}
L^a=L_0^a+\cdots L_{m-1}^a.
\end{equation}
we have
$$\ord \rGCD (L_0^a,...,L_{m-1}^a)=
\ord \rGCD (L_0^a,...,L_{m-1}^a,L^a) \leq
\ord \rGCD (L_w^a,L^a),$$
where $L_w^a$ is the operator described in Theorem 5. Thanks to

a) $\ord L_w^a \leq \ord L^a$,

b) $L^a$ is primitive,

c) there exist $s,k$ such that $x^sD^k \in L^a$ and
$x^{\tilde s}D^k \notin L_w^a$ for any ${\tilde s}\geqs$,\\
we obtain $ \ord \rGCD (L_w^a,L^a)< \ord L^a$. \qed

\begin{Lemma}
\label{srss}
Let $L$ and $R$ have the form (\ref{two}), (\ref{seven}) and
$R=\cR L$. Let $R$ be an $m$-sparse and 0 be a non-singular point of $L$
(i.e. $p_r(0) \ne 0$). Then the equation $Ly=0$ has $r=\ord L$
linearly independent solutions in $\cS^{(m)}$.
\end{Lemma}
{\em Proof:\/}
At a non-singular point, any $r$ initial
coefficients $c_0,\dots,c_{r-1}$ determine
a series which satisfies the equation $L(y)=0$.
Obviously, it is possible to construct such a basis for the space
of vectors $(c_0,\dots,c_{r-1})\in \complex ^r$ which contains only
$m$-sparse vectors: we can take $c_i=\delta _{ij}$ in the $j$-th
element of the basis, $j=0,...,r-1$. Let us  extend every element
of the basis by elements $c_{-1}=0,\dots,c_l=0$. All the extended
vectors are $m$-sparse.  Applying $m$-sparse recurrence to an
$m$-sparse vector of initial elements, we obtain an infinite
$m$-sparse sequence. \qed

\begin{Lemma}
\label{nsing}
Let $Ly=0$ be a differential equation with constant coefficients.
Then either $L=L_1\o L_2$ where $L_1,L_2$, have constant
coefficients and $L_2$ is $m$-sparse (in such a case all points are
$m$-points), or there are no $m$-points at all.
\end{Lemma}
{\em Proof:\/}
It is obvious that if $L$ is an operator with constant
coefficients then either every point is an $m$-point, or there is no
$m$-point.  It is enough to investigate the point 0. It is easy to
see that $L_0,...,L_{m-1}$ are $m$-sparse operators with constant
coefficients. The operator $S=\rGCD(L_0,...,L_{m-1})$ is either a
constant or an $m$-sparse operator with constant coefficients,
$\ord S \ge 1$ (due to the Euclidean algorithm).  In the last case the
difference operator ${\cal R}S$ is $m$-sparse.
By Lemma \ref{srss} the point 0 will be
an $m$-point of $Ly=0$.\qed

Using the (right) Euclidean algorithm and, for instance, the
differential resultant and subresultants \cite{Char91} we can
construct a differential operator ${\bf S}_m(L)$ over
$\complex [a,x]$ and a finite set ${\bf T}_m(L)\subset \complex$ (the
set of {\em distinctive} values of the parameter $a$) such that  the
substitution of any $a_0\notin {\bf T}_m(L)$ into ${\bf S}_m(L)$ for
$a$ gives the operator $S^{a_0}= \rGCD(L_0^{a_0},...,L_{m-1}^{a_0})$
with $\ord S^{a_0}=\ord {\bf S}_m(L)$.
If $a_0\in {\bf T}_m(L)$ then either
$ S^{a_0} \neq \rGCD(L_0^{a_0},...,L_{m-1}^{a_0})$
or the leading coefficient of ${\bf S}_m(L)$ vanichs if $a=a_0$.
If $L$ is irreducible then $\ord {\bf S}_m (L)=0$ due to
(\ref{suma}).

\begin{Theorem}
\label{cfac}
Let $L$ be a primitive
differential operator. Then either the equation
$Ly=0$ has only finite set of $m$-points or $L=L_1\o L_2$ where
$L_2$ is an $m$-sparse operator with constant coefficients.
\end{Theorem}
{\em Proof:\/}
Let $$S={\bf S}_m(L),\;T={\bf T}_m(L).$$
If there exists only a finite set of
$a_0$ which satisfy (\ref{21})
(i.e. $\ord S=0$)
then there is nothing to prove.
Due to Theorem \ref{ncc}
$\ord S =r$ iff $L$ is an $m$-sparse operator with constant
coefficients. In such a case the theorem is true. Let
\begin{equation}
\label{30}
1\le \ord S < r.
\end{equation}
Use induction on $r$.
\begin{enumerate}
\item $r=1$. The statement is true because~(\ref{30}) is not possible.
\item $r>1$. Since $L^a=L_0^a+...+L_{m-1}^a$ then $L^a = U\circ S$,
where $U$ is a differential operator with coefficients
in $\complex (x,a)$.
If
$a_0 \notin  T$
then by Corollary on the specialization we have that
$\rGCD(L_0^{a_0},...,L_{m-1}^{a_0})$
is equal to the operator $S^{a_0}$ which is the result of the
substitution of $a_0$ in $S$ for $a$. It means that if the equation
$L^{a_0}y=0$ has a solution in $\cS ^{(m)}$ then the equation
$S^{a_0}y=0$ will have the same solution.  We can
%clear denominators of the coefficients by multiplying the
%equation by a polynomial $h(x)$ and then
divide the equation by the polynomial $h(x)$ which is equal to gcd
of the coefficients of the equation. We obtain the equation $S'y=0$,
where $S'$ is a primitive operator
with polynomial coefficients
such that $\ord S'<r$.  By the inductive hypothesis the statement
of the theorem is true for this last equation.  If the equation has a
finite set of $m$-points $\{\alpha _0,\beta _0,...,\gamma _0\}$ then 
due to $L^a$ is equal to (\ref{five}), all $m$-points of the original 
equation are in
\begin{equation}
\label{31}
T \cup \{\alpha _0-a_0,\beta _0-a_0,...,\gamma _0-a_0\}.
\end{equation}
In the case when $S'$ has a right $m$-sparse factor $C$ with constant
coefficients (i.e. $S'=S''\o C$) the original operator $L$ has
the same factor.  Indeed, denote by $U^{a_0}$ the result of the
substitution of $a_0$ into $U$ for $a$. Then
$$ L^{a_0}=U^{a_0}\o h \o S'= U^{a_0}\o
h \o S''\o C, $$
and $L^{a_0}=V\o C$, where $V$ is an operator with polynomial
coefficients.  Let the substitution of $x-a_0$ for $x$ into $V$ gives
$V'$. Then $L=V'\o C$.
\end{enumerate} \qed

This last Theorem provides us with an algorithm to find all
$m$-points of the given equation. The process which was described
recursively in the last Theorem gives either an $m$-sparse operator
$C$ with constant coefficients (then all points are $m$-points) or
a finite set of candidates (each of them can be investigated by
the approach described in Section 2). Note that
in the first case it is possible to obtain, together with $C$,
a finite set of distinctive values collected on all the
levels of the recursive process (shifted like we demonstrated in
(\ref{31}). If the set is not empty then its
elements are of special interest because the number of $m$-sparse
linearly independent solutions can increase at those points.


{\bf Algorithm}

Input: a differential operator $L$;

Output: a finite set of points $M$ and an $m$-sparse operator $C$ 
with constant coefficients;

{\bf 1.} If $\ord L=0$ then return
$$the empty set,\; L;$$

{\bf 2.} $S={\bf S}_m(L),\;T={\bf T}_m(L);$

{\bf 3.} If $\ord S=\ord L$ then return
$$T,\;L;$$

{\bf 4.} Pick up a point $a_0\notin T$; construct $S'$ which is the 
result of dividing $S^{a_0}$  by the greatest common divisor of its 
coefficients; 

{\bf 5.} Apply the algorithm recursvely to $S'$, let
$$  \{\alpha ,\beta ,...,\gamma \}, C$$
be the result of the recursive call;

{\bf 6.} Return
$$T \cup \{\alpha _0-a_0,\beta _0-a_0,...,\gamma _0-a_0\},\;C$$ \qed

After this algorithm was applied we get a finite set of point $M$ and 
an $m$-sparse linear differential operator $C$ with constant 
coefficinets (it is possible that $\ord C=0$, i.e. $C=1$). All the 
point belonging to $M$ must be investigated by the approach described 
in Section 2.

An irreducible operator $L$ can not have a right factor $S$ with
(\ref{30}) and due to Theorems \ref{ncc}, \ref{cfac}
we deduce the following Theorem.


\begin{Theorem}
Let $L$ be a primitive
irreducible operator of the form~(\ref{two}), and
there is $i, 0\leq i \leq r,$ such that
$\deg p_i(x)>0$. Let
$p_{s_1}(x),...,p_{s_k}(x)$ be all nonzero coefficients of $L$ and
let $t_1,...,t_k$ be their degrees. Let $t=\deg p_r(x)$ and
$u_{ji}(a)$ be the coefficient of $x^iD^j$ in $L^a$. Then we have the
following necessary conditions.
\begin{enumerate} \item If there
exists an $m$-point of $Ly=0$ then $s_1-t_1=...=s_k-t_k \pmod m$.
\item If 0 is an $m$-point of $Ly=0$ then
$$
\forall _{i,j}
((x^iD^j\in L)\Rightarrow(j-i = r-t \pmod m)).
$$
\item If $a_0$ is an $m$-point of $Ly=0$ then
$$
\forall _{i,j}
((x^iD^j\in L^a,j-i \ne r-t \pmod m)\Rightarrow( u_{ji}(a_0)=0)).
$$
\end{enumerate}
If 0 or $a_0$ are non-singular points of the equation $Ly=0$ then
the two last
conditions are also sufficient conditions.
\end{Theorem}
{\em Proof:\/}
Since $L$ is irreducible, it can not have a factor $S$ such that
(\ref{30}) holds. Therefore if $a_0$ is an $m$-point then
$\rGCD(L_0^{a_0},...,L_{m-1}^{a_0})=L^{a_0}$. In the case of a
primitive $L$ it is possible only if one of
$L_0^{a_0},...,L_{m-1}^{a_0}$
is equal to $L^{a_0}$ and all others are equal to zero. Then $\cR
L^{a_0}$ is $m$-sparse. This prove the necessity of condition 2, 3.
In turn to prove the necessity of condition 1 we have additionaly to
take into account that the leading coefficients of all polynomial
coefficient of $L^{a_0}$ do not depend on the value of $a_0$. The
suffciency of conditions 2, 3 in the case of ordinary point $a_0$
follows from Lemma \ref{nsing}. \qed

\noindent {\bf Example 2}. $y''+(x-1)y=0$. We have $2\le m\le 3$. The
operator $L=D^2+(x-1)$ is irreducible over $\complex(x)$; $L^a=
D^2+(x-a-1)$.

$m=2$. The first of the necessary conditions formulated in Theorem 8 is not
satisfied: $2-0\ne 0-1 \pmod 2$. The equation has no 2-points.

$m=3$. the first the necessary conditions formulated in Theorem 8 is
satisfied: $2-0=0-1 \pmod 3$; 0 is not an $m$-point because
$2-0\ne 0-0 \pmod 3$. By the third condition $a_0=1$.
This is a non-singular point,
hence it is a 3-point.

By Lemma (\ref{nsing})
the given equation has two linearly independent 3-sparse solutions.
In this case they can be taken as 3-hypergeometric:
$$
1+\sum_{n=1}^\infty \frac{(-1)^n(x-1)^{3n}}
{2\cdot 3\cdot 5\cdot 6\dots (3n-1)\cdot 3n},
$$
$$
(x-1)+\sum_{n=1}^\infty \frac{(-1)^n(x-1)^{3n+1}}
{3\cdot 4\cdot 6\cdot 7\dots 3n\cdot (3n+1)}.
$$

\begin{thebibliography}{99}
\bibitem{ABP}
S.~Abramov, M.~Bronstein, M.~Petkov\v sek (1995):
{\em On Polynomial Solutions of Linear Operator Equations}.
Proc. ISSAC'95, 290-296.

\bibitem{AP}
S.~Abramov, M.~Petkov\v sek (1996):
{\em Special Power Series Solutions of Linear Differential Equation}.
Proc. FPSAC'96, 1-10.

\bibitem{APR}
S.~Abramov, M.~Petkov\v sek, A.~Ryabenko (1996):
{\em Special Formal Series Solutions of Linear Operator Equations}.
Submitted to Discret Math.

\bibitem{Char91}
M.~Chardin (1991):
{\em Differential Resultants and Subresultants}.
Lecture notes in computer science, {\bf 529}.

\bibitem{PS}
M.~Petkov\v sek, B.~Salvy (1993):
{\em Finding All Hypergeometric Solutions of Linear Differential
Equations}.  Proc. ISSAC'93, 27-33.

\end{thebibliography}
\end{document}


