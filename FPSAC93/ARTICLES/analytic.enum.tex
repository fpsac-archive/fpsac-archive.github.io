\documentstyle[12pt]{article}
\input amssym.def
\input amssym.tex
\setlength{\textwidth}{6.2in}
\setlength{\textheight}{9in}
\setlength{\oddsidemargin}{.2in}
\setlength{\topmargin}{-0.25in}
\setlength{\headheight}{0in}
\newcommand{\dd}{\ldots}
\newcommand{\hsp}{\hspace{\parindent}}
\newcommand{\tht}{\theta}
\newcommand{\af}{\alpha}
\newcommand{\be}{\beta}
\newcommand{\la}{\lambda}
\newcommand{\cd}{\cdots}
\newcommand{\mem}{\in}
\newcommand{\lt}{\left(}
\newcommand{\rt}{\right)}
\newcommand{\lf}{\lfloor}
\newcommand{\rf}{\rfloor}
\def\binom#1#2{{#1}\choose{#2}}
\newcommand{\dis}{\displaystyle}
\newcommand{\df}{\displaystyle\frac}
\newcommand{\ds}{\displaystyle\sum}
\newcommand{\In}{\infty}
\newcommand{\wg}{\sim}
\newcommand{\Tht}{\Theta}
\newcommand{\sq}{\sqrt}
\newcommand{\ep}{\epsilon}
\newcommand{\RR}{{\Bbb R}}
\newcommand{\CC}{{\Bbb C}}
\newcommand{\ZZ}{{\Bbb Z}}
\newcommand{\Om}{\Omega}
\newcommand{\beq}{\begin{equation}}
\newcommand{\beql}[1]{\begin{equation}\label{#1}}
\newcommand{\eeq}{\end{equation}}
\newcommand{\eqn}[1]{(\ref{#1})}
\newtheorem{lemma}{Lemma}[section] 
\newtheorem{prop}{Proposition}[section]
\newtheorem{theorem}{Theorem}[section]
\newtheorem{coro}{Corollary}[section]
\newtheorem{conj}{Conjecture}[section] 
\newtheorem{defn}{Definition}[section]
\newtheorem{exam}{Example}[section]
\renewcommand{\theequation}{\arabic{section}.\arabic{equation}}

\catcode`\@=11
\renewcommand{\section}{
	\setcounter{equation}{0}
	\@startsection {section}{1}{\z@}{-3.5ex plus -1ex minus
	-.2ex}{2.3ex plus .2ex}{\large\bf}
	}
\catcode`@=12
\thispagestyle{empty}
\begin{document}
\begin{center}
{\Large {\bf Analytic Methods in Asymptotic Enumeration}} \\
\vspace{\baselineskip}
{\em A. M. Odlyzko} \\
\vspace{0.25\baselineskip}
AT\&T Bell Laboratories \\
Murray Hill, New Jersey 07974 \\
\vspace{1.5\baselineskip}
{\large {\bf Abstract}}
\vspace{1\baselineskip}
\end{center}
\setlength{\baselineskip}{1.5\baselineskip}

Analytic methods give powerful tools for obtaining asymptotic estimates
in combinatorial enumeration.
When they can be used, they usually provide extremely precise results.
However, there are also many situations where they do not apply, and
one has to use elementary or probabilistic reasoning.
Problems in various classes are presented, and general
principles of what kinds of methods are best for various situations are
discussed.

\vspace*{.1in}
\section{Introduction}
\hsp
Analytic methods provide extremely powerful tools for asymptotic enumeration.
Their reach is steadily being extended by new research.
However, there are also many cases where analytic methods
have failed to yield useful
information,
even when it seemed that they ought to apply.
The goal of this paper is to illustrate both successes and failures of analytic
methods, indicate where additional research is needed, and draw some
conclusions about the applicability of such methods to various problems.

The most frequent and obvious reason for failure of analytic methods to yield enumeration information is the lack of a useful analytic generating function.
Of course, for any sequence $a_0 , a_1 , \dd$, we can find another explicit sequence
$b_0, b_1 , \dd$, such that the generating function
\beql{eq101}
f(z) = \sum_{n=0}^\In a_n b_n^{-1} z^n
\eeq
is analytic near $z=0$.
However, what is needed for a function to be useful for
asymptotic enumeration is for it to be in a form that can be used to 
deduce the analytic behavior of that function.
Usually this requires a generating function (typically ordinary or exponential
one) that reflects the combinatorial structure of the sequence
that is being enumerated.
In many situations no such generating function is known, and so it is no surprise that analytic methods do not apply.

This paper will concentrate on situations where an explicit
analytic generating function is known,
but there are difficulties in exploiting this feature.
We first briefly review the standard analytic methods.
We then illustrate with a series of examples of recent successes and failures
of such techniques.

The purpose of this note is not to present a general introduction
to the use of analytic methods in combinatorial
enumeration and analysis of algorithms.
That is done in the author's recent survey and tutorial \cite{AMO5}.
This paper was inspired and is to some extent based on that
work.
There are many other sources for basic asymptotic methods,
such as \cite{Bender74,BenWil,NGB,Egor84,Erdelyi,Fed89a,Flajolet92,GrKnPa89,GK,HaPa73,Hofri,Kemp84,DEK,DEK1,DEK2,Mahmoud,ViFl90,Wilf90,Wyman}.
\section{Standard analytic methods}
\hsp
When a sequence has an explicit single-variable generating
function, there is a wealth of techniques that usually suffice to determine
the asymptotics of that sequence with little effort.
The coefficients of generating functions $f(z)$ arising in
combinatorics or analysis of algorithms are usually nonnegative.
Hence there is typically a single singularity at $z=x_0 \in \RR$
on the circle of convergence of 
\beql{eq201}
f(z) = \sum_{n=0}^\In a_n z^n
\eeq
whose influence dominates the behavior of $a_n$.
It is referred to as the {\em dominant singularity}.
If it is the only singularity on $|z| = x_0$, everything simplifies.
When there are other singularities, the situation is more complicated but
often still tractable.

In almost all cases a pretty good upper bound can be obtained
for the coefficients of an analytic generating function with nonnegative coefficients by a trivial argument.
For any real $x > 0$ such that the power series \eqn{eq201}
converges, we have
\beql{JU202}
a_n x^n \le \sum_{k=0}^\In a_k x^k = f(x) ~,
\eeq
so
\beql{JU203}
a_n \le f(x) x^{-n}~.
\eeq
In particular,
\beql{JU204}
a_n \le \min_x f(x) x^{-n}~,
\eeq
where the minimum is taken over $x > 0$ for which $f(x)$ converges.
In many situations one does not even need to find the $x$ that
minimizes the bound \eqn{JU203}, as any $x$ in a substantial range near the
minimizing one will give a satisfactory result.
The bound \eqn{JU204} is often only a factor of $n^{1/2}$ or so
away from the correct one.

The bound \eqn{JU203} relied only on the nonnegativity of the coefficients $a_n$ and not on the analyticity of $f(x)$, and it can be generalized
to other kinds of generating functions.
It is also possible to prove that the bound \eqn{JU204} is close to
best possible, at least for summatory functions of the coefficients
\cite{AMO4}.
(Real-variable proofs usually cannot obtain lower bounds for
individual coefficients.)
For precise information about the $a_n$, however, one usually has to use analytic methods.

There are two basic classes of methods that apply to single-variable analytic generating functions.
If the dominant singularity is small, so that $f(z)$ in an appropriate
neighborhood of $z=x_0$
behaves like $(z-x_0)^\af$ or $( \log (z- \af ))^\beta$, then
transfer methods are very effective.
For example, the generating function of 2-regular graphs is known \cite{Comtet74} to be
\beql{eq202}
f(z) = (1-z)^{-1/2} \exp (-z/2 - z^2 /4) ~,
\eeq
so that the dominant singularity is at $z=1$ and there are no other singularities on $|z| =1$.
Transfer theorems, such as those of \cite{FAMO3} immediately show that
\beql{eq203}
a_n \sim \pi^{-1/2} \exp (-3/4) n^{-1/2} ~~~\mbox{as}~~~
n \to \In ~.
\eeq

If the dominant singularity is large, so that $f(z)$ blows up rapidly as $z$ approaches $x_0$ along the real axis from the left, then another class
of methods, based largely on the saddle point method, are most productive.
The saddle point, $z=x_0 = x_0 (n)$, is that value of $x$ which
where the minimum is achieved in \eqn{JU204}.
(We assume here that $a_n \ge 0$ for all $n$,
and that there is a single dominant singularity that is large enough
to guarantee there is only this one saddle point that matters.
For precise definitions, see the references mentioned at the end of Section~1.)
Then usually one obtains an estimate of the form
\beql{JU206}
a_n \sim (2 \pi b(x_0))^{-1/2} f(x_0 ) x_0^{-n} ~~~\mbox{as}~~~
n \to \In ~,
\eeq
where
\beql{JU207}
b(x) = x \lt x~ \frac{f'(x)}{f(x)} \rt ' ~.
\eeq
In general, proving that this estimate holds is rather laborious, as there are many
sometimes messy details to check.
A surprisingly large number of problems can be solved by using
theorems of Hayman \cite{WKH},
which present conditions that guarantee that the saddle point
estimates do apply.
For example, the Bell numbers $B_n$, which
have the exponential generating function
\beql{eq204}
B(z) = \sum_{n=0}^\In B_n \frac{z^n}{n!} =
\exp ( \exp (z) -1) ~,
\eeq
satisfy
\beql{eq205}
B_n \sim n! (2 \pi x_0^2 ~\exp (x_0))^{-1/2}
\exp ( \exp (x_0)-1) x_0^{-n} ~~~\mbox{as}~~~
n \to \In ~,
\eeq
where $x_0 \exp (x_0) = n$.
The estimate \eqn{eq205} can be derived immediately from the
results of \cite{WKH}.

Hayman's results are all based on the saddle point method.
He defines a class of functions, which are now called Hayman-admissible, and proves that the estimate \eqn{JU206}
holds for them.
Most important, he shows that there is structure among the Hayman-admissible functions,
so that if $f(z)$ and $g(z)$ are admissible, then so are $\exp (f(z))$ and
$f(z) g(z)$, for example.
The operations on admissible functions that keep them admissible
correspond to frequently used operations on combinatorial
structures that those functions enumerate.
Therefore one obtains, almost for free, asymptotics
of a variety of interesting objects, just by
applying Hayman's operations.

The transfer theorem and saddle point
methods for determining the asymptotics of sequences from univariate generating functions suffice for most examples in books such as
\cite{Comtet74}.
They are sufficiently well understood that they can be incorporated into automated systems for determining asymptotics, such as that of \cite{FlaSaZi91}.
(See also \cite{HaRoSc75}.)
However, there are many problems where more sophisticated approaches
are needed.
\section{Recent successes and failures}
\hsp
Single variable asymptotics are well understood.
That is not the case when multivariate generating functions are required.
Both transfer theorems and saddle point methods can be generalized (see \cite{AMO5}),
but their applicability is more limited than in the univariate case.
However, there have been remarkable successes in extending some of these methods even to cases where the number of variables in the problem
grows rapidly.
A good example comes from the work of McKay and Wormald
\cite{McKay90,McKayW90}.
\begin{exam}
\label{Simple}
Simple labeled graphs of high degree.
{\rm Let $G(n; d_1 , \dd , d_n )$ be the number of labeled simple graphs on $n$ vertices with degree sequence
$d_1 , d_2 , \dd , d_n$.
Then $G(n; d_1 , \dd , d_n )$ is the coefficient of $z_1^{d_1} z_2^{d_2} \cdots z_n^{d_n}$ in
\beql{JC1310}
F = \prod_{j,k=1 \atop j < k}^n (1+z_j z_k ) ~,
\eeq
and so by Cauchy's theorem
\beql{JC1311}
G(n; d_1 , \dd , d_N ) = (2 \pi i)^{-n}
\int \cdots \int F z_1^{-d_1 -1} \cdots z_n^{-d_n -1} d z_1 \cdots d z_n ~,
\eeq
where each integral is on a circle centered at the origin.
The difficulty of this problem arises from the large number of
variables that are used.
The general principle of the saddle point method is to choose an appropriate contour of integration so that a small region of the space of integration gives the dominant contribution to the integral.
This method, which is well understood for a single variable, can also
be extended easily to a fixed number of variables.
When the number of variables increases, though,
formidable difficulties arise, and it was a great achievement for McKay and Wormald to overcome all the technical problems.
Let all the radii be equal to some $r > 0$.
The integrand takes on its maximum absolute value on the product of these circles at precisely the two points $z_1 = z_2 = \cdots = z_n = r$ and $z_1 = z_2 = \cdots = z_n = -r$.
If $d_1 = d_2 = \cdots = d_n =d$,
so that we consider only regular graphs, McKay and Wormald \cite{McKayW90} show that for an appropriate choice of the radius $r$, these
two points are actually saddle points of the integrand, and succeed through careful analysis
in proving that if $dn$ is even, and $\min (d,n-d-1) > cn( \log n )^{-1}$ for some $c > 2/3$, then
\beql{JC1312}
G(n,d,d, \dd , d) = 2^{1/2} ( 2 \pi n \la^{d+1} (1- \la)^{n-d} )^{-n/2}
\exp
\lt \frac{-1 + 10 \la - 10 \la^2}{12 \la (1- \la)} + O( n^{- \zeta}) \rt
\eeq
as $n \to \In$ for any $\zeta < \min (1/4 , 1/2 - 1/(3c))$, where $\la = d/(n-1)$. \hfill $\blacksquare$
}
\end{exam}

In general, multivariate asymptotics when the dimension grows
rapidly is full of unsolved problems.
\begin{exam}
\label{exISRP}
Increasing subsequences in random permutations.
{\rm The distribution of the length of the longest
increasing subsequence of a random permutation of $\{1, \dd , n \}$ has been under study for a long time,
and it is known that almost always it is very close
to some $\tau_n$, where $\tau_n \sim 2n^{1/2}$ as $n \to \In$.
(See \cite{BB,OPWW} for references.)
However, more precise asymptotics of $\tau_n$ are not known.
A new approach, developed in \cite{OPWW}, shows that if $f(n,k)$ is the number of
permutations of $\{1, \dd , n \}$ which have no increasing
subsequences of length $> k$, then
\beql{RP304}
f(n,k) = \frac{2^{2n} (n!)^2}{(2 \pi )^k k! (2n)!}
\int_{- \pi}^\pi \cdots \int_{- \pi}^\pi
\lt \sum_{j=1}^k \cos \theta_j \rt^{2n}
\prod_{1 \le h,j \le k \atop h \neq j}
|e^{i \theta_h} - e^{i \theta_j} | d \theta_1 \cdots d \theta_k~.
\eeq
This leads to asymptotic estimates for
$f(n,k)$ for $k=o(n^{1/2})$, which is far beyond previous bounds.
However, so far this integral has not been estimated
rigorously for $k \sim cn^{1/2}$ for a fixed $c > 0$, which
is the region of interest.
There are serious technical difficulties, since the region where the integrand gives the main contribution changes,
and so the usual methods have not worked yet. \hfill $\blacksquare$
}
\end{exam}

Even univariate problems still present challenges.
Some of the problems arise when the relevant generating
functions do not have explicit forms,
but are defined by recursions.
\begin{exam}
\label{Heights}
Heights of binary trees.
{\rm We let $B_n$ denote the number of binary trees of size $n$,
so that $B_0 =1$ (by convention), $B_1 =1$, $B_2 =2$, $B_3 = 5, \dd \,$.
Let
\beql{Apreq151}
B(z) = \sum_{n=0}^\In B_n z^n ~.
\eeq
Since each nonempty binary tree consists of the root and two binary trees
(the left and right subtrees), we obtain the functional equation
\beql{Apreq152}
B(z) = 1+ z B(z)^2~.
\eeq
This implies that
\beql{Apreq153}
B(z) = \frac{1- (1-4z)^{1/2}}{2z} ~,
\eeq
so that
\beql{Apreq154}
B_n = \frac{1}{n+1} {{2n}\choose{n}} ~,
\eeq
and the $B_n$ are the Catalan numbers.
Stirling's formula then shows that
\beql{Apreq155}
B_n \sim \pi^{-1/2} n^{-3/2} 4^n ~~~\mbox{as}~~~ n \to \In ~.
\eeq

Let $B_{h,n}$ be the number of binary trees of size $n$ and height $\le h$,
and let
\beql{Apreq156}
b_h (z) = \sum_{n=0}^\In B_{h,n} z^n ~.
\eeq
Then
\beql{Apreq157}
b_0 (z) =0 , ~~~b_1 (z) =1 ~,
\eeq
and an extension of the argument that led to the relation \eqn{Apreq152} yields
\beql{Apreq158}
b_{h+1} (z) =1 + z b_h (z)^2 ~,~~~
h \ge 0 ~.
\eeq
The $b_h (z)$ are polynomials in $z$ of degree $2^{h-1} -1$ for $h \ge 1$.
Unfortunately there is no simple formula for them like
Eq.~\eqn{Apreq153} for $B(z)$, and one has to work with the recurrence \eqn{Apreq158} to obtain most of the results about heights of binary trees.
Different problems involve study of the recurrence in different ranges of values of $z$, and the behavior of the recurrence varies drastically.

For any fixed $z$ with $|z| \le 1/4$, $b_h (z) \to B(z)$ as $h \to \In$.
For $|z| > 1/4$ the behavior of $b_h (z)$ is more complicated,
and is a subject of nonlinear
dynamics.
For any real $z$ with $z > 1/4$, $b_h (z) \to \In$ as $h \to \In$.
To study the distribution of the $B_{h,n}$ as $n$ varies for $h$ fixed, but large,
it is necessary to investigate this range of rapid
growth.
It can be shown \cite{FAMO1} that for any $\la_1$ and $\la_2$ with $0 < \la_1 < \la_2 < 1/2$,
\beql{skp159}
B_{h,n} = \frac{\exp (2^{h-1} ( \be (r) - r \be' (r) \log r ))}{2^{(h-1)/2} ( 2 \pi (r^2 \be'' (r) + r \be' (r)))^{1/2}}
(1+ O(2^{-h/2} ))
\eeq
uniformly as $h,n \to \In$ with
\beql{skp1510}
\la_1 < n/2^h < \la_2 ~.
\eeq
The function $\be (x)$
and the number $r$ are defined by
transcendental functions that are easy to compute.
The proof of the estimate \eqn{skp159} is derived from a precise estimate
for $b_h (z)$
valid in a region along the half-axis
$x > 1/4$, and
the saddle point method.

The methods that are used to study the average height are different
from those used for trees of a fixed height.
The basic approach of \cite{FAMO} is to let
$$H_n = \sum_{T \atop |T| =n} ht (T) ~,$$
where the sum is over the binary trees $T$ of size $n$,
and $ht (T)$ is the height of $T$.
Then the average height is just $H_n / B_n$.
The generating function for the $H_n$ is
\beq\label{RPe16}
H(z) = \sum_{n=0}^\In H_n z^n =
\sum_{h \ge 0} (B(z) - b_h (z)) ~,
\eeq
and the analysis of \cite{FAMO} proceeds by investigating the behavior of $H(z)$ near
$z=1/4$.
If we let
\begin{eqnarray}
\label{RPe17}
\ep (z) & = & (1-4z)^{1/2} ~, \\
e_h (z) & = & (B(z) - b_h (z)) / (2B (z)) ~, \label{Rpe18}
\end{eqnarray}
then the recurrence \eqn{Apreq158} yields
\beq\label{RPe19}
e_{h+1} (z) = (1- \ep (z)) e_h (z) (1- e_h (z)) ~,~~
e_0 (z) = 1/2 ~.
\eeq
Extensive analysis of this relation yields an approximation
to $e_h (z)$ of the form
\beq\label{RPe20}
e_h (z) \approx \frac{\ep (z) (1- \ep (z))^h}{1-(1- \ep (z))^h} ~,
\eeq
valid for $| \ep (z) |$ sufficiently small,
$| \mbox{Arg}~ \ep (z) | < \pi /4 + \delta$ for a fixed $\delta > 0$.
(The precise error terms in this approximation are complicated,
and are given in \cite{FAMO}.)
This then leads to an expansion for $H(z)$ in a sector
$|z- 1/4 | < \af$,
$\pi /2 - \beta < | \mbox{Arg} (z- 1/4 ) | < \pi /2 + \beta$ of the form
\beq\label{RPe21}
H(z) = - 2 \log (1-4z) + K + O (| 1-4z|^v ) ~,
\eeq
where $v$ is any constant, $v < 1/4$, and $K$ is a fixed constant.
Transfer theorems then yield the asymptotic estimate
\beq\label{RPe22}
H_n \sim 2 n^{-1} 4^n ~~\mbox{as}~~ n \to \In ~.
\eeq
When we combine \eqn{RPe22} with \eqn{Apreq155},
we obtain the desired result that the average height of a binary tree of size $n$
is $\sim 2 ( \pi n )^{1/2}$ as $n \to \In$.

For extremely small and large heights, different methods are used.
It follows from \cite{FGOR} that
\beql{RPe25}
\frac{B_{h,n} - B_{h-1,n}}{B_n}
\le \exp (-c (h^2 /n + n/h^2))
\eeq
for a constant $c>0$, which shows that extreme heights are infrequent.
(The estimates in \cite{FGOR} are more precise than \eqn{RPe25}.)
Bounds of the above form for small heights are obtained in \cite{FGOR}
by studying the behavior of the $b_h (z)$ almost on the boundary
between convergence and divergence.
Let $x_h$ be the unique positive root of $b_h (z) =2$.
Note that $B(1/4) =2$,
and each coefficient of the $b_h (z)$ is nondecreasing as $h \to \In$.
Therefore $x_2 > x_3 > \cdots > 1/4$.
More effort shows \cite{FGOR}
that $x_h$ is approximately
$1/4 + \af h^{-2}$ for a certain $\af > 0$.
This leads to an upper bound for $B_{h,n}$.
Bounds for trees of large heights are even easier to obtain,
since they only involve upper bounds for the $b_h (z) - b_{h-1} (z)$ inside the disk of convergence
$|z| < 1/4$. \hfill $\blacksquare$
}
\end{exam}

There are some univariate generating functions which so far have not yielded
to analytic approaches.
\begin{exam}
\label{HRandom}
Heights of random binary search trees.
{\rm Example~\ref{Heights} is set in the standard combinatorial
counting model, in which all trees of a given size are counted
equally.
In many applications it is desirable to have different weights
for different trees.
For example, if random permutations are used to construct binary
search trees, then the two trees of maximal heights will have probability of occurring of $1/n!$ each, whereas more balanced trees will have exponentially
larger probabilities.
The average height in this model turns out to be $\sim c \log n$
as $n \to \infty$, where $c=4.311 \dd$ is a certain constant.
This was proved by Devroye \cite{Devroye86,Devroye87} using branching
processes methods.
(See also \cite{Mahmoud}.)
It would be nice to develop the analytic generating
function approach to this problem, since it might then be used to obtain new information about the distribution of heights, for example.
To study heights, we can consider a model in which at time $n \ge 1$
there will be $n$ pebbles on the nonnegative integers.
At time $n=1$, the single pebble will be at 0.
At time $n$, one of the $n$ pebbles will be chosen at random and removed from its position,
say from $k$, and replaced by two pebbles at $k+1$.
The question then is, what is the distribution of the pebbles
at time $n$.
By Devroye's results we know that the position of the rightmost pebble will on average be asymptotic to $c \log n$.
Combined with earlier results of Pittel (see \cite{Mahmoud}),
this shows that this rightmost pebble is at about $c \log n$ most of the time.
The methods of Devroye and Pittel are probabilistic,
and so far nobody has developed an analytic approach to this problem.
This is in contrast to the situation where we ask for the average number of pebbles on a given integer,
where simple recurrences show that the distribution is given explicitly
in terms of Stirling numbers, whose asymptotics are well understood. \hfill $\blacksquare$
}
\end{exam}

There are some problems with rather simple generating functions where analytic
methods fail, not because of the complexity of the function, but rather
because of the basic limitations of known analytic
methods.
This occurs in a set partition problem described below.
\begin{exam}
\label{Set}
Set partitions with distinct block sizes.
{\rm Let $a_n$ be the number of partitions of a set of $n$ elements into blocks of distinct sizes.
Then
$a_n = b_n \cdot n!$,
where $b_n = [z^n] f(z)$, with
\beql{JCeq1}
f(z) = \prod_{k=1}^\In \lt 1 + \frac{z^k}{k!} \rt ~.
\eeq
The function $f(z)$ is entire and has nonnegative coefficients,
so it might appear as an ideal candidate
for an application of the methods for dealing with large
singularities, such as the saddle point technique.
However,
on circles $|z| = (n+1/2) /e$, $n \in \ZZ^+$,
$f(z)$ does not vary much, so there are technical
problems in applying these analytic methods.
The saddle point method succeeds when the contour of integration can be chosen so that the integrand is large only in a small range,
and its behavior in that small range is given by an
approximation by a scaled Gaussian function.
Because of the peculiar distribution of the roots of $f(z)$, no such contour
can be found.
More than that, the results one obtains are very different from those that the saddle point method yields.
The estimates obtainable from the saddle point method
give (for the usual type of function encountered in combinatorial enumeration)
estimates for the coefficients that vary smoothly with the index of the coefficient.
On the other hand, combinatorial estimates
can be used to show \cite{KORSW} that
the $b_n$ behave in a ``regularly irregular'' way, so that, for example,
\begin{eqnarray}
\label{JCeq2}
b_{m(m+1)/2-1} & \sim & b_{m(m+1)/2} ~~~\mbox{as}~~~ m \to \In ~, \\
b_{m(m+1)/2} & \sim & m b_{m(m+1)/2 +1} ~~~\mbox{as}~~~
m \to \In ~.
\label{JCeq3}
\end{eqnarray}
The term $b_n z^n$ for $n= m(m+1)/2$ for example, comes almost
entirely from the product of $z^k / k!$, $1 \le k \le m$, all other products
contributing an asymptotically negligible amount.
There do not seem to be any analytic methods that obtain these results
without essentially redoing the combinatorial estimates. \hfill $\blacksquare$
}
\end{exam}
\clearpage
\begin{thebibliography}{99}
%\begin{thebibliography}{AndrewsCMOM}
%\bibitem[Bender74]{Bender74}
\bibitem{Bender74}
E.~A. Bender,
Asymptotic methods in enumeration,
{\em SIAM Review},
{\em 16} (1974),
pp.~485--515.
%\bibitem[BenWil]{BenWil}
\bibitem{BenWil}
E.~A. Bender
and S.~G. Williamson,
{\em Foundations of Applied Combinatorics},
Addison-Wesley, 1991.
%\bibitem[BB]{BB}
\bibitem{BB}
B. Bollob\'{a}s and G. Brightwell,
The height of a random partial order:
concentration of measure, {\em Ann. Appl. Prob. 2} (1992), 1009--1018.
%\bibitem[NGB]{NGB}
\bibitem{NGB}
N.~G. de~Bruijn,
{\em Asymptotic Methods in Analysis},
North--Holland, Amsterdam, 1958.
%\bibitem[Comtet74]{Comtet74}
\bibitem{Comtet74}
L. Comtet,
{\em Advanced Combinatorics},
Reidel, Dordrecht, 1974.
%Article{Devroye86,
%\bibitem[Devroye86]{Devroye86}
\bibitem{Devroye86}
L. Devroye,
A note on the expected height of binary search trees,
{\em J. ACM},
{\em 33}
(1986),
pp.~489--498.
%Article{Devroye87,
%\bibitem[Devroye87]{Devroye87}
\bibitem{Devroye87}
L. Devroye,
Branching processes in the analysis of the heights of trees,
{\em Acta Informatica},
{\em 24}
(1987),
pp.~277--298.
%\bibitem[Egor84]{Egor84}
\bibitem{Egor84}
G.~P. Egorychev,
{\em Integral Representation and the Computation of Combinatorial Sums},
Amer. Math. Soc. 1984.
%\bibitem[Erdelyi]{Erdelyi}
\bibitem{Erdelyi}
A. Erd{\'e}lyi,
{\em Asymptotic Expansions}, Dover reprint, 1956.
%\bibitem[Fed89a]{Fed89a}
\bibitem{Fed89a}
M.~V. Fedoryuk,
Asymptotic methods in analysis, pp.~83--191 in
{\em Analysis I},
R.~V. Gamkrelidze, ed., Springer 1989.
%\bibitem[Flajolet92]{Flajolet92}
\bibitem{Flajolet92}
P. Flajolet,
Analytic analysis of algorithms,
{\em Proc. ICALP '92}, Springer Lecture Notes in Computer Science, 1992, to be published.
%\bibitem[FGOR]{FGOR}
\bibitem{FGOR}
P. Flajolet, Z. Gao,
A.~M. Odlyzko,
and B. Richmond,
The height of binary trees and other simple trees,
{\em Combinatorics, Probability, and Computing} (1993), to appear.
%\bibitem[FAMO]{FAMO}
\bibitem{FAMO}
P. Flajolet and A.~M. Odlyzko,
The average height of binary trees and other simple trees,
{\em J. Comput. System Sci., 25} (1982), pp.~171--213.
%\bibitem[FAMO1]{FAMO1}
\bibitem{FAMO1}
P. Flajolet and A.~M. Odlyzko,
Limit distributions for coefficients of iterates of
polynomials with application
to combinatorial enumeration,
{\em Math. Proc. Cambridge Phil. Soc., 96} (1984), pp.~237--253.
%\bibitem[FAMO3]{FAMO3}
\bibitem{FAMO3}
P. Flajolet and A.~M. Odlyzko,
Singularity analysis of generating function,
{\em SIAM J. Discrete Math., 3} (1990), pp.~216--240.
%Article{FlaSaZi91,
%\bibitem[FlaSaZi91]{FlaSaZi91}
\bibitem{FlaSaZi91}
P. Flajolet, B. Salvy, and P. Zimmermann,
Automatic average--case analysis of algorithms,
{\em Theoretical Computer Science},
{\em 79} (1991), 37--109.
%\bibitem[GonnetBY]{GonnetBY}
\bibitem{GonnetBY}
G.~H. Gonnet and R. Baeza-Yates,
{\em Handbook of Algorithms and Data Structures},
2nd ed., Addison-Wesley, 1991.
%\bibitem[GJ]{GJ}
\bibitem{GJ}
I. Goulden and D. Jackson,
{\em Combinatorial Enumeration},
John Wiley, New York, 1983.
%Book{GrKnPa89,
%\bibitem[GrKnPa89]{GrKnPa89}
\bibitem{GrKnPa89}
R.~L. Graham, D.~E. Knuth, and O. Patashnik,
{\em Concrete Mathematics},
Addison Wesley, 1989.
%\bibitem[GK]{GK}
\bibitem{GK}
D. H. Greene and D. E. Knuth,
{\em Mathematics for the Analysis of Algorithms},
2nd ed., Birkh\"{a}user, Boston, 1982.
%Book{HaPa73,
%\bibitem[HaPa73]{HaPa73}
\bibitem{HaPa73}
F. Harary and E.~M. Palmer,
{\em Graphical Enumeration},
Academic Press, 1973.
%Article{HaRoSc75,
%\bibitem[HaRoSc75]{HaRoSc75}
\bibitem{HaRoSc75}
F. Harary, R.~W. Robinson, and A.~J. Schwenk,
Twenty--step algorithm for determining the asymptotic
number of trees of various species,
{\em J. Austral. Math. Soc. {\rm (Series A)}},
{\em 20} (1975),
pp.~483--503.
%\bibitem[WKH]{WKH}
\bibitem{WKH}
W.~K. Hayman,
A generalization of Stirling's formula,
{\em J. reine angew. Math., 196} (1956), pp.~67--95.
%\bibitem[Hofri]{Hofri}
\bibitem{Hofri}
M. Hofri,
{\em Probabilistic Analysis of Algorithms},
Springer, 1987.
%\bibitem[Kemp84]{Kemp84}
\bibitem{Kemp84}
R. Kemp,
{\em Fundamentals of the Average Case Analysis of Particular Algorithms},
Wiley, 1984.
%\bibitem[KORSW]{KORSW}
\bibitem{KORSW}
A. Knopfmacher, A. Odlyzko, B. Richmond, G. Szekeres,
and N. Wormald, manuscript in preparation.
%\bibitem[DEK]{DEK}
\bibitem{DEK}
D.~E. Knuth,
{\em The Art of Computer Programming Vol. 1}:
{\em Fundamental Algorithms},
2nd ed., Addison--Wesley, Reading, 1973.
%\bibitem[DEK1]{DEK1}
\bibitem{DEK1}
D.~E. Knuth,
{\em The Art of Computer Programming Vol. 2}:
{\em Semi--Numerical Algorithms},
2nd ed.,
Addison--Wesley, Reading, 1981.
%\bibitem[DEK2]{DEK2}
\bibitem{DEK2}
D.~E. Knuth,
{\em The Art of Computer Programming Vol. 3}:
{\em Sorting and Searching},
Addison--Wesley, Reading, 1973.
%\bibitem[Mahmoud]{Mahmoud}
\bibitem{Mahmoud}
H.~S. Mahmoud,
{\em Evolution of Random Search Trees},
Wiley, 1992.
%\bibitem[McKay90]{McKay90}
\bibitem{McKay90}
B.~D. McKay,
The asymptotic numbers of regular tournaments,
eulerian digraphs, and eulerian and oriented graphs,
{\em Combinatorica, 10} (1990), 367--377.
%\bibitem[McKayW90]{McKayW90}
\bibitem{McKayW90}
B.~D. McKay and N.~C. Wormald,
Asymptotic enumeration by degree sequence of graphs of high degree,
{\em European J. Combinatorics, 11} (1990), 565--580.
%\bibitem[AMO4]{AMO4}
\bibitem{AMO4}
A.~M. Odlyzko,
Explicit Tauberian estimates for functions with positive coefficients,
{\em J. Comput. Appl. Math., 41} (1992), 187--197.
%\bibitem[AMO5]{AMO5}
\bibitem{AMO5}
A.~M. Odlyzko,
Asymptotic enumeration methods,
in {\em Handbook of Combinatorics}, R.~L. Graham,
M. Gr\"{o}tschel, and L. Lov\'{a}sz, eds., North-Holland, to appear.
%\bibitem[OPWW]{OPWW}
\bibitem{OPWW}
A.~M. Odlyzko, B. Poonen, H. Widom, and H.~S. Wilf,
manuscript in preparation.
%\bibitem[JRIO]{JRIO}
\bibitem{JRIO}
J. Riordan,
{\em Introduction to Combinatorial Analysis},
John Wiley, New York, 1958.
%\bibitem[Riordan]{Riordan}
\bibitem{Riordan}
J. Riordan,
{\em Combinatorial Identities},
Wiley, 1968.
%\bibitem[RPS]{RPS}
\bibitem{RPS}
R.~P. Stanley,
{\em Enumerative Combinatorics}, vol.~1,
Wadsworth and Brooks/Cole, Monterey, 1986.
%incollection{ViFl90,
%\bibitem[ViFl90]{ViFl90}
\bibitem{ViFl90}
J. Vitter and P. Flajolet,
Analysis of algorithms and data structures,
{\em Handbook of Theoretical Computer Science},
Vol.~A: Algorithms and Complexity,
Ch.~9,
pp.~432--524,
J. Van Leeuwen, ed., North Holland, 1990.
%\bibitem[Wilf90]{Wilf90}
\bibitem{Wilf90}
H.~S. Wilf,
{\em Generatingfunctionology},
Academic Press, 1990.
\bibitem{Wyman}
M. Wyman,
The asymptotic behavior of the Laurent coefficients,
{\em Canad. J. Math., 11} (1959), 534--555.
\end{thebibliography}
\end{document}
