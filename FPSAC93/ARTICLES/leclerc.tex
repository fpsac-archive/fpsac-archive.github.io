\input a4new
\magnification=1200
\overfullrule=0pt
\def\boxit#1#2{\setbox1=\hbox{\kern#1{#2}\kern#1}%
\dimen1=\ht1 \advance\dimen1 by #1 \dimen2=\dp1 \advance\dimen2 by #1
\setbox1=\hbox{\vrule height\dimen1 depth\dimen2\box1\vrule}%
\setbox1=\vbox{\hrule\box1\hrule}%
\advance\dimen1 by .4pt \ht1=\dimen1
\advance\dimen2 by .4pt \dp1=\dimen2 \box1\relax}
\def\tab#1{\boxit{1pt}{$\matrix{#1}$}}
\def\bitab#1{\boxit{1pt}{\boxit{1pt}{$\matrix{#1}$}}}
\def\bo#1{\boxit{1pt}{$#1$}}
\def\bol#1{\boxit{4pt}{$#1$}}
\def\bop#1{\boxit{3pt}{$#1$}}
\def\he{\hskip 1mm}
\def\th{\noalign{\smallskip}\noalign{\hrule}\noalign{\smallskip}}
\def\tvi{\vrule height 1pt depth 0pt width 0pt}
\font\petcap=cmcsc10
\font\tpetcap=cmcsc8
\font\petit=cmti7
\font\twelvebf=cmbx12
\centerline{\twelvebf POWERS OF STAIRCASE SCHUR FUNCTIONS}
\centerline{\twelvebf AND SYMMETRIC ANALOGUES OF BESSEL POLYNOMIALS}
\vskip 1cm
\centerline{Bernard LECLERC \footnote{$\dagger$}{
\sevenrm Institut Gaspard Monge, 
Universit\'e de Marne-la-Vall\'ee, 2 rue de la Butte-Verte 93160 
Noisy-le-Grand Cedex France}}
\vskip 1cm
{\sevenbf Abstract. }{\sevenrm  We present several identities involving 
staircase Schur functions. These identities are then interpreted in terms of 
a sequence of orthogonal polynomials in one variable ${\scriptstyle x}$, with 
coefficients in the ring of symmetric functions. By an appropriate 
specialization these polynomials reduce to Bessel polynomials. This leads to 
a new determinantal expression for Bessel polynomials and suggests that their 
combinatorics might be linked to Young tableaux or shifted Young 
tableaux.}\bigskip 

\bf1  Introduction
\rm\medskip
Schur functions $S_I$ and skew Schur functions $S_{I/J}$ are indexed by 
partitions $I$ or skew partitions $I/J$, which are visualized graphically by 
a diagram of boxes. Several families of Schur functions associated with 
special diagrams are known to satisfy particular identities. Thus, Lascoux 
and Pragacz provided a determinantal expression of Schur functions and skew 
Schur functions in terms of ribbon functions, which generalizes the classical 
decomposition into hook functions given by Giambelli \bf[LP]\rm.
\smallskip
In this paper we are interested in Schur functions whose diagram is 
a staircase. 
% like 
%\smallskip
%$$\matrix{
%\bol{}& \bol{}& & & & & & \cr
%\bol{}&\bol{}&\bol{}& \bol{}& & & &\cr
%\bol{}&\bol{}&\bol{}&\bol{}&\bol{}&\bol{}& &\cr
%\bol{}&\bol{}&\bol{}&\bol{}&\bol{}&\bol{}&\bol{}&\bol{}\cr
%}\ \ .$$
%\medskip
Our starting point is the following formula for the square of a staircase
Schur function
$$S_{12\ldots n}^2=S_{12\ldots n+1/1}S_{12\ldots n-1} 
- S_{12\ldots n+1}S_{12\ldots n-1/1}\ ,$$
which first arose in the context of the Korteweg-de-Vries equation {\bf[AM]}.
It shows that the polynomials $\theta_d$ constructed by Adler and Moser
to represent the rational solutions of the KdV equation are, up to
normalization, nothing but staircase Schur functions 
\footnote{*}{\sevenrm We thank D. Svrtan
for pointing out this fact to us and kindly providing references on
this subject.}.
In a recent article Goulden has provided a combinatorial proof of this identity 
\bf[Go]\rm. We shall 
present more general formulae, expressing any power of a staircase Schur 
function.
\smallskip
An interpretation of these formulae in terms of orthogonal polynomials is then given. 
Indeed, it is shown that the family of staircase Schur functions $S_{12\ldots 
n}({\bb{E}}+x),\ n\geq0$, is a family of orthogonal polynomials in the variable $x$, 
which may be seen as a symmetric analogue of the family of Bessel 
polynomials. This provides a new determinantal expression for Bessel 
polynomials. It also shows that the powerful machinery of symmetric functions 
might be of some help in the study of this sequence of polynomials. Thus, 
using the fact that staircase $S$-functions are special cases of Schur $Q$-functions, 
Thibon and the author derived from a recent multiplication formula
due to Worley the computation of the linearization coefficients \bf[LT]\rm which 
had been conjectured by Favreau \bf[Fa]\rm.  
\smallskip
Finally, we show that our formulae are also linked to an old problem first 
studied by Borchardt \bf[Bo] \rm and Laguerre \bf[La]\rm, and further investigated by 
numerous authors including P\'olya \bf[Po] \rm and Foulkes \bf[Fo]\rm. The problem is to 
express any symmetric polynomial of $n$ indeterminates as a rational function 
of the power sums of odd degree. Set ${\bb{X}}=\{x_1,\ldots ,x_n\}$. The expression 
given by Polya and Foulkes for the $k^{th}$ elementary symmetric polynomial 
$\Lambda_k({\bb{X}})$ is the coefficient of $(-x)^{n-k}$ in the ratio $S_{12\ldots 
n}({\bb{X}}-x)/S_{12\ldots n-1}({\bb{X}})$. We obtain a similar expression 
for the complete 
symmetric polynomial $S_k({\bb{X}})$. 
\bigskip
\bf2  Schur functions and minors identities
\rm\medskip
In this section we shall define our notations and recall some basic facts 
about Schur functions and minors identities. General references are \bf[LS]\rm, 
\bf[Mc] \rm and \bf[Le]\rm.
\smallskip
Let ${\bb{E}}$ denote a set of indeterminates or \it alphabet\rm. The 
\it complete symmetric 
functions \rm$S_i({\bb{E}})$ and the \it elementary symmetric functions 
\rm$\Lambda_i({\bb{E}})$ are 
defined by means of the generating series
$$\sum_iS_i({\bb{E}})t^i = \prod_{e\in{\bb{E}}}(1-te)^{-1}\ ,\eqno (1)$$
$$\sum_i\Lambda_i({\bb{E}})t^i = \prod_{e\in{\bb{E}}}(1+te)\ .\eqno (2)$$
When there is no danger of confusion we shall omit the mention of the 
alphabet. For $I=(i_1,\ldots ,i_n)\in{\bb{N}}^n$, the \it Schur function \rm$S_I$ 
is defined 
by Jacobi--Trudi identity
$S_I={\rm det}[S_{i_l+l-k}]_{1\leq k,l\leq n}\ .$
In other words, denoting by $\bf S\rm$ the infinite Toeplitz matrix 
$[S_{j-i}]_{i,j\geq0}$ where $S_k=0$ for $k<0$, $S_I$ is the minor of $\bf S\rm$ taken 
on the lines $0,\ 1,\ldots ,\ n-1$ and the columns $i_1,\ i_2+1,\ldots ,\ 
i_n+n-1$. More generally, given $J=(j_1,\ldots,\ j_n)\in{\bb{N}}^n$ one defines the 
\it skew Schur function \rm$S_{I/J}$ as the minor of $\bf S\rm$ taken on the lines 
$j_1,\ j_2+1,\ldots ,\ j_n+n-1$ and the columns $i_1,\ i_2+1,\ldots ,\ i_n+n-1$
, \it i. e. \rm 
$S_{I/J}={\rm det}[S_{i_l+l-j_k-k}]_{1\leq k,l\leq n}\ .$
Note that this definition makes sense even if the sequences $I$ and $J$ are 
not arranged in increasing order. However, by permutation of columns and rows 
one can always assume that  $I$ and $J$ are \it partitions \rm i.e. weakly 
\it increasing 
\rm sequences of nonnegative integers. A partition $I$ is represented by a 
\it diagram \rm of boxes having $i_1$ boxes on its first 
row, $i_2$ boxes on its second row, etc.. $|I|=i_1+\ldots +i_n$ is called the 
\it weight \rm of $I$. The \it skew diagram \rm associated with $S_{I/J}$ 
is the complement 
of the diagram $J$ in the diagram $I$.
\smallskip
Schur functions may also be expressed in terms of elementary symmetric 
functions $\Lambda_i$. Indeed, defining by analogy the functions $\Lambda_{I/J}$ by
$\Lambda_{I/J}={\rm det}[\Lambda_{i_l+l-j_k-k}]_{1\leq k,l\leq n}\ ,$
one has the relation 
$$\Lambda_{I/J}=S_{I^{\sim}/J^{\sim}},\eqno (3)$$
where $I^{\sim}$ denotes the \it conjugate partition \rm of $I$ i.e. the partition 
whose diagram is obtained by interchanging the rows and columns of the 
diagram of $I$.
\smallskip
Suppose ${\bb{F}}$ is a second alphabet and denote by ${\bb{E}}+{\bb{F}}$ the 
union of ${\bb{E}}$ and 
${\bb{F}}$. It follows from (1), (2) that
$$S_n({\bb{E}}+{\bb{F}})=\sum_{0\leq i\leq n}S_i({\bb{E}})S_{n-i}({\bb{F}})\ ,\eqno (4)$$
$$\Lambda_n({\bb{E}}+{\bb{F}})=\sum_{0\leq i\leq n}\Lambda_i
({\bb{E}})\Lambda_{n-i}({\bb{F}})\ .\eqno (5)$$
We shall also define the difference ${\bb{E}}-{\bb{F}}$ by setting
$$\sum_iS_i({\bb{E}}-{\bb{F}})t^i = {\prod_{f\in{\bb{F}}}(1-tf)\over 
\prod_{e\in{\bb{E}}}(1-te)}\ ,
\qquad \sum_i\Lambda_i({\bb{E}}-{\bb{F}})t^i = {\prod_{e\in{\bb{E}}}(1+te)\over 
\prod_{f\in{\bb{F}}}(1+tf)}\ ,$$
so that in particular 
$$\Lambda_i({\bb{E}}) = (-1)^i S_i(-{\bb{E}})\ ,\eqno (6)$$
which is another way of expressing the duality between complete and 
elementary symmetric functions. This yields a compact expression for the 
polynomial whose set of zeros is the alphabet ${\bb{A}}$ of cardinality $n$
$$\prod_{a\in{\bb{A}}} (x-a) = \sum_ix^{n-i}(-1)^i\Lambda_i({\bb{A}}) 
= \sum_iS_{n-i}(x)S_i(-{\bb{A}}) = 
S_n(x-{\bb{A}})=(-1)^n\Lambda_n({\bb{A}}-x)\ .$$
\smallskip
Taking different alphabets ${\bb{E}}_1,\ldots ,\ {\bb{E}}_n$ in the columns of the 
determinant of Jacobi--Trudi formula, one gets a \it multi-Schur function
\rm$S_I({\bb{E}}_1,\ldots ,\ {\bb{E}}_n)=
{\rm det}[S_{i_l+l-k}({\bb{E}}_l)]_{1\leq k,l\leq n}\ .$
An example of multi-Schur function considered in this paper is the $n^{th}$ 
orthogonal polynomial associated with the linear functional $\mu$ of moments 
$\mu(x^i)=S_i({\bb{E}})$. (Note that, since the $S_i$ are algebraically independent, 
one may always define a formal alphabet ${\bb{E}}$ by assigning to each $S_i$ a 
given value $S_i({\bb{E}})$ in a commutative ring.) Indeed this polynomial 
$p_n(x)$ has, up to a constant factor, the expression 
$$p_n(x)=S_{n^n0}({\bb{E}},{\bb{E}},\ldots ,{\bb{E}},x)=
\left|\matrix{
S_n({\bb{E}})& S_{n+1}({\bb{E}})& \ldots & S_{2n-1}({\bb{E}})& x^n\cr
S_{n-1}({\bb{E}})& S_n({\bb{E}})& \ldots & S_{2n-2}({\bb{E}})& x^{n-1}\cr
\vdots&\vdots&      & \vdots&\vdots\cr
S_0({\bb{E}})&S_1({\bb{E}})&\ldots&S_{n-1}({\bb{E}}) & 1\cr
 }\right|\ ,\eqno (7)$$
(see \bf[Sz] \rm p.27). By subtraction of lines in the determinant, this polynomial 
may also be written as an ordinary Schur function on the alphabet ${\bb{E}}-x$. 
Indeed, one has  
$p_n(x)=S_{n^n}({\bb{E}}-x)\ .$
This is a consequence of the following important lemma.
\smallskip
{\petcap Lemma 2.1} {\it Let $m,\ n$ be two integers $m\leq n$, ${\bb{F}}$ an alphabet 
of cardinality $m$, and $I=(i_1,\ldots ,\ i_n)$ a partition. Then}
$$S_I({\bb{E}}_1,\ldots ,{\bb{E}}_n)=
\left|\matrix{
S_{i_1}({\bb{E}}_1-{\bb{F}})&\ldots &S_{i_n+n-1}({\bb{E}}_n-{\bb{F}})\cr
\vdots& &\vdots\cr
S_{i_1-n+m+1}({\bb{E}}_1-{\bb{F}})&\ldots &S_{i_n+m}({\bb{E}}_n-{\bb{F}})\cr
S_{i_1-n+m}({\bb{E}}_1)&\ldots &S_{i_n+m-1}({\bb{E}}_n)\cr
\vdots & &\vdots\cr
S_{i_1-n+1}({\bb{E}}_1)&\ldots &S_{i_n}({\bb{E}}_n)\cr
}\right|\ .$$
\smallskip
\it Proof. \rm By (4), we have 
$S_k({\bb{E}}_j-{\bb{F}})=
S_k({\bb{E}}_j)+S_{k-1}({\bb{E}}_j)S_1(-{\bb{F}})+\ldots +S_k(-{\bb{F}})\ .$
But $S_k(-{\bb{F}})=(-1)^k\Lambda_k({\bb{F}})=0$ if $k>m$. Thus, the 
transformation consists in 
adding to each of the first $n-m$ rows a linear combination of the next $m$ 
rows.\quad $\qed$
\medskip
The Jacobi--Trudi formula expresses Schur functions as minors of the matrix 
$\bf S\rm$ of complete symmetric functions $S_i$. This shows that many algebraic 
relations for Schur functions may be obtained by merely specializing the 
numerous identities satisfied by minors of a generic matrix. We have shown 
elsewhere [{\bf Le}] that a great number of these identities are easily derived from a 
single one by Turnbull. We shall briefly recall this identity for the convenience
of the reader. We first need some appropriate notations.
\smallskip
Let $M$ be a $n \times \infty$ matrix. We shall be interested in identities 
satisfied by maximal minors of $M$. Let $a,b,\ldots,c$ be $n$ column vectors 
of $M$. The maximal minor of $M$ taken on these $n$ columns will be denoted 
by either a bracket or a one--line tableau
$$[ab\ldots c] = \tab{a &b &\ldots &c\cr}\he .$$
\smallskip
A product of $p$ minors of $M$ will be designated by a $p \times n$ tableau
$$[ab\ldots c].[de\ldots f].[gh\ldots i] = 
\tab{
a &b &\ldots &c\cr\th
d &e &\ldots &f\cr\th
g &h &\ldots &i\cr
}\he.$$
To denote certain alternating sums of products of minors, we shall use 
tableaux with boxes enclosing some particular vectors. They are defined in 
the following way.
\smallskip
Let $T$ be a $p \times n$ tableau and $A$ a subset of elements of $T$. Given 
a permutation $\sigma$ of the elements of $A$, denote by $\sigma(T)$ the tableau in 
which the elements of $A$ are permuted by $\sigma$. Now, boxing in $T$ the 
elements of $A$, we get a new tableau $\tau$ which will represent the 
alternating sum of all the tableaux $\sigma(T)$, taking into account the fact that 
a permutation of elements of the same row gives a trivial action. More 
precisely, 
$\tau=\sum_\sigma sign(\sigma).\sigma(T) \he ,$
where $\sigma$ runs through the cosets of the symmetric group ${\goth{S}}(A)$ modulo the 
subgroup of the permutations which leave unchanged the rows of $T$.\par
Let $i_k,\ k=1,\ldots p$ denote the number of elements of $A$ lying in the 
$k^{th}$ row of $\tau$. The number of terms in this sum will be $(i_1+\ldots 
+i_p)!/i_1!\ldots i_p!$. In particular, if all the enclosed elements of $\tau$ 
lie in the same row, it reduces to the single tableau $T$.
\smallskip
To illustrate these notations, let us write a well--known minor identity 
(Pl\"ucker's relations) and then its transcription with tableaux.
\smallskip
Let $M$ be the matrix 
$$M=\pmatrix{
a_1 &b_1 &c_1 &d_1 &e_1 &f_1 &\ldots\cr
a_2 &b_2 &c_2 &d_2 &e_2 &f_2 &\ldots\cr
a_3 &b_3 &c_3 &d_3 &e_3 &f_3 &\ldots\cr
 }\he .$$
We have
$$\displaylines{\qquad\left|\matrix{
a_1 &b_1 &c_1\cr
a_2 &b_2 &c_2\cr
a_3 &b_3 &c_3\cr
}\right| .
\left|\matrix{
d_1 &e_1 &f_1\cr
d_2 &e_2 &f_2\cr
d_3 &e_3 &f_3\cr
}\right|\he -\he
\left|\matrix{
e_1 &b_1 &c_1\cr
e_2 &b_2 &c_2\cr
e_3 &b_3 &c_3\cr
}\right| .
\left|\matrix{
d_1 &a_1 &f_1\cr
d_2 &a_2 &f_2\cr
d_3 &a_3 &f_3\cr
}\right|\hfill\cr
\hfill -\he
\left|\matrix{
f_1 &b_1 &c_1\cr
f_2 &b_2 &c_2\cr
f_3 &b_3 &c_3\cr
}\right| .
\left|\matrix{
d_1 &e_1 &a_1\cr
d_2 &e_2 &a_2\cr
d_3 &e_3 &a_3\cr
}\right|
\he =\he\left|\matrix{
a_1 &e_1 &f_1\cr
a_2 &e_2 &f_2\cr
a_3 &e_3 &f_3\cr
}\right| .
\left|\matrix{
d_1 &b_1 &c_1\cr
d_2 &b_2 &c_2\cr
d_3 &b_3 &c_3\cr
}\right|\he .\qquad\cr}$$
\smallskip
Denoting the column vectors of $M$ by $a,b,c,d,e,f$, this identity written 
using tableaux takes the following form
\smallskip
$$\tab{
\bo{a} &b &c\cr\th
d &\bo{e} &\bo{f}\cr
}\he = \he 
\tab{
a &e &f\cr\th
d &b &c\cr
}\he .$$
\smallskip
It is readily deduced from Turnbull's identity, that we shall now state.
\smallskip
Let $\tau$ be a tableau and denote by $A$ the set of its boxed elements. Let 
$B$ be the set of elements of a given row, for example the first one, which 
do not belong to $A$. Finally let $C$ denote the set of all other elements of 
$\tau$. One can build another tableau $\upsilon$ equal to $\tau$ 
by merely exchanging the 
roles of $A$ and $B$ as shown by the following diagram:
\setbox3=\hbox{$\vcenter{\offinterlineskip\cleartabs
\def\cc#1{\hfill#1\hfill}
\def\tvi{\vrule height 12pt depth 5pt width 0pt}
\def\tv{\tvi\vrule}
\def\th{\vrule height 0.4pt width 2em}
\+ \th&\th&\th&\th&\th&\th&\cr
\+ \tv&\tv&   &   &\cc{B}& &\tv\cr
\+ &\th&\th&\th&\th&\th&\cr
\+ \tv&\cc{\bf A\rm} &\tv&&&&\tv\cr
\+ &\th&   &   &   &   &\cr
\+ \tv&\tv&&&\cc{C}&&\tv\cr
\+ \th&\th&\th&\th&\th&\th&\cr
}$}
\setbox4=\hbox{ = }
\setbox5=\hbox{$\vcenter{\offinterlineskip\cleartabs
\def\cc#1{\hfill#1\hfill}
\def\tvi{\vrule height 12pt depth 5pt width 0pt}
\def\tv{\tvi\vrule}
\def\th{\vrule height 0.4pt width 2em}
\+ \th&\th&\th&\th&\th&\th&\cr
\+ \tv&   &\cc{A} &   &\tv &\cc{\bf B\rm} &\tv\cr
\+ \th&\th&\th&\th&\th&\th&\cr
\+ \tv&\cc{\bf B\rm} &\tv&&&&\tv\cr
\+ &\th&   &   &   &   &\cr
\+ \tv&\tv&&&\cc{C}&&\tv\cr
\+ \th&\th&\th&\th&\th&\th&\cr
}$}
$$\line{\hfill{$\tau =$}\box3\box4\box5 {$= \nu$}\hfill}$$
More precisely, we have
\medskip\
{\petcap Theorem} 2.1. {\it (Turnbull's identity, \bf[Tu]\rm, p.209.)}
\smallskip
{\it Let $k$ be the number of elements of $A$.}
\smallskip
(i) {\it If  $k>n, \he \tau = 0.$}
\smallskip
(ii) {\it If  $k\leq n$, and $\upsilon$ is built by}\par
\hskip 1cm \bf a\rm. {\it exchanging each element of $A$ which does not belong to 
the first row, with any element of $B$;}\par
\hskip 1cm \bf b\rm. {\it boxing the elements of $B$;}\par
\hskip 1cm \bf c\rm. {\it removing the boxes of the elements of $A$;}\par
{\it then $\tau=\upsilon$.}
\smallskip
For instance, taking $n=6,\ k=4,\ A=\{ a, b, c, d \},\ B=
\{ \alpha,\beta,\gamma,\delta,\varepsilon\},$ one 
has
$$\tau=\tab{
\bo{ a}& \alpha& \beta& \gamma& \delta& \varepsilon\cr\th
\bo{ b}& \bo{ c}& f& g&h& i\cr\th
\bo{ d}& j& k& l& m& o\cr
} =
\tab{
{ a}& { b}& { c}& { d}& \bo{\delta}& \bo{\varepsilon}\cr\th
\bo{\alpha}& \bo{\beta}& f& g& h& i\cr\th
\bo{\gamma}& j& k& l& m& o\cr
}=\upsilon\ .$$
\smallskip
We shall now deduce from this identity some formulae for the power of 
a staircase Schur function.
\bigskip
\bf3  Powers of a staircase Schur function and symmetric analogues of Bessel 
polynomials
\rm\medskip
Let $\rho(k,m)$ denote the staircase partition $(m,\ 2m,\ldots ,\ km)$. The 
$n^{th}$ power of $S_{\rho(k,m)}$ admits the following expression:
\medskip
{\petcap Theorem 3.1. }
$$(S_{\rho(k,m)})^n=\sum_{I\subset\rho(n-1,m)}(-1)^{|\rho(n-1,m)|-|I|}\ 
S_{\rho(n+k-1,m)/I}\ 
\Delta_I\ ,$$
{\it where $\Delta_I$ is the minor taken on the lines $i_1+1,\ i_2+2,\ldots 
,\ i_{n-1}+n-1$ of the following $(m+1)(n-1)\times (n-1)$ matrix:}
$$M=\pmatrix{
S_{\rho(k-1,m)/1^m}& S_{\rho(k-1,m)/1^{2m+1}}&\ldots& S_{\rho(k-1,m)/1^{(n-1)m+n-2}}
\cr
S_{\rho(k-1,m)/1^{m-1}}& S_{\rho(k-1,m)/1^{2m}}&\ldots& S_{\rho(k-1,m)/1^{(n-1)m+n-3}}
\cr
\vdots&\vdots&&\vdots\cr
S_{\rho(k-1,m)}& S_{\rho(k-1,m)/1^{m+1}}&\ldots& S_{\rho(k-1,m)/1^{(n-2)(m+1)}}
\cr
0& S_{\rho(k-1,m)/1^{m}}&\ldots& S_{\rho(k-1,m)/1^{(n-2)m+n-3}}
\cr
0& S_{\rho(k-1,m)/1^{m-1}}&\ldots& S_{\rho(k-1,m)/1^{(n-2)m+n-4}}
\cr
\vdots& \vdots& & \vdots\cr
0& S_{\rho(k-1,m)}&\ldots& S_{\rho(k-1,m)/1^{(n-3)(m+1)}}
\cr
0& 0&\ldots& S_{\rho(k-1,m)/1^{(n-3)m+n-4}}\cr
\vdots& \vdots& &\vdots\cr
0& 0&\ldots& S_{\rho(k-1,m)}\cr
}\ .$$
Note that we allow the first parts of $I$ to be zero. Before proving the 
theorem, let us illustrate it by a few examples.
\medskip
{\petcap Example 3.2. } We take $n=3$, $m=1$ and $k=4$. The theorem states 
that
$$(S_{1234})^3=\sum_{I\subset12}(-1)^{3-|I|}\ S_{123456/I}\ \Delta_I\ ,$$
where $I$ runs over the set $\{00,\ 01, \ 02,\ 11,\ 12\}$ and $\Delta_I$ is the 
minor taken on the lines $i_1+1$, $i_2+2$ of the $4\times 2$ matrix:
$$M=\pmatrix{
S_{123/1}& S_{123/1^3}\cr
S_{123}& S_{123/1^2}\cr
0& S_{123/1}\cr
0& S_{123}\cr
}\ 
=\ \pmatrix{
S_{1234/1112}& S_{1234/1114}\cr
S_{1234/1111}& S_{1234/1113}\cr
S_{1234/1110}& S_{1234/1112}\cr
S_{1234/111(-1)}& S_{1234/1111}\cr
}\ .$$
The second expression of $M$ shows more clearly the regularity of its columns 
(we recall that for any partition $J$, the determinants $S_{J/1110}$ and 
$S_{J/111(-1)}$ are null having two identical rows). Expanding this sum and 
the determinants $\Delta_I$ yields the following relation:
$$\displaylines{\quad
(S_{1234})^3=S_{123456}\ S_{123}\ S_{123/111}-S_{123456}\ S_{123/1}
\ S_{123/11}+S_{123456/1}\ S_{123/1}\ S_{123/1}\hfill\cr
\hfill -S_{123456/11}\ S_{123}\ S_{123/1}-
S_{123456/2}\ S_{123}\ S_{123/1}+S_{123456/12}\ S_{123}\ S_{123}\ .\quad\cr}$$
\medskip
{\petcap Example 3.3. } We take $n=2$, so that theorem 3.1 reads
$$(S_{\rho(k,m)})^2=\sum_{0\leq i\leq m}(-1)^{m-i}\ S_{\rho(k+1,m)/i}\ 
S_{\rho(k-1,m)/1^{m-i}}\ 
,$$
which is the formula investigated by Goulden \bf[Go] \rm for the square of a 
staircase Schur function. 
\medskip
{\it Proof of Theorem} 3.1. We shall apply Turnbull's identity 2.1. Set 
$N=k+(n-1)(m+1)$ and consider the $N\times \infty$ matrix 
$$M=\pmatrix{
S_0 &S_1 &S_2 &S_3 &\ldots &1 &0 &0 &\ldots &0\cr
0   &S_0 &S_1 &S_2 &\ldots &0 &1 &0 &\ldots &0\cr
0   &0   &S_0 &S_1 &\ldots &0 &0 &1 &\ldots &0\cr
\vdots&\vdots&\vdots& \vdots&&\vdots&\vdots &\vdots &\ddots &\vdots\cr
0   &0   &0   &0   &\ldots &0 &0 &0 &\ldots &1\cr 
 }\he \it.\rm$$
Let $0,\ 1,\ 2,\ 3,\ldots $ denote the column vectors of the left part of 
$M$, and $1^*,\ 2^*,\ 3^*,\ldots N^*,$ denote the last $N$ column vectors of 
$M$. For brevity we shall write $m_p=pm+p-1$. The Schur function $S_{\rho(k,m)}$ 
may be expressed as a maximal minor of $M$ in $n$ different ways
$$\eqalign{
S_{\rho(k,m)} &=[m_1\ m_2\ \ldots \ m_k\ (k+1)^*\ (k+2)^*\ \ldots \ N^*]\cr
&=[0\ 1\ \ldots \ (m_1-1)\ m_1\ m_2\ \ldots \ m_{k+1}\ 
(k+m_1+1)^*\ (k+m_1+2)^*\ \ldots \ N^*]\cr
&\vdots\cr
&= [0\ 1\ 2\ \ldots \ (m_{n-1}-1)\ m_{n-1}\ m_n\ \ldots \ 
m_{k+n-1}]\ ,\cr
}$$
so that its $n^{th}$ power is equal to the product of these $n$ minors
$$\displaylines{
(S_{\rho(k,m)})^n= \hfill\cr
\hfill\tab{
m_1&\ldots &m_k    &(k+1)^*&\ldots & & &\ldots &N^*\cr\th
0  &\ldots &(m_1-1)& m_1   &\ldots &m_{k+1}& (k+m_1+1)^*& \ldots & N^*\cr\th
\vdots& &&&\vdots &&& &\vdots\cr\th
0 &\ldots& & & \ldots &(m_{n-1}-1)&m_{n-1}& \ldots &m_{k+n-1}\cr
}\ .\cr
}$$
Now, recalling that a tableau having two equal letters on the same row is 
equal to zero, we may box the letters $m_1,\ m_2,\ \ldots, m_k$ in the first 
row, the letter $m_{k+1}$ in the second row, $\ldots $, the letter 
$m_{k+n-1}$ in the $n^{th}$ row, and apply Turnbull's identity. This yields
$$\displaylines{
(S_{\rho(k,m)})^n\hfill\cr
\hfill =\ \tab{
\bo{m_1}&\ldots &\bo{m_k}    &(k+1)^*&\ldots & & &\ldots &N^*\cr\th
0  &\ldots &(m_1-1)& m_1   &\ldots &\bo{m_{k+1}}& (k+m_1+1)^*& \ldots & 
N^*\cr\th
\vdots& &&& \vdots&&& &\vdots\cr\th
0 &\ldots& & &\ldots &(m_{n-1}-1)&m_{n-1}& \ldots &\bo{m_{k+n-1}}\cr
}\ \ \cr
\ \cr
\hfill =\ \tab{
m_1&\ldots &m_k    &m_{k+1}&\ldots &m_{k+n-1} &\bo{(k+n)^*} &\ldots 
&\bo{N^*}\cr\th
0  &\ldots &(m_1-1)& m_1   &\ldots &\bo{(k+1)^*}& (k+m_1+1)^*& \ldots & 
N^*\cr\th
\vdots& &&& \vdots&&& &\vdots\cr\th
0 &\ldots& & &\ldots &(m_{n-1}-1)&m_{n-1}& \ldots &\bo{(k+n-1)^*}\cr
}\ .\cr
}$$
This last tableau represents a sum of products of Schur functions of the 
following type
$$\pm S_{\rho(n+k-1,m)/I}\ S_{\rho(k-1,m)/1^{j_1}}\ldots \ S_{\rho(k-1,m)/1^{j_{n-1}}} 
\ .$$
The coefficient of a given $S_{\rho(n+k-1,m)/I}$ in this sum is in fact a 
tableau having $n-1$ rows with only one letter boxed in each row, that is a 
determinant of order $n-1$. There is no difficulty in recognizing this 
determinant as the minor $\Delta_I$ defined above. $\qed$
\medskip
The formula for the square of $S_{\rho(k,1)}$ may be presented as the 
computation of a determinant of Schur functions
$$\left|\matrix{
S_{\rho(k-1,1)}& S_{\rho(k+1,1)}\cr
S_{\rho(k-1,1)/1}& S_{\rho(k+1,1)/1}\cr
}\right|
=(S_{\rho(k,1)})^2\ .\eqno (8)$$
The same may be done for theorem 3.1, which has also a determinantal 
formulation. We shall write for brevity $\rho(k)$ in place of $\rho(k,1)$.
\medskip
{\petcap Theorem 3.4.} 
$$\left|\matrix{
S_{\rho(k-1)}& 0& \ldots & 0& \ldots & 0& S_{\rho(k+1)}\cr
S_{\rho(k-1)/1}& 0& \ldots & 0& \ldots & S_{\rho(k+2)}& S_{\rho(k+1)/1}\cr
S_{\rho(k-1)/1^2}& S_{\rho(k,1)}& \ldots & 0& \ldots & S_{\rho(k+2)/1}& 
S_{\rho(k+1)/1^2}\cr
\vdots & \vdots & & \vdots & & \vdots &\vdots\cr
S_{\rho(k-1)/1^{2n-3}}& S_{\rho(k-1)/1^{2n-5}}&\ldots &  S_{\rho(k+n-1)/1^{n-1}}& 
\ldots & S_{\rho(k+2)/1^{2n-4}}& S_{\rho(k+1)/1^{2n-3}}\cr
}\right|$$
$$=(S_{\rho(k)})^n\ S_{\rho(k+1)}\ S_{\rho(k+2)}\ldots S_{\rho(k+n-2)}\ .$$
\medskip
{\petcap Example 3.5.} For $n=3$ and $k=4$ one has
$$\left|\matrix{
S_{123}& 0& 0& S_{12345}\cr
S_{123/1}& 0&S_{123456}& S_{12345/1}\cr
S_{123/1^2}& S_{123}& S_{123456/1}& S_{12345/1^2}\cr
S_{123/1^3}& S_{123/1}& S_{123456/1^2}& S_{12345/1^3}\cr
}\right|
=(S_{1234})^3\ S_{12345}\ .$$
\medskip
{\petcap Example 3.6.} For $n=4$ and $k=6$ one has
$$\left|\matrix{
S_{12345}    &0        &0  & 0& 0& S_{1234567}\cr
S_{12345/1}  &0        &0  & 0&S_{12345678}& S_{1234567/1}\cr
S_{12345/1^2}&S_{12345} &0  &S_{123456789}& S_{12345678/1}& S_{1234567/1^2}\cr
S_{12345/1^3}&S_{12345/1} &0 &S_{123456789/1}&S_{12345678/1^2}& 
S_{1234567/1^3}\cr
S_{12345/1^4}&S_{12345/1^2}&S_{12345}&S_{123456789/1^2}&S_{12345678/1^3}&
S_{1234567/1^4}\cr
S_{12345/1^5}&S_{12345/1^3}&S_{12345/1}&S_{123456789/1^3}&S_{12345678/1^4}&
S_{1234567/1^5}\cr
}\right|$$
$$=(S_{123456})^4\ S_{1234567}\ S_{12345678}\ .$$
\medskip
{\it Proof.} The proof of this last example will illustrate enough the 
principle of our computation. We consider the $12\times 30$ matrix
$$M=\pmatrix{
S_0 &S_1 &S_2 &S_3 &\ldots &S_{17}&1 &0 &0 &\ldots &0\cr
0   &S_0 &S_1 &S_2 &\ldots &S_{16}&0 &1 &0 &\ldots &0\cr
0   &0   &S_0 &S_1 &\ldots &S_{15}&0 &0 &1 &\ldots &0\cr
\vdots&\vdots&\vdots& \vdots&&\vdots&\vdots &\vdots &\ddots &\vdots\cr
0   &0   &0   &0   &\ldots &S_6&0 &0 &0 &\ldots &1\cr 
 }\he ,$$
whose column vectors are denoted $0,\ 1,\ 2,\ldots 17,\ 1^*,\ 2^*,\ldots\ 
12^*$. The determinant $D$ of example 3.6 is represented by the following 
tableau
$$D=
\tab{
0& 1& 3& 5& 7&  9& 11& \bo{7^*}& 9^*& 10^*& 11^*& 12^*\cr\th
0& 1& 2& 3& 5&  7&  9&  11     &  13& \bo{ 8^*}& 11^*& 12^*\cr\th
0& 1& 2& 3& 4&  5&  7&   9&  11&   13&   15& \bo{9^*} \cr\th
1& 3& 5& 7& 9& 11& 13&  15&  17& \bo{10^*}& 11^*& 12^*\cr\th
1& 3& 5& 7& 9& 11& 13&  15&  17&    0& \bo{11^*}& 12^*\cr\th
1& 3& 5& 7& 9& 11& 13&  15&  17&    0&    2& \bo{12^*}\cr
}\ .$$
Now we transform this tableau by Turnbull's identity and note that all boxes 
may be removed because of the presence of identical letters
$$D=
\tab{
0& 1& 3& 5& 7&  9& 11& \bo{13}& 9^*& 10^*& 11^*& 12^*\cr\th
0& 1& 2& 3& 5&  7&  9&  11     &  13& \bo{15}& 11^*& 12^*\cr\th
0& 1& 2& 3& 4&  5&  7&   9&  11&   13&   15& \bo{17} \cr\th
1& 3& 5& 7& 9& 11& 13&  15&  17& \bo{0}& 11^*& 12^*\cr\th
1& 3& 5& 7& 9& 11& 13&  15&  17&    0& \bo{2}& 12^*\cr\th
1& 3& 5& 7& 9& 11& 7^*&  8^*&  9^*&  10^*& 11^*& 12^*\cr
}$$
$$=\tab{
0& 1& 3& 5& 7&  9& 11& 13& 9^*& 10^*& 11^*& 12^*\cr\th
0& 1& 2& 3& 5&  7&  9&  11     &  13& 15& 11^*& 12^*\cr\th
0& 1& 2& 3& 4&  5&  7&   9&  11&   13&   15& 17 \cr\th
1& 3& 5& 7& 9& 11& 13&  15&  17& 0& 11^*& 12^*\cr\th
1& 3& 5& 7& 9& 11& 13&  15&  17&    0& 2& 12^*\cr\th
1& 3& 5& 7& 9& 11& 7^*&  8^*&  9^*&  10^*& 11^*& 12^*\cr
}$$
$$=(S_{123456})^4\ S_{1234567}S_{12345678}\qquad \qed$$
\medskip
%{\it Remark.} For convenience we limited ourself to staircases of type 
%$\rho(k,1)$ in the statement of theorem 3.4. However a similar formula holds for 
%general staircases $\rho(k,m),\ m\geq1$, and it may be easily induced from the 
%following example:
%$$\left|\matrix{
%S_{24}    &0        &0  & 0& 0& S_{1357}\cr
%S_{24/1}  &0        &0  & 0&S_{2468}& S_{1357/1}\cr
%S_{24/1^2}&0        &0  &S_{13579}& S_{2468/1}& S_{1357/1^2}\cr
%S_{24/1^3}&S_{24} &S_{2468.10} &S_{13579/1}&S_{2468/1^2}& S_{1357/1^3}\cr
%S_{24/1^4}&S_{24/1}&S_{2468.10/1}&S_{13579/1^2}&S_{2468/1^3}&S_{1357/1^4}\cr
%S_{24/1^5}&S_{24/1^2}&S_{2468.10/1^2}&S_{13579/1^3}&S_{2468/1^4}&
%S_{1357/1^5}\cr
%}\right|$$
%$$=(S_{246})^3\ S_{1357}\ S_{2468}\ S_{13579}\ .$$
%\medskip
We shall end this section with an interpretation of theorem 3.4 in terms of 
orthogonal polynomials. First we shall give an immediate generalization of 
formula (8).
\medskip
{\petcap Proposition 3.7. }{\it For $1\leq p\leq k+1$ we have}
$$\left|\matrix{
S_{\rho(k-1)}& S_{\rho(k+1)}\cr
S_{\rho(k-1)/1^p}& S_{\rho(k+1)/1^p}\cr
}\right|
=S_{\rho(k)}\ S_{\rho(k)/1^{p-1}}\ ,$$
{\it where $\rho(k)$ is short for the partition $12\ldots k$.}
\smallskip
{\it Proof. } This is an instance of Pl\"ucker relations. We consider again 
the matrix $M$ of the proof of theorem 3.1, taking $N=k+2$. The determinant 
to be computed is then represented by the $2\times (k+2)$ tableau
$$(-1)^p\ \tab{
0 &1 &3 &5 &\ldots &2k-1 &\bo{N^*}\cr \th
1 &3 &5 &7 &\ldots &2k+1 &\bo{(N-p)^*}\cr
}\ .$$
Applying Turnbull's identity, this tableau is transformed into
$$(-1)^p\ \tab{
0 &1 &3 &5 &\ldots &2k-1 &\bo{2k+1}\cr \th
\bo{1} &\bo{3} &\bo{5} &\bo{7} &\ldots &N^* &(N-p)^*\cr
}$$
$$=\ (-1)^p\ \tab{
0 &1 &3 &5 &\ldots &2k-1 &2k+1\cr \th
1 &3 &5 &7 &\ldots &N^* &(N-p)^*\cr
}
=S_{\rho(k)}\ S_{\rho(k)/1^{p-1}}\ .\qquad \qed$$
\smallskip
>From this proposition we deduce the following result:
\medskip
{\petcap Theorem 3.8. }{\it Let ${\bb{E}}$ be an alphabet and let us denote by $\pi_k$ 
the polynomial of degree $k$ in $x$ with coefficients in the ring of symmetric functions
of ${\bb{E}}$:
$\pi_k(x)=S_{\rho(k)}({\bb{E}}-x)$.
Then $\pi_k$ satisfies the three term recurrence relationship $(k\geq1)$
$$S_{\rho(k-1)}({\bb{E}})\pi_{k+1}(x)+xS_{\rho(k)}({\bb{E}})\pi_{k}(x)-
S_{\rho(k+1)}({\bb{E}})\pi_{k-1}(x)=0\ .$$
In other words $\big(\pi_k(x)\big)_{k\geq0}$ is a family of orthogonal 
polynomials. }

\medskip
{\it Proof. } The expansion of $\pi_k$ is obtained by considering it as a 
multi-Schur function and expanding this determinant along its last column. 
Indeed, by lemma 2.1,
$$\pi_k(x)=S_{12\ldots k0}({\bb{E}},\ldots ,{\bb{E}},x)=\sum_{0\leq i\leq k}S_{12\ldots 
k/1^i}({\bb{E}})(-x)^i\ .$$
By linearity we deduce from proposition 3.7 that
$$\left|\matrix{
S_{\rho(k-1)}& S_{\rho(k+1)}\cr
\pi_{k-1}(x)& \pi_{k+1}(x)\cr
}\right|
=-xS_{\rho(k)}\ \pi_k(x)\ ,$$
which is the required relationship.
\ \quad $\qed$
\smallskip
Thus, theorem 3.8 states that the elimination of $x^0$ between $\pi_{k-1}(x)$ 
and $\pi_{k+1}(x)$ produces a polynomial proportional to $x\pi_k(x)$. As a 
consequence the elimination of $x^0,\ x^1,\ x^2$ between $\pi_{k-1}(x),\ 
\pi_{k+1}(x),\ x\pi_{k+2}(x),\ x^2\pi_{k-1}(x)$ will produce a polynomial 
proportional to $x^3\pi_k(x)$. This elimination is expressed by the following 
formula
$$\left|\matrix{
S_{\rho(k-1)}& 0& 0& S_{\rho(k+1)}\cr
S_{\rho(k-1)/1}& 0&S_{\rho(k+2)}& S_{\rho(k+1)/1}\cr
S_{\rho(k-1)/1^2}& S_{\rho(k-1)}& S_{\rho(k+2)/1}& S_{\rho(k+1)/1^2}\cr
\pi_{(k-1)}(x)&x^2\pi_{k-1}(x)& -x\pi_{k+2}(x)& \pi_{k+1}(x)\cr
}\right|
=-(S_{\rho(k)})^2\ S_{\rho(k+1)}\ x^3\pi_k(x).$$
Picking up the coefficient of $x^3$ on each side we find again the case $n=3$ 
of theorem 3.4. And this process may clearly be carried out for larger values 
of $n$.
\medskip
The computation of the moments of the sequence of orthogonal polynomials 
$\big(\pi_k(x)\big)_{k\geq0}$ will result from the following identity. We shall 
denote by $\rho(n)+m$ the partition $(1,\ 2,\ldots ,\ n-1,\ n+m)$.
\medskip
{\petcap Theorem 3.9. }{\it Let k, n be integers such that $1\leq k\leq n$. There 
holds:
$$\left|\matrix{
S_{\rho(k)/1^{k-1}}& S_{\rho(k+1)/1^{k-1}}& \ldots & S_{\rho(n)/1^{k-1}}\cr
S_{\rho(k)/1^k}& S_{\rho(k+1)/1^k}& \ldots & S_{\rho(n)/1^k}\cr
\vdots&\vdots&&\vdots \cr
S_{\rho(k)/1^{n-1}}& S_{\rho(k+1)/1^{n-1}}& \ldots & S_{\rho(n)/1^{n-1}}\cr
}\right|
=S_{\rho(k-1)}P_{\rho(k)+(n-k+1)}S_{\rho(k+1)}S_{\rho(k+2)}\ldots S_{\rho(n-1)}\ ,$$
where $P_{\rho(k)+(n-k+1)}=\sum_{0\leq2i\leq n-k+1}\Lambda_{\rho(k)+2i}S_{n-k+1-2i}$.}
\medskip
{\petcap Example 3.10. } For $n=6,\ k=3$ one has
$$\left|\matrix{
S_{123/1^2}& S_{1234/1^2}& S_{12345/1^2} & S_{123456/1^2}\cr
S_{123/1^3}& S_{1234/1^3}& S_{12345/1^3} & S_{123456/1^3}\cr
S_{123/1^4}& S_{1234/1^4}& S_{12345/1^4} & S_{123456/1^4}\cr
S_{123/1^5}& S_{1234/1^5}& S_{12345/1^5} & S_{123456/1^5}\cr
}\right|
=S_{12}P_{127}S_{1234}S_{12345}\ ,$$
where $P_{127}=\Lambda_{123}S_4+\Lambda_{125}S_2+\Lambda_{127}$.
\smallskip
\it Proof. \rm The proof is similar to the proof of theorem 3.4, the only difference 
being that one has to apply Turnbull's identity several times in order to 
obtain the required factorization.
\medskip
{\petcap Corrolary 3.11. }{\it The $n^{th}$ moment of the sequence of 
orthogonal polynomials $\big(\pi_k(x)\big)_{k\geq0}$ is the sum of all hook Schur 
functions of weight $n+1$:}
$\mu(x^n)=\sum_{0\leq i\leq n}S_{1^i(n+1-i)}({\bb{E}})\ .$
\smallskip
\it Proof. \rm Let $q_k(x)$ denote the monic polynomial 
$q_k(x)=S_{\rho(k)}({\bb{E}}-x)/(-1)^kS_{\rho(k-1)}({\bb{E}})$. 
Theorem 3.9 may be seen as the 
resolution of the linear system in the unknown $a_{nk}$:
$$ x^n = \sum_k a_{nk}q_k(x)\ . \eqno (9)$$
The result is that 
$a_{nk}=P_{\rho(k+1)+n-k}({\bb{E}})/ S_{\rho(k+1)}\ .$
Denote by $Sym({\bb{E}})$ the ring of symmetric functions of the alphabet
${\bb{E}}$ and let $\mu$ be the functional on $Sym({\bb{E}})[x]$ associated 
with the sequence of 
orthogonal polynomials $\big(\pi_k(x)\big)_{k\geq0}$. In fact $\mu$ is defined up to 
a constant factor, so that we can add the condition $\mu(1)=S_1({\bb{E}})$. Then, 
recalling that by definition $\mu(q_k(x))=0,\ k\geq1$ and applying $\mu$ to the 
equality (9), one obtains that 
$\mu(x^n)=P_{n+1}({\bb{E}})=\sum_{0\leq i\leq n}S_{1^i(n+1-i)}({\bb{E}})\ ,$
the last equality resulting easily from Pieri rule for the multiplication of 
a Schur function by an elementary symmetric function. $\ \qed$
\medskip
\it Remarks. (i) \rm It can be shown that the symmetric function 
$$\sum_{0\leq2i\leq n-k+1}\Lambda_{\rho(k)+2i}S_{n-k+1-2i}$$
is nothing but the Schur $P$-function $P_{\rho(k)+n-k+1}$, hence the notation
used above. More generally it has been shown in [{\bf LLT}] that there 
exists for every Schur $P$-function a similar quadratic expansion in terms of 
ordinary Schur $S$-functions.
\smallskip
\it(ii) \rm It is easily checked that $S_{\rho(n)/1^m}=S_{\rho(n)/m}$. 
It follows that the 
polynomials $\phi_k(x)=S_{\rho(k)}({\bb{E}}+x)$ are equal to the polynomials $\pi_k(-x)$, 
and therefore also satisfy a three term recurrence relationship: 
$$S_{\rho(k-1)}({\bb{E}})\phi_{k+1}(x)-xS_{\rho(k)}({\bb{E}})\phi_{k}(x)-
S_{\rho(k+1)}({\bb{E}})\phi_{k-1}(x)=0\ .$$
They are orthogonal for the moments 
$\nu(x^n)=(-1)^n\sum_{0\leq i\leq n}S_{1^i(n+1-i)}({\bb{E}})\ .$
\medskip
We must now emphasize on the differences between the two Schur function 
expressions of orthogonal polynomials hitherto encountered. As recalled in 
section 2, the family of orthogonal polynomials associated with a given 
functional $\mu$ may \it always \rm be represented up to a constant factor by the 
sequence of rectangle Schur functions $S_{n^n}({\bb{E}}-x),\ n\geq0,$ 
where ${\bb{E}}$ is the 
alphabet formally defined by $S_k({\bb{E}})=\mu(x^k),\ k\geq0$. In contrast Schur 
functions of the type $S_{12\ldots n}({\bb{E}}-x),\ n\geq0$ or 
$S_{12\ldots n}({\bb{E}}+x),\ 
n\geq0$ represent only particular families of orthogonal polynomials. Indeed, 
the recurrence relationship of theorem 3.8 is not generic, for the 
coefficient of $\pi_n(x)$ in this relationship is of the type $\alpha_nx$ while the 
general form of this coefficient is $\alpha_nx+\beta_n$. The following example shows 
that these polynomials may be seen as symmetric analogues of Bessel 
polynomials.
\medskip
{\petcap Example 3.12.}
\smallskip
Let ${\cal{E}}$ be the alphabet formally defined by 
$S_k({\cal{E}})=1/ k!,\ k\geq0.$
It is a classical result that for any partition $I$ of weight $n$, 
there holds $S_I({\cal{E}})=f_I/n!$, where $f_I$ is the number of standard Young 
tableaux of shape $I$, \it i.e. \rm the dimension of the irreducible representation 
of ${\goth{S}}_n$ indexed by $I$. It follows that the moments associated to the 
sequence $\big(\phi_k(x)\big)_{k\geq0}$ are equal to
$\nu(x^n)=(-2)^n/ (n+1)!\ ,$
that is, are equal to the moments of Bessel polynomials (see for example 
\bf[Ch])\rm.
The corresponding polynomials (suitably normalized) 
$$y_n(x)=S_{12\ldots n}({\cal{E}}+x)/S_{12\ldots n}({\cal{E}}),\quad n\geq0,\eqno (10)$$
are therefore the \it Bessel polynomials\rm. The first ones are
$$\eqalign{
y_0(x)=&1,\cr
y_1(x)=&1+x,\cr
y_2(x)=&1+3x+3x^2,\cr
y_3(x)=&1+6x+15x^2+15x^3,\cr
y_4(x)=&1+10x+45x^2+105x^3+105x^4.\cr
}$$
Formula (10) provides the following determinantal expression
$$y_n(x)=\prod_{1\leq i\leq n}(2(n-i)+1)^i
\left|\matrix{
{1/1!}& {1/3!}& {1/5!}& \ldots & {1/(2n-1)!}& (-x)^n\cr
{1/0!}& {1/2!}& {1/4!}& \ldots & {1/(2n-2)!}& (-x)^{n-1}\cr
0& {1/1!}& {1/3!}& \ldots & {1/(2n-3)!}& (-x)^{n-2}\cr
\vdots&\vdots &\vdots &&\vdots&\vdots \cr
0 & &    &  & {1/n!}& -x\cr
0 & &    &  & {1/(n-1)!}& 1\cr
}\right|\ .$$

There exists a fairly extensive literature devoted to these polynomials. For 
a detailed account up to 1978, the interested reader is referred to \bf[Gr]\rm. We 
shall also mention the recent work of Dulucq and Favreau who have presented a 
combinatorial model for these polynomials, based upon weighted involutions 
\bf[DF]\rm.
\smallskip
Note that our point of view provides at once a q-analogue for Bessel 
polynomials, by merely replacing ${\cal{E}}$ by the alphabet ${\cal{E}}_q=\{1,\ q,\ 
q^2,\ldots \}$. More precisely, setting $v=x/(1-q)$ we define 
$$y_n(x,q)=q^{n\choose 2}S_{12\ldots n}({\cal{E}}_q+v)/S_{12\ldots n}({\cal{E}}_q).$$
The $y_n(x,q)$ are orthogonal polynomials in $x$, whose coefficients are 
polynomials in $q$. The first ones are
$$\eqalign{
y_0(x,q)=&1,\cr
y_1(x,q)=&1+x,\cr
y_2(x,q)=&q+(1+q+q^2)x+(1+q+q^2)x^2,\cr
y_3(x,q)=&q^3+q(1+q^2)(1+q+q^2)x+(1+q+q^2)(1+q+q^2+q^3+q^4)x^2\cr
&\hfill +(1+q+q^2)(1+q+q^2+q^3+q^4)x^3.\cr
}$$
These $q$-analogues are equal to those obtained by Dulucq by means of the 
combinatorial model of
Dulucq and Favreau {\bf [D]}.
\bigskip
\bf4 Expression of a symmetric polynomial in terms of the power sums of odd 
degree
\rm\medskip
Let ${\bb{X}}$ denote a finite set of variables ${\bb{X}}=\{x_1,\ldots ,x_n\}$. 
It is well 
known that any symmetric polynomial $F({\bb{X}})$ may be expressed as a polynomial 
function of the power sums $\psi_k({\bb{X}})=\sum_{x\in{\bb{X}}}x^k,\ k=1,2\ldots n$. 
It was 
shown in the last century that $F({\bb{X}})$ may also be expressed as a rational 
function of $\psi_k({\bb{X}}),\ k=1,3,5,\ldots 2n-1$ (\bf[Bo]\rm, \bf[La]\rm). 
P\'olya \bf[Po] \rm and 
Foulkes \bf[Fo]\rm, among others, have proposed explicit expressions for some 
particular symmetric polynomials $F({\bb{X}})$. For instance
\medskip
{\petcap Proposition 4.1.} {\it The $k^{th}$ elementary symmetric function is 
equal to
$\Lambda_k({\bb{X}})=S_{\rho(n)/1^{n-k}}({\bb{X}})/ S_{\rho(n-1)}({\bb{X}})\ \it,\rm$
the right-hand side being a rational function of the power sums of odd 
degree.}
\smallskip
\it Proof. \rm For sake of completeness we sketch the proof. It is known that
$S_{\rho(n-1)}({\bb{X}})=\prod_{i<j}(x_i+x_j)\ .$
Taking an additionnal variable z we have also
$$S_{\rho(n)}({\bb{X}}+z)=\prod_{i<j}(x_i+x_j)\prod_i(x_i+z)
=S_{\rho(n-1)}({\bb{X}})\sum_kz^{n-k}\Lambda_k({\bb{X}})
\ .$$
Thus, comparing the coefficients of $z^{n-k}$, we get
$\Lambda_k({\bb{X}})=S_{\rho(n)/{n-k}}({\bb{X}})/ S_{\rho(n-1)}({\bb{X}})
=S_{\rho(n)/1^{n-k}}({\bb{X}})/ S_{\rho(n-1)}({\bb{X}})\ .$
Finally, we recall that the staircase Schur function $S_{\rho(n)}$ and all its 
derivatives $S_{\rho(n)/I}$ depend only on the odd power sums $\psi_{2p+1}$, that 
is, belong to the subring generated by $\psi_1,\ \psi_3,\ \psi_5,\ldots $. $\ \qed$
\medskip
Now theorem 3.9 provides at once a similar expression for the complete 
symmetric functions $S_k({\bb{X}})$.
\medskip
{\petcap Proposition 4.2.} {\it The $k^{th}$ complete symmetric function is 
equal to
$$S_k({\bb{X}})={\sum_{0\leq2i\leq k}
\Lambda_{\rho(n-1)+2i}({\bb{X}})S_{k-2i}({\bb{X}})\over S_{\rho(n-1)}({\bb{X}})}\ ,$$
the right-hand side being a rational function of the power sums of odd 
degree.}
\smallskip
\it Proof. \rm Theorem 3.9 shows that the function $
\sum_{0\leq2i\leq k}\Lambda_{\rho(n-1)+2i}S_{k-2i}$ may be expressed in terms of Schur 
functions of the type $S_{\rho(m)/1^p}$, which all belong to 
${\bb{Z}}[\psi_1,\psi_3,\psi_5,\ldots ]$. On the other hand, since ${\bb{X}}$ 
is finite of 
cardinality $n$, $\Lambda_{\rho(n-1)+2i}({\bb{X}})=0$ for all $i>0$. $\ \qed$
\bigskip
\centerline{\petcap References}
\bigskip
\bf[AM] \rm\hskip 5mm {\petcap M. Adler and J. Moser}, {\it On a class of polynomials
connected with the Korteweg-de-Vries equation}, Commun. math. Phys. 61, (1978), 1--30.
\medskip
\bf[Bo] \rm\hskip 5mm {\petcap C.W. Borchardt}, {\it  \"Uber eine Eigenschaft der 
Potenzsummen ungerader Ordnung}, Monastber. Akad. Berlin 1857, 301--311.
\medskip
\bf[Ch]\rm\hskip 5mm {\petcap T.S. Chihara}, {\it  Introduction to Orthogonal 
Polynomials}, Gordon and Breach, New-York, 1978.
\medskip
\bf[D] \rm\hskip 5mm {\petcap S. Dulucq}, {\it Un q-analogue des polyn\^omes de Bessel},
Actes du 25-\`eme S\'eminaire Lotharingien de Combinatoire, 1990.
\medskip
\bf[DF] \rm\hskip 5mm {\petcap S. Dulucq and L. Favreau}, \it Un mod\`ele combinatoire 
pour les polyn\^omes de Bessel\rm, Actes du 25-\`eme S\'eminaire Lotharingien 
de Combinatoire, 1990.
\medskip
\bf[Fa]\rm\hskip 5mm {\petcap L. Favreau}, {\it  Combinatoire des tableaux 
oscillants et des polyn\^omes de Bessel}, Publ. L.A.C.I.M., U.Q.A.M., 
Montr\'eal, 1991.
\medskip
\bf[Fo]\rm\hskip 5mm {\petcap H.O. Foulkes}, {\it  Theorems of P\'olya and Kakeya 
on power-sums}, Math. Zeitschr. 65, 345--352, 1956.
\medskip
\bf[Go] \rm\hskip 5mm {\petcap I. P. Goulden}, \it Quadratic forms of skew Schur 
functions\rm, Europ. J. Combinatorics (1988) \bf9\rm, 161--168.
\medskip
\bf[Gr] \rm\hskip 5mm {\petcap E. Grosswald}, \it Bessel Polynomials\rm, Springer, 1978.
\medskip
\bf[La]\rm\hskip 5mm {\petcap E. Laguerre}, {\it  Sur un probl\`eme d'alg\`ebre}, 
Bull. Soc. Math. France 5, 26--30, 1877.
\medskip 
\bf[LLT]\rm\hskip 5mm {\petcap A. Lascoux, B. Leclerc et J.Y. Thibon}, {\it  Une 
nouvelle expression des fonctions $P$ de Schur}, C. R. Acad. Sci. Paris, t. 
316, S\'erie I, 221-224, 1993.
\medskip
\bf[LT]\rm\hskip 5mm {\petcap B. Leclerc et J.Y. Thibon}, {\it  Analogues 
sym\'etriques des polyn\^omes de Bessel}, C. R. Acad. Sci. Paris, t. 315, S\'erie 
I, 527--530, 1992.
\medskip
\bf[LP] \rm\hskip 5mm {\petcap A. Lascoux and P. Pragacz}, \it Ribbon Schur functions, 
\rm Europ. J. Combinatorics (1988) \bf9\rm, 561--574.
\medskip
\bf[LS] \rm\hskip 5mm {\petcap A. Lascoux and M.P. Sch\"utzenberger}, \it Formulaire 
raisonn\'e de fonctions sym\'etriques\rm, Publ. Math. Univ. Paris 7, 1985.
\medskip
\bf[Le] \rm\hskip 5mm {\petcap B. Leclerc}, \it On identities satisfied by minors of a 
matrix\rm, Adv. in Math., 100, (1993), 101-132.
\medskip
\bf[Mc] \rm\hskip 5mm {\petcap I. G. Macdonald}, \it Symmetric functions and Hall 
polynomials\rm, Oxford Math. Monographs, 1979.
\medskip
\bf[Po]\rm\hskip 5mm {\petcap G. P\'olya}, {\it  Remarques sur un probl\`eme 
d'alg\`ebre \'etudi\'e par Laguerre}, J. Math. Pur. Appl. 31, 37--47, 1951.
\medskip
\bf[Sz]\rm\hskip 5mm {\petcap G. Szeg\"o}, {\it  Orthogonal polynomials}, 
A.M.S. Colloquium Publications, Vol. 23, Providence, RI, 1975.
\medskip
\bf[Tu]\rm\hskip 5mm {\petcap H.W. Turnbull}, {\it  The irreducible concomitants 
of the quadratics in n variables}, Transac. Cambridge Philos. Soc., xxi pp. 
197--240, 1909.
\medskip
\bye

