% AMS-LaTeX FPSAC'01 template file, January 24, 2001.
% For AMS-LaTeX documentation, see http://www.ams.org/tex/amslatex.html
\documentclass{amsart} % Do not use A4 option; 10-point type preferred
\setlength{\textheight}{8.5in}   % These are our page sizes; 
\setlength{\textwidth}{6in}      % please do not change them
\setlength{\topmargin}{0pt}      % The actual margin will be one inch
\setlength{\oddsidemargin}{.5in} % larger than these,
\setlength{\evensidemargin}{0in} % We will bind double-sided
% Comment this line out if you do not use Fraktur (\mathfrak replaces the old
% \frak) or blackboard bold (\mathbb replaces the old \Bbb)
\usepackage{amsfonts} 
% Change these lines as appropriate
\title[Short Running Title]{Sample paper subimitted to FPSAC'01, Arizona
State University}
\author{Sample Author \and Paul Erd\H os}\thanks{Supported by a grant
from the NSF} % Note: no period at the end of \thanks
\email{your\_address@your.institution.edu} % Note how to get underscore
\address{Department of Mathematics, Your Institution, Tempe, AZ
85287-1804, USA}
% Include your own definitions here; some common examples are included
\newtheorem{thm}{Theorem}
\newtheorem{cor}[thm]{Corollary} % share numbers with theorems
\newtheorem{lemma}{Lemma}
\newcommand{\R}{{\mathbb R}}
\newcommand{\Z}{{\mathbb Z}}
\newcommand{\sumkinf}{\sum_{k=0}^\infty}
\newcommand{\sumtinf}{\sum_{t=0}^\infty}
\begin{document}
\begin{abstract} % Short abstract

This is the abstract of the paper in English; please include an abstract
in a second language if possible, introduced by the appropriate
translation of ``abstract.'' 

\noindent{\sc R\'esum\'e.}  C'est le r\'esum\'e fran\c cais.

\end{abstract}
\maketitle % This line must be here, following the abstract

% Note that the Plain TeX commands \over and \atop do not work in
% AMS-LaTeX; fractions must be specified as \frac{1}{2} and 
% binomial coefficients as \binom{n}{k}


% You will probably want to delete all of the sample text below; it is
% used only to illustrate the AMS-LaTeX format.

\section{Introduction}

This is a main section; we will cite a theorem in a journal
article~\cite[Theorem 2]{S}.

\section{Results}

This is another main section, in which we prove a result about the real
numbers $\R$ and the integers $\Z$.

\subsection{Elementary cases} Note that subsection titles appear
on the same line as the rest of the subsection.

\begin{thm}\label{samplethm}
Here is a sample theorem.
\end{thm}

\begin{proof}
This is the proof of the sample theorem.  Note that the proof environent
is already defined; if you have defined your own proof environment in
\LaTeX, please remove it.
\end{proof}

\begin{cor}
This is a corollary of Theorem~\ref{samplethm}; note that corollaries
and theorems share numbering. 
\end{cor}

\begin{lemma}
However, we have a different set of numbers for lemmas.
\end{lemma}

Here is an example of a mathematical formula defining the hyperbolic
Bessel functions, along with a citation of the source~\cite{WW}.
\begin{eqnarray*}
  \exp(x(u+u^{-1})) 
    &=& \sumkinf\frac{x^k}{k!} \sum_{j=-k}^k \binom{k}{j} u^{k-2j}\\
    &=& \sum_{m=-\infty}^{\infty} u^m \sumkinf 
	     \frac{x^k}{k!}\binom{k}{(k+m)/2}\\
    &=& \sum_{m=-\infty}^{\infty} u^m \sumtinf \frac{x^{2t+m}}{t!(t+m)!}\\
    &=& \sum_{m=-\infty}^{\infty} u^m I_m(2x),
\end{eqnarray*}

\begin{figure}
\vspace{2.5in}
\begin{picture}(5,5)(-1,-1)
\setlength{\unitlength}{.5in}
\put(0,-1){\vector(0,1){6}}
\put(-1,0){\vector(1,0){6}}
\multiput(0,0)(1,0){5}{\line(0,1){4}}
\multiput(0,0)(0,1){5}{\line(1,0){4}}
\end{picture}
\vspace{.5in}
\caption{This is a blank graph.}\label{samplefig}
\end{figure}

Figures which were drawn using picture mode in \LaTeX, such as
Figure~\ref{samplefig} below, should work with $\mathcal{AMS}$-\LaTeX.

There is a bibliography at the end of this sample.

This is additional text, which is added so that the sample paper will be
three pages long and you can see the format of the third page.
This is additional text, which is added so that the sample paper will be
three pages long and you can see the format of the third page.
This is additional text, which is added so that the sample paper will be
three pages long and you can see the format of the third page.
This is additional text, which is added so that the sample paper will be
three pages long and you can see the format of the third page.
This is additional text, which is added so that the sample paper will be
three pages long and you can see the format of the third page.
This is additional text, which is added so that the sample paper will be
three pages long and you can see the format of the third page.
This is additional text, which is added so that the sample paper will be
three pages long and you can see the format of the third page.
This is additional text, which is added so that the sample paper will be
three pages long and you can see the format of the third page.

This is additional text, which is added so that the sample paper will be
three pages long and you can see the format of the third page.
This is additional text, which is added so that the sample paper will be
three pages long and you can see the format of the third page.
This is additional text, which is added so that the sample paper will be
three pages long and you can see the format of the third page.
This is additional text, which is added so that the sample paper will be
three pages long and you can see the format of the third page.
This is additional text, which is added so that the sample paper will be
three pages long and you can see the format of the third page.
This is additional text, which is added so that the sample paper will be
three pages long and you can see the format of the third page.
This is additional text, which is added so that the sample paper will be
three pages long and you can see the format of the third page.
This is additional text, which is added so that the sample paper will be
three pages long and you can see the format of the third page.

This is additional text, which is added so that the sample paper will be
three pages long and you can see the format of the third page.
This is additional text, which is added so that the sample paper will be
three pages long and you can see the format of the third page.
This is additional text, which is added so that the sample paper will be
three pages long and you can see the format of the third page.
This is additional text, which is added so that the sample paper will be
three pages long and you can see the format of the third page.
This is additional text, which is added so that the sample paper will be
three pages long and you can see the format of the third page.
This is additional text, which is added so that the sample paper will be
three pages long and you can see the format of the third page.
This is additional text, which is added so that the sample paper will be
three pages long and you can see the format of the third page.
This is additional text, which is added so that the sample paper will be
three pages long and you can see the format of the third page.

\begin{thebibliography}{9}

\bibitem{S} Simion, R., ``Noncrossing Partitions.'' {\em Discrete Math.}
{\bf 217}(2000), 367--409.

\bibitem{WW} Whittaker, E. T.,  and Watson, G. N., {\em A Course of Modern
Analysis.} Cambridge University Press, Cambridge, 1927. 

\end{thebibliography}

\end{document}




