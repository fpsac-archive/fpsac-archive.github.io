% Dear Prof. Stanton:
%
%   I hope this manuscript is still in time to be considered for the
% proceedings of the FPSAC'96 conference next summer.  I think Vic =
Reiner's
% submission might be on a related topic.
%
%                            Yours, Peter Magyar

%  **  compile using Latex 2.09  **
%
\documentstyle{article}
%
 \def\Bbb{\bf}
 \def\frak{\cal}
 \def\boxtimes{\Box\hspace{-0.76em}\times}
 \def\bigtimes{\prod}
%
%
%
% New operators
\newcommand{\sgn}{\mathop{\rm sgn}}
\newcommand{\Row}{\mathop{\rm Row}}
\newcommand{\Col}{\mathop{\rm Col}}
\newcommand{\Ker}{\mathop{\rm Ker}}
\newcommand{\IM}{\mathop{\rm Im}}
\newcommand{\Sym}{\mathop{\rm Sym}\mbox{}}
\newcommand{\St}{\mathop{\rm St}}
\newcommand{\Gr}{\mathop{\rm Gr}}
\newcommand{\id}{\mathop{\rm id}}
\newcommand{\incl}{\mathop{\rm incl}}
\newcommand{\rest}{\mathop{\rm rest}\mbox{}}
\newcommand{\rank}{\mathop{\rm rank}}
\newcommand{\pr}{ {\mathop{\rm pr}} }
\newcommand{\wt}{ {\mathop{\rm wt}} }
\newcommand{\tr}{\mathop{\rm tr}}
\newcommand{\Hom}{\mathop{\rm Hom}}
\newcommand{\Char}{ {\mathop{\rm char}} }
\newcommand{\Span}{ {\mathop{\rm Span}} }
\newcommand{\diag}{\mathop{\rm diag}}
%
% Theorems
\newtheorem{thm}{Theorem}
\newtheorem{lem}{Lemma}
\newtheorem{conj}{Conjecture}
%
% Other mathematical symbols
\newcommand{\eqdef}{\stackrel{\rm def}{=3D}}
\newcommand{\CC}{{\Bbb C}}
\newcommand{\C}{{\Bbb C}}
\newcommand{\R}{{\Bbb R}}
\newcommand{\Z}{{\Bbb Z}}
\newcommand{\ZZ}{{\Bbb Z}}
\newcommand{\NN}{{\Bbb N}}
\newcommand{\PP}{{\Bbb P}}
\newcommand{\al}{{\alpha}}
\newcommand{\be}{{\beta}}
\newcommand{\om}{{\varpi}}
\newcommand{\Sig}{{\Sigma}}
\newcommand{\sig}{{\sigma}}
\newcommand{\alD}{{\alpha_D}}
\newcommand{\beD}{{\beta_D}}
\newcommand{\VV}{ {\cal V} }
\newcommand{\Vn}{ V^{\otimes n} }
\newcommand{\lam}{ \lambda }
\newcommand{\FF}{ { \cal F } }
\newcommand{\FD}{ \FF_D }
\newcommand{\FDB}{ \FF_D^B }
\newcommand{\SVD}{ S_D^{\VV} }
\newcommand{\II}{ { \cal I } }
\newcommand{\ID}{ \II_D }
\newcommand{\LL}{ {\cal L} }
\newcommand{\LD}{ \LL_D }
\newcommand{\EE}{ {\cal E} }
\newcommand{\OO}{ {\cal O} }
\newcommand{\TT}{ {\cal T} }
\newcommand{\SS}{ {\frak S} }
\newcommand{\DV}{ \Delta_D V}
\newcommand{\DVV}{ \Delta_D^{\VV} V}
\newcommand{\Dh}{ \widehat{D} }
\newcommand{\Dt}{ \widetilde{D} }
\newcommand{\Ch}{ \widehat{C} }
\newcommand{\tB}{ \stackrel{B}{\times} }
\newcommand{\tw}{ \tilde{w} }
\newcommand{\tu}{ \tilde{u} }

% \newcommand{\}{  }

%
\flushbottom
%
\title{Four New Formulas for Schubert Polynomials}
\author{Peter Magyar \\[.1em]
Northeastern University\\[.1em]
{ pmagyar@lynx.neu.edu}}
%
 \catcode`@=3D11
 \def\@date{November 1995}
 \catcode`@=3D12
%

\begin{document}

\maketitle

%  \begin{center} {\bf Abstract} \end{center}
%
%  {\small \noindent   }

\noindent
We  state several new combinatorial formulas for
the Schubert polynomials.
They are generalizations of well-known expressions
for the Schur polynomials:
(1) the Demazure character formula;
(2) the realization as the generating function of
semi-standard tableaux of a given shape;
and (3), (4) the Weyl character formula.
Our formulas appear surprising
from a combinatorial point of view because
their derivation and proof
involve a new geometric model, the configuration varieties.

The results we state here are a special case of
formulas for a broad class of Schur-type polynomials,
the (flagged) Schur polynomials of strictly separated diagrams =
\cite{LZ},
\cite{BFZ}.
These include skew Schur and key polynomials
\cite{LS3}, and the Schur polynomials of northwest diagrams
\cite{Mag1},
\cite{Mag2}, \cite{RS1}, \cite{RS2}.

\section{Schubert polynomials}

The Schubert polynomials $\SS(w)$ of
permutations $w \in \Sig_n$ are polynomials
in variables $x_1,\ldots,x_n$.
They were originally considered as representatives of
Schubert classes in the Borel picture of the cohomology of
the flag variety $GL(n)/B$, though we will give a completely
different geometric interpretation in the later sections of
this note.

They are constructed in terms of
the following {\em divided difference operators}
\cite{Dem}, \cite{LS1}, \cite{MacD}.
First, the operator $\partial_i$ is defined by
$$
\partial_i f\,(x_1,\ldots,x_n) =3D
{f(x_1,\ldots,x_i,x_{i+1},\ldots,x_n) - =
f(x_1,\ldots,x_{i+1},x_i,\ldots,x_n)
\over x_i - x_{i+1}
}.
$$
Then for a reduced decomposition of a
permutation $u =3D s_{i_1} s_{i_2} \cdots$,
the operator
$\partial_u =3D \partial_{i_1} \partial_{i_2} \cdots$
is independent of the reduced decomposition chosen.
Also, take $\partial_e =3D \mbox{id}$.

Now we may define the Schubert polynomials as follows.
Let $w_0 =3D n, n-1, \ldots, 2, 1$
be the longest permutation,
and take $u =3D w^{-1} w_0$, so that $wu =3D w_0$.
Then
$$
\SS(w) =3D
\partial_u(x_1^{n-1} x_2^{n-2} \cdots x_{n-2}^2 x_{n-1}).
$$
We have $\mbox{deg } \SS(w) =3D \ell(w)$.
\\[1em]
{\bf Example.}
For the permutation $w =3D 24153 \in \Sig_5$,
by inverting first ascents we get
$w s_1 s_3 s_2 s_1 s_4 s_3 =3D w_0$,
so
$$
\SS(w) =3D \partial_1 \partial_3 \partial_2 \partial_1 \partial_4
\partial_3(x_1^4 x_2^3 x_3^2 x_4)
=3D x_1 x_2\, (x_1 x_2 + x_1 x_3 + x_2 x_3 + x_1 x_4 + x_2 x_4).
$$

\section{Orthodontia on a Rothe diagram}

Consider the {\em Rothe diagram}
of a permutation $w \in \Sig_n$,
$$
D =3D D(w) =3D\{(i,j) \in \NN \times \NN
\mid i < w^{-1}(j), \ \ j < w(i)\}.
$$
Its elements are called {\em squares}.
We shall often think of $D$ as a sequence
$(C_1,C_2,\ldots,C_r)$ of columns $C_j \subset \NN$,
by projecting squares $(i,j)$ to their first coordinate.
We omit any empty columns from the sequence.

In the sequel, our main interest is in Rothe
diagrams, but our analysis will involve more
general diagrams $D \subset \NN \times \NN$
of squares in the plane.  In fact, to any such diagram
one can associate a ``Schur polynomial'', which is the character
of its flagged Schur module (see Section 7).
The formulas we state will apply not only to
the Schubert polynomials (associated to Rothe diagrams),
but to the Schur polynomials of any ``strongly separated''
diagram.  (See Section 9.)

Now, let $D$ be a Rothe diagram.
For our formulas, we will require a sequence of permutations
$w_1, w_2, \ldots, w_r$ which is compatible
with $D$ in the sense that
$w_j\, \{1,2,\ldots, |\,C_j|\} =3D C_j$ for all $j$.
We also demand that the sequence be
monotone in the weak order:
that is, for some $u_1, u_2, \ldots , u_r$,
we must have $w_1 =3D u_1,\, w_2 =3D u_1 u_2,\, \ldots\, ,
w_r =3D u_1 \cdots u_r$,
with $\ell(w_j) =3D \ell(u_1) + \ell(u_2) + \cdots + \ell(u_j)$.
(It is enough to require $w_r =3D \ell(u_1) + \cdots + \ell(u_r)$.)

This can be done by means of the following algorithm.
Given a column $C \subset \{1,\ldots,n\}$, a
{\em missing tooth} of $C$ is an integer $i$ such
that $i \not\in C$, but $i' \in C$ for some $i' > i$.
The only $C$ without any missing teeth are
$\{1,2,3,\ldots,j\}$.  Given a diagram $D =3D (C_1, \ldots, C_r)$,
let $(i_0,j_0)$ denote a {\em special} missing tooth in $D$ which is
in the leftmost column possible, and as high as possible in this column
subject to the condition that $(i_0+1,j_0) \in D$.

Now perform {\em orthodontia} on $D$ to get a new
diagram $D'$ with fewer missing teeth,
by switching rows $i_0$ and $i_0+1$ in the columns weakly
right of $(i_0,j_0)$.
That is, change $D$ to
$$
\begin{array}{rcl}
D' & =3D & \{ (i,j) \mid (i,j) \in D \mbox{ and } j < j_0 \} \\
& & \cup \{ (s_{i_0}i,j) \mid (i,j) \in D \mbox{ and } j \geq j_0 \} .
\end{array}
$$
Next, locate the special missing tooth $(i_1,j_1)$ of
$D'$, and perform this procedure again on $D'$ to
get $D''$ and $(i_2,j_2)$, and so on until we reach a diagram
with no missing teeth.  Notice that $j_0 \leq j_1 \leq \cdots $.

Finally, define the {\em orthodontic sequence} of
$D =3D D(w)$ to be $w_1, w_2, \ldots$, where
$w_j =3D \prod_{k \, : \
% \mbox{\tiny  s.t. }
j_k \leq j} s_{i_k}$, the product being taken
over all $k$ such that $(i_k,j_k)$ is weakly left of column $j$.
It is easily seen that this sequence
has the desired properties.
\\[1em]
{\bf Example.}
For the same
$ w =3D 24153$, we have
$$
D =3D D(w) =3D \
\begin{array}{cccc}
1 & \Box &      &      \\
2 & \Box &      & \Box \\
3 &      &      &      \\
4 &      &      & \Box
\end{array}
\ \
=3D
\ \
\begin{array}{cc}
\Box &  \circ \\
\Box &  \Box \\
     &       \\
     &  \Box
\end{array}
$$
$$
D' =3D
\begin{array}{ccc}
1 & \Box & \Box \\
2 & \Box &      \\
3 &      & \circ \\
4 &      & \Box
\end{array}
\ \ \ \
 D'' =3D
\begin{array}{ccc}
1 & \Box &  \Box \\
2 & \Box & \circ \\
3 &      &  \Box
\end{array} \
\ \ \ \
 D''' =3D
\begin{array}{ccc}
1 & \Box &  \Box \\
2 & \Box &  \Box \\
\end{array} \
$$
so that the special missing teeth (as indicated by $\circ$) are
$(i_0,j_0) =3D (1,2)$, $(i_1,j_1) =3D (3,2)$, $(i_2,j_2) =3D (2,2)$,
and $w_1 =3D e$, \, $w_2 =3D s_{i_0} s_{i_1} s_{i_2} =3D s_1 s_3 s_2$.

Note that $w_r =3D w_2 =3D s_1 s_3 s_2$ is a reduced subword
of the first-ascent sequence $s_1 s_3 s_2 s_1 s_4 s_3$
which raises $w$ to the maximal permutation $w_0$,
as in the previous section.  This is always the case,
and we can give an algorithm for extracting this subword.

\section{Demazure character formula}

The definition of $\SS(w)$ involves {\em descending} induction
(lowering the degree), but we give the following
{\em ascending} algorithm.

Let $D(w) =3D (C_1, \ldots, C_r)$ (omitting empty columns),
and let $c_j =3D |\, C_j|$.
Take a monotone compatible sequence $w_1, \ldots, w_r$ for $D(w)$,
such as the orthodontic sequence defined above, and
let $u_j =3D w_{j-1}^{-1} w_j$, so that
$w_1 =3D u_1$,\, $w_2 =3D u_1 u_2, \cdots$.
Furthermore, let $\lam_i =3D x_1 x_2 \cdots x_i$.

Define the {\em Demazure operators} (isobaric divided differences)
$ \pi_i =3D \partial_i x_i$ and $\pi_u =3D \pi_{i_1} \pi_{i_2} \cdots$,
for $u =3D s_{i_1} s_{i_2} \cdots$ a reduced decomposition.
(See \cite{Dem}.)
These are analogous to the $\partial$ operators, but
do not change the degree of a homogeneous polynomial.

Finally, let $\SS_0(w) =3D 1$
and
$$
\SS_k(w) =3D \pi_{u_k}(\lam_{c_k} \SS_{k-1}(w)).
$$

\begin{thm} \mbox{} \\
\mbox{} \hfill $\SS_r(w) =3D \SS(w)$.\hfill \mbox{}
\end{thm}
{\bf Example.}
For our permutation $w =3D 24153$,
we have $c_1 =3D c_2 =3D 2$, $u_1 =3D e$, $u_2 =3D s_1 s_3 s_2$,
and we may verify that
$$
\SS(w) =3D x_1 x_2 \, \pi_1 \pi_3 \pi_2(x_1 x_2) .
$$
Note that this makes the factorization evident.

\section{Young tableaux}

The work of
Lascoux-Schutzenberger \cite{LS3} and Littlemann
\cite{Lit} allows us to ``quantize'' our Demazure
formula, realizing the terms of the polynomial by
certain tableaux endowed with a crystal graph
structure.   Reiner and Shimozono have shown that our construction
gives the same
non-commutative Schubert polynomials as those in
\cite{LS2}.  In fact, A. Lascoux has informed me that
some of the contents of this section were known to him, and motivated
\cite{LS2}, though not explicitly stated there.
Our tableaux are different, however, from the
``balanced tableaux'' of Fomin, Greene, Reiner,
and Shimozono.

Recall that a {\em column-strict filling} of a diagram $D$
(with entries in $\{1,\ldots, n\}$) is
a map $t$ filling the squares of $D$ with numbers from
$1$ to $n$,
strictly increasing down each column.
The {\em content} of a filling is a monomial
$x^t =3D \prod_{(i,j) \in D} x_{t(i,j)}$, so that the
exponent of $x_i$ is the number of times $i$ appears in the
filling.
We will define a set of fillings
$\TT$ of the Rothe diagram $D(w)$ which satisfy
$$
\SS(w) =3D \sum_{t \in \TT} x^t.
$$
The set $\TT$ will be defined recursively,
$\TT_0,\,  \TT_1, \ldots, \TT_r =3D \TT$,
so that
$$
\SS_k(w) =3D \sum_{t \in \TT_k} x^t.
$$

We will need the {\em root operators} first defined in
\cite{LS3}.  These are operators $f_i$ which
take a filling $t$ of a diagram $D$ either to another filling of $D$
or to the empty set $\emptyset$.  To
define them we first encode a filling $t$ in terms of its
{\em reading word}:  that is, the sequence
of its entries starting at the upper left corner,
and reading down the columns one after another:
$t(1,1), t(2,1), t(3,1), \ldots, t(1,2), t(2,2), \ldots $.

The lowering operator $f_i$ either takes a word $t$ to the
empty word $\emptyset$,
or it changes one of the $i$ entries to $i+1$,
according to the following rule.  First, we ignore all the entries
in $t$ except those containing $i$ or $i+1$;  if an $i$ is
followed by an $i+1$ (ignoring non $i$ or $i+1$ entries in between),
then henceforth we ignore that pair of entries; we look again
for an $i$ followed (up to ignored entries) by an $i+1$,
and henceforth ignore this pair; and iterate until we obtain
a subword of the form $i+1,\, i+1,\ldots, i+1,\, i,\, i, \ldots, i$.
If there are {\em no} $i$ entries in this word, then
$f_i(t) =3D \emptyset$, the empty word.  If there are some $i$ entries,
then the {\em leftmost} is changed to $i+1$.

For example, we apply $f_2$ to the word
$$
\begin{array}{cccccccccccccc}
 t & =3D & 1 & 2 & 2 & 1 & 3 & 2 & 1 & 4 & 2 & 2 & 3 & 3 \\
   &   & . & 2 & 2 & . & 3 & 2 & . & . & 2 & 2 & 3 & 3 \\
   &   & . & 2 & . & . & . & 2 & . & . & 2 & . & . & 3 \\
   &   & . & {\bf 2} & . & . & . & {\bf 2} & . & . & . & . & . & .
\\[.2em]
f_2(t) & =3D &
         1 & {\bf 3} & 2 & 1 & 3 & {\bf 2} & 1 & 4 & 2 & 2 & 3 & 3  =
\\[.2em]
f_2^2(t) & =3D &
         1 & {\bf 3} & 2 & 1 & 3 & {\bf 3} & 1 & 4 & 2 & 2 & 3 & 3  =
\\[.2em]
f_2^3(t) & =3D & \emptyset
           &   &   &   &   &   &   &   &   &   &   &
\end{array}
$$
Decoding the image word back into a filling of the same diagram
$D$, we have defined our operators.

Moreover, consider the column
$\phi_m =3D \{1,2,\ldots,m\}$
and its minimal
column-strict filling
$t_m$ ($i$th level filled with $i$).
For a filling $t$ of any diagram
$D =3D (C_1,\ldots,C_r)$,
define in the obvious way
the composite filling $t_m \sqcup\, t$ of the
juxtaposed diagram $\phi_i \sqcup\, D =3D (\phi_i, C_1, \ldots, C_r)$.
In terms of words, this means
concatenating the words $(1,2,\ldots,m)$ and $t$.

Now we can define our sets of tableau.
Let our notation be as in the Demazure
character formula,
$D(w) =3D (C_1,\ldots, C_r)$, etc,
and take a reduced decomposition
$u_k =3D s_{i_1} \cdots s_{i_l}$.  Define
$\TT_0 =3D \{ \emptyset \}$, and
$$
\TT_{k} =3D \langle f_{i_1} \rangle \cdots
\langle f_{i_l} \rangle( t_{c_k} \!\sqcup \TT_{k-1} ),
$$
where $\langle f_{i} \rangle$ means the set of powers
$\{\id, f_i, f_i^2, \ldots\} $.

\begin{thm}  The Schubert polynomial $\SS(w)$
is the generating function
for the tableaux $\TT =3D \TT_r$:
$$
\SS(w) =3D \sum_{t \in \TT} x^t.
$$
Furthermore, the crystal graph structure of
$\TT$ reflects the splitting of $\SS(w)$ into key polynomials:
$$
\SS(w) =3D \sum_{t \in \mbox{Yam}(\TT)} \kappa_{w_t(x^t)},
$$
where $\mbox{Yam}(\TT)$ is the set of Yamanouchi words in
$\TT$, and the $w_t$ are some permutations.
\end{thm}
{\bf Example.}
As above, when $c_1 =3D c_2 =3D 2$, $u_1 =3D e$, $u_2 =3D s_1 s_3 s_2$,
the set of tableaux (words) grows as follows:
$$
\TT_0 =3D \{\emptyset\}
\stackrel{t_2 \sqcup}{\rightarrow} \{ 12 \}
\stackrel{\langle f_2 \rangle}{\rightarrow} \{12, 13\}
\stackrel{\langle f_3 \rangle}{\rightarrow}  \{12, 13, 14\}
\stackrel{\langle f_1 \rangle}{\rightarrow} \TT_1 =3D \{12, 13, 14, 23, =
24\}
$$
$$
\stackrel{t_2 \sqcup}{\rightarrow}
\ \ \TT =3D \TT_2 =3D \{1212, 1213, 1214, 1223, 1224\}.
$$
This clearly gives us the Schubert polynomial
as generating function,
and furthermore we see the crystal graph
(with vertices the tableaux in $\TT$ and edges all pairs of the
form $(t, f_i t)$\ ):
$$
\begin{array}{ccc}
1223 \!\! & \! \stackrel{1}{\leftarrow}  \! & \!\! 1213 \\
\mbox{\small 3}\! \downarrow & & \downarrow\! \mbox{\small 3} \\
1224 \!\!& \! \stackrel{1}{\leftarrow} \! &\!\! 1214
\end{array}
\ \ \  \ \ 1212
$$
The highest-weight elements in each component are the
Yamanouchi words

$\mbox{Yam}(\TT) =3D \{1213, 1212\}$, and by looking at
the corresponding lowest elements, we may deduce
$\SS(w) =3D \kappa_{x_1 x_2^2 x_4} + \kappa_{x_1^2 x_2^2} =3D
\kappa_{1201} + \kappa_{2200}$.
Lascoux and Schutzenberger $\cite{LS3}$
have obtained another characterization of such lowest-weight
tableaux.


\section{Weyl character formula I}

Our next character formula mixes the rational terms
of the Weyl character formula with the chains of Weyl group
elements in the Standard Monomial Theory of Lakshmibai-Seshadri-Musili.
% \cite{Lak}
For computations, this formula is very inefficient:
a related expression with much fewer terms, more directly
generalizing the Weyl formula, is given in
the following section.  The main advantage of the current expression
is that one can use it to obtain character formulas for
certain analogous polynomials associated with other root systems
(though unfortunately these analogous polynomials do not seem to include
the Schubert polynomials of other root systems).

Suppose we have any permutation with choice of reduced decomposition,
$u =3D s_{i_1} \cdots s_{i_l}$,
and a sequence of zeroes and ones
$\epsilon =3D (\epsilon_1, \epsilon_2, \cdots )$, $\epsilon_j \in =
\{0,1\}$.
Denote
$$
u^{\epsilon} =3D s_{i_1}^{\epsilon_1} \cdots s_{i_l}^{\epsilon_l},
$$
a subword of $u$, not necessarily reduced.

Consider as before
a monotone sequence of permutations $w_1, w_2,\ldots,w_r$
compatible with the Rothe diagram $D =3D D(w)$, and
choose a reduced decomposition for
$u_j =3D w_{j-1}^{-1} w_j$.
This gives a choice of reduced decomposition for each $w_j =3D u_1 =
\cdots u_j$,
and in particular for $w_r =3D s_{i_1} \cdots s_{i_l}$.
Let $v_k =3D s_{i_1} \cdots s_{i_k}$ for $k \leq l$.
Recall that $\lam_{c_j} =3D x_1 x_2 \cdots x_{c_j}$, the fundumental
weight associated to the length $c_j$ of the $j$th column of $D$
(not counting empty columns).

\begin{thm}
$$
\SS(w) =3D
\sum_{\epsilon}
{ \prod_{j =3D 1}^r \,  w_j^{\epsilon}(\lam_{c_j})
  \over
  \prod_{k =3D 1}^l \,
  (1- v_k^{\epsilon}(x_{i_k}^{-1} x_{i_k+1})) ,
}
$$
where the summation is over all $2^l$  sequences
of zeroes and ones $\epsilon =3D (\epsilon_1, \ldots, \epsilon_l)$.
\end{thm}
This follows straight-forwardly from our Demazure formula,
though it also has a geometric interpretation in terms of
Bott-Samelson varieties (see below).
{\bf Example.}  For the same $w =3D 24153$,
we have the reduced decomposition $w_r =3D w_2 =3D s_1 s_3 s_2$,
$l =3D 3$,
so the Schubert polynomial is the following sum of
8 terms corresponding
to $\epsilon =3D (000), (001), (010), (011), \ldots$\ :
$$
\begin{array}{rcl}
S(w) & =3D &
% (000)
{ x_1^2 x_2^2
\over
(1 - x_1^{-1} x_2) (1 - x_3^{-1} x_4) (1 - x_2^{-1} x_3) }
+
% (001)
{ x_1^2 x_2 x_3
\over
(1 - x_1^{-1} x_2) (1 - x_3^{-1} x_4) (1 - x_3^{-1} x_2)  }
\\[.7em]
& & \
+
% (010)
{ x_1^2 x_2^2
\over
(1 - x_1^{-1} x_2) (1 - x_4^{-1} x_3) (1 - x_2^{-1} x_4) }
+
%(011)
{ x_1^2 x_2 x_4
\over
(1 - x_1^{-1} x_2) (1 - x_4^{-1} x_3) (1 - x_4^{-1} x_2)  }
\\[.7em]
& & \ \ \
+
% (100)
{ x_1^2 x_2^2
\over
(1 - x_2^{-1} x_1) (1 - x_3^{-1} x_4) (1 - x_1^{-1} x_3)  }
+
% (101)
{ x_1 x_2^2 x_3
\over
(1 - x_2^{-1} x_1) (1 - x_3^{-1} x_4) (1 - x_3^{-1} x_1)  }
\\[.7em]
& & \ \ \ \ \
+
% (110)
{ x_1^2 x_2^2
\over
(1 - x_2^{-1} x_1) (1 - x_4^{-1} x_3) (1 - x_1^{-1} x_4)  }
+
% (111)
{ x_1 x_2^2 x_4
\over
(1 - x_2^{-1} x_1) (1 - x_4^{-1} x_3) (1 - x_4^{-1} x_1)  } .
\end{array}
$$


\section{Weyl Character Formula II}
\label{Weyl II}

Finally, we state a result directly generalizing
the Weyl character formula (Jacobi bialternant), reducing to
it in case $\SS(w)$ is a Schur polynomial.

The formula involves
certain extensions of the Rothe diagram $D =3D D(w)$.
Define the Young diagram
$\Phi =3D \{ (i,j) \mid 1 \leq i \leq j \leq n-1 \}$.
Let the {\em flagged diagram }
$\Phi \sqcup D$ be the concatenation
of the two diagrams placed horizontally next to each other:
that is, the columns of $\Phi \sqcup D$
are those of $\Phi$ followed by those of $D$.

Now, given $\Phi \sqcup D =3D (C_1,\ldots,C_r)$,
define the {\em blowup} of the flagged diagram
$\widehat{\Phi \sqcup D} =3D (C_1, \ldots, C_r,
{C}_1', {C'}_2', \ldots)$,
where the extra columns  are the intersections
$\tilde{C} =3D  C_{i_1} \cap C_{i_2} \cap \cdots \subset \NN$,
for all lists $C_{i_1}$, $C_{i_2}, \ldots$ of columns of
$\Phi \sqcup D $;
but if an intersection $C_{i_1} \cap C_{i_2} \cap \cdots =3D C_k$
is already a column of $\Phi \sqcup D$, then we do not append it.

Now let
$ \Dt =3D \widehat{\Phi \sqcup D} $.
Define a {\em standard tabloid} $t$ of
$ \Dt $
to be a column-strict filling
such that if $C, C'$ are columns of
$ \Dt $
with $C$ horizontally contained in $C'$,
then the numbers filling $C$
all appear in the filling of $C'$.
In symbols,
$t : \Dt \rightarrow \{1,\ldots,n\}$ ,
$t(i,j) < t(i+1,j)$ for all $i, j$,
and
$C \subset C' \Rightarrow t(C) \subset t(C')$.

For $1 \leq i \neq j \leq n$
and a tabloid $t$ of
$ \Dt $,
we
define certain integers: $d_{ij}(t)$
 is the number of connected components of
the following graph.
The vertices are columns $C$ of
$ \Dt $
such that $i \in t(C)$,  $j \not\in t(C)$;
the edges are $(C,C')$ such that $C \subset C'$
or $C' \subset C$.

Finally, since there are inclusions of diagrams
$D,\, \Phi \subset \Dt =3D  \widehat{\Phi \sqcup D}$,
we have the {\em restrictions} of
a tabloid $t$ for $\Dt$
to $D$ and $\Phi$, which we denote $t|D$ and $t|\Phi$.

\begin{thm} \mbox{} \\
$$
\SS(w) =3D
\sum_{t}
{  x^{(t|D \! )}
  \over \prod_{i < j}\,
(1-x_i^{-1} x_j)^{d_{ij}(t)-1}\
(1-x_j^{-1} x_i)^{d_{ji}(t)}  },
$$
where $t$ runs over the standard tabloids for
$\widehat{\Phi \sqcup D}$ such that
$(t|\Phi)(i,j) =3D i$ for all $(i,j) \in \Phi$.
\end{thm}
{\bf Example.}  For the same $w =3D 24153$,
$$
D =3D D(w) =3D
\begin{array}{ccc}
1 & \Box &      \\
2 & \Box & \Box \\
3 &      &      \\
4 &      & \Box
\end{array}
\ \ \ \ \ \ \ \ \ \ \ \ \
\Phi \sqcup D =3D
\begin{array}{cccccc}
\Box & \Box & \Box & \Box & \Box &      \\
     & \Box & \Box & \Box & \Box & \Box \\
     &      & \Box & \Box &      &      \\
     &      &      & \Box &      & \Box
\end{array}
$$
$$
\widehat{\Phi \sqcup D} =3D
\begin{array}{cccccccc}
1 & \Box & \Box & \Box & \Box & \Box &       &       \\
2 &      & \Box & \Box & \Box & \Box &  \Box & \Box  \\
3 &      &      & \Box & \Box &      &       &       \\
4 &      &      &      & \Box &      &  \Box &
\end{array}.
$$
There are six standard
tabloids of the type occurring in the theorem.
Their restrictions to the last three columns of
$\widehat{\Phi \sqcup D}$ are:
$$
\begin{array}{ccc}
1 &   &   \\
2 & 1 & 1 \\
  &   &   \\
  & 2 &
\end{array},
\ \
\begin{array}{ccc}
1 &   &   \\
2 & 1 & 1 \\
  &   &   \\
  & 3 &
\end{array},
\ \
\begin{array}{ccc}
1 &   &   \\
2 & 1 & 1 \\
  &   &   \\
  & 4 &
\end{array},
\ \
\begin{array}{ccc}
1 &   &   \\
2 & 1 & 2 \\
  &   &   \\
  & 2 &
\end{array},
\ \
\begin{array}{ccc}
1 &   &   \\
2 & 2 & 2 \\
  &   &   \\
  & 3 &
\end{array},
\ \
\begin{array}{ccc}
1 &   &   \\
2 & 2 & 2 \\
  &   &   \\
  & 4 &
\end{array}.
$$
The integers $d_{ij}(t)$ are 0, 1, or 2,
and we obtain
$$
\begin{array}{rcl}
S(w) & =3D &
{ x_1^2 x_2^2
\over
(1 - x_1^{-1} x_2) (1 - x_2^{-1} x_3) (1 - x_2^{-1} x_4)  }
+
%
{ x_1^2 x_2 x_3
\over
(1 - x_1^{-1} x_2) (1 - x_3^{-1} x_4) (1 - x_3^{-1} x_2)  }
\\[.7em]
& & \ +
%
{ x_1^2 x_2 x_4
\over
(1 - x_1^{-1} x_2) (1 - x_4^{-1} x_2) (1 - x_4^{-1} x_3)  }
+
%
{ x_1^2 x_2^2
\over
(1 - x_1^{-1} x_3) (1 - x_1^{-1} x_4) (1 - x_2^{-1} x_1)  }
\\[.7em]
& & \ \ \ \ +
%
{ x_1 x_2^2 x_3
\over
(1 - x_2^{-1} x_1) (1 - x_3^{-1} x_4) (1 - x_3^{-1} x_1)  }
+
%
{ x_1 x_2^2 x_4
\over
(1 - x_2^{-1} x_1) (1 - x_4^{-1} x_1) (1 - x_4^{-1} x_3)  } .
\end{array}
$$
Note that some, but not all, of the above six terms are
among the eight terms of the previous example.
As in the case of
the original Weyl character formula,
it is not clear a priori why either of these rational functions
should simplify to a polynomial (with positive integer coefficients).

This formula has been implemented by the author in Mathematica, =
available
on request.

\section{Schur modules}

The above formulas arise naturally from a Borel-Weil
theory which relates Schubert polynomials with certain
algebraic varieties similar to the Schubert varieties
of $GL(n)$.  (See \cite{Mag2}.)
The starting point of our theory
is the result of Kraskiewicz and Pragacz
which realizes Schubert polynomials as characters
of ``flagged Schur modules''.

We shall write $G =3D GL(n,\CC)$,
$B =3D $ the subgroup
of upper triangular matrices,
$V =3D \CC^n$ the
defining representation,
and $V_i$ the subspace of $V$
spanned by the first $i$ coordinate vectors.

Let the diagram $D \subset \NN \times \NN$ be
any set of squares $(i,j)$ in the plane.
Let $\Sigma_D$ be the
symmetric group permuting the squares of $D$,
 $\Col(D) \subset \Sigma_D$ the subgroup permuting
the squares within each column, and $\Row(D)$
similarly for rows.
Define (almost) idempotents
$\al_D$, $\beta_D$ in the group algebra
$\CC[\Sigma_D]$ by
$$
\alD =3D  \sum_{\sigma \in \Row D} \sigma, \ \ \ \
\beD =3D  \sum_{\sigma \in \Col D} \sgn(\sigma) \sigma ,
$$
where $\sgn(\sigma)$ is the sign of the permutation.

$\Sigma_D$ acts on the right of the tensor product
$V^{\otimes D}$ by permuting factors,
and $G$ and  $B$ act on the left
by the diagonal action.
These two actions commute.
Define the {\em flagged Schur module}
to be the $B$-stable subspace
$$
S_D^B =3D (\! \! \bigotimes_{(i,j) \in D} V_i \ )
\ \alD \beD
\subset V^{\otimes D} .
$$

\begin{thm}{(Kraskiewicz-Pragacz)\ }
\label{KP}
The Schubert polynomial for a permutation
is the character of the flagged Schur module
of its Rothe diagram:
$$
\SS(w) =3D
\tr(\, \diag(x_1,\ldots,x_n) \mid S_{D(w)}^B).
$$
Here the constant $n$ is taken so that $D(w)$ has at
most $n$ rows.
\end{thm}
Reiner, Shimozono, and the author have given a proof
of this theorem using configuration varieties.
\\[.5em]
{\bf Example.}  For $w =3D 24153$, if we
number the squares of $D(w)$ as
$$
D =3D D(w) =3D
\begin{array}{cc}
 \Box &      \\
 \Box & \Box \\
      &      \\
      & \Box
\end{array}
\ \ \
=3D
\ \ \
\begin{array}{cc}
  1   &      \\
  2   &  3   \\
      &      \\
      &  4
\end{array}
,
$$
then we have
(in cycle notation for $\Sigma_D \cong \Sigma_4$),
$$
\begin{array}{rcl}
\alpha_D \beta_D & =3D &
 (1 + (23)) \ (1-(12))(1-(34)) \\
& =3D &
1 + (23) - (12) - (34)
- (132) - (234) + (12)(34) + (1324) .
\end{array}
$$

Take $n =3D 5$,
so that $V^{\otimes D} \cong (\CC^5)^{\otimes 4}$,
a 20-dimensional space with coordinate vectors
$e_{i_1 i_2 i_3 i_4} =3D
e_{i_1} \otimes e_{i_2} \otimes e_{i_3} \otimes e_{i_4}$,
$i_1, \ldots, i_4 \in \{1,\ldots,5\}$.
$\Sigma_D$ acts by
$e_{i_1 i_2 i_3 i_4} \cdot \sigma =3D
e_{i_{\sigma(1)} i_{\sigma(2)} i_{\sigma(3)} i_{\sigma(4)}}$,
and $GL(5)$ acts diagonally on the tensor factors.
By definition, $S_D^B$ is spanned by vectors of the form
$$
v_t =3D e_{t(1), t(2), t(3), t(4)}
\, \alpha_D \beta_D ,
$$
for all fillings $t  : D \rightarrow \{1,\ldots,5\}$
with $t(1) \leq 1$, $t(2),t(3) \leq 2$,
$t(4)\leq 4$, and
it should follow from our theory that we obtain a basis if
we take only the 6 fillings $t \in \TT$, the set of tableaux
defined in our second character formula.
For instance, if $t =3D (1213)$,
$$
\begin{array}{rcl}
v_t & =3D  & e_{1213} + e_{1123} - e_{2113} - e_{1231}
- e_{1123} - e_{1132} + e_{2131} + e_{1132} \\
& =3D & e_{1213} - e_{2113} - e_{1231} + e_{2131}.
\end{array}
$$
% We may verify that, as indicated by the crystal graph,
% this is a highest weight vector:  $b \cdot v_t \in \CC v_t$
% for any upper triangular matrix $b$.


\section{Configuration varieties}

Now we translate our algebra into geometry, realizing a
flagged Schur module $S_D^B$ as the space of sections
of a line bundle over a (possibly singular) algebraic
variety $\FDB$.  Because the singularities are sufficiently
tame, we can obtain non-trivial transformations of our problem
by considering desingularizations of $\FDB$ and applying known
character formulas for spaces of sections over smooth varieties.
The first three formulas come from a Bott-Samelson resolution
of $\FDB$ \cite{De1}, the last from an analog of Zelevinsky's resolution
\cite{Zel}.

Let $D \subset \NN \times \NN$ be a diagram with all its
squares in rows $i =3D 1,\ldots,n$.  Let $C_1,\ldots,C_r \subset
\{1,\ldots,n\}$ be the columns of $D$, and for each $C =3D C_j$,
let $V_C =3D \mbox{Span }\{e_i \mid i \in C\} \subset \CC^n$,
a coordinate subspace of dimension $c_j =3D |C_j|$.  Consider the =
$r$-tuple
$(V_{C_1},\ldots,V_{C_r})$ as a point in the product of Grassmannians
$\Gr(D) =3D \Gr(c_1, \CC^n) \times \cdots \times \Gr(c_r, \CC^n)$,
and define the {\em flagged configuration variety}
$$
\FDB =3D \overline{B \cdot (V_{C_1},\ldots,V_{C_r})}
\subset \Gr(D) ,
$$
the closure of the $B$-orbit of the above point.
This is an irreducible projective variety, and the
Schubert varieties of $GL(n)/B$ and $GL(n)/P$ are clearly
special cases.
$\FDB$ has a natural line bundle $\LD$ defined by restricting
the Plucker bundle $\OO(1,\ldots,1)$ over the product of Grassmannians
$\Gr(D)$.
These varieties are very tractable in the case of a Rothe diagram,
and we may state the following Borel-Weil-Bott theorem.

\begin{thm}{(Magyar-van der Kallen)\ }
Let $D =3D D(w)$ a Rothe diagram.  Then $\FDB$ has rational
singularities and is projectively normal with respect to $\LD$.
Furthermore, the space of global sections
$$
H^0(\FDB,\LD) \cong (S_D^B)^*
$$
as $B$-modules, and $H^i(\FDB,\LD) =3D 0$ for all $i>0$.
\end{thm}


Now let $w_1, \ldots, w_r$ be the orthodontic sequence of
$D =3D D(w)$ and $w_r =3D s_{i_1} \cdots s_{i_l}$ the associated
reduced decomposition.  Thus the initial subwords are reduced
decompositions $w_j =3D s_{i_1} \cdots s_{i_{l(j)}}$,
where $l(j) =3D l(w_j)$.
Consider the associated Bott-Samelson variety \cite{De1}
$$
Z =3D P_{i_1} \tB \cdots \tB P_{i_l} / B ,
$$
where $P_i$ denotes the maximal parabolic of $G =3D GL(n)$ such that
$G/P_i \cong \Gr(i,\CC^n)$.  Define the multiplication map
$$
\begin{array}{cccc}
\phi : & Z & \rightarrow &
\Gr(D) \cong G/P_{c_1} \times G/P_{c_2} \times \cdots \times G/P_{c_r} =
\\
& (p_1,\ldots,p_l) & \mapsto &
( p_1 p_2 \cdots p_{l(1)},\  p_1 p_2 \cdots p_{l(2)},\,
\ldots \, , \ p_1,\cdots p_{l(r)}) .
\end{array}
$$

\begin{thm}
The map $\phi$ maps $Z$ birationally onto $\FDB$, and so is a
resolution of singularities.  Furthermore, for all $i$,
$$
H^i(Z, \phi^* \LD) \cong H^i(\FDB, \LD) .
$$
\end{thm}   From this,
the computations of Demazure \cite{De1} on the Bott-Samelson
variety directly imply our formula (1),  and (2) follows by the theory
of root operators.  Formula (3) results from applying the
Atiyah-Bott-Lefschetz fixed-point formula to $Z$.

\begin{thm}
Let $D =3D D(w)$ and
$\widetilde{D} =3D \widehat{\Phi \sqcup D}$ the blowup diagram
of section \ref{Weyl II}.  Then the configuration variety
$\FF^B_{\widetilde{D}}$ is a smooth variety, and is a resolution
of singularities of $\FDB$ via the natural projection map
map $\psi : \Gr(\widetilde{D}) \rightarrow \Gr(D)$.
Furthermore, for all $i$,
$$
H^i(\FF^B_{\widetilde{D}}, \psi^* \LD) \cong H^i(\FDB, \LD) .
$$
\end{thm}
%
Formula (4) now results from Atiyah-Bott-Lefschetz applied to
$\FF^B_{\widetilde{D}}$.
\\[.5em]
{\bf Example.}
In our case $w =3D 24153$, we may take $n =3D 4$, so that we
have $\Gr(D) =3D \Gr(2,\CC^4) \times \Gr(2, \CC^4)$, which we
may think of as the variety of pairs of lines in $\PP^3$.
The $B$-orbit of the special point $(V_{12}, V_{24})$ is precisely
the pairs of the form $(V_{12}, W)$, where $W =3D \langle v_1, v_2 =
\rangle$,
$v_1 \in V_{12}$, and $v_1, v_2$ are linearly independent.
Thus
$$
\FDB \cong \{ W \in \Gr(2,\CC^4) \mid \mbox{ dim}(V_{12} \cap W)
 \geq 1 \},
$$
the variety of lines in $\PP^3$ which intersect the coordinate axis.
(We can give such a description of $\FDB$ as configurations
with intersection conditions for any Rothe diagram. )
This is the singular Schubert variety in $\Gr(2,\CC^4)$,
the resolutions mentioned are the original Bott-Samelson and
Zelevinsky resolutions,
and there exist regular maps
$Z \rightarrow \FF^B_{\widetilde{D}} \rightarrow \FDB$.

%  In general, we can realize the Bott-Samelson variety also as
%  a configuration variety, in this case corresponding to
%  $$
%  BS(D) =3D
%  \begin{\array}{cccc}
%  1  &      &      & \Box \\
%  2  & \Box & \Box & \Box \\
%  3  &  &  &  \\
%  4  &       & \Box & \Box
%  \end{array},
%  $$
%  the variety of flags
%  $$
%  \{ \mbox{ point }\subset \mbox{ line }\subset \mbox{ plane } \subset =
\PP^3
%  \mid \mbox{ point }\subset \mbox{ coordinate axis }\subset \mbox{ =
plane }\}.
%  $$
%  In this case $\

\section{Generalizations}

The above results can be used to compute the characters
of  Schur modules more general than those associated to
Rothe diagrams.  In fact, let us replace $D(w)$ by any diagram
which satisfies
the following {\em strictly separated} condition.
For two sets $S, S' \subset \NN$, we say $S < S'$ if $s < s'$ for all
$s \in S$, $s' \in S'$.
Now, a diagram $D =3D (C_1, \ldots , C_r)$ with
columns $C_j \subset \{1,\ldots,n\}$,
is strictly separtated if,
for any two columns $C, C'$ of $D$, we have
$$
(C \setminus C') < (C' \setminus C)
\ \ \ \mbox{or} \ \ \
(C \setminus C') > (C' \setminus C) \ ,
$$
where $C\setminus C'$ denotes the complement of $C'$ in $C$.
See \cite{LZ}, \cite{BFZ}.
Replace the Schubert polynomial by the character of the
flagged Schur module $S_D^B$.
Then our Theorems 1, 2, 3, 6, and 7 remain valid:
the orthodontic algorithm gives a sequence of Weyl group
elements compatible with $D$, the first three formulas compute the
character of $S_D^B$, and the associated configuration variety
$\FF_D^B$ satisfies the Borel-Weil Theorem and possesses a Bott-Samelson
resolution.

Suppose that $D$ satisfies an even stronger property, the
{\em northwest condition}:
$$
(i,j), (i',j') \in D \Rightarrow (\, \min(i,i'), \min(j,j')\, ) \in D \ =
.
$$
Then Theorems 4 and 8 are valid as well:
the fourth character formula is true for $S_D^B$, and the
variety has a Zelevinsky resolution smaller than the Bott-Samelson one.


\begin{thebibliography}{99}
%
\bibitem{BFZ} A. Berenstein, S. Fomin, and A. Zelevinsky,
{\em Parametrizations of canonical bases and totally positive matrices},
preprint 1995.
%
%
%  \bibitem{BGG} I.N. Bernstein, I.M. Gelfand, and S.I. Gelfand,
%  {\em Schubert cells and cohomology of the the spaces $G/P$},
%  Russ. Math. Surv. {\bf 28} (1973), 1-26.
%
% \bibitem{GGG} I.M. Gelfand, Yu.R. Gelfand, and E.S. Gelfand,
% {\em A Note on Gelfand's identity},
% Internat. J. of Gelf. {\bf 99} (2033).
%
 \bibitem{De1} M. Demazure, {\em D\'{e}singularisation des
 vari\'{e}t\'{e}s
 de Schubert g\'{e}n\'{e}ralis\'{e}s}, Ann. Sci. Ec. Norm. Sup
  {\bf 7} (1974), 53-88.
%
\bibitem{Dem} M. Demazure, {\em Une nouvelle formule des
caract\`{e}res}, Bull. Sci. Math. (2)
 {\bf 98} (1974), 163-172.
%
% \bibitem{Ed} P. Edelman,
% {\em Lexicographically first reduced words},
% to appear.
%
% \bibitem{Ful} W. Fulton, ``Young Tableaux with Applications'',
% to appear.
%
% \bibitem{Hil} H. Hiller, The Geometry of Coxeter Groups,
% Res. Notes in Math {\bf 54}, C.C. Pittman, Toronto (1982).
%
% \bibitem{Jantzen} J.C. Jantzen,
% Representations of Algebraic Groups,
% Pure and App. Math {\bf 116}, Academic Press, London (1987).
%
\bibitem{KP} W. Kraskiewicz and P. Pragacz, {\em Foncteurs de Schubert},
C.R. Acad. Sci. Paris {\bf 304} Ser I No 9 (1987), 207-211.
%
\bibitem{LS1} A. Lascoux and M.-P. Schutzenberger,
{\em Polynomes de Schubert}, C. R. Acad. Sci. Paris {\bf 294}
(1982), 447-450.
%
\bibitem{LS2} A. Lascoux and M.-P. Schutzenberger,
{\em Tableaux and non-commutative Schubert polynomials},
Funkt. Anal. {\bf 23}
(1989), 63-64.
%
\bibitem{LS3} A. Lascoux and M.-P. Schutzenberger,
{\em Keys and Standard Bases}, Tableaux and Invariant Theory,
IMA Vol. in Math. and App. {\bf 19},
ed. D. Stanton (1990), 125-144.
%
\bibitem{LZ} B. LeClerc and A. Zelevinsky, {\em Chamber sets}, preprint.
%
\bibitem{Lit} P. Littelmann, {\em A Littlewood-Richardson
rule for Symmetrizable Kac-Moody algebras}, Inv. Math.
{\bf 116} (1994), 329-346.
%
\bibitem{MacD} I.G. Macdonald, Notes on Schubert Polynomials,
Pub. LCIM {\bf 6}, Univ. du Qu\'{e}bec a Montr\'{e}al, 1991.
%
\bibitem{Mag1} P. Magyar, {\em Borel-Weil theorem for
configuration varieties and Schur modules}, preprint
alg-geom/9411014.
%
\bibitem{Mag2} P. Magyar,
{\em Bott-Samelson Varieties and
Configuration Spaces}, preprint alg-geom/9611019.
%
\bibitem{RS1} V. Reiner and M. Shimozono, {\em Specht series for
column-convex diagrams}, to appear in J. Algebra.
%
\bibitem{RS2} V. Reiner and M. Shimozono, {\em Key polynomials
and a flagged Littlewood-Richardson rule}, J. Comb. Th. Ser. A
{\bf 70} (1995), 107-143.
%
\bibitem{Zel} A. Zelevinsky, {\em Small resolutions of singularities
of Schubert varieties},  Funct. Anal. App. {\bf 17} (1983), 142-144.
%
%  \bibitem{}  , {\em  },  {\bf  } (19), .
%
\end{thebibliography}





\end{document}

------=_NextPart_000_000A_01BEB7DF.314443E0--


